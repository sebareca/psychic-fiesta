\documentclass[a4paper,oneside,fleqn,11pt,../tesis.tex]{subfiles}

\newcommand{\yetter}{\prescript{H}{H}{\mathcal{YD}}}
\newcommand{\yetterg}{\prescript{\field G}{\field G}{\mathcal{YD}}}

\begin{document}

El objetivo de este capitulo es introducir el concepto de álgebra de Nichols y exponer algunos ejemplos, entre 
ellos el super plano de Jordan que será nuestro objeto de estudio. Para poder hacer esto, primero haremos una introducción a las álgebras
de Hopf. A partir de ahora y por el resto de la tesis $\ox$
denota el producto tensorial sobre $\field$.

\section{Definiciones básicas}
Esta sección se basa en \cite{Mon}. 

\subsection{Álgebras y coálgebras}

Empecemos definiendo $\field$-álgebra de una manera alternativa a como usualmente se define en un
curso básico de estructuras algebraicas con el propósito de definir luego de manera análoga $\field$-coálgebras. 

\begin{definition}\label{defalgebra} Una $\field$-\emph{álgebra} con unidad es un $\field$-espacio vectorial $A$ junto con dos
transformaciones $\field$-lineales
\[
	m: A \ox A \to A \quad\text{y}\quad \eta: \field \to A
\]
tales que los siguientes diagramas conmutan:
\begin{enumerate}[(a)]
\item (asociatividad)
\begin{align*}
\xymatrix@=5em{
	A \ox A \ox A \ar[r]^{m \ox id} \ar[d]^{id \ox m} & A \ox A \ar[d]^{m} \\
	A \ox A \ar[r]^{m} & A
}
\end{align*}
\item (unidad)
\begin{align*}
\xymatrix@=3em{
	& A \ox A \ar[dd]^{m} & \\
	\field \ox A \ar[ur]^{\eta \ox id} \ar[dr] & & A \ox \field \ar[ul]_{id \ox \eta} \ar[dl] \\
	& A &
}
\end{align*}
\end{enumerate}
El morfismo $m$ se llama \emph{multiplicación} y el morfismo $\eta$ \emph{unidad}. 
Notar que el elemento unidad de $A$ viene dado por $1_{A} = \eta(1_{\field})$.
\end{definition}

\begin{definition}
	Sean $V$ y $W$ dos $\field$-espacios vectoriales. Definimos la \emph{transposición}
	\[
		\tau_{V,W}: V \ox W \to W \ox V
	\]
	como la única transformación $\field$-lineal tal que
	$\tau_{V,W}(v \ox w) = w \ox v$ para todo $v \in V$ y $w \in W$.
\end{definition}

Notar que $A$ es conmutativa si y solo si $m \circ \tau_{A, A} = \tau_{A, A}$.
\begin{obs}
	Si $A$ y $B$ son dos $\field$-álgebras, entonces $A \ox B$
	también lo es con multiplicación $m_{A \ox B}$ definida en los elementos homogéneos como
	\[
		m_{A \ox B} ((a_1 \ox b_1) \ox (a_2 \ox b_2)) = a_1 a_2 \ox b_1 b_2
	\]
	y unidad $\eta_{A \ox B}$ definida como la única transformación $\field$-lineal tal que
	\[
		\eta_{A \ox B}(1_{\field}) = \eta_A(1_{\field}) \ox \eta_B(1_{\field}).
	\]
\end{obs}

Esta reformulación de la definición de $\field$-álgebra mediante morfismos y diagramas nos permite dualizar fácilmente esta estructura.

\begin{definition}\label{defcoalgebra}
	Una $\field$-\emph{coálgebra} con unidad es un $\field$-espacio vectorial $C$
	junto con dos transformaciones $\field$-lineales
	\[
		\Delta: C \to C \ox C \quad\text{y}\quad \varepsilon: C \to \field
	\]
	tales que los siguientes diagramas conmutan:
	\begin{enumerate}[(a)]
		\item (coasociatividad)
		\begin{align*}
		\xymatrix@=5em{
			C \ar[r]^{\Delta} \ar[d]^{\Delta} & C \ox C \ar[d]^{\Delta \ox id}\\
			C \ox C \ar[r]^{id \ox \Delta} & C \ox C \ox C
		}
		\end{align*}
		\item (counidad)
		\begin{align*}
		\xymatrix@=3em{
			& C \ar[dd]^{\Delta} \ar[dl] \ar[dr] & \\
			\field \ox C  & & C \ox \field \\
			& C \ox C \ar[ul]^{\veps \ox id} \ar[ur]_{id \ox \veps} &
		}
		\end{align*}
	\end{enumerate}
	El morfismo $\Delta$ se llama \emph{comultiplicación} y el morfismo $\veps$ se llama \emph{counidad}.
	Decimos que $C$ es \emph{coconmutativo} si $\tau_{C \ox C} \circ \Delta = \Delta$.
\end{definition}

\begin{definition} \label{defmorfcoalegbras}
	Sean $C$ y $D$ coálgebras con comultiplicaciónes $\Delta_C$ y $\Delta_D$ y counidades $\veps_C$ y $\veps_D$ respectivamente.
	Decimos que $f:C \to D$ es un \emph{morfismo de coálgebras} si es una transformación $\field$-lineal y los siguientes diagramas
	conmutan.
	\begin{align*}
	\xymatrix@=5em{
		C \ar[r]^{f} \ar[d]^{\Delta_C} & D \ar[d]^{\Delta_D} && C \ar[r]^{f} \ar[d]^{\veps_C} & D \ar[dl]^{\veps_D} \\
		C \ox C \ar[r]^{f \ox f} & D\ox D && \field	
	}
	\end{align*}
\end{definition}
Notar que si damos vuelta las flechas en estos diagramas, cambiamos comultiplicación por multiplicación y counidad
por unidad obtenemos la definición de morfismo de álgebras.

\begin{definition}
	Sea $C$ una coálgebra. Un subespacio $I \subseteq C$ es un \emph{coideal} si 
	\[
		\Delta(I) \subseteq I \ox C + C \ox I \text{ y } \veps(I) = 0.
	\]
\end{definition}

Si $I$ es un coideal, el espacio vectorial $C/I$ con la comultiplicación y la counidad
inducidas por $\Delta$ y $\veps$ respectivamente es una coálgebra.

\begin{definition}
	Sea $C$ una coálgebra. La \emph{coálgebra coopuesta} $C^{cop}$ tiene a $C$ como espacio vectorial subyacente, 
	comultiplicación $\Delta^{cop} = \tau_{C \ox C} \circ \Delta$ y counidad $\veps^{cop} = \veps$.
\end{definition}

\begin{obs}
	Análogo a lo que sucede para las álgebras, si $C$ y $D$ son coálgebras, entonces $C \ox D$ es una coálgebra
	con comultiplicación
	\[
		\Delta_{C \ox D} = (id \ox \tau_{C \ox D} \ox id) \circ (\Delta_C \ox \Delta_D)
	\] y counidad \[
		\veps_{C \ox D} = m_{\field} \circ (\veps_C \ox \veps_D).
	\]
\end{obs}
\subsection{Biálgebras}
Combinaremos ahora la noción de álgebra y coálgebra.
\begin{definition}
	Un $\field$-espacio vectorial $B$ es una \emph{biálgebra} si $\left(B, m, \eta\right)$ es un álgebra,
	$\left(B, \Delta, \veps \right)$ es una coálgebra y alguna de las siguientes condiciones equivalentes
	se cumple:
	\begin{itemize}
		\item $\Delta$ y $\veps$ son morfismos de álgebras
		\item $m$ y $\eta$ son morfismos de coálgebras.
	\end{itemize}
\end{definition}

\begin{definition}
	Sean $B$ y $E$ dos biálgebras. Decimos que $f: B \to E$ es un \emph{morfismo de biálgebras} si
	es un morfismo de álgebras y de coálgebras.
\end{definition}

\begin{example}
	Sea $G$ un grupo. El álgebra de grupo $\field G$ resulta ser una biálgebra definiendo $\Delta(g) = g \ox g$
	y $\veps(g) = 1$ para todo $g \in G$. Esta biálgebra resulta ser coconmutativa pero solo es conmutativa
	cuando $G$ es abeliano. 
\end{example}

\begin{example}
	Sea $\mathfrak{g}$ un álgebra de Lie. El álgebra envolvente $U(\mathfrak{g})$ es una biálgebra definiendo
	$\Delta(x) = x \ox 1 + 1 \ox x$ y $\veps(x) = 0$ para todo $x \in \mathfrak{g}$. Esta biálgebra también
	resulta ser coconmutativa y solo es conmutativa cuando $\mathfrak{g}$ es abeliana.
\end{example}

Estos dos ejemplos motivan las siguientes definiciones.

\begin{definition}
	Sea $C$ una coálgebra y $c \in C$. Decimos que $c$ es \emph{tipo grupo}
	si
	\[
		\Delta(c) = c \ox c
	\]
\end{definition}

\begin{definition}
	Sean $C$ una coálgebra y $c \in C$. Decimos que $c$ es \emph{primitivo}
	si
		\[\Delta(c) = c \ox 1 + 1 \ox c\]
\end{definition}

\begin{example}\label{exampleqp}
	Sean $q \in \field$, $q \neq 0$ y $B = \field \left\langle x, y \mid xy = qyx \right\rangle$. Si definimos
	$\Delta(x) = x \ox x$, $\Delta(y) = y \ox 1 + 1 \ox y$, $\veps(x) = 1$ y $\veps(y) = 0$, $B$ resulta ser una
	biálgebra conocida como el \emph{plano cuántico}. Notar que $x$ es un elemento tipo grupo e $y$ es primitivo.
	El conjunto $\{ y^ix^j : i, j \geq 0\}$ es una base de $B$ como $\field$-espacio vectorial.
\end{example}

\begin{definition}
	Sea $B$ una biálgebra. Un subespacio $I \subseteq B$ es un \emph{biideal} de $B$ si es un ideal y un coideal
	simultáneamente.
\end{definition}

\begin{example}
	Sea $B$ el plano cuántico del Ejemplo \ref{exampleqp}. El ideal bilátero $I \subseteq B$ generado por $y$
	resulta ser un coideal y por lo tanto un biideal. Para ver esto
	primero verifiquemos que $\Delta(I) \subseteq I \ox B + B \ox I$. Sea $p \in I$, como $\Delta$ es lineal
	podemos suponer que $p$ es un elemento de la base, es decir $p = y^nx^m$, con $n \geq 1$. Luego
	\begin{align*}
		\Delta(p) &= \Delta(	\lambda y^nx^m) = \lambda \Delta(y)^n \Delta(x)^m  \\
		&= \lambda ((y\ox 1 + 1 \ox y)^n) (x^m \ox x^m)
			= \lambda \left(\sum_{i = 0}^n y^i \ox y^{n - i}\right) (x^m \ox x^m) \\
		&= \lambda \sum_{i = 0}^n y^ix^m \ox y^{n - i}x^m 
	\end{align*}
	Como $y^ix^m \ox y^{n - i}x^m \in I \ox B$ para todo $i \geq 1$ y $x^m \ox y^nx^m \in B \ox I$,
	obtenemos que $\Delta(p) \in I \ox B + B \ox I$. Es fácil ver que $\veps(I) = 0$ usando que $\veps$
	es morfismo de álgebras. 
\end{example}

\subsection{Notación de Sweedler y producto de convolución}
Antes de continuar con nuevas definiciones y ejemplos vamos a introducir una nueva notación muy útil a la hora
de trabajar con coálgebras. Si $C$ es una coálgebra y $c \in C$, $\Delta(c) \in C \ox C$, es decir que
$c = \sum_{i = 1}^n a_i \ox b_i$, donde $a_i$, $b_i \in C$ para todo $i$. Lo que sugiere Sweedler es que no usemos nuevas
letras $a$ y $b$ sino subíndices $c_{(1)}$, $c_{(2)}$ de modo que
\[
	\Delta(c) = \sum_{i = 1}^n c_{(1)i} \ox c_{(2)i}.
\]

Para ciertas manipulaciones que involucren operaciones lineales no necesitamos tener en cuenta el índice $i$ de la
sumatoria y por lo tanto podemos escribir:

\[
	\Delta(c) = \sum c_{(1)} \ox c_{(2)}.
\]

o más aún podemos no escribir el símbolo de la suma. Esta notación se vuelve muy útil cuando aplicamos $\Delta$
varias veces. La coasociatividad de la comultiplicación dice (en notación de Sweedler) que
\[
	c_{(1)} \ox {c_{(2)}}_{(1)} \ox {c_{(2)}}_{(2)} = (id \ox \Delta) \circ \Delta (c)
	= (\Delta \ox id) \circ \Delta(c) = {c_{(1)}}_{(1)} \ox {c_{(1)}}_{(2)} \ox c_{(2)}.
\]
Esto nos permite escribir
\[
	(\Delta \ox id) \circ \Delta (c) = (id \ox \Delta) \circ \Delta(c) = c_{(1)} \ox c_{(2)} \ox c_{(3)}
\]
y más en general
\[
	\Delta_n(c) = c_{(1)} \ox c_{(2)} \ox \ldots \ox c_{(n + 1)},
\]
donde $\Delta_1 = \Delta$ y $\Delta_{n + 1}(c) = (\Delta \ox id^{\ox^n}) \circ \Delta_{n}$.

Utilizando esta nueva notación el axioma de counidad se puede expresar de la siguiente forma:
\[
	c = \veps(c_{(1)}) c_{(2)} = \veps(c_{(2)}) c_{(1)}.
\]

\begin{definition}
	Sea $\left(A, m, \eta \right)$ un álgebra y $\left(C, \Delta, \veps \right)$ una coálgebra. Definimos el
	\emph{producto convolución} $\ast$ sobre $\Hom_{\field}(C, A)$ de la siguiente manera: Dados $f$, $g$ $\in \Hom_{\field}(C, A)$,
	\[
		f \ast g = m \circ (f \ox g) \circ \Delta. 	
	\]
	En notación de Sweedler el producto esta dado por
	\[
		(f \ast g)(c) = f(c_{(1)})g(c_{(2)}) \text{ para todo } c \in C.
	\]
\end{definition}

Este producto y la unidad $\eta \circ \veps$ le dan a $\Hom_{\field}(C, A)$ estructura de álgebra.

\subsection{Álgebras de Hopf}

\begin{definition}
	Una biálgebra $\left(H, m, \eta, \Delta, \veps \right)$ es una $\field$-\emph{álgebra de Hopf} si existe
	un morfismo $S \in \Hom_{\field}(H, H)$, llamado \emph{antípoda}, tal que
	\[
		S \ast id_{H}  = id_{H} \ast S = \eta \circ \veps.
	\]
	Es decir, $S$ es la inversa de $id_H$ para el producto convolución.
\end{definition}

En notación de Sweedler la antípoda $S$ cumple que 
\[
	S(h_{(1)})h_{(2)} = \veps(h) 1_{H} = h_{(1)}S(h_{(2)}) \text{ para todo } h \in H.
\]

\begin{definition}
	Sean $H, K$ álgebras de Hopf con antípodas $S_H$ y $S_K$ respectivamente. Decimos que $f: H \to K$
	es un \emph{morfismo de álgebras de Hopf} si es un morfismo de biálgebras y $f \circ S_H = S_K \circ f$.
\end{definition}

\begin{definition}
	Un subespacio $I \subseteq H$ es un \emph{ideal de Hopf} si es un biideal y $S(I) \subseteq I$. En este caso
	$H / I$ es un álgebra de Hopf con la estructura inducida por $H$.
\end{definition}

\begin{example}
	El álgebra de grupo $H = \field G$ es un álgebra de Hopf definiendo $Sg = g^{-1}$ para todo $g \in G$. En general
	si $H$ es un álgebra de Hopf y $h \in H$ es un elemento tipo grupo, entonces $S(h) = h^{-1}$ debido a que
	$S(h)h = 1_{H} \veps(h) = 1_{H}$.
\end{example}

\begin{example}
	El álgebra asociativa envolvente $H = U(\mathfrak{g})$ es un álgebra de Hopf definiendo $S(x) = -x$ para todo $x \in \mathfrak{g}$.
	En general si $H$ es un álgebra de Hopf y $h \in H$ es un elemento de primitivo, entonces $S(h) = -h$.
\end{example}

\begin{example}
	(Álgebra de Taft) Hasta ahora todos los ejemplos que dimos son coconmutativos. El álgebra de Hopf más chica no conmutativa y no coconmutativa
	tiene dimensión 4 y es única si $car(\field) \neq 2$. Se trata del álgebra
	\[
		\field \left\langle 1, g, x \mid g^2, x^2, xg + gx \right\rangle,
	\]
	con estructura de álgebra de Hopf dada además por $\Delta(g) = g \ox g$, $\Delta(x) = x \ox 1 + g \ox x$,
	$\veps(g) = 1$, $\veps(x) = 0$, $S(g) = g = g^{-1}$ y $S(x) = -gx$.
\end{example}

Las siguientes dos propiedades de la antípoda son muy importantes. Daremos la demostración de la primera para mostrar la utilidad de
la notación de Sweedler.

\begin{prop} Sea $H$ un álgebra de Hopf con antípoda $S$.
	\begin{enumerate}[(1)]
		\item La antípoda $S$ es un antimorfismo de álgebras, es decir,
		\[
			S(hk) = S(k)S(h) \text{ para todo }	h,k \in H \text{ y } S(1_{H}) = 1_{H}.	
		\]
		\item La antípoda $S$ es un antimorfismo de coálgebras, es decir,
		\[
			\Delta \circ S = \tau \circ	(S \ox S) \circ \Delta \text{ y } \veps \circ S = \veps.	
		\]
	\end{enumerate}
\end{prop}
\begin{proof}
	Notemos primero que como $H \ox H$ es una coálgebra, $\Hom_{\field}(H \ox H, H)$ es un álgebra con el producto de convolución.
	Sean $f$ y $g \in \Hom_{\field}(H \ox H, H)$ los únicos morfismos $\field$-lineales que cumplen que
	$f(h \ox k) = S(hk)$ y $g(h \ox k) = S(k) S(h)$ para todo $h$, $k \in H$. Para ver que $f = g$ vamos a probar que $f$ es la inversa
	a izquierda de $m_{H}$ y $g$ la inversa a derecha, es decir,
	\[
		f \ast m_{H} = m_{H} \ast g = \eta_{H} \circ \veps_{H \ox H}.
	\]
	Sean $h$ y $k \in H$,
	\begin{align*}
		\left(f \ast m_H\right)\left(h \ox k\right) &= f\left((h \ox k)_{(1)}\right)m_H\left(((h \ox k)_{(2)}\right)
			= f\left(h_{(1)} \ox k_{(1)}\right) m_{H}\left(h_{(2)} \ox k_{(2)}\right) \\
		&= S\left(h_{(1)} k_{(1)}\right)h_{(2)}k_{(2)} = S\left((hk)_{(1)}\right)(hk)_{(2)} \\
		&= \eta_H \circ \veps_{H}(hk) = \eta_{H} \circ \veps_{H\ox H} (h \ox k). 
	\end{align*}
	De manera similar,
	\begin{align*}
		(m_{H} \ast g)(h \ox k) &= m_{H}\left((h \ox k)_{(1)}\right)g\left((h \ox k)_{(2)}\right) \\
		&= m_{H}\left(h_{(1)} \ox k_{(2)}\right)g\left(h_{(1)} \ox k_{(2)}\right)
			= h_{(1)}k_{(1)}S\left(k_{(2)}\right)S\left(h_{(2)}\right) \\
		&= h_{(1)} (\eta \circ \veps) (k) S\left(h_{(2)}\right) = h_{(1)} S\left(h_{(2)}\right) \veps(k) 1_{H} \\
		&= \veps(h) \veps(k) 1_{H} =\eta_{H} \circ \veps_{H\ox H} (h \ox k).
	\end{align*}
	Por lo tanto $f = g$. Para ver que $S(1_{H}) = 1_{H}$ alcanza con evaluar la expresión
	\[
		S \ast id_{H} = \eta \circ \veps
	\]
	en $1_{H}$. Esto prueba (1), para probar (2) se utiliza un argumento similar.
\end{proof}

\subsection{Módulos y comódulos}

Al igual que para las coálgebras primero vamos a dar una definición alternativa de módulo que facilite dualizar
la estructura.

\begin{definition}
	Sea $A$ una $\field$-álgebra. Un $A$-\emph{módulo} es un $\field$-espacio vectorial M junto con una transformación
	$\field$-lineal
	\[
		\gamma: A \ox M \to M	
	\]
	tal que los siguientes diagramas conmutan:
	\begin{align*}
		\xymatrix@=5em{
			A \ox A \ox M \ar[r]^{m_A \ox id} \ar[d]^{id \ox \gamma} & A \ox M \ar[d]^{\gamma}
				& \field \ox M \ar[rd] \ar[r]^{\eta \ox id} & A \ox M \ar[d]^{\gamma}\\
			A \ox M \ar[r]^{\gamma} & M
				&& M
		}
	\end{align*}
	Denotamos por $\prescript{}{A}{\mathcal{M}}$ a la categoría de $A$-módulos a izquierda. Los $A$-módulos a derecha
	se definen de manera análoga.
\end{definition}

\begin{definition}
	Sea $C$ una $\field$-coálgebra. Un $C$-\emph{comódulo a derecha} es un $\field$-espacio vectorial $M$
	junto con una transformación $\field$-lineal
	\[
		\rho: M \to M \ox C	
	\]	
	tal que los siguientes diagramas conmutan:
	\begin{align*}
		\xymatrix@=5em{
			M \ar[r]^{\rho} \ar[d]^{\rho} & M \ox C \ar[d]^{id \ox \Delta}
				& M \ar[rd] \ar[r]^{\rho} & M \ox C \ar[d]^{id \ox \veps}\\
			M \ox C \ar[r]^{\rho \ox id} & M \ox C \ox C
				&& M \ox \field
		}
	\end{align*}
	Denotamos por $\mathcal{M}^{C}$ a la categoría de $C$-comódulos a derecha.
\end{definition}

La notación de Sweedler se puede extender a comódulos a derecha escribiendo
\[
	\rho(m) = m_{(0)} \ox m_{(1)},
\]
manteniendo la convención de que $m_{(i)} \in C$ para todo $i \neq 0$. El primer diagrama
de la definición de comódulo dice que $(id \ox \Delta) \circ \rho = (\rho \ox id) \circ \rho$.
Si escribimos esta condición en notación de Sweedler obtenemos que
\[
	{m_{(0)}}_{(0)} \ox {m_{(0)}}_{(1)} \ox m_{(1)} = m_{(0)} \ox {m_{(1)}}_{(1)} \ox {m_{(1)}}_{(2)}.
\]
Esto permite escribir
\[
	(id \ox \Delta) \circ \rho(m) = (\rho \ox id) \circ \rho(m) = m_{(0)} \ox m_{(1)} \ox m_{(2)}.
\]
El segundo diagrama dice, en notación de Sweedler, que
\[
	m = m_{(0)} \veps(m_{(1)}).
\]
De manera análoga se define $\field$-\emph{comódulo a izquierda} mediante un morfismo
\[
	\rho': M \to C \ox M.
\]
En este caso escribimos $\rho'(m) = m_{(-1)} \ox m_{(0)}$.

\begin{definition}
	Sean $M$ y $N$ dos $C$-comódulos a derecha con morfismos de estructura $\rho_M$ y $\rho_N$ respectivamente.
	Una transformación $\field$-lineal $f: M \to N$ se dice \emph{morfismo de $C$-comódulos a derecha}
	si $\rho_N \circ f = (f \ox id) \circ \rho_M$.
\end{definition}

\begin{example}
	Sea $C$ una coálgebra, el espacio vectorial $M = C$ es un cómodulo a derecha definiendo $\rho = \Delta$.
\end{example}


\begin{example} \label{exgroupgrading}
	Si $G$ es un grupo y $M$ un $\field$-espacio vectorial, entonces $M$ es un $\field G$-comódulo si y solo si $M$ es $G$-graduado.
	Para ver esto empecemos suponiendo que $M$ es un $\field G$-comódulo. Debido a que los elementos de $G$ forman
	una base de $\field G$, podemos escribir $\rho(m) = \sum_{g \in G} m_g \ox g$ de manera única para todo $m \in M$.
	Esto define, para cada $g \in G$, el subespacio $M_g = \left\lbrace m_g \mid m \in M\right\rbrace$. Como
	\begin{align*}
		\xymatrix@=3em{
			m \ar@{|->}[r]^(.3){\rho} & \sum_{g \in G} m_g \ox g \ar@{|->}[r]^{id \ox \veps}
				& \sum_{g \in G} m_g \ox 1 \ar@{|->}[r]^(.5){\sim} & \sum_{g \in G} m_g
		}
	\end{align*}		 
	para todo $m \in M$ y esta composición de morfismos resulta ser la identidad, entonces $M = \sum_{g \in G} M_g$.
	Nos resta ver que la suma es directa. Si usamos que
	\[
	(id \ox \Delta) \circ \rho = (\rho \ox id) \circ \rho
	\]
	obtenemos que para todo $m \in M$
	\[
	\sum_{g \in G} m_g \ox g \ox g  = \sum_{g \in G}	\sum_{h \in G} \left({m_g}\right)_{h} \ox g \ox h. 
	\]
	Como los elementos de $G$ forman una base de $\field G$, entonces $({m_g})_h = \delta_{g, h} m_{g}$ y luego, $M = \oplus_{g \in G} M_g$.
	Si empezamos suponiendo que $M = \oplus_{g \in G} M_g$ definimos $\rho(m) = m \ox g$ para todo $m \in M$.
\end{example}

\subsection{Invariantes y coinvariantes}
Dado un grupo $G$ actuando sobre un $\field$-espacio vectorial $M$, los invariantes y los coinvariantes de $M$ por esta acción son
isomorfos cuando el orden de $G$ no es cero en $\field$. Vamos a ver como estos objetos se generalizan al caso de la acción
de un álgebra de Hopf.

\begin{definition}
	\begin{enumerate}[(1)]
		\item
		Sean $H$ un álgebra de Hopf y $M$ un $H$-módulo a izquierda. Los \emph{invariantes de $M$ por las acción de $H$} son los elementos $m \in M$
		tales que $h \cdot m = \veps(h)m$ para todo $h \in H$. Este conjunto lo denotamos por $M^{H}$.
		\item
		Sean $H$ un álgebra de Hopf y $M$ un $H$-comódulo a derecha. Los \emph{coinvariantes de $H$ en $M$} son los elementos $m \in M$
		tales que $\rho(m) = m \ox 1_{H}$. Este conjunto los denotamos por $M^{coH}$
	\end{enumerate}
\end{definition}

\begin{example}
	Sea $H = \field G$. Si $M$ es un $H$-módulo a izquierda, entonces $M^{H} = M^{G}$. Si $M$ es un $H$-comódulo a derecha, entonces
	$M^{coH} = M_1$, donde $M_1$ es la componente identidad de la graduación del Ejemplo \ref{exgroupgrading}. 
\end{example}

\begin{example}
	Sea $H = U(\mathfrak{g})$. Si $M$ es un $H$-módulo a izquierda, entonces
	\[
		M^{H} = \left\lbrace m \in M \mid x \cdot m = 0 \text{ para todo } x \in \mathfrak{g} \right\rbrace.
	\]
\end{example}

\subsection{Producto tensorial de módulos y comódulos}

Sea $H$ un álgebra de Hopf y sean $V$ y $W$ dos $H$-módulos a izquierda. El espacio vectorial $V \ox W$ resulta ser un $H$-módulo a izquierda definiendo
\[
	h \cdot (v \ox w) = h_{(1)} \cdot v \ox h_{(2)} \cdot w.	
\]
para todo $h \in H$, $v \in V$ y $w \in W$. Lo mismo sucede para $H$-módulos a derecha.
	
Si $H$ es coconmutativo, entonces $V \ox W \cong W \ox V$, donde el isomorfismo está dado por la transposición $\tau_{V, W}$.
Si removemos la hipótesis de que $H$ sea coconmutativo este resultado es falso. Por ejemplo, sea $H = \Hom_{\field}\left(\field G, \field\right)$,
donde $G$ es un grupo finito no conmutativo. Veamos primero que $H$ es un álgebra de Hopf. Su estructura de álgebra está dada por 
$m_{H}\left(f \ox g\right)(x) = f(x)g(x)$ y $\eta_{H}(\lambda) = \lambda\veps_{\field G}$ para todo $f, g \in H$, $x \in G$ y $\lambda \in \field$.
Para describir la estructura de coálgebra consideramos la base $\left\lbrace p_g \mid g \in G \right\rbrace$ de $H$, dual a la base de elementos de $G$, es decir, $p_g(h) = \delta_{g, h}$ para todo $g, h \in G$. La comultiplicación se define en esta base como $\Delta_H(p_g) = \sum_{h \in G} p_h \ox p_{h^{-1}g}$ y se extiende linealmente a $H$. La counidad $\veps_{H}$ está dada por $\veps_{H}(f) = f \circ \eta_{\field G}(1_{\field})$. Por último la
antípoda se define como $S_{H} = S_{\field G}^{\ast}$. Sean $g, h \in G$ tales que $gh \neq hg$ y sean
$\left\langle p_g \right\rangle \subseteq H$ y $\left\langle p_h \right\rangle \subseteq H$ los subespacios generados por $p_g$ y $p_h$ respectivamente.
Estos subespacios son $H$-módulos con la acción dada por la multiplicación a izquierda. Es fácil ver que
$p_{gh}(\left\langle p_g \right\rangle \ox \left\langle p_h \right\rangle) = \left\langle p_g \right\rangle \ox \left\langle p_h \right\rangle$ y que
$p_{gh}(\left\langle p_h \right\rangle \ox \left\langle p_g \right\rangle) = 0$. Luego
$\left\langle p_g \right\rangle \ox \left\langle p_h \right\rangle \ncong \left\langle p_h \right\rangle \ox \left\langle p_g \right\rangle$
como $H$-módulos.

Vimos que si $H$ es un álgebra de Hopf y $V$ y $W$ son dos $H$-módulos a izquierda, entonces $V\ox W$ resulta ser un $H$-módulo a izquierda y dimos
una formula para la acción. Reescribamos esta formula en términos de morfismos para que resulte más sencillo dualizar. Si $\phi_{V}: H \ox V \to V$ y 
$\phi_W: H \ox W \to W$ son las acciones de $H$ sobre $V$ y $W$ respectivamente, entonces la acción de $H$ sobre $V \ox W$ esta dada por
\[
	\phi_{V \ox W} = \left(\phi_V \ox \phi_W \right) \circ \left(id \ox \tau_{H, V} \ox id \right) \circ \left(\Delta \ox id \ox id \right).
\]

Sea $H$ un álgebra de Hopf y sean $V$ y $W$ dos $H$-comódulos a derecha con coacciones $\rho_V$ y $\rho_W$ respectivamente. El espacio vectorial $V \ox W$
resulta ser un $H$-comódulo a derecha mediante el morfismo
\[
	\rho_{V\ox W} = \left(id \ox m\right) \circ \left(id \ox \tau_{H, W} \ox id\right) \circ \left(\rho_V \ox \rho_W \right).
\]
En notación de Sweedler $\rho(v\ox w) = v_{(0)} \ox w_{(0)} \ox v_{(1)} w_{(1)}$ para todo $v \in V$ y $w \in W$.

\subsection{Módulo álgebras, comódulo álgebras y productos smash}

A partir de ahora todos los módulos son módulos a izquierda y todos los comódulos son comódulos a derecha.

\begin{definition}
	Sea $H$ un álgebra de Hopf. Un álgebra $A$ es un $H$-\emph{módulo álgebra} si:
	\begin{enumerate}[(1)]
		\item $A$ es un $H$-módulo con acción $\cdot: H \ox A \to A$,
		\item $h \cdot (ab) = (h_{(1)}a)(h_{(2)}b)$,
		\item $h \cdot 1_A = \veps(h)1_A$,
	\end{enumerate}
	para todo $h \in H$, $a, b \in A$.
	Las condiciones (2) y (3) dicen que la multiplicación $m_A$ y la unidad $\eta_A$ son morfismos de $H$-módulos.
\end{definition}

\begin{definition}
	Sea $H$ un álgebra de Hopf. Un álgebra $A$ es un $H$-\emph{comódulo álgebra} si:
	\begin{enumerate}
		\item $A$ es un $H$-comódulo con coacción $\rho: A \to A\ox H$,
		\item la multiplicación $m_A$ y la unidad $\eta_A$ son morfismos de $H$-comódulos.
	\end{enumerate}
	En notación de Sweedler la condición (2) dice que
	\[
		(ab)_{(0)}\ox (ab)_{(1)} = \rho(ab) = a_{(0)}b_{(0)} \ox a_{(1)}b_{(1)}
	\] para todo $a, b \in A$ y que $\rho(1) = 1\ox 1$.
\end{definition}

\begin{definition}
	Sean $H$ un álgebra de Hopf y $A$ un $H$-módulo álgebra.
	Se define el \emph{producto smash} $A\#H$ de la siguiente manera:
	\begin{enumerate}[(1)]
		\item El espacio vectorial subyacente es $A \ox H$ y escribimos $a\#h$ en vez de $a\ox h$.
		\item La multiplicación esta dada en los tensores elementales por
		\[
			(a\#h)(b\#k) = a(h_{(1)}b)\#h_{(2)}k.
		\]
	\end{enumerate}
\end{definition}

\begin{example}
	Sean $H$ un álgebra de Hopf y $A$ un álgebra. La acción trivial $h \cdot a = \veps(h) a$ para todo $h \in A$
	y $a \in A$ le da estructura a $A$ de $H$-módulo álgebra. Es fácil ver que en este caso $A\#H \cong A\ox H$.
\end{example}

\begin{example}
	Sea $A$ un $\field G$-módulo álgebra. Como $\Delta(g) = g \ox g$ para todo $g \in G$ y $m_A$ es un morfismo de $\field G$-módulos,
	entonces $g \cdot (ab) = (g \cdot a)(g \cdot b)$ para todo $g \in G$ y $a,b \in A$.
	Es decir que cada $g$ actúa como endomorfismo de $A$. Más aún como
	$g$ es inversible, $g$ actúa como un automorfismo. Luego la estructura de $\field G$-módulo álgebra de $A$ induce
	un morfismo de grupos $G \to \Aut_{\field}A$. Recíprocamente un morfismo de grupos de $G$ en los automorfismos
	de $A$ induce una estructura de $\field G$-módulo álgebra en $A$. En este caso $A \# \field G$ es el producto semidirecto
	de $A$ con $G$.
\end{example}

\begin{example}
	Sea $A$ un $\field G$-comódulo álgebra. Ya sabemos que $A = \oplus_{g \in G} A_g$ como $\field$-espacio vectorial
	y que para todo $a_g \in A_g$, $\rho(a_g) = a_g \ox g$. Como $m_A$ es un morfismo de $H$-comódulos, entonces
	$\rho(a_g b_h) = a_g b_h \ox gh$, es decir, $a_g b_h \in A_{gh}$ o lo que es lo mismo $A_g A_h \subseteq A_{gh}$.
	Más aún como $\eta_A$ es un morfismo de $H$-comódulos, entonces $\rho(1_A) = 1_A \ox 1_G$ y por lo tanto
	$A$ es un álgebra $G$-graduada. 
\end{example}

\begin{example}
	Sea $A$ un $U(\mathfrak{g})$-módulo álgebra. Como $\Delta(x) = x \ox 1 + 1 \ox x$ para todo $x \in \mathfrak{g}$,
	entonces $x\cdot (ab) = (x \cdot a)b + a (x \cdot b)$ para todo $x\in \mathfrak{g}$ y $a, b \in A$, es decir,
	los elementos de $\mathfrak{g}$ actúan como $\field$-derivaciónes y esto
	induce un morfismo de álgebras de Lie $U(\mathfrak{g}) \to \Der_{\field}A$.
\end{example}

\subsection{Dimensión de Gelfand-Kirillov}
En esta subsección definimos la dimensión de Gelfand-Kirillov que será mencionada al final de la próxima sección.
Nos basamos en \cite[Capítulo 8]{McR}. Todas las demostraciones omitidas se encuentran en ese capítulo.

Empezamos con una definición auxiliar.
\begin{definition}
	Sea $f:\NN \to \RR_{\geq 1}$ una función. Decimos que $f$ tiene crecimiento polinomial si existe $\nu \in \RR$
	tal que $f(n) \leq n^{\nu}$ para todo $n \gg 0$. Si este es el caso llamamos
	\[		
		\gamma(f) = \inf \left\lbrace \nu \mid f(n) \leq n^{\nu} \text{ para todo } n \gg 0  \right\rbrace.
	\]
	Si $f$ no tiene crecimiento polinomial decimos que $\gamma(f) = \infty$.
\end{definition}

Sea $A$ una $\field$-álgebra finitamente generada y sea $V \subseteq A$ un subespacio generador de dimensión finita.
Llamamos $V^n$ al subsespacio de $A$ generado por los elementos de la forma $x_{1} \ldots x_n$, donde $x_i \in V$ para todo $i$
y llamamos $R_n = \sum_{i = 0}^n V^i$, donde $R_0 = V^0 = \field$. El conjunto $\left\lbrace R_n \right\rbrace_{n \geq 0}$ es una
filtración de $A$. Resulta que $\gamma(\dim(R_n))$ no depende del subespacio generador $V$ y eso permite hacer la siguiente definición.

\begin{definition}
	Sea $A$ una $\field$-álgebra finitamente generada, sea $V \subseteq A$ un subsespacio generador y sea $\left\lbrace R_n \right\rbrace_{n \geq 0}$
	la filtración asociada a $V$. Se define la \emph{dimensión de Gelfand-Kirillov} de $A$ como
	\[
		\text{GK}(A) = \gamma(\dim(R_n)).	
	\]
	Esta definición se extiende a $\field$-álgebras arbitrarias $A$ definiendo
	\[
		\text{GK}(A) = \sup\left\lbrace \text{GK}(R) \mid R \text{ es una subálgebra de }A\text{ finitamente generada} \right\rbrace.
	\]
\end{definition}

\begin{example}	Damos algunos ejemplos sencillos:
	\begin{itemize}		
		\item Si $A = \field\left[x_1, \ldots, x_s\right]$, entonces GK$(A) = s$.
		\item Si $\mathfrak{g}$ es un álgebra de Lie de dimensión finita, entonces GK$(U(\mathfrak{g})) = \dim(\mathfrak{g})$
		\item Si $A$ es el álgebra libre $\field <x_1, \ldots, x_s>$, con $s \geq 2$, entonces GK$(A) = \infty$.
	\end{itemize}
\end{example}


\section{Álgebras de Nichols}
Esta sección se basa en \cite{AS2}. Todas las demostraciones omitidas se encuentran en ese artículo
salvo que aclaremos lo contrario.

\subsection{Espacios vectoriales trenzados y módulos de Yetter-Drinfeld}

En esta subsección estudiaremos cuándo el producto smash de un álgebra de Hopf $H$
por un $H$-módulo álgebra resulta ser un álgebra de Hopf.
Un teorema de Radford \cite{Ra} dice que si $H$ es un álgebra de Hopf y $B$ es un 
álgebra de Hopf en la categoría de módulos de Yetter-Drinfeld sobre H, entonces $B\#H$ es un álgebra
de Hopf. En general que $B$ sea un álgebra, coálgebra, biálgebra o álgebra de Hopf en una categoría $\field$-lineal $\mathcal{C}$
quiere decir que los morfismos de estructura de $B$ son morfismos en $\mathcal{C}$. Las categorías que nos interesan
son las de módulos y comódulos sobre un álgebra de Hopf $H$. Por ejemplo, un álgebra $A$ en $\prescript{}{H}{\mathcal{M}}$
es simplemente un $H$-módulo álgebra a izquierda. Análogamente una coálgebra $C$ en $\prescript{}{H}{\mathcal{M}}$
es un $H$-módulo coálgebra. Para que $B$ sea una biálgebra en una categoría $\mathcal{C}$ no solo se requiere que $B$ sea
un álgebra y una coálgebra en $\mathcal{C}$, sino que también la comultiplicación $\Delta_B:B \to B\ox B$ debe ser un morfismo en la categoría.
En particular $B\ox B$ debe ser un álgebra en $\mathcal{C}$. La multiplicación usual del producto tensorial de dos álgebras
$A$ y $B$ está dada por 
\[
	m_{A\ox B} = \left(m_A \ox m_B\right) \circ \left(id \ox \tau_{A, B} \ox id\right).
\]
Notar que antes de usar los morfismos $m_A$ y $m_B$ se aplica la transposición $\tau_{A, B}$. Esto sugiere
que en la categoría $\mathcal{C}$ debe haber una forma de "torcer" productos tensoriales.

\begin{definition}
	Sean $V$ un $\field$-espacio vectorial y $c:V \ox V \to V \ox V$ un isomorfismo lineal. Decimos
	que $(V,c)$ es un \emph{espacio vectorial trenzado} si $c$ es solución de la \emph{ecuación de trenzas}, es decir,
	\[
		(c \ox id)\circ(id \ox c)\circ(c \ox id) = (id \ox c)\circ(c \ox id)\circ(id \ox c).
	\] 
\end{definition}

\begin{example}
	Sea $V$ un $\field$-espacio vectorial con base $\left\lbrace x_i \right\rbrace_{i \in I}$
	y sea $\left\lbrace q_{i, j} \right\rbrace_{i, j \in I} \subseteq \field$ un conjunto de escalares no nulos.
	El isomorfismo $c:V \ox V \to V \ox V$ dado por
	\[
		c(x_i \ox x_j) = q_{i, j} x_j \ox x_i
	\] es solución de la ecuación de trenzas.
\end{example}

En esta sección nos vamos a concentrar en estudiar ejemplos de espacios vectoriales trenzados que provengan de módulos
de Yetter-Drinfeld.

\begin{definition}
	Sea $H$ un álgebra de Hopf. Un $\field$-espacio vectorial $V$ es un \emph{módulo de Yetter-Drinfeld a izquierda sobre $H$}
	si $\left(V, \cdot\right)$ es un $H$-módulo a izquierda, $(V, \rho)$ es un $H$-comódulo a izquierda
	y se satisface la siguiente condición de compatibilidad:
	\[
		\rho(h \cdot v) = h_{(1)}v_{(-1)}S(h_{(3)}) \ox h_{(2)} \cdot v_{(0)}	
	\]
	para todo $h \in H$ y para todo $v \in V$.
	La categoría de módulos de Yetter-Drinfeld a izquierda sobre $H$ se denota por $\yetter$ y
	sus morfismos preservan la acción y la coacción.
\end{definition}

Sea $H$ un álgebra de Hopf y sean $V, W \in \yetter$. El espacio vectorial $V \ox W$ resulta ser un módulo de Yetter-Drinfeld
a izquierda sobre $H$ definiendo
\[
	\rho(v \ox w) = v_{(-1)}w_{(-1)} \ox (v_{(0)} \ox w_{(0)}) \quad\text{y}\quad h\cdot (v \ox w) = h_{(1)}\cdot v \ox h_{(2)} \cdot w
\]
para todo $v \in V$, $w \in W$ y $h \in H$.

\begin{definition}
	Sea $H$ un álgebra de Hopf y sean $V, W \in \yetter$. La trenza
	\[
		c_{V, W}: V\ox W \to W\ox V
	\] es la única transformación $\field$-lineal que cumple que
	\[
		c_{V, W}(v \ox w) = v_{(-1)} \cdot w \ox v_{(0)}
	\]
	para todo $v \in V$ y $w \in W$.
\end{definition}

Los módulos de Yetter-Drinfeld sobre álgebras de grupo son muy importantes por sus aplicaciones al estudio de álgebras de Hopf punteadas.
\begin{example}\label{examplegroupyetter}
	Sean $G$ un grupo abeliano y $V \in \yetterg$. Análogamente a lo que sucede para $\field G$-comódulos a derecha,
	resulta que $V = \oplus_{g \in G} V_g$, donde $V_g = \left\lbrace v \in V \mid \rho(v) = g \ox v \right\rbrace$.
	Debido a la condición de compatibilidad, resulta que
	$\rho(h \cdot v) = hgh^{-1} \ox h \cdot v = g \ox h \cdot v$ para todo $v \in V$ y $h \in H$.
	Luego para cada $g \in G$, el subespacio vectorial $V_g$ es un $G$-módulo.
	En este caso la trenza asociada a $V \ox V$ está dada por $c_{V, V}(x \ox y) = g \cdot y \ox x$ para todo $g \in G$, $x \in V_g$,
	$y \in V$.
	
	Si suponemos que para cada $g \in G$ el grupo $G$ actúa por caracteres sobre $V_g$, entonces $V_g = \oplus_{\xi \in \hat{G}} V_{g}^{\xi}$,
	donde
	\[
		V_{g}^{\xi} = \left\lbrace v \in V_g \mid h \cdot v = \xi(h)v \text{ para todo } h \in G \right\rbrace.	
	\]
	En este caso decimos que $V$ es un módulo de Yetter-Drinfeld de \emph{tipo diagonal}.
	Si $\field$ es algebraicamente cerrado, la característica de $\field$ es cero
	y $G$ es un grupo finito abeliano, entonces todo módulo de Yetter-Drinfeld sobre $\field G$ de dimensión finita es de tipo diagonal.
\end{example}

\subsection{Álgebras de Hopf trenzadas}
El hecho de que haya productos tensoriales en la categoría de módulos de Yetter-Drinfeld permite definir allí la noción de álgebra y coálgebra.

\begin{definition}
	Sean $H$ un álgebra de Hopf y $B \in \yetter$. Decimos que $B$ es un álgebra en $\yetter$ si $\left(B,m_B,\eta_B\right)$ es una $\field$-álgebra
	tal que la multiplicación $m_B$ y la unidad $\eta_B$ son morfismos en $\yetter$.
\end{definition}

\begin{definition}
	Sean $H$ un álgebra de Hopf y $B \in \yetter$. Decimos que $B$ es una coálgebra en $\yetter$ si
	$\left(B,\Delta_B,\veps_B\right)$ es una $\field$-coálgebra
	tal que la comultiplicación $\Delta_B$ y la counidad $\veps_B$ son morfismos en $\yetter$.
\end{definition}

Sean $H$ un álgebra de Hopf y $B$ un álgebra en $\yetter$.  La trenza $c_{B, B}: B \ox B \to B \ox B$ le da a $B \ox B$ estructura de álgebra en $\yetter$ definiendo
\[
	m_{B \ox B} = \left(m_B \ox m_B \right) \circ \left(id \ox c_{B, B} \ox id \right).
\]
Esta estructura de álgebra trenzada que tiene el producto tensorial de álgebras permite hacer la siguiente definición.

\begin{definition}
	Sea $H$ un álgebra de Hopf. Decimos que $B$ es un \emph{álgebra de Hopf trenzada en $\yetter$} si
	\begin{itemize}
	\item $\left(B, m_B, \eta_B\right)$ es un álgebra en $\yetter$,
	\item $\left(B, \Delta_B, \veps_B \right)$ es una coálgebra en $\yetter$,
	\item $\Delta_B$ y $\veps_B$ son morfismos de álgebras en $\yetter$,
	\item existe un morfismo $S_B:B \to B$ en $\yetter$ que resulta ser la inversa de $id_B$
	para el producto de convolución.	
	\end{itemize}
	Para no confundir el coproducto de $B$ con el de $H$ escribimos $\Delta_B(b) = b^{(1)}\ox b^{(2)}$ para todo $b \in B$.
	Un álgebra de Hopf trenzada \emph{graduada} en $\yetter$ es un álgebra de Hopf trenzada $B$ en $\yetter$ junto
	con una graduación $B = \oplus_{n \geq 0}B(n)$, donde cada $B(n)$ es un módulo de Yetter-Drinfeld y con esta 
	graduación $B$ es un álgebra graduada y una coálgebra graduada.
\end{definition}

\begin{example}
	Si $H$ es el álgebra del grupo trivial, entonces $B$ es un álgebra de Hopf trenzada en $\yetter$ si y solo si $B$ es un álgebra de Hopf.
\end{example}

\begin{example} \label{exampletensoralgebra}
	Sean $H$ un álgebra de Hopf y $V \in \yetter$. El álgebra tensorial $T(V) = \oplus_{n \geq 0}V^{\ox n}$ es un álgebra de Hopf trenzada en $\yetter$
	con comultiplicación y counidad dadas por
	\[
		\Delta(v) = v \ox 1 + 1 \ox v, \quad \veps(v) = 0 \quad\text{para todo } v \in V.	
	\]
	La existencia de la antípoda se puede encontrar en \cite[5.2.10]{Mon}. Si $H = \field G$, $V$ es de tipo diagonal, $g \in G$, $v \in V$ y
	$w \in W$, entonces
	\[
		\Delta(vw) = 1 \ox vw + (g \cdot w) \ox v + v \ox w + vw \ox 1.
	\]
\end{example}

\begin{prop} (Biproduct de Radford, bosonización de Majid) Sean $H$ un álgebra de Hopf y $B \in \yetter$ un álgebra de Hopf trenzada.
	El product smash $B\#H$ resulta ser un álgebra de Hopf definiendo:
	\begin{align*}
		\Delta(b \ox h) &= \left( b^{(1)}\ox {(b^{(2)})}_{(-1)} h_{(1)} \right) \ox \left( (b^{(2)})_{(0)} \ox h_{(2)} \right),\\	
		S(b \ox h) &= \left(1 \ox S(h) S\left( b_{(-1)}\right) \right) \left(S_B\left(b_{(0)}\right) \ox 1\right)
	\end{align*}
	para todo $b \in B$ y para todo $h \in H$.
\end{prop}
\begin{proof}
	Ver \cite{Ra}
\end{proof}

\subsection{Definiciones y ejemplos de Álgebras de Nichols}

Una herramienta esencial para el estudio de álgebras de Hopf punteadas es la \emph{filtración coradical}
\[
	A_0 \subset A_1 \subset \cdots \subset A, \cup_{n \geq 0}A_n = A
\]
donde $A$ es un álgebra de Hopf, $A_0$ es la suma de todas las subcoalgebras simples de $A$ y
$A_{n + 1} = \Delta^{-1}\left( A_n \ox A + A \ox A_0\right)$, para todo $n \geq 1$.
En el caso en el que $A$ sea un álgebra de Hopf punteada, esta filtración
resulta ser una filtración de álgebras de Hopf. Por un teorema de Radford \cite{Ra} el graduado
asociado gr$(A)$ a esta filtración es el biproducto $R\#\field\Gamma$, donde $\Gamma$ es el grupo de todos los elementos tipo
grupo de $A$ y $R$ es un álgebra de Hopf trenzada graduada en la categoría de módulos a izquierda de Yetter-Drinfeld  de $\field\Gamma$.
El espacio vectorial $V$ formado por todos los elementos primitivos de $R$ es un submódulo de Yetter-Drinfeld de $R$.
La subálgebra $\mathfrak{B}(V)$ de $R$ generada por los elementos de $V$ es una subálgebra de Hopf trenzada.
En su tesis \cite{Ni}, Nichols estudió las álgebras de Hopf de la forma $\mathfrak{B}(V) \# \field\Gamma$.
El álgebra $\mathfrak{B}(V)$ se llama \emph{álgebra de Nichols de $V$}.

Para poder definir formalmente álgebra de Nichols necesitamos enunciar el siguiente lema:
\begin{lema}
	Sean $H$ un álgebra de Hopf, $V \in \yetter$ y $T(V)$ el álgebra de Hopf trenzada del Ejemplo \ref{exampletensoralgebra}.
	Existe un único coideal maximal $\mathfrak{J}(V)$ entre todos los coideales de $T(V)$ que están contenidos en $\oplus_{n \geq 2}V^{\ox n}$. Este coideal es homogéneo con respecto a la graduación de $T(V)$ y resulta ser un ideal.
\end{lema}

\begin{definition}
	La coálgebra cociente $\mathfrak{B}(V) = T(V)/\mathfrak{J}(V)$ se llama \emph{álgebra de Nichols de V}.
\end{definition}

\begin{Teorema}
	La coálgebra $\mathfrak{B}(V)$ es un álgebra de Hopf trenzada $\NN_{0}$-graduada.
\end{Teorema}

Los siguientes son los dos ejemplos son los más sencillos de álgebras de Nichols. Para simplificar
no vamos a escribir los elementos de $T(V)$ con el símbolo del producto tensorial.
\begin{example}
	Si $H$ es el álgebra del grupo trivial, entonces $V = V_1$ y $1\cdot v = v$ para todo $v \in V$.
	Utilizando la formula que del Ejemplo \ref{exampletensoralgebra} obtenemos que
	\[
		\Delta(vw - wv) = 1\ox (vw -wv) + (vw - wv)\ox 1
	\]
	y por lo tanto $vw - wv \in \mathfrak{J}(V)$ para todo $v, w \in V$. Se puede ver que $\mathfrak{J}(V)$ es el ideal
	generado por los elementos de la forma $vw - wv$ con $v, w \in V$ y luego
	$\mathfrak{B}(V)$ resulta ser el álgebra simétrica $S(V)$.
\end{example}

\begin{example}
	Sea $H = \field \ZZ_{2}$ y supongamos que $V = V_{-1}$ y que $-1 \cdot v = -v$ para todo $v \in V$.
	Utilizando la misma formula que en el ejemplo anterior obtenemos que
	\[
		\Delta(v^2) = 1\ox v^2 + v^2 \ox 1
	\]
	y por lo tanto $v^2 \in \mathfrak{J}(V)$. Nuevamente se puede ver que $\mathfrak{J}(V)$
	está generado por los elementos de la forma $v^2$ con $v \in V$.
	En este caso $\mathfrak{B}(V)$ resulta ser el álgebra exterior $\bigwedge(V)$.
\end{example}
Un problema importante y difícil es determinar el coideal $\mathfrak{J}(V)$ como hicimos en los ejemplos
anteriores.

A partir de ahora $H = \field G$, donde $\field$ es un cuerpo algebraicamente cerrado, car$(\field) = 0$ y
$G$ es un grupo abeliano. Sea $V$ un módulo de Yetter-Drinfeld a izquierda sobre $H$. Ya sabemos, por lo que vimos en el Ejemplo \ref{examplegroupyetter}, que $V = \oplus_{g \in G} V_g$ y que $V_g$ es un $G$-módulo para cada $g \in G$.

En los artículos \cite{AAH2} y las referencias incluidas, los autores estudian la siguiente versión del problema que mencionamos antes. Supongamos que la dimensión de $V$ es finita. Se quiere determinar cuando
la dimensión de Gelfand-Kirillov de $\mathfrak{B}(V)$ es finita y describir el coideal $\mathfrak{J}(V)$.
Por hipótesis, $V$ es suma directa de indescomponibles, lo cual permite suponer
que $V$ es indescomponible y por lo tanto $V \subseteq V_g$ para algún $g \in G$. A su vez esto nos permite suponer
que $V \in \prescript{\field \ZZ}{\field \ZZ}{\mathcal{YD}}$ donde $\ZZ = \left\langle g \right\rangle$. Como restringimos escalares, $V$ puede dejar de ser indescomponible. Sea $\mathcal{V}_n(\veps, l)$ un objeto en $\prescript{\field \ZZ}{\field \ZZ}{\mathcal{YD}}$, homogéneo de grado $g^n$ y dimensión $l > 1$, 
donde la acción de $g$ está dada por un bloque de Jordan de tamaño $l$ y autovalor $\veps$.
En este caso, vale el siguiente teorema \cite{AAH}.

\begin{Teorema}
La dimensión de Gelfand-Kirillov de $\mathfrak{B}\left(\mathcal{V}_n(\veps, l)\right)$ es finita si y solo si $l = 2$ y $\veps = \pm 1$.
Además,
\begin{itemize}
	\item si $\veps = 1$, entonces
	\[
		\mathfrak{B}\left(\mathcal{V}_n(1, 2)\right) = \field\left\langle x, y \mid yx - xy + \frac{1}{2}x^2 \right\rangle.
	\]
	\item Si $\veps = -1$, entonces
	\[
		\mathfrak{B}\left(\mathcal{V}_n(-1, 2)\right) = \field\left\langle x, y \mid x^2, y^2x - xy^2 - xyx \right\rangle.
	\]
	El álgebra $\mathfrak{B}\left(\mathcal{V}_n(1, 2)\right)$ se conoce como \emph{el plano de Jordan}
	y $\mathfrak{B}\left(\mathcal{V}_n(-1, 2)\right)$ se conoce como \emph{el super plano de Jordan}.
\end{itemize}
\end{Teorema}
En el mismo articulo en el que se encuentra el teorema recién mencionado, se demuestra que el super plano de Jordan tiene como base al conjunto
\[
	\mathcal{B} = \left\lbrace x^{a}(yx)^{b}y^c \mid a \in \lbrace 0, 1 \rbrace, b, c \geq 0\right\rbrace.
\]


\end{document}








