\documentclass[fleqn,../tesis.tex]{subfiles}
%\externaldocument{tesis}

\begin{document}
La cohomología de Hochschild es un invariante relevante: es invariante por equivalencias Morita, por procesos inclinantes y 
por equivalencias derivadas. Las técnicas que utilizamos han permitido calcular la cohomología de Hochschild de numerosas 
álgebras.

El cálculo de estos invariantes requiere de una resolución del álgebra considerada como bimódulo sobre sí misma. 
Siempre se cuenta con una 
resolución canónica, la resolución bar, muy útil desde el punto de vista teórico, pero poco satisfactoria en la práctica: la  
complejidad de 
esta resolución rara vez permite llevar a cabo cálculos explícitos. 
Existen familias de álgebras que poseen una resolución minimal, única a menos de isomorfismo, por ejemplo los cocientes de álgebras tensoriales 
de espacios vectoriales de dimensión finita por ideales que contienen el radical de Jacobson, o por ideales graduados \cite{BK}. Una resolución minimal permite obtener inmediatamente algunos invariantes del álgebra considerada. Por ejemplo, su longitud es igual a la dimensión global del álgebra. Sin embargo, aún cuando su existencia está asegurada, una resolución minimal no es sencilla de construir.

Las álgebras de Nichols son generalizaciones de las álgebras  simétricas  en el contexto de las categorías tensoriales trenzadas. 
Se trata de álgebras graduadas, que aparecieron por primera vez en un artículo de Nichols \cite{Ni} en 1978, en el cual el autor buscaba ejemplos 
de álgebras de Hopf. Estas álgebras fueron redescubiertas más tarde por Woronowicz \cite{Wo}.
Se trata de objetos fundamentales para la clasificación de las álgebras de Hopf 
punteadas, como fue puesto de manifiesto por los trabajos de Andruskiewitsch y Schneider (ver por ejemplo \cite{AS}).

Heckenberger clasificó las álgebras de Nichols de dimensión finita de tipo diagonal. La clasificación separa a las álgebras de Nichols en 
diversas clases: álgebras de Nichols de tipo Cartan, relacionadas esencialmente con los grupos cuánticos finitos de Lusztig; 
álgebras de Nichols relacionadas con supergrupos cuánticos finitos; 
y una tercera clase relacionada con superálgebras de Lie contragradientes. 
Posterioremente, Angiono describió las relaciones  de definición de las álgebras de Nichols de la lista de Heckenberger \cite{Ang}. 
Estos resultados abrieron el camino para  el estudio nuevos problemas en teoría de representación y en la clasificación de las álgebras de 
Hopf punteadas.
La estructura general de estas últimas no es aún conocida totalmente, aunque hay resultados parciales, ver \cite{AAH}. 

La categoría de módulos sobre un álgebra de Hopf $H$ es, gracias a la existencia de la comultiplicación, de la counidad y de la antípoda, una categoría
tensorial con objeto unidad y duales. La cohomología de $H$ se define como la generalización de la cohomología de un grupo, es decir
$\Hy^{\bullet}(H, \field) = \Ext_H^{\bullet}(\field, \field)$. Tal como sucede para el caso de la cohomología de un grupo, $\Hy^{\bullet}(H, \field)$
es un álgebra conmutativa graduada con el producto cup.

La generación finita del anillo de cohomología de las álgebras de Hopf de dimensión finita es un problema abierto e interesante por sus 
consecuencias sobre la categoría de representaciones y porque permite el uso de métodos geométricos. En algunas situaciones este resultado 
es conocido, como sucede en el caso de las álgebras 
de grupo (finito) en característica positiva. Se trata de resultados ya clásicos de Venkov \cite{Ve}, Golod \cite{Go} y Evens \cite{Ev} y también en el de las álgebras de 
Hopf coconmutativas de dimensión finita (Friedlander y Suslin \cite{FrSu}). En 1993, Ginzburg y Kumar demostraron la generación finita de 
la cohomología de grupos 
cuánticos peque\~nos cuando el cuerpo de base son los números complejos y bajo ciertas restricciones de los parámetros \cite{GiKu}. Más recientemente, Bendel, Nakano, Parshall, y Pillen \cite{BNPP} calcularon la cohomología de grupos cuánticos pequeños bajo condiciones mucho 
más generales sobre los parámetros. 
Todos estos resultados llevan a preguntarse si es cierto que el álgebra de cohomología de cualquier álgebra de Hopf finito dimensional es 
finitamente generada. 
En 2004 Etingof y Ostrik conjeturaron este resultado en el contexto más general de las categorías tensoriales finitas \cite{EtOs}. 
En un artículo reciente de Mastnak, Pevtsova, Schauenburg y Witherspoon \cite{MPSW}, los autores obtuvieron resultados parciales para álgebras de 
Hopf no conmutativas de dimensión finita y punteadas sobre un cuerpo de característica $0$ cuyo grupo de elementos de tipo grupo es abeliano, bajo algunas restricciones sobre el orden del grupo. En este trabajo los autores usaron la clasificación de \cite{AAH}.
El articulo trata de álgebras de Hopf cuya estructura de álgebra es una deformación de un producto smash, en el sentido que
puede darse una filtración del álgebra  de Hopf cuyo graduado asociado es isomorfo al producto smash $\mathfrak{B}(V)\#\field G$,
donde $\mathfrak{B}(V)$ es el álgebra de Nichols de un módulo de Yetter-Drinfeld $V$ sobre $\field G$.

Este problema se relaciona con la cohomología de Hochschild, ya que para todo álgebra $A$ con aumentación $\veps:A \to \field$,
el espacio graduado
$\Ext_{A}^{\bullet}(\field, \field)$ es isomorfo a la cohomología de Hochschild de $A$ con coeficientes triviales,
es decir $\Ext_{A^{e}}^{\bullet}(A, \field)$. Más aún, si $A$ es un álgebra de Hopf cuya antípoda es biyectiva, $\Ext_{A}^{\bullet}(\field, A^{ad})$
es isomorfo como álgebra a $\Ext_{A^e}^{\bullet}(A, A)$ y por lo tanto $\Ext_A^{\bullet}(\field, \field)$ es isomorfa a un sumando directo
de la cohomología de Hochschild $\Hy^{\bullet}(A,A)$. Como consecuencia, si esta última es finitamente generada como álgebra, lo mismo
sucederá con la primera.

El objetivo de esta tesis es calcular la homología y cohomología de Hochschild del álgebra 
$A = \Bbbk\langle x, y | x^2, y^2x - xy^2 -xyx \rangle$, llamado el super plano de Jordan cuando $\field$
es un cuerpo de característica 0 y algebraicamente cerrado. Se trata del álgebra de Nichols $\mathfrak{B}(V(-1,2))$,
cuya dimensión de Gelfand-Kirillov es $2$. Este es el caso más simple, después del plano de Jordan, de la familia de álgebras 
que queremos estudiar posteriormente: la familia de álgebras de Nichols $\mathfrak{B}(V)$ con dimensión de Gelfand-Kirillov finita.

Los resultados principales obtenidos son los siguientes. En el Teorema \ref{teo_homologia} damos bases explicitas para los
espacios de homología de Hochschild $\Hy_{i}(A, A)$, para todo $i \geq 0$. En el Teorema \ref{teo_cohomologia} hacemos
lo mismo para los espacios de cohomología $\Hy^{i}(A, A)$. Los espacios $\Hy_{i}(A, A)$ y $\Hy^{i}(A, A)$ son todos de dimensión
infinita salvo el centro $\Hy^{0}(A,A)$ que es isomorfo a $\field$. En la Tabla $\ref{table:1}$ describimos el producto
cup entre los generadores de los espacios de cohomología $\Hy^{i}(A, A)$. A partir de esta descripción vemos que el ismorfismo
entre $\Hy^{2p}(A, A)$ y $\Hy^{2p + 2}(A, A)$, donde $p > 0$, esta dado por la multiplicación con $u_0^{2}$. Para los
grados impares sucede lo mismo. Finalmente, en la Sección \ref{section_lie} describimos la estructura de álgebra de Lie
de  $\Hy^{1}(A,A)$ con el corchete de Gerstenhaber (Teorema \ref{teo_lie}). La misma resulta isomorfa a una subálgebra
de Lie de Virasoro: $\Hy^{1}(A,A)$ es isomorfo como álgebra de Lie a $\mathfrak{h} \oplus \Vir^{-}$. Queda por
describir la estructura de módulo de Lie de $\Hy^{n}(A,A)$ para todo $n \geq 0$ y ver cuáles son las representaciones
del álgebra de Virasoro que estos inducen.

Los contenidos de la tesis son los siguientes. En el Capítulo $1$ damos una introducción a las álgebras de Hopf con
el objetivo de introducir el concepto de álgebra de Nichols y exponer algunos ejemplos. En el Capítulo $2$ resumimos
las definiciones y los resultados del álgebra homológica que serán necesarios para el siguiente capítulo. En el capítulo
$3$ calculamos la homología y cohomología de Hochschild del super plano de Jordan. También describimos
parte de la estructura de álgebra de Gerstenhaber que tiene la cohomología de Hochschild. Por último,
en el Apéndice incluimos las reglas de conmutación del super plano de Jordan.
\end{document}








