\documentclass[fleqn,../tesis.tex]{subfiles}
\externaldocument{apendice}
\externaldocument{tesis}
\externaldocument{hochschild}

\begin{document}

\section{Homología de Hochschild}
Sea $\mathbf{X}_{\bullet,\bullet}$ el siguiente complejo doble ubicado en el primer cuadrante
\begin{align*}
\xymatrix{
	& & \\
	& A \ox k\{x^4\} \ox A \ar[d]^{\delta} \ar@{<--}[u]_{\partial} & A \ox k\{y^2x^4\} \ox A \ar[l]_d \ar[d]^{\delta'} \ar@{<--}[u]_{\partial'} \\
	& A \ox k\{x^3\} \ox A \ar[d]^{\partial} & A \ox k\{y^2x^3\} \ox A \ar[l]_d \ar[d]^{\partial'} \\
	& A \ox k\{x^2\} \ox A \ar[d]^{\delta} & A \ox k \{y^2x^2\} \ox A \ar[l]_d \ar[d]^{\delta'} \\
	A \ox A & A \ox k\{x, y\} \ox A \ar[l]_{d_0} & A \ox k \{y^2x\} \ox A \ar[l]_{d_1} \\
}
\end{align*}
donde los diferenciales de $A$-bimódulos se definen de la siguiente manera:
\begin{align*}
	d_0(1 \ox v \ox 1) &= v \ox 1 - 1 \ox v, \quad\text{ para todo } v \in k\{x, y\}, \\
	\\
	d_1(1 \ox y^2x \ox 1) &= y^2 \ox x \ox 1 + y \ox y \ox x + 1 \ox y \ox yx \\
		&-(xy \ox y \ox 1 + x \ox y \ox y + 1 \ox x \ox y^2) \\
		&-(xy \ox x \ox 1 + x \ox y \ox x + 1 \ox x \ox yx), \\
	\\	
	d(1 \ox y^2x^n \ox 1) &= y^2 \ox x^n \ox 1 - xy \ox x^n \ox 1 - 1 \ox x^n \ox y^2 - 1 \ox x^n \ox yx,\\
	\\	
	\delta(1 \ox x^n \ox 1) &= x \ox x^{n -1} \ox 1 + 1 \ox x^{n - 1} \ox x,\\
	\delta'(1 \ox y^2x^n \ox 1) &= - (x \ox y^2 x^{n -1} \ox 1 + 1 \ox y^2x^{n - 1} \ox x),\\
	\\	
	\partial(1 \ox x^n \ox 1) &= x \ox x^{n -1} \ox 1 - 1 \ox x^{n - 1} \ox x,\\
	\partial'(1 \ox y^2x^n \ox 1) &= -(x \ox y^2 x^{n -1} \ox 1 - 1 \ox y^2x^{n - 1} \ox x). \\
\end{align*}

Aplicamos el funtor $A \ox_{A^e}(-)$ a $\mathbf{X}_{\bullet,\bullet}$ y obtenemos un nuevo
complejo doble tal que la homología de su complejo total es $HH_*(A)$. El complejo es el siguiente
\begin{align*}
\xymatrix{
	& & \\
	& A \ox_{A^e} (A \ox k\{x^4\} \ox A) \ar[d]^{id\ox\delta} \ar@{<--}[u]_{id\ox\partial}
		& A \ox_{A^e} (A \ox k\{y^2x^4\} \ox A) \ar[l]_{id\ox d} \ar[d]^{id\ox\delta'} \ar@{<--}[u]_{id\ox\partial'} \\
	& A \ox_{A^e} (A \ox k\{x^3\} \ox A) \ar[d]^{id\ox\partial}
		& A \ox_{A^e} (A \ox k\{y^2x^3\} \ox A) \ar[l]_{id\ox d} \ar[d]^{id\ox\partial'} \\
	& A \ox_{A^e} (A \ox k\{x^2\} \ox A) \ar[d]^{id\ox\delta}
		& A \ox_{A^e} (A \ox k \{y^2x^2\} \ox A) \ar[l]_{id\ox d} \ar[d]^{id\ox\delta'} \\
	A \ox_{A^e} (A \ox A) & A \ox_{A^e} (A \ox k\{x, y\} \ox A) \ar[l]_{id\ox d_0} & A \ox k \{y^2x\} \ox A \ar[l]_{id\ox d_1} \\
}
\end{align*}
Identificando de manera natural $A \ox_{A^e} (A \ox W \ox A)$ con $A \ox W$ para todo espacio vectorial $W$ resulta
el complejo doble
\begin{align*}
\xymatrix{
	& & \\
	& A \ar[d]^{\delta} \ar@{--}[u]_{\partial} & A\ar[l]_{d} \ar[d]^{-\delta} \ar@{--}[u]_{-\partial} \\
	& A \ar[d]^{\partial} & A \ar[l]_{d} \ar[d]^{-\partial} \\
	& A \ar[d]^{\overline{\delta}} & A \ar[l]_{d} \ar[d]^{-\delta} \\
	A & A \oplus A \ar[l]_{d_0} & A \ar[l]_{d_1} \\
}
\end{align*}
con los siguientes diferenciales $k$-lineales
\begin{align*}
&d_0(a,b) = \left[a,x\right] + \left[b,y\right(],\\
&d_1(a) = (\left[a,y^2\right] - (axy + yxa), \left[x,a\right]y + y\left[x,a\right] - xax),\\
&d(a) = \left[a,y^2\right] - (axy + yxa),\\
&\overline{\delta}(a) = (ax + xa, 0),\\
&\delta(a) = ax + xa,\\
&\partial(a) = \left[a, x\right].
\end{align*}
Para calcular la homología de Hochschild vamos a utilizar una sucesión espectral inducida por la filtración por columnas del bicomplejo.
Denotamos por $E$ a la sucesión espectral.


\subsection{Primera página}
Para poder calcular los espacios de homología de la primera página de la sucesión espectral
vamos a necesitar saber qué valores toman los morfismos $overline{\delta}$, $\delta$ y $\partial$
en los elementos de la base $\base$.

Empecemos con $\delta$. Sea $z = x^a(yx)^by^c \in \base$, luego $\delta(z) = x^a(yx)^by^cx + x^{a + 1}(yx)^by^c$. Nos va a
resultar conveniente escribir esta última expresión como combinación lineal de elementos de $\base$. Para eso vamos a necesitar considerar
diferentes casos.
\begin{itemize}
\item Si $a = 0$ y $c = 2k$,
\begin{align*}
	\delta(z) &= (yx)^by^{2k}x + x(yx)^by^{2k} = (yx)^b\left( \sum_{i = 0}^k\frac{k!}{i!}x(yx)^{k - i}y^{2i}\right) + x(yx)^by^{2k} \\
	&= \sum_{i = 0}^k\frac{k!}{i!}(yx)^bx(yx)^{k - i}y^{2i} + x(yx)^by^{2k}. 
\end{align*}
\begin{itemize}
\item Si $b = 0$,
\[
	\delta(z) = \sum_{i = 0}^k\frac{k!}{i!}x(yx)^{k - i}y^{2i} + xy^{2k}.
\]
\item Si $b \geq 1$, como $x^2 = 0$,
\begin{align*}
	\delta(z) &= \sum_{i = 0}^k\frac{k!}{i!}(yx)^{b-1}(yx)x(yx)^{k - i}y^{2i} + x(yx)^by^{2k} \\
	&= \sum_{i = 0}^k\frac{k!}{i!}(yx)^{b-1}yx^2(yx)^{k - i}y^{2i} + x(yx)^by^{2k} = x(yx)^by^{2k}.
\end{align*}
\end{itemize}
\item Si $a = 0$ y $c = 2k + 1$,
\begin{align*}
	\delta(z) &= (yx)^by^{2k + 1}x + x(yx)^by^{2k + 1} \\
	&= (yx)^b\left( \sum_{i = 0}^k\frac{k!}{i!}(yx)^{k - i + 1}y^{2i}\right) + x(yx)^by^{2k + 1} \\
	&= \sum_{i = 0}^k\frac{k!}{i!}(yx)^{b + k - i + 1}y^{2i} + x(yx)^by^{2k + 1}.
\end{align*}
\item Si $a = 1$ y $c = 2k$,
	\begin{align*}
		\delta(z) &= x(yx)^by^{2k}x + x^2(yx)^by^c = x(yx)^by^{2k}x \\
		&= x(yx)^b\left( \sum_{i = 0}^k\frac{k!}{i!}x(yx)^{k - i}y^{2i}\right) \\
		&= \sum_{i = 0}^k\frac{k!}{i!}x(yx)^bx(yx)^{k - i}y^{2i}. \\
		& = 0,
	\end{align*}
	ya que como $x^2 = 0$, resulta que $x(yx)^bx = 0$ para todo $b \geq 0$.
\item Si $a = 1$ y $c = 2k + 1$
	\begin{align*}
		\delta(x) &= x(yx)^by^{2k + 1}x = x(yx)^b\left( \sum_{i = 0}^k\frac{k!}{i!}(yx)^{k - i + 1}y^{2i}\right) \\
		&= \sum_{i = 0}^k\frac{k!}{i!}x(yx)^{b + k - i + 1}y^{2i}.
	\end{align*}
\end{itemize}
Con este cálculo probamos que
\begin{align*}
	\Ima(\delta) = \Bigg\langle &\sum_{i = 0}^k\frac{k!}{i!}x(yx)^{k - i}y^{2i} + xy^{2k},\ x(yx)^{b + 1}y^{2k},
		 \sum_{i = 0}^k\frac{k!}{i!}(yx)^{b + k - i + 1}y^{2i} + x(yx)^by^{2k + 1}, \\
		&\hspace{100pt}\sum_{i = 0}^k\frac{k!}{i!}x(yx)^{b + k - i + 1}y^{2i}\ :\ b,k \geq 0	\Bigg\rangle.
\end{align*}
Ahora que ya tenemos un conjunto de generadores de $\Ima(\delta)$, queremos extraer una base.
\begin{prop}
\label{imdelta}
El conjunto
\begin{align*}
		\left\{ x(yx)^by^{2k},
			\sum_{i = 0}^k\frac{k!}{i!}(yx)^{b + k - i + 1}y^{2i} + x(yx)^by^{2k + 1} :b, k \geq 0 \right\}
\end{align*}
es una base de $\Ima(\delta)$.
\end{prop}
\begin{proof}
Vamos a usar las siguientes notaciones:
\begin{align*}
	\eta_k &=  \sum_{i = 0}^k\frac{k!}{i!}x(yx)^{k - i}y^{2i} + xy^{2k}, \\
	\theta_{b,k} &= x(yx)^{b + 1}y^{2k},\\
	\lambda_{b,k} &= \sum_{i = 0}^k\frac{k!}{i!}(yx)^{b + k - i + 1}y^{2i} + x(yx)^by^{2k + 1}, \\
	\mu_{b,k} &= \sum_{i = 0}^k\frac{k!}{i!}x(yx)^{b + k - i + 1}y^{2i}.
\end{align*}
Sabemos que el conjunto $\{\eta_k,\ \theta_{b,k},\ \lambda_{b,k},\ \mu_{b,k}\ :\ b,k \geq 0\}$ genera $\Ima(\delta)$.
Como
\begin{align*}
	\eta_k &= \sum_{i = 0}^{k}\frac{k!}{i!}x(yx)^{k - i}y^{2i} + xy^{2k}
		= \sum_{i = 0}^{k-1}\frac{k!}{i!}x(yx)^{k - i}y^{2i} + 2xy^{2k} \\
	&\hspace{100pt}= \sum_{i = 0}^{k -1}\frac{k!}{i!}\theta_{k - i, i} + 2xy^{2k}
\end{align*}
y la característica de $\field$ es cero, resulta que $xy^{2k}$ pertenece a $\Ima(\delta)$ para todo $k \geq 0$. Por otro lado,
\begin{align*}
	\mu_{b,k} &= \sum_{i = 0}^k\frac{k!}{i!}x(yx)^{b + k - i + 1}y^{2i}
		= \sum_{i = 0}^k\frac{k!}{i!}\theta_{b + k - i + 1, i}.
\end{align*}
Por lo tanto
\begin{align*}
	&\left\{\theta_{b,k},\ xy^{2k},\ \lambda_{b,k}\ :\ b,k \geq 0 \right\} \\
	&\hspace{60pt}= \left\{ x(yx)^by^{2k},
			\sum_{i = 0}^k\frac{k!}{i!}(yx)^{b + k - i + 1}y^{2i} + x(yx)^by^{2k + 1} :b, k \geq 0 \right\}
\end{align*}
genera $\Ima(\delta)$. Más aún, se trata de una base ya que cualquier combinación lineal de elementos de este conjunto es una
combinación lineal de elementos distintos de $\base$.
\end{proof}
\begin{prop}
El conjunto
\begin{align*}
		\left\{ \left(x(yx)^by^{2k}, 0\right),
			\left(\sum_{i = 0}^k\frac{k!}{i!}(yx)^{b + k - i + 1}y^{2i} + x(yx)^by^{2k + 1},0\right) :b, k \geq 0 \right\}
\end{align*}
es una base de $\Ima(\overline{\delta})$.
\end{prop}
\begin{proof}
	Se deduce de que para todo $z \in A$, $\overline{\delta}(z) = (\delta(z), 0)$.
\end{proof}

Nos resta calcular los valores que toma $\partial$ en los elementos de la base y obtener una base de su imagen.
Calculemos $\partial(z)$ para $z = x^a(yx)^by^c \in \mathcal{B}$. Como \[\partial(z) = x^a(yx)^by^cx - x^{a + 1}(yx)^by^c,\] resulta que
\begin{itemize}
\item si $a = 0$ y $c = 2k$,
	\begin{align*}
		\partial(z) &= (yx)^by^{2k}x - x(yx)^by^{2k} = \sum_{i = 0}^k\frac{k!}{i!}(yx)^bx(yx)^{k - i}y^{2i} - x(yx)^by^{2k}.
	\end{align*}
	\begin{itemize}
	\item Si $b = 0$,
		\begin{align*}
			\partial(z) = \sum_{i = 0}^k\frac{k!}{i!}x(yx)^{k - i}y^{2i} - xy^{2k} = \sum_{i = 0}^{k - 1}\frac{k!}{i!}x(yx)^{k - i}y^{2i}.
		\end{align*}
	\item Si $b \geq 1$, usando nuevamente que $x^2 = 0,$
		\begin{align*}
			\partial(z) &= \sum_{i = 0}^k\frac{k!}{i!}(yx)^{b-1}(yx)x(yx)^{k - i}y^{2i} - x(yx)^by^{2k} \\
			&=  -x(yx)^by^{2k}.
		\end{align*}
	\end{itemize}
\item Si $a = 0$ y $c = 2k + 1$,
	\begin{align*}
		\partial(z) &= (yx)^by^{2k + 1}x - x(yx)^by^{2k + 1}
			= \sum_{i = 0}^k\frac{k!}{i!}(yx)^{b + k - i + 1}y^{2i} - x(yx)^by^{2k + 1}.
	\end{align*}
\item Si $a = 1$ y  $c = 2k$,
	\begin{align*}
		\partial(z) &= x(yx)^by^{2k}x - x^2(yx)^by^{2k} 
			= \sum_{i = 0}^k\frac{k!}{i!}x(yx)^bx(yx)^{k - i}y^{2i} = 0,
	\end{align*}
ya que 	$x(yx)^bx = 0$ para todo $b \geq 0$.
\item Si $a = 1$ y  $c = 2k + 1$,
	\begin{align*}
		\partial(z) &= x(yx)^by^{2k}x - x^2(yx)^by^{2k} = \sum_{i = 0}^k\frac{k!}{i!}x(yx)^{b + k - i + 1}y^{2i}.
	\end{align*}
\end{itemize}
Con este cálculo probamos que
\begin{align*}
	\Ima(\partial) = \Bigg\langle & \sum_{i = 0}^{k - 1}\frac{k!}{i!}x(yx)^{k - i}y^{2i},
		\ x(yx)^{b + 1}y^{2k},
		\ \sum_{i = 0}^k\frac{k!}{i!}(yx)^{b + k - i + 1}y^{2i} - x(yx)^by^{2k + 1}, \\
		&\hspace{100pt}\sum_{i = 0}^k\frac{k!}{i!}x(yx)^{b + k - i + 1}y^{2i} : b,k \geq 0 \Bigg\rangle.
\end{align*}
\begin{prop}
\label{impartial}
El conjunto
\begin{align*}
		\left\{ x(yx)^{b + 1}y^{2k},
			\sum_{i = 0}^k\frac{k!}{i!}(yx)^{b + k - i + 1}y^{2i} - x(yx)^by^{2k + 1} :b, k \geq 0 \right\}
\end{align*}
es una base de $\Ima(\partial)$.
\end{prop}
\begin{proof}
Llamamos como antes
\begin{align*}
	\eta_{k - 1} &=  \sum_{i = 0}^{k-1}\frac{k!}{i!}x(yx)^{k - i}y^{2i}, \\
	\theta_{b,k} &= x(yx)^{b + 1}y^{2k},\\
	\lambda_{b,k} &= \sum_{i = 0}^k\frac{k!}{i!}(yx)^{b + k - i + 1}y^{2i} - x(yx)^by^{2k + 1}, \\
	\mu_{b,k} &= \sum_{i = 0}^k\frac{k!}{i!}x(yx)^{b + k - i + 1}y^{2i}.
\end{align*}
Como $k - i \geq 1$ para todo $i$,  $0 \leq i \leq k -1$, entonces
\begin{align*}
	\eta_{k - 1} &= \sum_{i = 0}^{k-1}\frac{k!}{i!}x(yx)^{k - i}y^{2i} = \sum_{i = 0}^{k-1}\frac{k!}{i!}\theta_{k-1,i}.
\end{align*}
Por otro lado como $b + k - i + 1 \geq 1$ para todo $i$, $0 \leq i \leq k$, entonces
\begin{align*}
	\mu_{b,k} &= \sum_{i = 0}^k\frac{k!}{i!}x(yx)^{b + k - i + 1}y^{2i}
		= \sum_{i = 0}^k\frac{k!}{i!}\theta_{b + k - i + 1, i}.
\end{align*}
Luego el conjunto
\begin{align*}
	\left\{ x(yx)^{b + 1}y^{2k},
		\sum_{i = 0}^k\frac{k!}{i!}(yx)^{b + k - i + 1}y^{2i} - x(yx)^by^{2k + 1} :b,k \geq 0	\right\}
\end{align*}
 genera $\Ima(\partial)$ y es fácil ver que es una base.
\end{proof}

Ya estamos en condiciones de calcular la primera página de la sucesión espectral. 
\begin{prop}
	El espacio vectorial $E_{1,0}^1 = \frac{A \oplus A}{\Ima(\overline{\delta})}$
	tiene como base al conjunto
	\begin{align*}
		\left\{ \left[((yx)^by^c, 0)\right]\ :\ b,c \geq 0\right\}
			\cup \left\{ \left[(0, x^a(yx)^by^c)\right]\ :\ a \in \{0, 1\}\ y\ b,c \geq 0\right\}. 
	\end{align*}
\end{prop}
\begin{proof}
	Debido a que
	\begin{align*}
		&\left[(x(yx)^by^{2k}, 0)\right] = 0, \\
		&\left[(-x(yx)^by^{2k + 1}, 0)\right] = \sum_{i = 0}^{k}\frac{k!}{i!}\left[(yx)^{b + k - i + 1}y^{2i}, 0)\right],
	\end{align*}
	es claro que el conjunto que aparece en el enunciado de la proposición genera $E_{1, 0}^1$. Es fácil verificar
	que también es linealmente independiente.
\end{proof}
Vamos a identificar a $E_{1,0}^1 = \frac{A \oplus A}{\Ima(\overline{\delta})}$ con $\frac{A}{\Ima(\delta)} \oplus A$
y vamos escribir a los elementos $\left[\left(a, b\right)\right] \in E_{1,0}^1$ como $\left(\left[a\right], \left[b\right]\right)$ para poder
trabajar más fácil con cada coordenada por separado.
\begin{prop}
	El espacio vectorial $E_{2,0}^1 = \frac{A}{\Ima(\delta)}$ tiene como base al conjunto
	\begin{align*}
		\left\{ \left[(yx)^by^c\right] \ :\ b,c \geq 0\right\}. 
	\end{align*}
\end{prop}
\begin{proof}
	Se deduce con un razonamiento análogo al anterior.
\end{proof}
\begin{prop}
	Los espacios vectoriales
	\begin{align*}
		E_{1,2i + 1}^1 = E_{2, 2i + 1}^1 = \frac{\Ker(\delta)}{\Ima(\partial)}
	\end{align*}
	tienen como base al conjunto $\left\{ \left[xy^{2n}\right] \ :\ n \geq 0\right\}$ , para todo $i \geq 0$.
\end{prop}
\begin{proof}
	Es claro que el núcleo y la imagen de $\delta$ coinciden respectivamente con el núcleo
	y la imagen de $-\delta$. Lo mismo ocurre con los morfismos
	$\partial$ y $-\partial$. Luego $E_{1,2i + 1}^1 = E_{2, 2i + 1}^1$, para todo $i \geq 0$.
	Sea $z \in \Ker(\delta)$, como $\delta(w) = 0$ o $\gr(\delta(w)) = \gr(w) + 1$ para todo $w \in A$
	y las relaciones que definen a $A$ son homogéneas, podemos suponer que $z$ es homogéneo.
	Separamos el cálculo de $z$ según la paridad de su grado.
	\begin{itemize}
		\item Si $\gr(z) = 2n$, entonces
		\begin{align*}
			z = \sum_{l = 0}^n\alpha_l(yx)^{n - l}y^{2l} + \sum_{i = 1}^n\beta_ix(yx)^{n - i}y^{2i - 1}.
		\end{align*}
		Por la Proposición \ref{impartial}, $x(yx)^{n - i}y^{2i - 1} - \sum_{j = 0}^{i - 1}\frac{(i-1)!}{j!}(yx)^{n - j}y^{2j}$
		pertenece a la imagen de $\partial$ para todo $1 \leq i \leq n$.
		Por lo tanto, reduciendo módulo borde, podemos suponer que $z = \sum_{l = 0}^n\alpha_l(yx)^{n - l}y^{2l}$.
		Si $z$ pertenece al núcleo de $\delta$, entonces
		\begin{align*}
			0 &= \delta(z) = \delta\left(\sum_{l = 0}^n\alpha_l(yx)^{n - l}y^{2l}\right) = \sum_{l = 0}^n\alpha_l\delta((yx)^{n - l}y^{2l}) \\
			&= \sum_{l = 0}^{n - 1}\alpha_l\delta((yx)^{n - l}y^{2l}) + \alpha_n\delta(y^{2n}) \\
				&= \sum_{l = 0}^{n - 1}\alpha_lx(yx)^{n - l}y^{2l} + \alpha_n\sum_{l = 0}^n \frac{n!}{l!}x(yx)^{n - l}y^{2l} + \alpha_nxy^{2n} \\
				&= \sum_{l = 0}^{n - 1}(\alpha_l + \alpha_n\frac{n!}{l!}) x(yx)^{n - l}y^{2l} + 2\alpha_nxy^{2n}.
		\end{align*}	
		Como el monomio $xy^{2n}$ no aparece en la sumatoria, resulta que $\alpha_n = 0$.
		Luego
		\[
			\sum_{l = 0}^{n - 1}\alpha_lx(yx)^{n - l}y^{2l} = 0
		\]
		y como los monomios que aparecen en la suma son linealmente independientes, entonces $\alpha_l = 0$ para todo $l$, $0 \leq l \leq n -1$.
		Es decir, $z = 0$.
		\item Si $\gr(z) = 2n + 1$, entonces
		\begin{align*}
			z = \sum_{l = 0}^n\alpha_l(yx)^{n - l}y^{2l + 1} + \sum_{l = 0}^n\beta_lx(yx)^{n - l}y^{2l}.
		\end{align*}
		Por la Proposición \ref{impartial}, $x(yx)^{n - l}y^{2l} \in \Ima(\partial)$ para todo $0 \leq l \leq n - 1$.
		Por lo tanto, reduciendo módulo borde, podemos suponer que $z = \sum_{l = 0}^n\alpha_l(yx)^{n - l}y^{2l + 1} + \beta xy^{2n}$.
		Si $z$ pertenece a el núcleo de $\delta$, entonces
		\begin{align*}
			0 &= \delta(z) = \sum_{l = 0}^n\alpha_l\delta((yx)^{n - l}y^{2l + 1}) + \beta \delta(xy^{2n}) \\
			&= \sum_{l = 0}^n\alpha_l \left( \sum_{i = 0}^l\frac{l!}{i!}(yx)^{n + 1 - i}y^{2i} + x(yx)^{n - l}y^{2l + 1} \right) \\
			&= \sum_{l = 0}^n \sum_{i = 0}^l\alpha_l\frac{l!}{i!}(yx)^{n + 1 - i}y^{2i}
				+ \sum_{l = 0}^n\alpha_l x(yx)^{n - l}y^{2l + 1}.
		\end{align*}	
		En particular $\sum_{l = 0}^n\alpha_l x(yx)^{n - l}y^{2l + 1}  = 0$, por lo tanto $\alpha_l = 0$ para todo $l$, $0 \leq l \leq n$.
		Es decir, $z = \beta xy^{2n}$ y
		\[
			E_{1,2i + 1}^1 = E_{2, 2i + 1}^1 = \left\langle \left[xy^{2n}\right]\ :\ n \geq 0 \right\rangle.
		\]
	\end{itemize}
		Veamos que los generadores que encontramos forman una base. Sea $[w] \in \frac{\Ker(\delta)}{\Ima(\partial)}$ una combinación
		lineal de elementos de la forma $[xy^{2n}]$ y supongamos que $[w] = 0$, es decir, $w \in \Ima(\partial)$.
		Como los elementos que generan $\Ima(\delta)$ son homogéneos podemos suponer que $w$ también lo es y por lo tanto $w = \lambda xy^{2n}$
		para algún $n \geq 0$ y para algún $\lambda \in k$.
		Pero los únicos elementos de la forma $x(yx)^by^c$ que pertenecen a $\Ima(\delta)$ cumplen que $b \geq 1$.
		Por lo tanto $\lambda = 0$ y el conjunto $\left\{ [xy^{2n}] \ :\ n \geq 0\right\}$ es una base de $\frac{\Ker(\delta)}{\Ima(\partial)}$.	 
\end{proof}
\begin{prop}
	Para todo $i \geq 1$, los espacios vectoriales
	\begin{align*}
		E_{1,2i}^1 = E_{2, 2i}^1 = \frac{\Ker(\partial)}{\Ima(\delta)}
	\end{align*}
	tienen como base al conjunto $\left\{ \sum_{l = 0}^n \frac{n!}{l!}\left[(yx)^{n - l}y^{2l}\right] \ :\ n \geq 0\right\}$.
\end{prop}
\begin{proof}
	Sea $z \in \Ker(\partial)$, nuevamente podemos suponer que $z$ es homogéneo por la misma razón que antes.
	Separamos el cálculo de $z$ según la paridad de su grado.
	\begin{itemize}
		\item Si $\gr(z) = 2n, n \geq 0$,
		\begin{align*}
			z = \sum_{l = 0}^n\alpha_l(yx)^{n - l}y^{2l} + \sum_{i = 1}^n\beta_ix(yx)^{n - i}y^{2i - 1}.
		\end{align*}
		Por la Proposición \ref{imdelta}, $x(yx)^{n - i}y^{2i - 1} + \sum_{j = 0}^{i - 1}\frac{(i-1)!}{j!}(yx)^{n - j}y^{2j} \in \Ima(\delta)$
		para todo $1 \leq i \leq n$.
		Por lo tanto, reduciendo módulo borde, podemos suponer que $z = \sum_{l = 0}^n\alpha_l(yx)^{n - l}y^{2l}$.
		Si $z \in \Ker(\partial)$, entonces
		\begin{align*}
			0 &= \sum_{l = 0}^n\alpha_l\partial((yx)^{n - l}y^{2l})
				= \sum_{l = 0}^{n - 1}\alpha_l\partial((yx)^{n - l}y^{2l}) + \alpha_n\partial(y^{2n})\\
			&= \sum_{l = 0}^{n - 1}\alpha_l(-x(yx)^{n - l}y^{2l}) + \alpha_n\sum_{l = 0}^{n - 1}\frac{n!}{l!}x(yx)^{n - l}y^{2l}
				= \sum_{l = 0}^{n - 1}(-\alpha_l + \alpha_n\frac{n!}{l!})x(yx)^{n - l}y^{2l}.
		\end{align*}
		Por lo tanto $\alpha_l = \alpha_n\frac{n!}{l!}$ para todo $0 \leq l \leq n$ y
		\[
			z = 	\sum_{l = 0}^n\alpha_l(yx)^{n - l}y^{2l} = \alpha_n\sum_{l = 0}^n\frac{n!}{l!}(yx)^{n - l}y^{2l}.
		\]
		\item Si $\gr(z) = 2n + 1, n \geq 0$,
		\begin{align*}
			z = \sum_{l = 0}^n\alpha_l(yx)^{n - l}y^{2l + 1} + \sum_{l = 0}^n\beta_lx(yx)^{n - l}y^{2l}.
		\end{align*}
		Por la Proposición \ref{imdelta}, $x(yx)^{n - l}y^{2l} \in \Ima(\delta)$ para todo $l$, $0 \leq l \leq n$.
		Por lo tanto, reduciendo módulo borde, podemos suponer que $z = \sum_{l = 0}^n\alpha_l(yx)^{n - l}y^{2l + 1}$.
		Si $z$ pertenece al núcleo de $\partial$, entonces
		\begin{align*}
			0 &= \sum_{l = 0}^n\alpha_l\partial((yx)^{n - l}y^{2l + 1})
				= \sum_{l = 0}^n\alpha_l\left(\sum_{i = 0}^l\frac{l!}{i!}(yx)^{n - i + 1}y^{2i} -x(yx)^{n - l}y^{2l + 1}\right)\\
			&= \sum_{l = 0}^n\sum_{i = 0}^l\alpha_l\frac{l!}{i!}(yx)^{n - i + 1}y^{2i} - \sum_{l = 0}^n\alpha_lx(yx)^{n - l}y^{2l + 1}.
		\end{align*}
		En particular $\sum_{l = 0}^n\alpha_lx(yx)^{n - l}y^{2l + 1} = 0$ y por lo tanto $\alpha_l = 0$ para todo $0 \leq l \leq n$.
		Luego
		\[
			E_{1,2i}^1 = E_{2, 2i}^1 = \left\langle \sum_{l = 0}^n \frac{n!}{l!}\left[(yx)^{n - l}y^{2l}\right] \ :\ n \geq 0\right\rangle.
		\]
	\end{itemize}
	Nuevamente para verificar que estos generadores forman una base alcanza con probar que para  todo $n \geq 0$,
	$w = \sum_{l = 0}^n \frac{n!}{l!}(yx)^{n - l}y^{2l}$ no pertenece al subespacio $\Ima(\delta)$.
	Como el grado de $w$ es par, si $w \in \Ima(\delta)$, entonces necesariamente $w$ debería pertenecer al subespacio
	\[
		\left\langle \sum_{i = 0}^k\frac{k!}{i!}(yx)^{b + k - i + 1}y^{2i} + x(yx)^by^{2k + 1} : b,k \geq 0 \right\rangle.
	\]
	Pero esto es absurdo debido a que aparecen términos de la forma $x(yx)^by^{2k + 1}$.
\end{proof}

El siguiente diagrama describe la primera página de la sucesión espectral:
\begin{align*}
\xymatrix{
	& & \\
	& \left\langle \left[xy^{2n}\right] \right\rangle \ar@{--}[u] & \left\langle \left[xy^{2n}\right] \right\rangle \ar[l]_{d^{(1)}} \ar@{--}[u]\\
	& \left\langle \sum_{l = 0}^n \frac{n!}{l!}\left[(yx)^{n - l}y^{2l}\right] \right\rangle \ar@{--}[u]
		& \left\langle \sum_{l = 0}^n \frac{n!}{l!}\left[(yx)^{n - l}y^{2l}\right] \right\rangle \ar[l]_{d^{(1)}} \ar@{--}[u]\\
	& \left\langle \left[xy^{2n}\right] \right\rangle \ar@{--}[u] & \left\langle \left[xy^{2n}\right] \right\rangle \ar[l]_{d^{(1)}} \ar@{--}[u]\\
	A & \left\langle \left(\left[(yx)^by^c\right], 0\right) \right\rangle \bigoplus A \ar[l]_(.7){d_0^{(1)}} \ar@{--}[u]
		& \left\langle \left[(yx)^by^c\right] \right\rangle \ar[l]_{d_1^{(1)}} \ar@{--}[u] \\
}
\end{align*}
donde $d_0^{(1)}$, $d_1^{(1)}$ y $d^{(1)}$ son los morfismos inducidos en la homología por $d_0$, $d_1$ y $d$ respectivamente.



\subsection{Segunda Página}
Debido a la forma del complejo doble, la sucesión espectral se estaciona rápidamente y $E^2 = E^{\infty}$. Para el cálculo de la segunda página procederemos de igual manera que para la primera. Calcularemos
qué valores toman los morfismos $d_0^{(1)}, d_1^{(1)}$ y $d^{(1)}$ en los elementos de las bases correspondientes y luego obtendremos una descripción
de las imágenes de estos morfismos. Por último calcularemos las homologías en cada lugar.
\begin{prop} \label{prop_imad1impar}
    Para todo $k \geq 0$, el morfismo $d^{(1)}: E_{2, 2k + 1}^1 \to E_{1, 2k + 1}^1$ es nulo.
\end{prop}
\begin{proof}
	Sea $\left[xy^{2n}\right]$ un elemento de la base de $E_{2, 2k + 1}^1$, entonces
	\begin{align*}
		d(xy^{2n}) &= \left[ xy^{2n}, y^2 \right] - xy^{2n}xy
			= \left[ xy^{2n}, y^2 \right] - x\sum_{l = 0}^n\frac{n!}{l!}x(yx)^{n - l}y^{2l + 1}  \\
		&= \left[ xy^{2n}, y^2 \right] = \left[x, y^2 \right]y^{2n} + x\left[y^{2n}, y^2\right] =  \left[x, y^2 \right]y^{2n} \\
		&= (xy^2 - y^2x)y^{2n} = (-xyx)y^{2n}. 
	\end{align*}
	Luego $d(xy^{2n}) \in \Ima(\partial)$ y por lo tanto $d^{(1)}(\left[xy^{2n}\right]) = 0$
	\end{proof}
\begin{prop}  \label{prop_imad1par}
Para todo $k \geq 1$, el morfismo $d^{(1)}: E_{2, 2k}^1 \to E_{1, 2k}^1$ es nulo.
\end{prop}
\begin{proof}
	Sea $w = \sum_{l = 0}^n \frac{n!}{l!}(yx)^{n - l}y^{2l}$, de modo que $[w]$ es un elemento de la base de $E_{2,2k}^1$.
	\begin{itemize}
		\item Si $n = 0$, entonces
			\[
				d(w) = d(1) = \left[1, y^2 \right] - (xy + yx) = -xy -yx.			
			\]
			Luego $d(w) \in \Ima(\delta)$ y $d^{(1)}(\left[w\right]) = 0$.
			De hecho, utilizando la notación de la demostración de la Proposición \ref{imdelta}, resulta que
			$d(w) = -\lambda_{0,0}$.
		\item Si $n \geq 1$ podemos suponer que $n = m + 1$, donde $m \geq 0$. Debido a la Proposición \ref{cp_conmutatividad}
		del Apéndice
			\begin{align*}
				w &= \sum_{l = 0}^{m + 1} \frac{(m + 1)!}{l!}(yx)^{m + 1 - l}y^{2l}
					= \left((m + 1)\sum_{l = 0}^m\frac{m!}{l!}(yx)^{m + 1 - l}y^{2l}\right) + y^{2m + 2} \\
				&= (m + 1)y^{2m + 1}x + y^{2m + 2}.
			\end{align*}	
			Aplicando $d$ obtenemos que
			\begin{align*}
				d(w) &= d((m + 1)y^{2m + 1}x + y^{2m + 2}) = (m + 1)d(y^{2m + 1}x) + d(y^{2m + 2}).		
			\end{align*}
			Calculemos $d(y^{2m + 1}x)$ y $d(y^{2m + 2})$ por separado:
			\begin{align*}
				d(y^{2m + 1}x) &= \left[y^{2m + 1}x, y^2\right] - yxy^{2m + 1}x = y^{2m + 1}\left[x,y^2\right] - yxy^{2m + 1}x\\
				&= y^{2m + 1}(-xyx) - yxy^{2m + 1}x = -y^{2m + 1}xyx - yxy^{2m + 1}x\\
				&= -\sum_{i = 0}^m\frac{m!}{i!}(yx)^{m -i + 1}y^{2i +1}x - \sum_{j = 0}^{m}\frac{m!}{j!}(yx)^{m -j + 2}y^{2j} \\
				&= -\sum_{i = 0}^m\frac{m!}{i!}(yx)^{m -i + 1}\left(\sum_{j = 0}^{i}\frac{i!}{j!}(yx)^{i - j + 1}y^{2j}\right)
					- \sum_{j = 0}^{m}\frac{m!}{j!}(yx)^{m -j + 2}y^{2j} \\
				&= -\sum_{i = 0}^m\sum_{j = 0}^i\frac{m!}{j!}(yx)^{m -j + 2}y^{2j}
					- \sum_{j = 0}^{m}\frac{m!}{j!}(yx)^{m -j + 2}y^{2j} \\
				&= -\sum_{j = 0}^m\sum_{i = j}^m\frac{m!}{j!}(yx)^{m -j + 2}y^{2j}
					- \sum_{j = 0}^{m}\frac{m!}{j!}(yx)^{m -j + 2}y^{2j} \\
				&= -\sum_{j = 0}^m(m - j + 1)\frac{m!}{j!}(yx)^{m -j + 2}y^{2j}
					- \sum_{j = 0}^{m}\frac{m!}{j!}(yx)^{m -j + 2}y^{2j} \\
				&= -\sum_{j = 0}^m(m - j + 2)\frac{m!}{j!}(yx)^{m -j + 2}y^{2j}
			\end{align*}
			y
			\begin{align*}
				d(y^{2m + 2}) &= \left[y^{2m + 2}, y^2\right] - (y^{2m +2}xy + yxy^{2m + 2}) = - y^{2m +2}xy - yxy^{2m + 2} \\
				&= -\sum_{l = 0}^{m + 1}\frac{(m + 1)!}{l!}x(yx)^{m - l + 1}y^{2l + 1} - yxy^{2m + 2}.
			\end{align*}
			Luego
			\begin{align*}
				d^{(1)}(\left[w\right]) &=
					-\sum_{j = 0}^m(m - j + 2)\frac{(m + 1)!}{j!}\left[(yx)^{m -j + 2}y^{2j}\right]\\
				&\hspace{70pt}-\sum_{l = 0}^{m + 1}\frac{(m + 1)!}{l!}\left[x(yx)^{m - l + 1}y^{2l + 1}\right]
						- \left[yxy^{2m + 2}\right]\\
				&= -\sum_{j = 0}^{m + 1}(m - j + 2)\frac{(m + 1)!}{j!}\left[(yx)^{m -j + 2}y^{2j}\right]\\
				& \hspace{70pt}-\sum_{l = 0}^{m + 1}\frac{(m + 1)!}{l!}\left[x(yx)^{m - l + 1}y^{2l + 1}\right]			
			\end{align*}
			Como $\sum_{i = 0}^k\frac{k!}{i!}(yx)^{b + k - i + 1}y^{2i} + x(yx)^by^{2k + 1} \in \Ima(\delta)$
			para todo $b,k \geq 0$, eligiendo $k = l$, $b = m - l + 1$, obtenemos que
				$-\left[x(yx)^{m - l + 1}y^{2l + 1}\right]
					= \sum_{j = 0}^l\frac{l!}{j!}\left[(yx)^{m - j + 2}y^{2j}\right].$
			Por lo tanto
			
			%& \hspace{70pt}-\sum_{l = 0}^{m + 1}\frac{(m + 1)!}{l!}\left[x(yx)^{m - l + 1}y^{2l + 1}\right] \\
			\begin{align*}
				d^1(\left[w\right]) &=
					\sum_{j = 0}^{m + 1}(m - j + 2)\frac{(m + 1)!}{j!}\left[(yx)^{m -j + 2}y^{2j}\right]\\
				& \hspace{70pt}-\sum_{l = 0}^{m + 1}\frac{(m + 1)!}{l!}\left[x(yx)^{m - l + 1}y^{2l + 1}\right] \\
				&=-\sum_{j = 0}^{m + 1}(m - j + 2)\frac{(m + 1)!}{j!}\left[(yx)^{m -j + 2}y^{2j}\right]\\
				& \hspace{70pt} + \sum_{l = 0}^{m + 1}\frac{(m + 1)!}{l!}
					\left(\sum_{j = 0}^l\frac{l!}{j!}\left[(yx)^{m - j + 2}y^{2j}\right]\right) \\
				&= -	\sum_{j = 0}^{m + 1}(m - j + 2)\frac{(m + 1)!}{j!}\left[(yx)^{m -j + 2}y^{2j}\right]\\
				& \hspace{70pt} + \sum_{l = 0}^{m + 1}\sum_{j = 0}^{l}\frac{(m + 1)!}{j!}
					\left[(yx)^{m - j + 2}y^{2j}\right]\\
				&= -	\sum_{j = 0}^{m + 1}(m - j + 2)\frac{(m + 1)!}{j!}\left[(yx)^{m -j + 2}y^{2j}\right]\\
				& \hspace{70pt} + \sum_{j = 0}^{m + 1}\sum_{l = j}^{m + 1}\frac{(m + 1)!}{j!}
					\left[(yx)^{m - j + 2}y^{2j}\right]\\
				&= -	\sum_{j = 0}^{m + 1}(m - j + 2)\frac{(m + 1)!}{j!}\left[(yx)^{m -j + 2}y^{2j}\right]\\
				& \hspace{70pt} + \sum_{j = 0}^{m + 1}(m - j + 2)\frac{(m + 1)!}{j!}
					\left[(yx)^{m - j + 2}y^{2j}\right]\\
				&= 0.
			\end{align*}	 
	\end{itemize}
\end{proof}

El siguiente paso va a ser calcular que valores toma $d_1^{(1)}$ en los elementos de la base de $E^1_{2, 0}$. Llamemos
$f$ a la composición de $d_1$ con la proyección en la primera coordenada y llamemos $g$ a la composición de $d_1$ con la proyección en la segunda coordenada, de modo que
\[
	d_1(a) =  \left(\left[a,y^2\right] - (axy + yxa), \left[x,a\right]y + y\left[x,a\right] - xax\right) = (f(a), g(a)), \text{ para todo } a \in A.
\]
Para no complicar la notación notaremos $f$ al morfismo inducido por $f$ en la homología. Lo mismo haremos con $g$.
Empecemos calculando que valores toma $f$ en los elementos de la base de $E^1_{2, 0}$. Sean $b, c \geq 0$ y $z = (yx)^by^c$, de modo que $\left[z\right]$
sea un elemento de la base de $E^1_{2, 0}$. Luego
\begin{align*}
	f(z) &= \left[(yx)^by^c, y^2\right] - \left((yx)^by^cxy + yx(yx)^by^c\right)\\
	&= \left[(yx)^b, y^2\right]y^c - (yx)^b(y^cx)y - (yx)^{b + 1}y^c \\
	&= (yx)^by^{c + 2} - y^2(yx)^by^c - (yx)^b(y^cx)y - (yx)^{b + 1}y^c \\
\end{align*}
y debido a la Proposición \ref{prop_conmutatividad2} del Apéndice,
\begin{align*}
	f(z) &= (yx)^by^{c + 2} - ((yx)^by^2 + b(yx)^{b + 1})y^c - (yx)^b(y^cx)y - (yx)^{b + 1}y^c \\
	&= - (b + 1)(yx)^{b + 1}y^c - (yx)^b(y^cx)y.
\end{align*}
En la homología, resulta que
\begin{itemize}
	\item si $c = 2k$,
	\[
		f([z]) = -(b + 1)\left[(yx)^{b + 1}y^{2k}\right] - \sum_{i  = 0}^k\frac{k!}{i!}\left[(yx)^bx(yx)^{k - i}y^{2i + 1}\right].
	\]	
	
	\begin{itemize}
		\item Si $b = 0$
		\[
			f([z]) = -\left[(yx)y^{2k}\right] - \sum_{i  = 0}^k\frac{k!}{i!}\left[x(yx)^{k - i}y^{2i + 1}\right].
		\]
		Como $\left(\sum_{l = 0}^n\frac{n!}{l!}(yx)^{d + n - l + 1}y^{2l} + x(yx)^by^{2n + 1}, 0\right) \in \Ima(\overline{\delta})$
		para todo $n,d \geq 0$, podemos elegir $n = i$, $d = k - i$ y obtenemos que 
		\[
			-\left[x(yx)^{k - i}y^{2i + 1}\right] = \sum_{l = 0}^i\frac{i!}{l!}\left[(yx)^{k - l + 1}y^{2l}\right].
		\]
		Luego
		\begin{align*}
			f([z]) &= -\left[(yx)y^{2k}\right] + \sum_{i  = 0}^k\sum_{l = 0}^i\frac{k!}{l!}\left[(yx)^{k - l + 1}y^{2l}\right]\\
			&= -\left[(yx)y^{2k}\right] + \sum_{l = 0}^{k}(k - l + 1)\frac{k!}{l!}\left[(yx)^{k - l + 1}y^{2l}\right]\\
			&= \sum_{l = 0}^{k - 1}(k - l + 1)\frac{k!}{l!}\left[(yx)^{k - l + 1}y^{2l}\right].
		\end{align*}
		\item Si $b \geq 1$,
		\[		
		f([z]) = -(b + 1)\left[(yx)^{b + 1}y^{2k}\right].
		\]	
	\end{itemize}
	\item Si $c = 2k + 1$,
	\begin{align*}
	    f([z]) &= -(b + 1)\left[(yx)^{b + 1}y^{2k + 1}\right] - \sum_{i  = 0}^k\frac{k!}{i!}\left[(yx)^{k + b - i + 1}y^{2i + 1}\right]\\
	    &= -(b + 2)\left[(yx)^{b + 1}y^{2k + 1}\right] - \sum_{i  = 0}^{k - 1}\frac{k!}{i!}\left[(yx)^{k + b - i + 1}y^{2i + 1}\right]
	\end{align*}
\end{itemize}

Ahora vamos a calcular qué valores toma $g$. Dados $b, c$ y $z$ como antes,
\begin{align*}
    g(z) &= \left[x,z\right]y + y\left[x,z\right] - xzx = \left[x, (yx)^by^c\right]y + y\left[x, (yx)^by^c\right] - x(yx)^by^c x\\
    &= x(yx)^by^{c + 1} - (yx)^by^{c}xy + (yx)^{b + 1}y^c - y(yx)^by^cx - x(yx)^by^cx.
\end{align*}
\begin{itemize}
    \item Si $b = 0$,
        \[
            g(z) = xy^{c + 1} - y^{c}xy + yxy^{c} - y^{c + 1}x - xy^{c}x.
        \]
        \begin{itemize}
            \item Si $c = 2k$,
            \begin{align*}
                g(z) &= xy^{2k + 1} - y^{2k}xy + yxy^{2k} - y^{2k + 1}x - xy^{2k}x\\
                &= xy^{2k + 1} - \sum_{i = 0}^{k}\frac{k!}{i!}x(yx)^{k - i}y^{2i + 1} + yxy^{2k}
                    - \sum_{i = 0}^{k}\frac{k!}{i!}(yx)^{k + 1 - i}y^{2i}\\
                &= - \sum_{i = 0}^{k - 1}\frac{k!}{i!}x(yx)^{k - i}y^{2i + 1} -  \sum_{i = 0}^{k - 1}\frac{k!}{i!}(yx)^{k + 1 - i}y^{2i}.
            \end{align*}
            \item Si $c = 2k + 1$,
            \begin{align*}
                g(z) &= xy^{2k + 2} - y^{2k + 1}xy + yxy^{2k + 1} - y^{2k + 2}x - xy^{2k + 1}x\\
                &= xy^{2k + 2} - \sum_{i = 0}^{k}\frac{k!}{i!}(yx)^{k + 1 - i}y^{2i + 1} + yxy^{2k + 1}
                    - \sum_{i = 0}^{k + 1}\frac{(k+ 1)!}{i!}x(yx)^{k + 1 -i}y^{2i} \\
                    &\qquad - \sum_{i = 0}^{k}\frac{k!}{i!}x(yx)^{k + 1 -i}y^{2i}\\
                &= - \sum_{i = 0}^{k - 1}\frac{k!}{i!}(yx)^{k + 1 - i}y^{2i + 1}
                    - \sum_{i = 0}^{k}\left(\frac{(k+ 1)!}{i!} + \frac{k!}{i!}\right)x(yx)^{k + 1 -i}y^{2i}\\
                &= - \sum_{i = 0}^{k - 1}\frac{k!}{i!}(yx)^{k + 1 - i}y^{2i + 1}
                    - \sum_{i = 0}^{k}\frac{(k + 2)k!}{i!}x(yx)^{k + 1 -i}y^{2i}.\\
            \end{align*}
        \end{itemize}
    \item Si $b \geq 1$, debido a la Proposición \ref{prop_conmutatividad2} del Apéndice,
    \[
       g(z) = x(yx)^by^{c + 1} - (yx)^by^{c}xy + (yx)^{b + 1}y^c - x(yx)^{b -1}y^{c + 2}x - (b + 1)x(yx)^by^cx.
    \]
    \begin{itemize}
        \item Si $c = 2k$,
        \begin{align*}
            g(z) = x(yx)^by^{2k + 1} + (yx)^{b + 1}y^{2k}.
        \end{align*}
        \item Si $c = 2k + 1$,
        \begin{align*}
            g(z) &= x(yx)^by^{2k + 2} - (yx)^by^{2k + 1}xy + (yx)^{b + 1}y^{2k + 1} \\
            &\qquad - x(yx)^{b -1}y^{2k + 3}x - (b + 1)x(yx)^by^{2k + 1}x\\
            &= x(yx)^by^{2k + 2} - \sum_{i = 0}^{k}\frac{k!}{i!}(yx)^{b + k + 1 - i}y^{2i + 1} + (yx)^{b + 1}y^{2k + 1}\\
                &\qquad -\sum_{i = 0}^{k + 1}\frac{(k + 1)!}{i!}x(yx)^{b + k + 1 - i}y^{2i}
                    - (b + 1)\sum_{i = 0}^{k}\frac{k!}{i!}x(yx)^{b + k + 1 - i}y^{2i}\\
            &= -\sum_{i = 0}^{k - 1}\frac{k!}{i!}(yx)^{b + k + 1 - i}y^{2i + 1}
                    - \sum_{i = 0}^{k}\frac{(k + b + 2)k!}{i!}x(yx)^{b + k + 1 - i}y^{2i}.
        \end{align*}
    \end{itemize}
\end{itemize}
Con este cálculo encontramos generadores de $\Ima\left(d_1^{(1)}\right)$. Nos resta obtener una base de este espacio.

Para enunciar la próxima proposición vamos a utilizar la siguiente notación:
\begin{align*}
    \theta^{1}_{b, k} &= -(b + 1)\left[(yx)^{b + 1}y^{2k}\right],\\
    \theta^{2}_{b, k} &= \left[x(yx)^by^{2k + 1}\right] + \left[(yx)^{b + 1}y^{2k}\right],\\
    \lambda^{1}_{b, k} &= (b + 1)\left[(yx)^{b}y^{2k + 1}\right]
        +\sum_{i  = 0}^{k - 1}\frac{k!}{i!}\left[(yx)^{k + b - i}y^{2i + 1}\right]\\
    \lambda^{2}_{b, k} &= \sum_{i = 0}^{k - 1}\frac{k!}{i!}\left[(yx)^{b + k - i}y^{2i + 1}\right]
        + \sum_{i = 0}^{k}\frac{(k + b + 1)k!}{i!}\left[x(yx)^{b + k - i}y^{2i}\right]
\end{align*}
para todo $k \geq 0$, $b \geq 1$.
\begin{prop}\label{prop_imad11}
    El conjunto $\left\lbrace\left(\theta^{1}_{b, k}, \theta^{2}_{b, k}\right),
        \left(\lambda^{1}_{b, k}, \lambda^{2}_{b, k}\right) \mid k \geq 0, b\geq 1\right\rbrace$ es una base de $\Ima\left(d_1^{(1)}\right)$.
\end{prop}
\begin{proof}
Llamamos
\begin{align*}
    \eta^{1}_k &= \sum_{l = 0}^{k - 1}(k - l + 1)\frac{k!}{l!}\left[(yx)^{k - l + 1}y^{2l}\right],\\
    \eta^{2}_k &= -\sum_{i = 0}^{k - 1}\frac{k!}{i!}\left[x(yx)^{k - i}y^{2i + 1}\right]
        - \sum_{i = 0}^{k - 1}\frac{k!}{i!}\left[(yx)^{k + 1 - i}y^{2i}\right]
\end{align*}
para todo $k \geq 0$. Sabemos que el conjunto
    \[
        \left\lbrace \left(\theta^{1}_{b, k}, \theta^{2}_{b, k}\right),
            \left(\lambda^{1}_{b, k}, \lambda^{2}_{b, k}\right),  \left(\eta^{1}_{k}, \eta^{2}_{k}\right)
                \mid k \geq 0, b\geq 1\right\rbrace
    \]
genera $\Ima\left(d_1^{(1)}\right)$. Si $k = 0$, entonces $\eta^1_k = 0$ y $\eta^2_k = 0$. Si $k \geq 1$, entonces
\[
    \left(\eta^{1}_k, \eta^{2}_k\right) = \sum_{l = 0}^{k - 1}\left(\theta^{1}_{k - l, l}, \theta^{2}_{k - l, l}\right).
\]
Por lo tanto el conjunto $\left\lbrace\left(\theta^{1}_{b, k}, \theta^{2}_{b, k}\right),
        \left(\lambda^{1}_{b, k}, \lambda^{2}_{b, k}\right) \mid k \geq 0, b\geq 1\right\rbrace$ genera $\Ima\left(d_1^{(1)}\right)$
y es fácil ver que es una base.
\end{proof}

Nos resta calcular que valores toma $d_{0}^{(1)}$ en los elementos de la base de $E_{1, 0}^1$ y obtener una base de su imagen.
Recordemos que $E_{1, 0}^1$ esta generado por los elementos de la forma $\left(\left[(yx)^by^c\right], 0\right)$ y $\left(0, \left[x^a(yx)^by^c\right]\right)$
para todo $a \in \{0, 1\}$ y para todo $b, c \geq 0$. Sea $z = ((yx)^by^c, 0) \in A\oplus A$, como $d_0(z) = \partial(z)$, resulta que:
\begin{itemize}
    \item si $c = 2k$ y $b = 0$,
         \[d_0(z) = \sum_{i = 0}^{k - 1}\frac{k!}{i!}x(yx)^{k - i}y^{2i}.\]
    \item Si $c = 2k$ y  $b \geq 1$,
         \[d_0(z) = -x(yx)^by^{2k}.\]
    \item Si c = $2k + 1$,
        \[d_0(z) = \sum_{i = 0}^k\frac{k!}{i!}(yx)^{k + b + 1 - i}y^{2i} - x(yx)^by^{2k + 1}.\]
\end{itemize}
Sea $z = (0, x^a(yx)^by^c)$, como $d_0(z) = d_0((0, x^a(yx)^by^c)) = x^a(yx)^by^{c + 1} - yx^a(yx)^{b}y^c$, resulta que:
\begin{itemize}
    \item si $a = 1$,
        \[
            d_0(z) = x(yx)^by^{c + 1} - (yx)^{b + 1}y^c.        
        \]
    \item Si $a = 0$ y $b = 0$,
        \[
            d_0(z) = 0.        
        \]
    \item Si $a = 0$ y $b \geq 1$,
        \[
            d_0(z) = (yx)^{b}y^{c + 1} - x(yx)^{b - 1}y^{c + 2} - bx(yx)^{b}y^c.
        \]
\end{itemize}
Estos valores que obtuvimos generan $\Ima(d_0^{(1)})$. Antes de encontrar una base de $\Ima(d_0^{(1)})$ vamos a darle nombres
a estos generadores, como hicimos antes, para simplificar la escritura del cálculo:
\begin{align*}
    \eta_k &= \sum_{i = 0}^{k - 1}\frac{k!}{i!}x(yx)^{k - i}y^{2i},\\
    \theta_{b, k} &= x(yx)^{b + 1}y^{2k},\\
    \lambda_{b, k} &= \sum_{i = 0}^k \frac{k!}{i!}(yx)^{k + b + 1 - i}y^{2i} - x(yx)^by^{2k + 1},\\
    \mu_{b, c} &= x(yx)^by^{c + 1} - (yx)^{b + 1}y^c,\\
    \nu_{b, c} &= (yx)^{b + 1}y^{c + 1} - x(yx)^{b}y^{c + 2} -(b + 1)x(yx)^{b + 1}y^c
\end{align*}
para todo $b, k, c \geq 0$. Sabemos que el conjunto
\[
   \left\lbrace \eta_k, \theta_{b, k}, \lambda_{b, k}, \mu_{b, c}, \nu_{b, c} \mid b, k, c \geq 0 \right\rbrace
\]
genera $\Ima\left(d_0^{(1)}\right)$.
Como $\mu_{b, c + 1} + \nu_{b, c} = -(b + 1)x (yx)^{b + 1}y^c$ para todo $b, c \geq 0$,
resulta que $x(yx)^{b + 1}y^c \in \Ima\left(d_0^{(1)}\right)$ y por lo tanto
$x(yx)^{b + 1}y^{c + 1} - \mu_{b + 1, c} = (yx)^{b + 2}y^c \in \Ima\left(d_0^{(1)}\right)$ para todo $b, c \geq 0$.
A partir de esta observación es fácil probar la siguiente proposición.

\begin{prop} \label{prop_imad01}
    El conjunto
    \[
        \left\lbrace \left[x(yx)^{b + 1}y^c\right], \left[(yx)^{b + 2}y^c\right], \left[xy^{c + 1} - (yx)y^c\right] \mid b,c \geq 0 \right\rbrace    
    \]
    es una base de $\Ima\left(d_0^{(1)}\right)$.
\end{prop}

Ya estamos en condiciones de calcular los espacios de homología de $E^{2}$. Vamos
a escribir $\ov{a}$, en vez de $\left[a\right]$, cuando nos refiramos a un elemento de la segunda pagina.

Las siguientes tres proposiciones son consecuencia inmediata de las Proposiciones \ref{prop_imad1impar}, \ref{prop_imad1par} y \ref{prop_imad01}
respectivamente.
\begin{prop} 
    Para todo $i \geq 0$, los espacios vectoriales $E_{1,2i + 1}^2 = E_{2, 2i + 1}^2$
	tienen como base al conjunto $\left\{ \ov{xy^{2n}} \ :\ n \geq 0\right\}$.
\end{prop}

\begin{prop}
    Para todo $i \geq 1$, los espacios vectoriales $E_{1,2i}^2 = E_{2, 2i}^2$
	tienen como base al conjunto $\left\{ \sum_{l = 0}^n \frac{n!}{l!}\ov{(yx)^{n - l}y^{2l}} \ :\ n \geq 0\right\}$.
\end{prop}

\begin{prop}
    El espacio vectorial $E^{2}_{0, 0}$ tiene como base al conjunto $\left\{ \ov{xy^{n}}, \ov{y^n} \ :\ n \geq 0\right\}.$
\end{prop}

Nos resta calcular $E^2_{1,0}$ y $E^2_{2, 0}$.

\begin{prop}
    El espacio vectorial $E^{2}_{2, 0}$ tiene como base al conjunto
    \[
        \left\{ \sum_{l = 0}^n \frac{n!}{l!}\ov{(yx)^{n - l}y^{2l}} \ :\ n \geq 0\right\}.
    \]
\end{prop}
\begin{proof}
    Nuevamente escribimos $d_1^{(1)} = (f, g)$. Sea $[z] \in \Ker\left(d_{1}^{(1)}\right)$,
    como $f(z) = 0$ o $\gr(f(z)) = gr(z) + 3$, $g(z) = 0$ o $\gr(g(z)) = gr(z) + 3$ y la graduación pasa a la homología, podemos
    suponer que $[z]$ es homogéneo. Separamos el cálculo de $[z]$ según la paridad de su grado.
    \begin{itemize}
        \item Si $\gr(z) = 2n$, $z = \sum_{l = 0}^n \alpha_l (yx)^{n - l}y^{2l}$.
        Como $[z]$ pertenece a $\Ker\left(d_{1}^{(1)}\right)$, en particular $f([z]) = 0$ y por lo tanto
        \begin{align*}
            0 &= f([z]) = \sum_{l = 0}^{n}\alpha_l f\left(\left[(yx)^{n - l}y^{2l}\right]\right)\\
            &= -\sum_{l = 0}^{n - 1} \alpha_l (n - l + 1)\left[(yx)^{n - l + 1}y^{2l}\right]
                + \sum_{l = 0}^{n - 1} \alpha_n (n - l + 1) \frac{n!}{l!}\left[(yx)^{n - l + 1}y^{2l}\right] \\
            &= \sum_{l = 0}^{n - 1} \left(\alpha_n \frac{n!}{l!} - \alpha_l\right)\left[(yx)^{n - l + 1}y^{2l}\right]. 
        \end{align*}
        Por lo tanto $\alpha_l = \alpha_n \frac{n!}{l!}$ para todo $l$, $0 \leq l \leq n$ y
        $[z] = \alpha_n \sum_{l = 0}^n \frac{n!}{l!}\left[(yx)^{n - l}y^{2l}\right]$. Es fácil verificar que si este el caso,
        $d_{1}^{(1)}([z]) = 0$.
        \item Si $\gr(z) = 2n + 1$, $z = \sum_{l = 0}^n \alpha_l (yx)^{n - l}y^{2l + 1}$. Al igual que antes, como $[z]$ pertenece a $\Ker\left(d_{1}^{(1)}\right)$, resulta que $f([z]) = 0$ y por lo tanto
        \begin{align*}
            0 &= f([z]) = \sum_{l = 0}^n \alpha_l f\left( \left[(yx)^{n - l}y^{2l + 1}\right] \right)\\
            &= -\sum_{l = 0}^n \alpha_l (n - l + 2)\left[(yx)^{n - l 1}y^{2l + 1}\right]
                - \sum_{l = 0}^{n} \sum_{i = 0}^{l - 1}\alpha_l \frac{l!}{i!} \left[(yx)^{n - i + 1}y^{2i + 1}\right]\\
            &= -\sum_{l = 0}^n \alpha_l (n - l + 2)\left[(yx)^{n - l 1}y^{2l + 1}\right]
                -\sum_{l = 0}^{n - 1}\sum_{j = l + 1}^{n}\frac{j!}{l!}\alpha_l\left[(yx)^{n - l + 1}y^{2l + 1}\right]\\
            &= -2\alpha_n (yx)y^{2n + 1}
                - \sum_{l = 0}^{n - 1}\left((n - l + 2) + \sum_{j = l + 1}^{n}\frac{j!}{l!}\right)\alpha_l\left[(yx)^{n - l + 1}y^{2l + 1}\right]
        \end{align*}
        Luego $\alpha_l = 0$ para todo $l$, $0 \leq l \leq n$.
    \end{itemize}
    De este cálculo se deduce que
        $E^{2}_{2, 0} = \left\langle \sum_{l = 0}^n \frac{n!}{l!}\left[(yx)^{n - l}y^{2l}\right] \mid n \geq 0 \right\rangle$.
    Más aún, como los generadores que encontramos son combinaciones lineales homogéneas no nulas de elementos de la base de $E^{1}_{2, 0}$
    y entre ellos tienen grados distintos, resulta que son linealmente independientes.
\end{proof}
\begin{prop}
    El espacio vectorial $E_{1, 0}^{2}$ tiene como base al conjunto
    \begin{align*}
    \Bigg\lbrace &\sum_{l = 0}^n \frac{n!}{l!}\left(\ov{(yx)^{n - l}y^{2l}}, 0\right), 
        \left(-\ov{(yx)^{2n}}, \ov{xy^{2n + 1} + (yx)y^{2n}}\right), \left(0, \ov{y^{n}}\right),\\
        &\qquad \left(\ov{y^{2n + 1}}, \sum_{l = 0}^n \frac{n!}{l!}\frac{n + 1}{n + 1 - l}\ov{x(yx)^{n - l}y^{2l}} 
            + \sum_{l = 0}^n \frac{n!}{l!}\frac{1}{n - l}\ov{(yx)^{n - l}y^{2l + 1}} \right) \mid n \geq 0 \Bigg\rbrace.
    \end{align*}
\end{prop}
\begin{proof}
    Sea $\left([z], [w]\right) \in \Ker(d_{0}^{(1)})$, nuevamente podemos suponer que $w$ y $z$ son homogéneos y tienen el mismo grado.
    Separamos el cálculo de $[z]$ y de $[w]$ según la paridad sus grados.
    \begin{itemize}
        \item Si el grado es $2n$,
            \[
                ([z], [w]) = \left( \sum_{l = 0}^{n} \alpha_l\left[(yx)^{n - l}y^{2l}\right],
                    \sum_{l = 0}^{n - 1}\beta_l \left[x (yx)^{n - l - 1}y^{2l + 1}\right]
                        +  \sum_{l = 0}^{n} \gamma_l\left[(yx)^{n - l}y^{2l}\right]\right)               
            \]
        y por lo tanto
        \begin{align*}
            0 &= d_{0}^{(1)}([z], [w]) = \alpha_n \sum_{l = 0}^{n - 1}\frac{n!}{l!}\left[x(yx)^{n - l}y^{2l}\right]
               + \sum_{l = 0}^{n - 1}\alpha_l \left[x(yx)^{n - l}y^{2l}\right]\\
            &\qquad  + \sum_{l = 0}^{n - 1}\beta_l \left(\left[x(yx)^{n - l - 1}y^{2l + 2}\right] + \left[(yx)^{n - l}y^{2l} \right] \right)\\
            &\qquad + \sum_{l = 0}^{n - 1}\gamma_l \left(\left[(yx)^{n - l}y^{2l + 1} \right]
                - \left[x(yx)^{n - l - 1}y^{2l + 2}\right] - (n - l)\left[x(yx)^{n - l}y^{2l} \right]\right)\\
            &= \sum_{l = 0}^{n - 1}(\gamma_l - \beta_l)\left[(yx)^{n - l}y^{2l + 1}\right]
                + \sum_{l = 0}^{n - 1}(\beta_l - \gamma_l)\left[ x(yx)^{n - l - 1}y^{2l + 2} \right] \\
            &\qquad + \sum_{l = 0}^{n - 1}\left( \alpha_n \frac{n!}{l!} - \alpha_l -(n - l)\gamma_l \right)\left[x(yx)^{n - l}y^{2l}\right].
        \end{align*}
        Como todos los monomios que aparecen en las sumas son linealmente independientes, resulta que $\beta_l = \gamma_l$
            y $\alpha_l = \alpha_n \frac{n!}{l!} - (n - l)\gamma_l$ para todo $l$, $0 \leq l \leq n - 1$. Luego
        \begin{align*}
            ([z], [w]) &= \Bigg(\sum_{l = 0}^{n}\left(\alpha_n \frac{n!}{l!} - (n - l)\gamma_l\right)\left[(yx)^{n - l}y^{2l}\right],\\
                &\qquad \sum_{l = 0}^{n - 1}\gamma_l\left[ x(yx)^{n - l-1 }y^{2l + 1} + (yx)^{n - l}y^{2l}\right]
                    + \gamma_n\left[y^{2n}\right]\Bigg)\\
            &= \alpha_n \sum_{l = 0}^n\frac{n!}{l!}\left( \left[(yx)^{n - l}y^{2l} \right], 0 \right)
                + \gamma_n\left(0, \left[y^{2n}\right]\right)\\
            &\qquad + \sum_{l = 0}^{n - 1}\gamma_l \left( -(n - l)\left[(yx)^{n - l}y^{2l}\right],
                    \left[x(yx)^{n - l - 1}y^{2l + 1}\right] + \left[(yx)^{n - l}y^{2l}\right] \right)\\
            &= \alpha_n \sum_{l = 0}^n\frac{n!}{l!}\left( \left[(yx)^{n - l}y^{2l} \right], 0 \right)
                + \gamma_n\left(0, \left[y^{2n}\right]\right)\\
            &\qquad + \gamma_{n - 1} \left( -\left[(yx)y^{2n - 2}\right],
                    \left[xy^{2n - 1}\right] + \left[yxy^{2n - 2}\right] \right).\\        
        \end{align*}
        La última igualidad se debe a que
            \[ 
                \left( -(n - l)\left[(yx)^{n - l}y^{2l}\right],
                    \left[x(yx)^{n - l - 1}y^{2l + 1}\right] + \left[(yx)^{n - l}y^{2l}\right] \right)
            \]
        pertenece a $\Ima\left(d_{1}^{(1)}\right)$ para todo $l$, $0 \leq l \leq n-2$. De hecho, utilizando
        la notación de la Proposición \ref{prop_imad11}, resulta que
        \[ 
             \left( -(n - l)\left[(yx)^{n - l}y^{2l}\right],
                 \left[x(yx)^{n - l - 1}y^{2l + 1}\right] + \left[(yx)^{n - l}y^{2l}\right] \right)
                     = \left(\theta^{1}_{n - l - 1, l}, \theta^{2}_{n - l - 1, l}\right).
        \]
    
    \item Si el grado es $2n + 1$,
    \[
        ([z], [w]) = \left(\sum_{l = 0}^{n}\alpha_l \left[(yx)^{n - l}y^{2l + 1}\right],
            \sum_{l = 0}^{n}\beta_l \left[x(yx)^{n - l}y^{2l}\right]
                + \sum_{l = 0}^{n}\gamma_l\left[(yx)^{n - l}y^{2l + 1}\right]\right).    
    \]
    Por la Proposición \ref{prop_imad11}, $\left(\lambda_{b, k}^{1}, \lambda_{b, k}^{2}\right)$ pertenece
    a la imagen de $d_1^{(1)}$  para todo $b\geq 1$ y para todo $k \geq 0$. Por lo tanto
    reduciendo módulo borde podemos suponer que 
    \[
        ([z], [w]) = \left(\alpha_n \left[y^{2n + 1}\right],
            \sum_{l = 0}^{n}\beta_l \left[x(yx)^{n - l}y^{2l}\right]
                + \sum_{l = 0}^{n}\gamma_l\left[(yx)^{n - l}y^{2l + 1}\right]\right). 
    \]
    Si $ ([z], [w])$ pertenece al núcleo de $d_{0}^{(1)}$, entonces
    \begin{align*}
        0 &= d_{0}^{(1)}([z], [w]) = \alpha \sum_{l = 0}^{n}\frac{n!}{l!}\left[(yx)^{n +1 - l}y^{2l}\right] - \alpha \left[xy^{2n + 1}\right]\\
        &\qquad +\sum_{l = 0}^{n}\beta_l \left( \left[x(yx)^{n - l}y^{2l + 1}\right] - \left[(yx)^{n - l + 1}y^{2l} \right] \right)\\
        &\qquad + \sum_{l = 0}^{n - 1}\gamma_l \left( \left[(yx)^{n - l}y^{2l + 2}\right]
            - \left[x(yx)^{n - l - 1}y^{2l + 3}\right] - (n - l)\left[x(yx)^{n - l}y^{2l+ 1}\right]\right)\\
        &= \left(\alpha n! - \beta_0\right)\left[(yx)^{n + 1}\right]
            + \sum_{l = 1}^{n}\left(\frac{n!}{l!}\alpha - \beta_l + \gamma_{l - 1}\right)\left[(yx)^{n + 1 - l}y^{2l}\right]\\
        &\qquad + \left(\beta_0 - \gamma_0 n\right)\left[x(yx)^{n}y\right]
            + \sum_{l = 0}^{n - 1}\left(\beta_l - \gamma_{l - 1} - \gamma_l(n - l)\right)\left[x(yx)^{n - l}y^{2l + 1}\right]\\
        &\qquad + \left(\beta_n - \gamma_{n - 1} - \alpha\right)\left[xy^{2n + 1}\right].
    \end{align*}
    Como todos los monomios son linealmente independientes obtenemos el siguiente sistema de ecuaciones:
    \begin{align*}
        \beta_0 &= n! \alpha,\\
        \beta_l &= \frac{n!}{l!}\alpha + \gamma_{l - 1} \text{ para todo }l, 1 \leq l \leq n,\\
        \beta_l &= \gamma_{l - 1} + \gamma_l(n - l) \text{ para todo } l, 1\leq n \leq n - 1,\\
        \gamma_0 &= \frac{1}{n}\beta_0,\\
        \gamma_{n - 1} &= \beta_n - \alpha.     
    \end{align*}
    De estas ecuaciones se deduce que $\gamma_l = \alpha \frac{n!}{l!} \frac{1}{n - l}$ para todo $l$, $0 \leq l \leq n - 1$ y
    que $\beta_l  = \alpha \frac{n!}{l!} \frac{n + 1}{n - l + 1}$ para todo $0 \leq l \leq n$. Por lo tanto, $([z], [w])$ es igual a
    \begin{align*}
     &\alpha \Bigg(\left[y^{2n + 1}\right],
            \sum_{l = 0}^{n}\frac{n!}{l!} \frac{n + 1}{n - l + 1}\left[x(yx)^{n -l}y^{2l}\right]
                + \sum_{l = 0}^{n}\frac{n!}{l!}\frac{1}{n - l}\left[(yx)^{n - l}y^{2l + 1}\right]\Bigg)\\
      &\qquad + \gamma_n \left(0, \left[y^{2n + 1}\right]\right).
    \end{align*}
    \end{itemize}
    Con este cálculo probamos que el conjunto que aparece en el enunciado de la proposición genera. Alcanza
    con mirar los grados de estos generadores para deducir que el conjunto es una base.
\end{proof}

De manera similar al ejemplo introductorio de la Sección \ref{sucesiones_espectrales} se tiene la siguiente sucesión
exacta corta
\begin{align*}
\xymatrix{
    0 \ar[r] & E_{1, p}^{2} \ar[r]^{\iota} & \Hy_{p + 1}(A) \ar[r]^{\pi} & E_{2, p - 1}^{2} \ar[r] & 0.
}
\end{align*}
Para encontrar una
base de $\Hy_{p + 1}(A, A)$ alcanza con completar una base de $\iota\left(E_{1, p}^{2}\right)$ con elementos
tales que sus imágenes, por medio de $\pi$, formen una base de $E_{2, p - 1}^{2}$. Más concretamente, si $\{\ov{a_i}\}_i$
es una base de $E_{1, p}^{2}$, el conjunto $\{\ov{b_j}\}_j$ es una base de $E_{2, p - 1}^{2}$ y $\{\ov{c_j}\}_j$ son
elementos tales que $f(c_j) = -d(b_j)$ para todo $j$, donde $f = \delta$ o $f = \partial$ según corresponda, entonces 
el conjunto $\{(\ov{a_i}, 0)\}_i \cup \{(\ov{b_j}, \ov{c_j})\}_j$ es una base de $\Hy_{p + 1}(A, A)$. A partir
de esta observación es sencillo verificar la siguiente proposición.

\begin{prop} Se tienen los siguientes isomorfismos:
\begin{align*}
    		&\Hy_0(A, A) \cong \left\langle \ov{xy^n}, \ov{y^n} \mid n \geq 0 \right\rangle, \\
    		&\Hy_1(A, A) \cong \\
    		    &\qquad\left\langle \left(\ov{y^{2n + 1}},
    			\sum_{i = 0}^n\frac{n!}{i!} \left(\frac{n + 1}{n -(i - 1)}\ov{x(yx)^{n - i}y^{2i}}
    				+ \frac{1}{n - i}\ov{(yx)^{n - i}y^{2i + 1}} \right) \right) \mid n \geq 0 \right\rangle\\
    			&\qquad\oplus \left\langle \left(\sum_{i = 0}^n\frac{n!}{i!}\ov{(yx)^{n - i}y^{2i}}, 0\right),
    				\left(0, \ov{y^{n}}\right), \left(\ov{-yxy^{2n}},\ov{xy^{2n + 1}} + \ov{yxy^{2n}}\right) \mid n \geq 0 \right\rangle, \\
    		&\Hy_2(A, A) \cong \left\langle \left(\ov{xy^{2n}}, 0\right), \left(0, \sum_{i = 0}^n\frac{n!}{i!}\ov{(yx)^{n - i}y^{2i}}\right) \mid n \geq 0 \right\rangle, \\
    		&\Hy_{2p + 1}(A, A) \cong \left\langle \left(\sum_{i = 0}^n\frac{n!}{i!}\ov{(yx)^{n - i}y^{2i}}, 0\right) ,
    			\left(-\ov{(yx)y^{2n}}, \ov{xy^{2n}}\right) \mid n \geq 0 \right\rangle, \\
    		&\Hy_{2p + 2}(A, A) \cong \left\langle \left(\ov{xy^{2n}}, 0 \right) ,
    			\left(\sum_{i = 0}^n\frac{n!}{i!}\ov{(yx)^{n - i}y^{2i + 1}}, \sum_{i = 0}^n\frac{n!}{i!}\ov{(yx)^{n - i}y^{2i}}\right)\mid
    				n \geq 0 \right\rangle, \\
    \end{align*}
    para todo $p \geq 1$. En cada caso los conjuntos generadores son bases del espacio que generan.
\end{prop}

\section{Cohomología de Hochschild}
Para calcular la cohomología de Hochschild procederemos de manera similar al cálculo de la homología. Nuevamente consideramos
el complejo doble $X_{\bullet, \bullet}$ y le aplicamos el funtor $\Hom_{A^e}(-, A)$. La homología total
de este nuevo complejo es isomorfa a $H^{\bullet}(A, A)$. Identificando de manera natural $\Hom_{A^{e}}(A\ox W \ox A)$ con
$\Hom_{\field}(W, A)$, $\Hom_{\field}(W, A)$ con $W^{\ast}\ox A$ y luego este último con $A^{dim(W)}$ para todo
espacio vectorial $W$ de dimension finita, resulta el complejo doble
\begin{align*}
\xymatrix{
	& & \\
	& A \ar@{-->}[u]^{\partial} \ar[r]^{d} & A \ar@{-->}[u]^{\partial'} \\
    & A \ar[u]^{\delta} \ar[r]^{d} & A \ar[u]^{\delta'} \\
    & A \ar[u]^{\partial} \ar[r]^{d} & A \ar[u]^{\partial'} \\
    A \ar[r]^{d^{0}} &A \oplus A \ar[r]^{d^{1}} \ar[u]^{\hat{\delta}} & A \ar[u]^{\delta'}
} 
\end{align*}
con los siguientes diferenciales $\field$-lineales
\begin{align*}
    d^{0}(a) &= \left([x, a], [y,a]\right),\\
    d^{1}(a, b) &= [y^2, a] + [yb + by, x] - (xya + ayx) - xbx,\\
    d(a) &= [y^2, a] - (xya + ayx),\\
    \hat{\delta}(a, b) &= xa + ax,\\
    \delta(a) &= xa + ax,\\
    \partial(a) &= [x, a].
\end{align*}
Para calcular al homología total de este complejo vamos a utilizar la sucesión espectral inducida por la filtración por columnas.
Denotamos por $E^{\bullet, \bullet}_{r}$ a la sucesión espectral.

\subsection{Primera página}
Esta vez no es necesario calcular las imágenes de los morfismos verticales
debido a que ya lo hicimos para el cálculo de la homología de Hochschild. Procederemos directamente al cálculo de la primera página.
Es claro que $E^{0, 0}_{1} = A$, $E^{1, 0}_{1} = \Ker(\hat{\delta}) = \Ker(\delta)\oplus A$ y $E^{2, 0}_{1} = \Ker(\delta') = \Ker(\delta)$,
por lo tanto empezamos calculando $\Ker(\delta)$.

\begin{prop}
El espacio vectorial $\Ker(\delta)$ tiene como base al conjunto
\begin{align*}
    \Bigg\lbrace \left[x(yx)^{b}y^{2k + 1}\right] - \sum_{i = 0}^k \frac{k!}{i!}\left[(yx)^{k + b + 1 - i}y^{2i}\right],
    \left[x(yx)^{b}y^{2k}\right] \mid b, k \geq 0\Bigg\rbrace
\end{align*}
\end{prop}
\begin{proof}
    Sea $z \in \Ker(\delta)$. Al igual que en todos los cálculos anteriores podemos suponer que $z$ es homogéneo y separar
    el cálculo según la paridad de su grado.
    \begin{itemize}
        \item Si $\gr(z) = 2n$,
            \[
                z= \sum_{l = 0}^{n}\alpha_l(yx)^{n - l}y^{2l}  +\sum_{l = 0}^{n - 1}\beta_l x(yx)^{n - l - 1}y^{2l + 1}.            
            \]
         Luego
         \begin{align*}
             0 &= \delta(z) = \sum_{l = 0}^{n}\alpha_l\delta\left((yx)^{n - l}y^{2l}\right) 
                 +\sum_{l = 0}^{n - 1}\beta_l \delta\left(x(yx)^{n - l - 1}y^{2l + 1}\right)\\
             &= \sum_{l = 0}^{n}\left( \alpha_l + \alpha_n \frac{n!}{l!} \right) x(yx)^{n - l}y^{2l}
                 + \sum_{l = 0}^{n - 1}\sum_{i = 0}^{l}\beta_l \frac{l!}{i!} x(yx)^{n - i}y^{2i} \\
             &= \sum_{l = 0}^{n}\left( \alpha_l + \alpha_n \frac{n!}{l!} \right) x(yx)^{n - l}y^{2l}
                 + \sum_{i = 0}^{n - 1}\sum_{l = i}^{n - 1}\beta_l\frac{l!}{i!} x(yx)^{n - i}y^{2i} \\
             &= \sum_{l = 0}^{n}\left(\alpha_l + \alpha_n \frac{n!}{l!} + \sum_{j = l}^{n - 1}\beta_j \frac{j!}{l!}\right)x(yx)^{n - l}y^{2l}.
         \end{align*}
         Como los monomios que aparecen en la suma son linealmente independientes, resulta que $\alpha_n = 0$
         y $\alpha_l = -\sum_{j = l}^{n - 1}\beta_j \frac{j!}{l!}$ para todo $l$, $0 \leq l < n$. Por lo tanto
            \begin{align*}
                z&= -\sum_{l = 0}^{n - 1}\left(\sum_{j = l}^{n - 1}\beta_j \frac{j!}{l!}\right)(yx)^{n - l}y^{2l} 
                    +\sum_{l = 0}^{n - 1}\beta_l x(yx)^{n - l - 1}y^{2l + 1}\\
                 &= -\sum_{j = 0}^{n - 1}\beta_j \left(\sum_{l = 0} ^{j}\frac{j!}{l!}(yx)^{n -l}y^{2l} + x(yx)^{n - j - 1}y^{2j + 1}\right).
            \end{align*}
         \item Si $\gr(z) = 2n + 1$,
         \[
             z= \sum_{l = 0}^{n}\alpha_l(yx)^{n - l}y^{2l + 1}  +\sum_{l = 0}^{n}\beta_l x(yx)^{n - l}y^{2l}.            
         \]
         Luego
         \begin{align*}
              0 &= \delta(z) = \sum_{l = 0}^{n}\alpha_l\delta \left((yx)^{n - l}y^{2l + 1}\right)
                  + \sum_{l = 0}^{n}\beta_l \delta \left( x(yx)^{n - l}y^{2l}\right)\\
                &= \sum_{l = 0}^{n}\sum_{i = 0}^{l}\alpha_l \frac{l!}{i!}(yx)^{n + 1 - i}y^{2i} + \sum_{l = 0}^{n}\alpha_l x(yx)^{n - l}y^{2l + 1}.
         \end{align*}
         y por lo tanto $\alpha_l = 0$ para todo $l$, $0 \leq l \leq n$, es decir,  $z=\sum_{l = 0}^{n}\beta_l x(yx)^{n - l}y^{2l}$.
    \end{itemize}
    Este cálculo prueba que el conjunto del enunciado genera $\Ker(\delta)$ y es claro que es linealmente independiente.
\end{proof}

Como $E_{1}^{1, 2i + 1} = E_{1}^{2, 2i + 1} = \Ker(\partial)/ \Ima(\delta)$ para todo $i \geq 0$
y $E_{1}^{1, 2i} = E_{1}^{2, 2i} = \Ker(\delta)/ \Ima(\partial)$, se obtienen las siguientes dos proposiciones.

\begin{prop}
    Los espacios vectoriales $E_{1}^{1, 2i + 1} = E_{1}^{2, 2i + 1} = \Ker(\partial)/ \Ima(\delta)$ tienen como base
    al conjunto $\left\lbrace \sum_{l = 0}^{n}\frac{n!}{l!}\left[(yx)^{n - l}y^{2l}\right] | n \geq 0 \right\rbrace$, para todo $i \geq 0$.
\end{prop}

\begin{prop}
    Los espacios vectoriales $E_{1}^{1, 2i} = E_{1}^{2, 2i} = \Ker(\delta)/ \Ima(\partial)$ tienen como base
    al conjunto $\left\lbrace \left[xy^{2n}\right] | n \geq 0 \right\rbrace$, para todo $i \geq 1$.
\end{prop}

El siguiente diagrama describe la primera página de la sucesión espectral
\begin{align*}
\xymatrix{
	& & \\
	& \left\langle \sum_{l = 0}^{n}\frac{n!}{l!}\left[(yx)^{n - l}y^{2l}\right] \right\rangle \ar[r]^{d_{(1)}} \ar@{--}[u]
	    & \left\langle \sum_{l = 0}^{n}\frac{n!}{l!}\left[(yx)^{n - l}y^{2l}\right] \right\rangle \ar@{--}[u] \\
    & \left\langle \left[xy^{2n}\right] \right\rangle \ar@{--}[u] \ar[r]^{d_{(1)}} & \left\langle \left[xy^{2n}\right] \right\rangle \ar@{--}[u] \\
    & \left\langle \sum_{l = 0}^{n}\frac{n!}{l!}\left[(yx)^{n - l}y^{2l}\right] \right\rangle \ar[r]^{d_{(1)}} \ar@{--}[u]
	    & \left\langle \sum_{l = 0}^{n}\frac{n!}{l!}\left[(yx)^{n - l}y^{2l}\right] \right\rangle \ar@{--}[u] \\ 
    A \ar[r]^{d^{0}_{(1)}} & \Ker(\delta) \oplus A \ar[r]^{d^{1}_{(1)}} \ar@{--}[u] & \Ker(\delta) \ar@{--}[u]
} 
\end{align*}
donde $d^{0}_{(1)}, d^{1}_{(1)}$ y $d_{(1)}$ son los morfismos inducidos en la homología por $d^{0}, d^{1}$ y $d$ respectivamente.

\subsection{Segunda página}
Al igual que para el cálculo de la homología de Hochschild, la sucesión espectral se estaciona rápidamente y $E_{2} = E_{\infty}$.
Vamos a empezar calculando bases para las imágenes de $d^{0}_{(1)}, d^{1}_{(1)}$ y $d_{(1)}$.

\begin{prop}
    Para todo $i \geq 0$, el morfismo $d_{(1)}: E^{1, 2i + 1}_{1} \to E^{2, 2i + 1}_{1}$ es nulo.
\end{prop}
\begin{proof}
Sea $w = \sum_{l = 0}^{n}\frac{n!}{l!}(yx)^{n - l}y^{2l}$, de modo que $[w]$ es un elemento de la base de $E^{1, 2i + 1}_{1}$.
\begin{itemize}
    \item Si $n = 0$, resulta que $d(w) = d(1) = \left[y^2, 1\right] - xy - yx = - xy - yx$. Luego $d(w)$ pertenece a la imagen de $\delta$ 
    y $d_{(1)}([w]) = 0$.
    \item Si $n \geq 1$, podemos suponer que $n = m + 1$, donde $m \geq 0$. Esto permite reescribir $w$ de la siguiente manera
    \[
        w = \sum_{l = 0}^{m + 1}\frac{(m + 1)!}{l!} (yx)^{m + 1 -l}y^{2l} = (m + 1)y^{2m + 1}x + y^{2m + 2}.
    \]
    Aplicando $d$ obtenemos que $d(w) = (m + 1)d\left(y^{2m + 1}x\right) + d\left(y^{2m + 2}\right)$.    
    Calculamos $d\left(y^{2m + 1}x\right)$ y $d\left(y^{2m + 2}\right)$ por separado:
    \begin{align*}
        d\left(y^{2m + 1}x\right) &= \left[y^{2}, y^{2m + 1}x\right] - xy^{2m + 2}x - y^{2m + 1}xyx \\
            &= y^{2m + 1}\left[y^{2}, x\right] - y^{2m + 1}xyx\\
        & = y^{2m + 1}\left(y^2x - xy^{2} - xyx\right) = 0
    \end{align*}
     y
     \begin{align*}
         d\left(y^{2m + 2}\right) &= \left[y^{2}, y^{2m + 2}\right] - xy^{2m + 3} - y^{2m + 3}x\\
         &= xy^{2m + 3} - \sum_{l =0}^{m + 1}\frac{(m + 1)!}{l!}(yx)^{m + 2 - l}y^{2l}.
     \end{align*}
     Como $xy^{2m + 3} - \sum_{l =0}^{m + 1}\frac{(m + 1)!}{l!}(yx)^{m + 2 - l}y^{2l}$ pertenece a la imagen de $\delta$,
     entonces $d_{(1)}([w]) = 0$.
\end{itemize}
\end{proof}


\begin{prop}
    Para todo $i \geq 1$, el morfismo $d_{(1)}: E^{1, 2i}_{1} \to E^{2, 2i}_{1}$ es nulo.
\end{prop}
\begin{proof}
Sea $\left[xy^{2n}\right]$ un elemento de la base de $E^{1, 2i}_{1}$, luego
\begin{align*}
    d(xy^{2n}) &= \left[y^{2}, xy^{2n}\right] - xyxy^{2n} - xy^{2n + 1}x \\
    &= y^{2}xy^{2n} - xy^{2n + 2} - xyxy^{2n} - \sum_{l = 0}^{n}\frac{n!}{l!}x(yx)^{n - l + 1}y^{2l} \\
    &= xy^{2n + 2} + xyxy^{2n} - xy^{2n + 2} - xyxy^{2n} - \sum_{l = 0}^{n}\frac{n!}{l!}x(yx)^{n - l + 1}y^{2l} \\
    &= -\sum_{l = 0}^{n}\frac{n!}{l!}x(yx)^{n - l + 1}y^{2l}.
\end{align*}
Como $x(yx)^{b}y^{c}$ pertenece a $\Ima(\partial)$ para todo $b \geq 1$ y para todo $c \geq 0$, resulta que
$d_{(1)}(\left[xy^{2n}\right]) = 0$ para todo $n \geq 0$.
\end{proof}

\begin{coro}
    Para todo $i \geq 0$, los espacios vectoriales $E_{2}^{1, 2i} = E_{2}^{2, 2i}$ tienen como base
    al conjunto
       $\left\lbrace \ov{xy^{2n}} | n \geq 0 \right\rbrace$.
\end{coro}

El siguiente paso consiste en calcular que valores toma $d_{(1)}^{0}$ en los elementos de la base de $E^{0, 0}_{1} = A$.
Por definición $d^{0}(z) = \left([x,z], [y,z]\right)$ para todo $z \in A$. Ya hemos escrito $[x, z]$ y $[y, z]$ en términos de la base de $A$
 cuando obtuvimos las bases de las imágenes de los morfismos $\partial$ y $d_0$ en el cálculo
de la homología de Hochschild. Si $z  = x^a(yx)^by^c\in B$, resulta que
\begin{itemize}
    \item si $a = 0$,
    \begin{itemize}
        \item si $c = 2k$,
        \begin{itemize}
            \item si $b = 0$,
            \[
                d^{0}_{(1)}([z]) = \left(-\sum_{l = 0}^{k -1}\frac{k!}{l!}\left[x(yx)^{k - l}y^{2l}\right], 0\right).
            \]
            \item Si $b \geq 1$,
            \[
                d^0_{(1)}([z]) = \left(\left[x(yx)^by^{2k}\right],\left[x(yx)^{b - 1}y^{2k + 2}\right]
                    + b\left[x(yx)^{b}y^{2k}\right] - \left[(yx)^{b}y^{2k + 1}\right]\right).
            \]
        \end{itemize}
        \item Si $c = 2k + 1$,
        \begin{itemize}
            \item si $b = 0$,
            \[
                d^{0}_{(1)}([z]) = \left(-\sum_{l = 0}^{k}\frac{k!}{l!}\left[(yx)^{k + 1 - l}y^{2l}\right] + \left[xy^{2k + 1}\right], 0\right).
            \]
            \item Si $b \geq 1$,
            \begin{align*}
                d^{0}_{(1)}([z]) &= \Bigg(-\sum_{l = 0}^{k}\frac{k!}{l!}\left[(yx)^{k + 1 + b - l}y^{2l}\right]
                    + \left[x(yx)^{b}y^{2k + 1}\right],\\
                &\qquad \left[x(yx)^{b - 1}y^{2k + 3}\right]
                    + b\left[x(yx)^{b}y^{2k + 1}\right] - \left[(yx)^{b}y^{2k + 2}\right]\Bigg).
            \end{align*}
        \end{itemize}
     \end{itemize}
     \item Si $a = 1$,
     \begin{itemize}
          \item si $c = 2k$,
          \[
               d^{0}_{(1)}([z]) = \left(0, \left[(yx)^{b + 1}y^{2k}\right] - \left[x(yx)^by^{2k + 1}\right] \right).          
          \]
          \item Si $c = 2k + 1$,
          \[
              d^{0}_{(1)}([z]) = \Bigg(-\sum_{l = 0}^{k}\frac{k!}{l!}\left[x(yx)^{k + 1 + b - l}y^{2l}\right],
                  \left[(yx)^{b + 1}y^{2k + 1}\right] - \left[x(yx)^{b}y^{2k + 2}\right]\Bigg)
          \]      
     \end{itemize}
\end{itemize}
Es posible simplificar el conjunto de generadores que acabamos de obtener notando que si $b \geq 0$ y $k \geq 0$, entonces
\begin{align*}
    d^{0}&\left((yx)^{b + 1}y^{2k + 1} + x(yx)^{b}y^{2(k + 1)}\right) = \\
    & \Bigg(-\sum_{l = 0}^{k}\frac{k!}{l!}(yx)^{k + 2 + b - l}y^{2l}
                    + x(yx)^{b + 1}y^{2k + 1}, (b + 1) x(yx)^{b + 1}y^{2k + 1}\Bigg)
\end{align*}
y
\begin{align*}
    d^{0}&\left(x(yx)^by^{2k + 1} + (yx)^{b + 1}y^{2k}\right) = \Bigg(-\sum_{l = 0}^{k - 1}\frac{k!}{l!}x(yx)^{k + 1 + b - l}y^{2l},
                  (b + 1)x(yx)^{b + 1}y^{2k}\Bigg).
\end{align*}
Para enunciar la próxima proposición vamos a utilizar la siguiente notación:
\begin{align*}
    \eta_k &= \left(\sum_{l = 0}^{k - 1}\frac{k!}{l!}\left[x(yx)^{k -l}y^{2l}\right], 0\right),\\
    \theta_{b,k} &= \left(\left[x(yx)^{b + 1}y^{2k}\right], \left[x(yx)^{b}y^{2k + 2}\right] + (b + 1)\left[x(yx)^{b  + 1}y^{2k}\right] 
        - \left[(yx)^{b + 1}y^{2k + 1}\right]\right),\\
    \lambda_{k} &= \left(\sum_{l = 0}^{k}\frac{k!}{l!}\left[(yx)^{k + 1 - l}\right] - \left[xy^{2k + 1}\right], 0\right),\\
    \mu_{b, k} &= \Bigg(-\sum_{l = 0}^{k}\frac{k!}{l!}\left[(yx)^{k + 2 + b - l}y^{2l}\right]
                    + \left[x(yx)^{b + 1}y^{2k + 1}\right], (b + 1) \left[x(yx)^{b + 1}y^{2k + 1}\right]\Bigg),\\
    \nu_{b , k} &=\left(0, \left[(yx)^{b + 1}y^{2k}\right] - \left[x(yx)^{b}y^{2k + 1}\right]\right),\\
    \xi_{b, k} &= \Bigg(-\sum_{l = 0}^{k - 1}\frac{k!}{l!}\left[x(yx)^{k + 1 + b - l}y^{2l}\right],
                  (b + 1)\left[x(yx)^{b + 1}y^{2k}\right]\Bigg).
\end{align*}

\begin{prop}\label{coh_ima_d01}
    El conjunto $\left\lbrace \eta_k, \theta_{b , k}, \lambda_k, \mu_{b , k}, \nu_{b, k}, \xi_{b, k} \mid b, k \geq 0 \right\rbrace$
    es una base de $\Ima\left(d^{0}_{(1)}\right)$.
\end{prop}
\begin{proof}
    Ya sabemos que el conjunto del enunciado genera $\Ima\left(d^{0}_{(1)}\right)$. Nos resta verificar que los generadores son linealmente
    independientes. Sea $w$ una combinación lineal de los generadores y supongamos que $w$ es igual a cero. Como todos los elementos son homogéneos, en el sentido que cada
    coordenada es homogénea y a su vez tienen el mismo grado, podemos suponer que $w$ es una combinación lineal homogénea. Si $\gr(w) = 2k + 1$,
    entonces
    \begin{align*}
        0 &= w = \alpha \eta_k + \sum_{i = 0}^{k - 1} \beta_i \theta_{k -i - 1, i} + \sum_{i = 0}^{k - 1}\gamma_i \xi_{k - i - 1, i}
        = \alpha \eta_k + \sum_{i = 0}^{k - 1}\gamma_i \xi_{k - i - 1, i}\\
        &\qquad + \sum_{i = 0}^{k - 1} \beta_i \Bigg(\left[x(yx)^{k - i}y^{2i}\right],
            \left[x(yx)^{k - i - 1}y^{2i + 2}\right] + (k - i)\left[x(yx)^{k - i}y^{2i}\right]\\
            &\qquad \qquad - \left[(yx)^{k - i}y^{2i + 1}\right]\Bigg)
    \end{align*}
    Los terminos de la forma $(yx)^{k - i}y^{2i + 1}$ no aparecen en los elementos $\xi_{\ast, \ast}$ o $\eta_{\ast}$ por lo tanto $\beta_i = 0$ para todo $i$, $0 \leq i \leq k - 1$ y obtenemos la siguiente igualdad
    \begin{align*}
        0 &= w = \alpha \eta_k +  \sum_{i = 0}^{k - 1}\gamma_i \xi_{k - i - 1, i} \\
        &= \alpha \eta_k + \sum_{i = 0}^{k - 1}\gamma_i \Bigg(-\sum_{l = 0}^{i - 1}\frac{i!}{l!}\left[x(yx)^{k - l}y^{2l}\right],
                  (k - i)\left[x(yx)^{k - i}y^{2i}\right]\Bigg).
    \end{align*}
    Como la segunda coordenada de $\eta_k$ es cero, entonces $\gamma_i = 0$ para todo $i$, $0 \leq i \leq k - 1$ y por lo tanto
    $\alpha = 0$. El caso $\gr(w) = 2k$ es análogo.
\end{proof}

Nos resta calcular una base de $\Ima\left(d^{1}_{(1)}\right)$. Sea $z = (z_1, z_2)$ un elemento de la base de $\Ker(\delta) \oplus A$.
Como ya obtuvimos una base de $\Ima\left(d^{0}_{(1)}\right)$ podemos reducir a $z$ módulo $\Ima\left(d^{0}_{(1)}\right)$ para simplificar el cálculo. Utilizando la notación de la proposición anterior, 
sabemos que $\theta_{b, k}, \lambda_k$ y $\mu_{b, k} \in \Ima\left(d^{0}_{(1)}\right)$ para todo $b, k \geq 0$ y por lo tanto podemos
restarle a $z$ una combinación lineal de estos elementos. Esta observación permite suponer que $z = (xy^{2k}, 0)$ o 
$z = (0, x^{a}(yx)^{b}y^{c})$. Si $z = (xy^{2k}, 0)$ para algún $k \geq 0$, entonces
\[
    d^1(z) = \left[y^{2}, xy^{2k}\right] - x(yx)y^{2k} - xy^{2k + 1}x = -\sum_{i = 0}^{k}\frac{k!}{i!}x(yx)^{k + 1- i}y^{2i}.
\]
Si $z = (0, x^{a}(yx)^{b}y^{c})$, separamos el cálculo en distintos casos.
\begin{itemize}
    \item Si $a = 0$,
        \[
            d^1(z) = \left[y(yx)^by^c + (yx)^by^{c + 1}, x\right] - x(yx)^by^cx.        
        \]
    \begin{itemize}
        \item Si $b = 0$,
        \[
            d^1(z) = 2\left[y^{c + 1}, x\right] - xy^cx = 2\left(y^{c + 1}x - xy^{c + 1}\right) - xy^cx.     
        \]
        \begin{itemize}
            \item Si $c = 2k$,
            \[
                d^1(z) = 2\left(y^{2k + 1}x - xy^{2k + 1}\right) = 2\left(\sum_{i = 0}^k \frac{k!}{i!}(yx)^{k + 1 - i}y^{2i} - xy^{2k + 1}\right)
            \]
            \item Si $c = 2k + 1$
            \begin{align*}
                d^1(z) &= 2\left(y^{2k + 2}x - xy^{2k + 2}\right) - xy^{2k + 1}x\\
                &= 2\sum_{i = 0}^{k}\frac{(k + 1)!}{i!}(yx)^{k + 1 - i}y^{2i} - \sum_{i = 0}^k \frac{k!}{i!}x(yx)^{k + 1 - i}y^{2i}\\
                & = (2k + 1)\sum_{i = 0}^k \frac{k!}{i!}x(yx)^{k + 1 - i}y^{2i}.
            \end{align*}
        \end{itemize}
        \item Si $b \geq 1$,
        \begin{align*}
            d^1(z) &= \left[x(yx)^{b - 1}y^{c + 2} + b x(yx)^{b}y^c + (yx)^{b}y^{c + 1}, x\right] - x(yx)^by^c \\
            &= x(yx)^{b - 1}y^{c + 2}x + bx(yx)^by^cx + (yx)^{b}y^{c + 1}x\\
            &\qquad - x(yx)^{b}y^{c + 1} - x(yx)^by^cx\\
            &= x(yx)^{b - 1}y^{c + 2}x + (b - 1)x(yx)^by^cx + (yx)^{b}y^{c + 1}x - x(yx)^{b}y^{c + 1}.
        \end{align*}
        \begin{itemize}
            \item Si $c = 2k$,
                \[
                    d^1(z) = \sum_{i =0}^{k}\frac{k!}{i!}(yx)^{k + b + 1 - i}y^{2i} - x(yx)^by^{2k + 1}.          
                \]
            \item Si $c = 2k + 1$,
                \[
                    d^1(z) = \sum_{i = 0}^{k}\frac{k!}{i!}(k + b)x(yx)^{k +  b + 1 - i}y^{2i}.
                \]
        \end{itemize}
    \end{itemize}
    \item Si $a = 1$
       \begin{align*}
           d^1(z) &= \left[(yx)^{b + 1}y^cx  + x(yx)^{b}y^{c + 1}, x\right]\\
           &= (yx)^{b + 1}y^{c}x - x(yx)^{b + 1}y^{c} + x(yx)^b y^{c + 1}x.
       \end{align*}
    \begin{itemize}
        \item Si $c = 2k$,
        \[
             d^1(z) = \sum_{i = 0}^{k - 1}\frac{k!}{i!} x(yx)^{k + b + 1 -i}y^{2i}.           
        \]
        \item Si $c = 2k + 1$,
        \[
            d^1(z) = \sum_{i = 0}^{k}\frac{k!}{i!} (yx)^{k  + b + 2 -i}y^{2i} - x(yx)^{b + 1}y^{2k + 1}.
        \]
    \end{itemize}
\end{itemize}
Como $\sum_{i = 0}^{k}\frac{k!}{i!}\left[x(yx)^{k +  b + 1 - i}y^{2i}\right]$ y $\sum_{i = 0}^{k - 1}\left[x(yx)^{k + b + 1 -i}y^{2i}\right]$
pertenecen a $\Ima\left(d^{1}_{(1)}\right)$, entonces su resta también, es decir, $\left[x(yx)^{b + 1}y^{2i}\right]$ pertenece a $\Ima\left(d^{1}_{(1)}\right)$
para todo $b, i \geq 0$. Con esta observación y el cálculo anterior probamos la siguiente proposición.
\begin{prop}
    El conjunto
    \[
        \left\lbrace \sum_{i = 0}^{k}\frac{k!}{i!}(yx)^{k + b + 1 - i}y^{2i} - x(yx)^{b}y^{2k + 1},
            x(yx)^{b + 1}y^{2i} \mid b, k \geq 0 \right\rbrace.    
    \]
\end{prop}
Es claro que este conjunto se trata de una base de $\Ima\left(d^{1}_{(1)}\right)$. El siguiente corolario
es consecuencia inmediata de la proposición anterior.
\begin{prop}
    El espacio vectorial $E^{2, 0}_{2}$ tiene como base al conjunto $\left\lbrace \ov{xy^{2n}} \mid n \geq 0\right\rbrace$.
\end{prop}

Nos resta calcular $E^{0,0}_{2}$ y $E^{1, 0}_{2}$.
\begin{prop}
    El espacio vectorial $E^{0, 0}_{2} = \Ker(d^{0}_{(1)})$ es isomorfo a $\field$.
\end{prop}
\begin{proof}
    Sea $[z] \in \Ker\left(d^{0}_{(1)}\right)$ homogéneo de grado $2n$.
    Por definición, $z \in \Ker\left(d^{0}\right)$ si y solo si $[x, z] = [y, z] = 0$.
    Veamos que condiciones impone sobre $z$ la ecuación $[y, z] = 0$. Como $\gr(z) = 2n$, resulta que
    $z= \sum_{l = 0}^{n}\alpha_l(yx)^{n - l}y^{2l} + \sum_{l = 0}^{n - 1}\beta_l x(yx)^{n - l - 1}y^{2l + 1}$ y por lo tanto
    \begin{align*}
        0 &= [y, z] = \sum_{l = 0}^{n}\alpha_l\left[y, (yx)^{n - l}y^{2l}\right]
            + \sum_{l = 0}^{n - 1}\beta_l\left[y, x(yx)^{n - l - 1}y^{2l + 1}\right] \\
        &= \sum_{l = 0}^{n - 1}\alpha_l \left(x(yx)^{n - l - 1}y^{2l +2} + (n - l)x (yx)^{n - l}y^{2l} - (yx)^{n - l}y^{2l  +1}\right)\\
        &\qquad + \sum_{l = 0}^{n - 1}\beta_l \left((yx)^{n - l}y^{2l + 1} - x(yx)^{n - l - 1}y^{2l + 2}\right)\\
        & = \sum_{l = 0}^{n - 1}\left(\beta_l - \alpha_l\right)(yx)^{n - l}y^{2l + 1} 
            + \sum_{l = -1}^{n - 2}\alpha_{l + 1} (n - l - 1) x(yx)^{n - l - 1}y^{2l + 2}\\
        &\qquad + \sum_{l = 0}^{n -1}\left(\alpha_l - \beta_l \right)x(yx)^{n - l - 1}y^{2l +2}.
    \end{align*}
    Debido a que en las dos últimas sumas no aparecen términos de la forma $(yx)^{n - l}y^{2l + 1}$, resulta que $\alpha_l = \beta_l$
    para todo $l$, $0 \leq l \leq n - 1$ y
    \[
        \sum_{l = -1}^{n - 2}\alpha_{l + 1} (n - l - 1) x(yx)^{n - l - 1}y^{2l + 2} = 0.
    \]
    Por lo tanto $\alpha_l = 0$ para todo $l$, $0 \leq l \leq n- 1$ y $z = \alpha_n y^{2n}$. Por otro lado $z$ debe cumplir
    la ecuación $[x, z] = 0$, es decir,
    \[
        0 = [x, z] = \alpha_n \left( xy^{2n} - \sum_{l = 0}^{n}\frac{n!}{l!}x(yx)^{n - l}y^{2l}\right).    
    \]
    Luego $\alpha_n = 0$ o $n = 0$ y entonces $z$ debe pertenecer a $\field$.
    
    Si el grado de $z$ es $2n + 1$, resulta que
    $z = \sum_{l = 0}^{n}\alpha_n (yx)^{n - l}y^{2l + 1} + \sum_{l = 0}^{n}\beta_l x (yx)^{n - l}y^{2l}$.
    Como $z$ pertenece a $\Ker\left(d^{0}\right)$, entonces
    \begin{align*}
        0 &= [y,z] = \sum_{l = 0}^{n - 1}\alpha_l \left(x (yx)^{n - l - 1}y^{2l + 3}
            + (n - l)x(yx)^{n - l}y^{2l + 1} - (yx)^{n - l}y^{2l + 2}\right) \\
        &\qquad + \sum_{l = 0}^{n}\beta_l\left((yx)^{n - l + 1}y^{2l} - x(yx)^{n - l}y^{2l + 1}\right)\\
        & = \sum_{l = 1}^{n}\alpha_{l - 1} x (yx)^{n - l}y^{2l + 1} + \sum_{l = 0}^{n - 1}\alpha_l (n - l)x(yx)^{n - l}y^{2l + 1}
           -\sum_{l = 1}^{n}\alpha_{l - 1}(yx)^{n - l + 1}y^{2l}\\
        &\qquad + \sum_{l = 0}^{n}\beta_l (yx)^{n - l + 1}y^{2l} -\sum_{l = 0}^{n}\beta_l x(yx)^{n - l}y^{2l + 1} \\
        &= \sum_{l = 1}^{n}\left(\beta_l - \alpha_{l - 1}\right)(yx)^{n - l + 1}y^{2l} + \beta_0 (yx)^{n + 1}
            + \left(\alpha_0 n - \beta_0\right)x(yx)^{n}y\\
        &\qquad + \sum_{l = 0}^{n - 1}\left(\alpha_{l - 1} + \alpha_l(n - l) - \beta_l\right)x(yx)^{n - l}y^{2l + 1}
            +\left(\alpha_{n - 1} - \beta_n \right)xy^{2n + 1}.
    \end{align*}
    Como todos los monomios son linealmente independientes obtenemos el siguiente sistema de ecuaciones:
    \begin{align}
        &\beta_0 = 0,\\
        &\beta_l - \alpha_{l - 1} = 0 \text{ para todo } l, 1 \leq l \leq n,\\
        &\alpha_0 n - \beta_0 = 0,\\
        &\alpha_{l - 1} +  \alpha_l(n - l) -\beta_l = 0 \text{ para todo } l, 1 \leq l \leq n - 1.
    \end{align}
    Sumando (1) y (2) obtenemos que $\alpha_l = 0$ para todo $l$, $0 \leq l \leq n - 1$. De (1) y (3) se deduce que $\alpha_0 = 0$
    y de (1) y (2) se deduce que $\beta_l = 0$ para todo $l$, $0\leq l \leq n$. Luego $z$ es igual a $\alpha_n y^{2n + 1}$. Además
    $[x, z]$ debe ser igual a cero, por lo tanto
    \[
        0 = [x, z] = [x, \alpha_n y^{2n + 1}] = \alpha_n \left(xy^{2n + 1} - \sum_{l = 0}^{n}\frac{n!}{l!}(yx)^{n + 1- l}y^{2l}\right).
    \]
    La única posibilidad es que $\alpha_n$ sea cero y por ende $z$ también.
\end{proof}
Para terminar de calcular la segunda página de la sucesión espectral debemos obtener una base de $E^{1, 0}_{2}$.
Sea $[z] = ([z_1], [z_2]) \in \Ker\left(d^{1}_{(1)}\right)$. Como siempre, podemos suponer que cada coordenada de $[z]$
es homogénea y que tienen el mismo grado. Al igual que en el cálculo de una base de $\Ima\left(d^{1}_{(1)}\right)$, podemos
reducir a $[z]$ módulo borde y suponer que se escribe como combinación lineal de elementos del conjunto
\[
    \left\lbrace \left(\left[xy^{2n}\right], 0\right), \left(0, \left[x^{a}(yx)^{b}y^c\right]\right)
        \mid n \geq 0, a \in \{0, 1\}, b, c \geq 0  \right\rbrace.
\] 
Si el grado de $z$ es $2k$, resulta que
\[
    z = \left(0, \sum_{i = 0}^{k}\alpha_i (yx)^{k - i}y^{2i}
+ \sum_{i = 0}^{k - 1}\beta_i x(yx)^{k - i - 1}y^{2i + 1}\right)
\]
y luego
\begin{align*}
    0 &= d^{1}(z) = \sum_{i = 0}^{k - 1}\alpha_i \left(\sum_{l = 0}^{i}\frac{i!}{l!}(yx)^{k + 1 - l}y^{2l} - x(yx)^{k - i}y^{2i + 1}\right)\\
    &\qquad + 2\alpha_k \left(\sum_{i = 0}^{k}\frac{k!}{i!}(yx)^{k + 1 -i}y^{2i} - xy^{2k + 1}\right)\\
    &\qquad+ \sum_{i = 0}^{k - 1}\beta_i\left(\sum_{l = 0}^{i}\frac{i!}{l!}(yx)^{k + 1 - l}y^{2l} - x(yx)^{k - i}y^{2i + 1}\right).
\end{align*}
Debido a que el término $(yx)y^{2k}$ solo aparece en la suma que acompaña al coeficiente $\alpha_k$, resulta que $\alpha_k = 0$ y por
lo tanto
\begin{align*}
    0 &=  \sum_{l =0}^{k - 1} \left(\sum_{i = l}^{k - 1}\frac{i!}{l!}(\alpha_i + \beta_i)\right)(yx)^{k + 1 -l}y^{2l}
        - \sum_{i = 0}^{k - 1}\left(\alpha_i + \beta_i\right)x(yx)^{k - i}y^{2i + 1}.
\end{align*}
De esta igualdad se deduce que $\alpha_i = -\beta_i$ para todo $i$, $0\leq i \leq k - 1$ y como $\alpha_k = 0$, resulta que
\[
    [z] = \sum_{i = 0}^{k - 1}\alpha_i\left(0, \left[(yx)^{k - i}y^{2i}\right] -\left[x (yx)^{k - i - 1}y^{2i + 1}\right]\right) = 0,
\]
donde la última igualdad proviene de que $\left[(yx)^{k - i}y^{2i}\right] - \left[x(yx)^{k - i - 1}y^{2i + 1}\right]$ pertenece a $\Ima\left(d^{0}_{(1)}\right)$ para todo $i$, $0\leq i \leq k - 1$.

Si el grado de $z$ es $2k + 1$,
\[
    z = \sum_{i = 0}^{k}\alpha_i \left(0, (yx)^{k - i}y^{2i + 1}\right) + \sum_{i = 0}^{k}\beta_i \left(0, x(yx)^{k - i}y^{2i}\right)
        +\gamma\left(xy^{2k}, 0\right)
\]
y luego
\begin{align*}
    0 &= d^{1}(z) = \sum_{i = 0}^{k - 1}\sum_{l = 0}^{i}\alpha_i \frac{i!}{l!}kx(yx)^{k + 1 - l}y^{2l}
        + \sum_{i = 0}^{k}\sum_{l = 0}^{i - 1}\beta_i\frac{i!}{l!}x(yx)^{k + 1 - l}y^{2l}\\
        &\qquad + \left(\alpha_k (2k + 1) - \gamma\right)\sum_{i = 0}^{k}\frac{k!}{i!}x(yx)^{k + 1- i}y^{2i}.
\end{align*}
Debido a que el término $x(yx)y^{2k}$ solo aparece en al última suma, resulta que $\alpha_k (2k + 1) = \gamma$ y por lo tanto
\begin{align*}
    0 &= \sum_{i = 0}^{k - 1}\sum_{l = 0}^{i}\alpha_i \frac{i!}{l!}kx(yx)^{k + 1 - l}y^{2l}
        + \sum_{i = 0}^{k}\sum_{l = 0}^{i - 1}\beta_i\frac{i!}{l!}x(yx)^{k + 1 - l}y^{2l}\\
    &=   \sum_{l = 0}^{k - 1}\left(\sum_{i = l}^{k - 1}\alpha_i \frac{i!}{l!}k\right)x(yx)^{k + 1 - l}y^{2l}
        + \sum_{l = 0}^{k - 1}\left(\sum_{i = l + 1}^{k}\beta_i\frac{i!}{l!}\right)x(yx)^{k + 1 - l}y^{2l}\\
\end{align*}
De esta igualdad se deduce que
$\sum_{i = l}^{k - 1}\alpha_i \frac{i!}{l!}k = -\sum_{i = l + 1}^{k}\beta_i\frac{i!}{l!} = -\sum_{i = l}^{k - 1}\beta_{i + 1}\frac{(i + 1)!}{l!}$
para todo $l$, $0\leq l \leq k - 1$. Si llamamos $c_l = \sum_{i = l}^{k - 1}\alpha_i i!k + \beta_{i + 1}(i + 1)! = 0$, resulta que
$0 = c_l - c_{l + 1} = \alpha_l l!k + \beta_{l + 1}(l + 1)!$ y por lo tanto $\beta_{l + 1} = -\alpha_l \frac{k}{l + 1}$ para todo $l$,
$0 \leq l \leq k - 2$. Como  $0 = c_{k - 1} = \alpha_{k - 1}k! + \beta_k k!$, entonces $\beta_{l + 1} = -\alpha_l \frac{k}{l + 1}$ para todo $l$,
$0 \leq l \leq k - 1$ y luego
\begin{align*}
    z &= \sum_{i = 0}^{k - 1}\alpha_i \left(0, (yx)^{k - i}y^{2i + 1}\right) - \sum_{i = 1}^{k} \alpha_{i - 1} \frac{k}{i}\left(0, x(yx)^{k - i}y^{2i}\right)
        + \alpha_k\left((2k + 1)xy^{2k}, y^{2k + 1}\right)\\
    &\qquad + \beta_0 \left(0, x(yx)^{k}\right)\\
    &= \sum_{i = 0}^{k - 1}\alpha_i \left(0, (yx)^{k - i}y^{2i + 1} - \frac{k}{i + 1}x(yx)^{k - 1 - i}y^{2i + 2}\right)
        + \alpha_k\left((2k + 1)xy^{2k}, y^{2k + 1}\right)\\
    &\qquad + \beta_0 \left(0, x(yx)^{k}\right).
\end{align*}
Con este cálculo probamos que
\begin{align*}
     E^{1,0}_{2} &= \Bigg\langle \left(0, \ov{(yx)^{b + 1}y^{2k + 1}} - \frac{b + k + 1}{k + 1}\ov{x(yx)^{b}y^{2k + 2}}\right),
         \left(0, \ov{x(yx)^{k}}\right),\\
      &\left((2k + 1)\ov{xy^{2k}}, \ov{y^{2k + 1}}\right) \mid k, b \geq 0\Bigg\rangle.
\end{align*}
\begin{prop}
    El conjunto $\left\lbrace\left(0, \ov{x}\right), \left((2k + 1)\ov{xy^{2k}}, \ov{y^{2k + 1}}\right) \mid k\geq 0 \right\rbrace$
    es una base de $E^{1, 0}_2$.
\end{prop}
\begin{proof}
Vamos a utilizar la notación de la Proposición \ref{coh_ima_d01}. Es un cálculo directo verificar las siguientes igualdades:
\begin{align*}
    &\left(0, \left[x(yx)^{k + 1}\right]\right) = \frac{1}{k + 1}\xi_{0, k},\\
    &\left(0, \left[(yx)y^{2k + 1}\right] - \left[xy^{2k + 2}\right]\right) =
    		-\theta_{0, k} + \xi_{0, k} + \frac{1}{k + 1}\eta_{k + 1},\\
    &\left(0, \left[(yx)^{b + 1}y^{2k + 1}\right] - \frac{b +k + 1}{k + 1}\left[x(yx)^{b}y^{2k + 2}\right]\right)
        = -\theta_{b, k} + \xi_{b, k} -\frac{1}{k + 1}\xi_{b - 1, k + 1}
\end{align*}
para todo $b \geq 1$ y para todo $k \geq 0$. Por lo tanto, el conjunto del enunciado genera $E^{1, 0}_2$. Sea $w$
una combinación lineal homogénea de estos generadores y supongamos que $w = 0$. Si el grado de $w$ es $1$, entonces
$0 = w = \alpha (0, \ov{x})$ + $\beta(\ov{x}, \ov{y})$. Como no aparecen monomios de grado $1$ en las segundas coordenadas
de los generadores que encontramos de $\Ima\left(d^{0}_{(1)}\right)$, resulta que $\alpha = \beta = 0$. Si el grado
de $w$ es mayor que 1, entonces $w = \left((2k + 1)\ov{xy^{2k}}, \ov{y^{2k + 1}}\right)$ para algún $k \geq 1$. El monomio
$\left[y^{2k + 1}\right]$ no aparece en las segundas coordenadas de los generadores de $\Ima\left(d^{0}_{(1)}\right)$, por lo tanto
$\alpha = 0$.
\end{proof}

Debido a la forma del complejo doble, es claro que $\Hy^{0}(A, A) = E^{0, 0}_{2}$ y $\Hy^{1}(A, A) = E^{1, 0}_{2}$. De manera
similar al ejemplo introductorio de la sección \ref{sucesiones_espectrales}, solo que con las flechas invertidas, se tiene la siguiente
sucesión exacta corta
\begin{align*}
\xymatrix{
    0 \ar[r] & E^{2, p - 1}_{2} \ar[r]^{\iota} & \Hy^{p + 1}(A, A) \ar[r]^{\pi} & E^{1, p}_2 \ar[r] & 0.
}
\end{align*}
donde $\iota(\ov{a}) = \ov{(0, a)}$ para todo $\ov{a} \in E^{2, p - 1}_{2}$ y $\pi(\ov{(a, b)}) = \ov{b}$ para todo
$\ov{(a, b)} \in \Hy^{p + 1}(A, A)$.Para encontrar una
base de $\Hy^{p + 1}(A, A)$ alcanza con completar una base de $\iota\left(E^{2, p - 1}_{2}\right)$ con elementos
tales que sus imágenes, por medio de $\pi$, formen una base de $E^{1, p}_{2}$.A partir
de esta observación es sencillo verificar la siguiente proposición.
\begin{prop}
    Se tienen los siguientes isomorfismos:
    \begin{align*}
        &H^0(A,A) \cong \Bbbk, \\
  		&H^1(A,A) \cong \left\langle \left(0, \ov{x}\right), \left((2n + 1)\ov{xy^{2n}}, \ov{y^{2n + 1}}\right): n \geq 0\right\rangle, \\
  		&HH^{2p}(A) \cong \left\langle \left(0, \ov{xy^{2n}}\right),
  			\left(\sum_{i = 0}^n\frac{n!}{i!}\ov{(yx)^{n - i}y^{2i}}, -\ov{y^{2n + 1}} \right): n \geq 0\right\rangle,\\
  		&HH^{2p + 1}(A) \cong \left\langle \left(0, \sum_{i = 0}^n\frac{n!}{i!}\ov{(yx)^{n - i}y^{2i}}\right),
  			\left(\ov{xy^{2n}}, \ov{xy^{2n + 1}}\right) : n \geq 0\right\rangle 
    \end{align*}
    para todo $p\geq 1$.
\end{prop}

\section{Estructura de álgebra de Gerstenhaber de $\Hy^{\bullet}(A, A)$}
En la Sección \ref{hochschild_gerstenhaber} vimos que la cohomología de Hochschild de un álgebra $A$ con
coeficientes en la misma álgebra tiene estructura de álgebra de Gerstenhaber. El objetivo de esta
sección es calcular la estructura de anillo graduado de la cohomología en el caso en el que $A$ es el 
super plano de Jordan. También calcularemos la estructura de álgebra de Lie de $\Hy^{1}(A, A)$.

\subsection{Morfismos de comparación}
Si bien la resolución $P_{\bullet}A$ resulto muy útil para calcular la homología y la cohomología de Hochschild,
el producto cup y el corchete de Gerstenhaber están definidos a partir de la resolución bar $B_{\bullet}A$.
Es por esto que necesitamos encontrar morfismos de comparación entre las dos resoluciones, es decir, morfismos
de complejos
\[
	f_{\bullet} : P_{\bullet}A \to B_{\bullet}A \text{ y } g_{\bullet} : B_{\bullet}A \to P_{\bullet}A
\]
que extiendan a $id_A$. Como las dos resoluciones son proyectivas, sabemos que estos morfismos existen.

\begin{prop}
	Sea $f_{\bullet} : P_{\bullet}A \to B_{\bullet}A$ la sucesión de morfismos de $A-A$ bimódulos
	$\left\lbrace f_{n} : P_{n}A \to B_{n}A \right\rbrace_{n \geq 0}$ definidos como
	\begin{itemize}
		\item si $n = 0$,
		\begin{align*}
			f_0 : A \ox A \to A \ox A, \quad f_0 = id_{A\ox A},
		\end{align*}
		\item si $n = 1$,
		\begin{align*}
			&f_1 : A \ox \field \left\lbrace x, y \right\rbrace \ox A \to A \ox A \ox A,\\
			&f_1(1 \ox v \ox 1) = 1 \ox v \ox 1 \quad\text{para todo } v \in \field \left\lbrace x, y \right\rbrace,
		\end{align*}
		\item si $n \geq 2 $,
		\begin{align*}
			&f_n : A \ox \field \left\lbrace x^{n}, y^2x^{n - 1} \right\rbrace \ox A \to A^{n} \ox A \ox A,\\
			&f_n(1 \ox x^{n} \ox 1) = 1 \ox x^{\ox n} \ox 1,\\
			&f_n(1 \ox y^2x^{n - 1} \ox 1) = y \ox y \ox x^{\ox n - 1}\ox 1 + 1 \ox y \ox yx \ox x^{ \ox n - 2}\ox 1\\
			&\qquad - x \ox y \ox y \ox x^{\ox n - 2}\ox 1 - 1 \ox x \ox y^{2}\ox x^{\ox n - 2} \ox 1\\
			&\qquad -x \ox y \ox x^{\ox n - 1} \ox 1 - 1 \ox x \ox yx \ox x^{\ox n - 2} \ox 1\\
			&\qquad + \sum_{i = 0}^{n - 3}(-1)^{i}\left(1\ox x^{\ox 2 + i} \ox y^2 \ox x^{\ox n - 3 - i} \ox 1
				+ 1 \ox x^{\ox 2 + i} \ox yx \ox x^{\ox n - 3 - i} \ox 1\right),
		\end{align*}
		donde la última suma es cero si $n = 2$.
	\end{itemize}
	La sucesión $f_{\bullet}$ resulta ser un morfismo de complejo que levanta a la identidad.
\end{prop}
\begin{proof}
	Es claro que $f_{\bullet}$ levanta a la identidad ya que $f_0 = id_{A \ox A}$. Para probar que
	$f$ es un morfismo de complejos debemos verificar que $b_n \circ f_{n + 1} = f_n \circ d_n$ para todo
	$n \geq 0$. Es cálculo directo, aunque tedioso, hacer esta verificación por inducción y omitiremos la demostración. 
\end{proof}
El morfismo $f_{\bullet}$ guarda la información de los distintos términos que aparecen al aplicarle las reglas
de re-escritura a las $n-$ambigüedades. Para ver estos términos alcanza con cambiar los tensores por productos
en la definición de $f_n(1 \ox x^{n} \ox 1)$ y de $f_n(1 \ox y^2x^{n - 1} \ox 1)$.

No vamos a dar una fórmula explicita de $g_{\bullet} : B_{\bullet}A \to P_{\bullet}A$ sobre todo el complejo bar
sino que solo sobre los términos que nos interesan para poder calcular el producto cup y el corchete de Gerstenhaber.
Como $B_{\bullet}A$ es una resolución libre, es posible definir, para cada $n \geq  0$, el morfismo $g_n$
sobre un conjunto linealmente independiente $B_n \subseteq A\ox A^{\ox n}\ox A$ siempre y cuando
$d_n \circ g_n(z) = g_n \circ b_n(z)$ para cada $z \in B_{n + 1}$. Notar que para que sea posible verificar esta
condición $b_n(B_{n + 1})$ debe estar incluido en $ B_{n}$. Al igual que para el morfismo $f_{\bullet}$
no daremos una demostración de la siguiente proposición ya que es una verificación directa por inducción.

\begin{prop}
	Existe un morfismo de complejos $g_{\bullet} : B_{\bullet}A \to P_{\bullet}A$ que cumple que
	\begin{itemize}
		\item $g_0 = id_{A\ox A}$,
		\item para todo $x^{a}(yx)^{b}y^c$ perteneciente a la base de $A$,
		\begin{align*}
			&g_1\left(x^{a}(yx)^{b}y^c\right) = a\ox x \ox (yx)^by^c
				+ \sum_{i = 0}^{b - 1} x^{a}(yx)^{i}\ox y \ox x(yx)^{b - 1 - i}y^c\\
			&\qquad + \sum_{i = 0}^{b - 1}x^{a}(yx)^{i}y \ox x \ox (yx)^{b - 1 - i}y^c
				+ \sum_{i =0}^{c - 1}x^{a}(yx)^{b} y^{i}\ox y \ox y^{c - 1 - i}
		\end{align*}
	\end{itemize}
	y para todo $n \geq 2$,
	\begin{itemize}
		\item $g_n\left(1\ox y \ox x^{\ox n - 1}\ox 1\right) = 0$,
		\item $g_n\left(1\ox y \ox yx \ox x^{ \ox n - 2} \ox 1\right) = 1\ox y^2 x^{n - 1} \ox 1$,
		\item $g_n\left(1\ox y \ox y \ox x^{\ox n - 2} \ox 1\right) = 0$,
		\item $g_n\left(1\ox x^{\ox n} \ox 1\right) = 1\ox x^n \ox 1$,
		\item $g_n\left(1\ox y^2 \ox x^{\ox n - 1}\ox 1\right) = 1 \ox y^{2}x^{n - 1 \ox 1}$,
		\item $g_n\left(1\ox x^{\ox i} \ox y^{2} \ox x^{n - 1 - i} \ox 1\right) =  0$
			para todo $i$, $1 \leq i \leq n  - 1$,
		\item $g_n\left(1\ox yx \ox x^{n - 1} \ox 1\right) = y\ox x^{n}\ox 1$,
		\item $g_n\left(1\ox x^{\ox i} \ox yx \ox x^{\ox n - 1 - i}\ox 1\right) = 0$
			para todo $i$, $1 \leq i \leq n  - 1$,
		\item $g_n\left(1 \ox xy^{2} \ox x^{\ox n - 1}\ox 1\right) = x \ox y^2x^{n - 1} \ox 1
			+ 1 \ox x^{n} \ox y^{2} + 1 \ox x^{n} \ox yx$
		\item $g_n\left(1 \ox x^{\ox i}\ox xy^{2} \ox x^{\ox n - 1 - i}\ox 1\right) =
			1 \ox x^{n} \ox y^{2} + 1 \ox x^{n} \ox yx$ para todo $i$,\\ $1 \leq i \leq n  - 2$,
		\item $g_n\left(1 \ox x^{\ox n - 1}\ox xy^{2} \ox 1\right) = 1 \ox x^{n} \ox y^{2}$,
		\item $g_n\left(1 \ox xyx \ox x^{\ox n - 1}\ox 1\right) = xy \ox x^{n} \ox 1$,
		\item $g_n\left(1 \ox x^{\ox i}\ox xyx \ox x^{\ox n - 1 - i}\ox 1\right) = 0$
			para todo $i$, $1 \leq i \leq n  - 2$,
		\item $g_n\left(1 \ox x^{\ox n - 1}\ox xyx \ox 1\right) = 1 \ox x^{n} \ox yx$.
	\end{itemize}
\end{prop}
El morfismo $g_{\bullet}$ también guarda la información de los distintos términos que aparecen al aplicarle las reglas
de re-escritura a las $n-$ambigüedades.

\subsection{Producto cup}
Como $f$ y $g$ son levantados de la identidad, resulta que los morfismos inducidos en la cohomología $f^{\ast}$
y $g^{\ast}$ son isomorfismos y más aún $\left(f^{\ast}\right)^{-1} = g^{\ast}$. Esta observación nos permite 
transportar la estructura de álgebra de Gerstenhaber de la cohomología calculada a partir de la resolución bar
a la cohomología calculada a partir de $P_{\bullet}A$. Sean $\varphi \in \Hom_{A^e}\left(P_n A, A\right)$
y $\phi \in \Hom_{A^e}\left(P_m A, A\right)$, el producto cup entre las clases de estos dos elementos
resula ser $\ov{\varphi} \smile \ov{\phi} = \ov{\left(\varphi g_n \smile \phi g_m\right) f_{n + m}}$.
Como todos los cálculos serán realizados en la cohomología, dejaremos de escribir la barra sobre los elementos
para indicar que es una clase en el cociente. Además, utilizaremos indistintamente la notación de morfismo
y la notación de vector de dos coordenadas para los elementos de $\Hy^{\bullet}(A, A)$.

Antes de calcular los productos entre los distintos generadores
de $\Hy^{\bullet}(A, A)$ vamos a introducir una notación para estos elementos:
\begin{align*}
	&c = \left(0,x\right) \in \Hy^1(A, A),\\
	&s_n = \left((2n + 1)xy^{2n}, y^{2n + 1}\right)\in \Hy^1(A, A),\\
	&t_n^{2p} = \left(0, xy^{2n}\right) \in 	\Hy^{2p}(A, A),\\
	&u_n^{2p} = \left(\sum_{i = 0}^{n}\frac{n!}{l!}(yx)^{n - i}y^{2i}, -y^{2n + 1}\right)
		\in 	\Hy^{2p}(A, A),\\
	&v_n^{2p + 1} = \left(0, \sum_{i = 0}^{n}\frac{n!}{l!}(yx)^{n - i}y^{2i}\right)
		\in 	\Hy^{2p + 1}(A, A),\\
	&w_n^{2p + 1} = \left(xy^{2n}, xy^{2n + 1}\right)
		\in 	\Hy^{2p + 1}(A, A)
\end{align*}
para todo $n \geq 0$. 

Es sencillo verificar que si $\lambda \in \Hy^{0}(A, A)$ y $\varphi \in \Hy^{n}(A, A)$,
entonces
\[
	\varphi \smile \lambda = \lambda \smile \varphi = \lambda \varphi.
\]
Sean $n, m \geq 1$
y sean $\varphi \in \Hy^{n}(A, A)$ y $\phi \in \Hy^{m}(A, A)$ elementos homogéneos de la cohomología.
Por definición del producto cup, el morfismo $\varphi \smile \phi$ pertenece a $\Hy^{n + m}(A, A)$,
por lo tanto para calcular $\varphi \smile \phi$ debemos saber que valor toma en $1\ox x^{n + m} \ox 1$
y en $1\ox y^{2}x^{n + m -1}\ox 1$. Como el producto cup es super conmutativo podemos
calcular los producto en el orden que nos resulte más conveniente.
\subsubsection{1) $c \smile s_n$:}
\begin{align*}
	c &\smile s_n\left(1\ox x^{2} \ox 1\right) = \left(c g_1 \smile s_n g_1\right)f_2\left(1\ox x^2 \ox 1\right)
		= \left(c g_1 \smile s_n g_1\right)(1 \ox x^{\ox 2}\ox 1)\\
	&= c(1\ox x \ox 1)s_n(1 \ox x \ox 1) = 0 (2n + 1)xy^{2n} = 0,
\end{align*}
\begin{align*}
	c &\smile s_n\left(1\ox y^2x \ox 1\right) = \left(c g_1 \smile s_n g_1\right)f_2\left(1\ox y^2x \ox 1\right)\\
	&= \left(c g_1 \smile s_n g_1\right)\bigg(y \ox y \ox x \ox 1 + 1 \ox y \ox yx \ox 1 
		- x \ox y \ox y \ox 1\\ &\qquad - 1\ox x \ox y^2 \ox 1 - x \ox y \ox x \ox 1 
		- 1\ox x \ox yx \ox 1\bigg) \\
	&= y c(1 \ox y \ox 1)s_n(1 \ox x \ox 1) + c(1\ox y \ox 1)s_n(y \ox x \ox 1 + 1 \ox y \ox x)\\
	&\qquad -xc(1\ox y \ox 1)s_n(1 \ox y \ox 1) -c(1\ox x \ox 1)s_n(y \ox y \ox 1 + 1\ox y \ox y)\\
	&\qquad -xc(1\ox y \ox 1)s_n(1 \ox x \ox 1) -c(1\ox x \ox 1)s_n(y \ox x \ox 1 + 1\ox y \ox x)\\
	&= yx(2n + 1)xy^{2n} + x\left(y(2n + 1)xy^{2n} + y^{2n + 1}x\right) - x^2y^{2n + 1} \\
	&\qquad - 0\left(y^{2n + 2} + y^{2n + 2}\right) - x^2(2n + 1)xy^{2n}
		- 0\left(y(2n + 1)xy^{2n} + y^{2n + 1} x\right)\\
	&= (2n + 1)xyxy^{2n} + xy^{2n + 1}x = (2n + 1)xyxy^{2n} + \sum_{i = 0}^{n}\frac{n!}{i!}x(yx)^{n  + 1 -i}y^{2i}.
\end{align*}
Si bien con este cálculo obtuvimos que
\[
	c \smile s_n = \left(0, (2n + 1)xyxy^{2n} + \sum_{i = 0}^{n}\frac{n!}{i!}x(yx)^{n  + 1 -i}y^{2i}\right),
\]
el producto no está escrito en términos de la base de $\Hy^{2}(A, A)$. Esto dice que es posible reducir
a $c \smile s_n$ módulo borde. Vamos a denotar por $d^{n}$ a los diferenciales del complejo
$\Hom_{A^e}\left(P_{\bullet}A, A\right)$. Sea $\alpha \in \Hom_{A^e}\left(A\ox \field\{x, y\}\ox A, A\right)$
tal que \[\alpha(1\ox x \ox 1) = -xy^{2n} \text{ y } \alpha(1\ox y \ox 1) = 0,\]
luego
\begin{align*}
	d^{1}&(\alpha)\left(1 \ox x^{2} \ox 1\right) = \alpha \left(d_1\left(1\ox x^{2} \ox 1\right)\right)
		= \alpha(x \ox  x \ox 1 + 1 \ox x \ox x) \\
	&= x \left(-xy^{2n}\right) - xy^{2n}x = 0
\end{align*}
y
\begin{align*}
	d^{1}&(\alpha)\left(1 \ox y^2x \ox 1\right) = \alpha \left(d_1\left(1\ox y^2x \ox 1\right)\right)\\
	&=\alpha\bigg(y^2 \ox x \ox 1 + y \ox y \ox x + 1 \ox y \ox yx - xy \ox y \ox 1 - x \ox y \ox y
		- 1 \ox x \ox y^2\\
 	&\qquad- xy \ox x \ox 1 - x \ox y \ox x - 1 \ox x \ox yx\bigg) \\
 	&= -y^2xy^{2n} + xy^{2n + 2} + xyxy^{2n} + x y^{2n + 1}x\\
 	&= -xy^{2n + 2} - xyxy^{2n}+ xy^{2n + 2} + xyxy^{2n} + \sum_{i = 0}^{n}\frac{n!}{i!}x(yx)^{n  + 1 -i}y^{2i}\\
 	&= \sum_{i = 0}^{n}\frac{n!}{i!}x(yx)^{n  + 1 -i}y^{2i}.
\end{align*}
Por otro lado, sea $\beta \in \Hom_{A^e}\left(A\ox \field\{x, y\} \ox A, A\right)$ tal que 
\[
	\beta(1\ox x \ox 1) = 0 \text{ y } \beta(1 \ox y \ox 1) = y^{2n + 1} - (2n + 1)xy^{2n},
\]
luego
\begin{align*}
	d^1(\beta)\left(1\ox x^{2} \ox 1\right) = \beta(x \ox x \ox 1 + 1 \ox x \ox x) = 0	
\end{align*}
y
\begin{align*}
	d^1&(\beta)\left(1\ox y^2x \ox 1\right) = \beta \left(d_1\left(1\ox y^2x \ox 1\right)\right)\\
	&=\beta\bigg(y^2 \ox x \ox 1 + y \ox y \ox x + 1 \ox y \ox yx - xy \ox y \ox 1 - x \ox y \ox y
		- 1 \ox x \ox y^2\\
 	&\qquad- xy \ox x \ox 1 - x \ox y \ox x - 1 \ox x \ox yx\bigg) \\
 	&= 2\left(y^{2n + 2}x - xy^{2n + 2}\right) - xy^{2n + 1}x
 		-(2n + 1)\left(xy^{2n + 1}x -xyxy^{2n}\right)\\
 	&= 2\left(\sum_{i = 0}^{n}\frac{(n + 1)!}{i!}x(yx)^{n + 1 -i}y^{2i}\right)
 		- \sum_{i = 0}^{n}\frac{n!}{i!}x(yx)^{n + 1 -i}y^{2i}\\
 	&\qquad - (2n + 1) \sum_{i = 0}^{n - 1}\frac{n!}{i!}x(yx)^{n + 1 -i}y^{2i}\\
 	&= \sum_{i = 0}^{n}\left(\frac{2(n + 1)!}{i!} - \frac{n!}{i!}\right)x(yx)^{n + 1 -i}y^{2i}
 		- (2n + 1) \sum_{i = 0}^{n - 1}\frac{n!}{i!}x(yx)^{n + 1 -i}y^{2i}\\
 	&= (2n + 1)\sum_{i = 0}^{n}\frac{n!}{i!}x(yx)^{n + 1 -i}y^{2i}
 		- (2n + 1) \sum_{i = 0}^{n - 1}\frac{n!}{i!}x(yx)^{n + 1 -i}y^{2i}\\
 	&= (2n + 1)xyxy^{2n}.
\end{align*}
Por lo tanto $c \smile s_n = d^1(\alpha + \beta) = 0$.
\subsubsection{2) $s_m \smile s_n$:}
\begin{align*}
	s_m &\smile s_n\left(1\ox x^{2} \ox 1\right) = \left(s_m g_1 \smile s_n g_1\right)f_2\left(1\ox x^2 \ox 1\right)\\
	&= \left(s_m g_1 \smile s_n g_1\right)(1 \ox x^{\ox 2}\ox 1) = s_m(1\ox x \ox 1)s_n(1 \ox x \ox 1)\\
	&= (2m + 1)xy^{2m} (2n + 1)xy^{2n} = 0,
\end{align*}
\begin{align*}
	s_m &\smile s_n\left(1\ox y^2x \ox 1\right) = \left(s_m g_1 \smile s_n g_1\right)f_2\left(1\ox y^2x \ox 1\right)\\
	&= \left(s_m g_1 \smile s_n g_1\right)\bigg(y \ox y \ox x \ox 1 + 1 \ox y \ox yx \ox 1 
		- x \ox y \ox y \ox 1\\ &\qquad - 1\ox x \ox y^2 \ox 1 - x \ox y \ox x \ox 1 
		- 1\ox x \ox yx \ox 1\bigg) \\
	&= y s_m(1 \ox y \ox 1)s_n(1 \ox x \ox 1) + s_m(1\ox y \ox 1)s_n(y \ox x \ox 1 + 1 \ox y \ox x)\\
	&\qquad -xs_m(1\ox y \ox 1)s_n(1 \ox y \ox 1) -s_m(1\ox x \ox 1)s_n(y \ox y \ox 1 + 1\ox y \ox y)\\
	&\qquad -xs_m(1\ox y \ox 1)s_n(1 \ox x \ox 1) -s_m(1\ox x \ox 1)s_n(y \ox x \ox 1 + 1\ox y \ox x)\\
	&= (2n + 1)y^{2m + 2}xy^{2n} + (2n + 1)y^{2m + 2}xy^{2n} + y^{2(m + n + 1)}x - xy^{2(m + n + 1)}\\
	&\qquad + 2(2m + 1)xy^{2(m + n + 1)} - (2n + 1)xy^{2m + 1}xy^{2n} - (2m + 1)(2n + 1)xy^{2m + 1}xy^{2n}\\
	&\qquad -(2m + 1)xy^{2(m + n) + 1}x\\
	&=(4n + 2)\sum_{i = 0}^{m + 1}\frac{(m + 1)!}{i!}x(yx)^{m + 1 - i}y^{2(i + n)}
		+\sum_{i = 0}^{m + n + 1}\frac{(m + n + 1)!}{i!}x(yx)^{m + n + 1 - i}y^{2i}\\
	&\qquad -(4m + 3)xy^{2(m + n + 1)}
		- (2m + 2)(2n + 1)\sum_{i = 0}^{m + 1}\frac{(m + 1)!}{i!}x(yx)^{m + 1 - i}y^{2(i + n)}\\
	&\qquad -(2m + 1)\sum_{i = 0}^{m + n}\frac{(m + n)!}{i!}x(yx)^{m + n + 1 - i}y^{2i}.
\end{align*}
Cuando escribimos a $c\smile s_n$ en términos de la base, encontramos un elemento
$\beta$, tal que $d^1(\beta)(1\ox x^2\ox 1) = 0$
y $d^1(\beta)(1\ox y^2x\ox 1) = (2n + 1)x(yx)^{2n}$. Esto permite borrar cualquier termino de la forma
$x(yx)y^{2i}$ de la segunda coordenada de $s_m \smile s_n$. Notar que es esencial que $d^1(\beta)(1\ox x^2\ox 1) = 0$.
Sean $b \geq  1$ y $i \geq 0$ y sea $\gamma_{b, i} \in \Hom_{A^e}\left(A\ox \field\{x, y\}\ox A, A\right)$ tal que
\[
	\gamma_{b, i}(1\ox x \ox 1) = 0 \text{ y }
		\gamma_{b, i}(1 \ox y \ox 1) = \frac{1}{b + i}(yx)^{b}y^{2i + 1} - x(yx)^{b}y^{2i},
\]
resulta que
\[
	d^1(\gamma_{b, i})\left(1\ox x^2 \ox 1\right) = 0
		\text{ y } d^1(\gamma_{b, i})\left(1\ox y^2x \ox 1\right) = x(yx)^{b + 1}y^{2i}.
\]
Por lo tanto podemos borrar cualquiera término de la forma $x(yx)^{b + 1}y^{2n}$ de la segunda coordenada
de $s_m \smile s_n$ y obtenemos que
\begin{align*}
	s_m &\smile s_n(1 \ox y^2x\ox 1) = (4n + 2)y^{2(m + n + 1)} + xy^{2( m + n + 1)} - (4m + 3)xy^{2(m + n + 1)}\\
	&= 4(n - m)xy^{m + n + 1},
\end{align*}
es decir, $s_m \smile s_n = 4(n - m)t_{n + m + 1}^{2}$.
\subsubsection{3) $c \smile c$:}
Como el producto cup es super conmutativo y el grado de $c$ es $1$, resulta que $c \smile c = 0$.

\subsubsection{4) $t_m^{2p} \smile c$:}
\begin{align*}
	t_m^{2p}&\smile c(1 \ox x^{2p + 1} \ox 1) = t_m^{2p}\left(1\ox x^{2p} \ox 1\right)c(1 \ox x \ox 1) = 0,
\end{align*}	
\begin{align*}
	t_m^{2p}&\smile c(1 \ox y^2x^{2p} \ox 1) = \left(t_m^{2p}g_{2p}\smile cg_1\right)
		\Bigg(y\ox y \ox x^{\ox 2p} \ox 1  + 1 \ox y \ox yx \ox x^{ \ox 2p - 1} \\
	&\qquad - x \ox y \ox y \ox x^{\ox 2p - 1} \ox 1 - 1\ox x \ox y^{2} \ox x^{\ox 2p - 1}\ox 1
		- x \ox y \ox x^{\ox 2p}\\
	&\qquad -1 \ox x \ox yx \ox x^{\ox 2p - 1} \ox 1
		+ \sum_{i = 0}^{2p - 2}(-1)^{i}\bigg(1\ox x^{\ox 2 + i} \ox y^2 \ox x^{\ox 2p - 2 - i} \ox 1\\
	&\qquad\qquad + 1 \ox x^{\ox 2 + i} \ox yx \ox x^{\ox 2p - 2 - i} \ox 1\bigg) \Bigg)\\
	&= yt_m^{2p}(0)c(1 \ox x \ox 1) + t_m^{2p}\left(1 \ox y^2x^{2p - 1} \ox 1\right) c(1 \ox x \ox 1)
		- x t_m^{2p}(0)c(1 \ox x \ox 1)\\	
	&\qquad - t_m^{2p}(0)c(1 \ox x \ox 1) -xt_m^{2p}(0)c(1 \ox x \ox 1) -t_m^{2p}(0)c(1 \ox x \ox 1)\\
	&\qquad + \sum_{i = 0}^{2p - 3}(-1)^{i}\bigg(t_m^{2p}(0)c(1 \ox x \ox 1) + t_m^{2p}(0)c(1 \ox x \ox 1)\bigg)\\
	&\qquad + t_m^{2p}\left(1 \ox x^{2p} \ox 1\right)c(y \ox y \ox 1 + 1 \ox y \ox y)\\
	&\qquad + t_m^{2p}\left(1 \ox x^{2p} \ox 1\right)c(y \ox x \ox 1 + 1 \ox y \ox x) = 0,
	\end{align*}	
ya que $c(1\ox x \ox 1) = t_m^{2p}\left(1 \ox x^{2p} \ox 1\right) = 0$ y por lo tanto
$t_m^{2p}\smile c$ es igual a $0$.
\subsubsection{5) $t_m^{2p} \smile s_n$:}
\begin{align*}
	t_m^{2p} \smile s_n\left(1 \ox x^{2 p + 1}\ox 1\right) = t_m^{2p}\left(1 \ox x^{2p} \ox 1\right)
		s_n(1 \ox x \ox 1) = 0,
\end{align*}
\begin{align*}
	t_m^{2p} &\smile s_n\left(1 \ox y^2x^{2 p}\ox 1\right) =
		t_m^{2p}\left(1 \ox y^{2}x^{2p - 1}\ox 1\right)s_n(1 \ox x \ox 1)\\
	&\qquad + t_m^{2p}\left(1 \ox x^{2p} \ox 1\right)s_n(y \ox y \ox 1 + 1 \ox y \ox y)\\
	&\qquad + t_m^{2p}\left(1 \ox x^{2p} \ox 1\right)s_n(y \ox x \ox 1 + 1 \ox y \ox x)\\
	&= xy^{2m}(2n + 1)xy^{2n} = 0.
\end{align*}
Por lo tanto $t_m^{2p} \smile s_n$ es igual a $0$.

\subsubsection{6) $u_m^{2p} \smile c$:}
\begin{align*}
	u_{m}^{2p} \smile c \left(1 \ox x^{2p + 1} \ox 1\right) = u_{m}^{2p}\left(1\ox x^{2p} \ox 1\right)c(1 \ox x \ox 1)
	=0,
\end{align*}
\begin{align*}
	u_{m}^{2p} &\smile c \left(1 \ox y^2x^{2p} \ox 1 \right) =
		u_m^{2p}\left(1 \ox y^{2}x^{2p - 1}\ox 1\right)c(1 \ox x \ox 1)\\
	&\qquad + u_m^{2p}\left(1 \ox x^{2p} \ox 1\right)c(y \ox y \ox 1 + 1 \ox y \ox y)\\
	&\qquad + u_m^{2p}\left(1 \ox x^{2p} \ox 1\right)c(y \ox x \ox 1 + 1 \ox y \ox x)\\
	&= \sum_{i = 0}^{m}\frac{m!}{i!}(yx)^{m - i}y^{2i}\left(yx + xy\right)\\
	&= \sum_{i = 0}^{m}\frac{m!}{i!}(yx)^{m - i}y^{2i + 1}x + \sum_{i = 0}^{m}\frac{m!}{i!}(yx)^{m - i}y^{2i}xy\\ 
	&= \sum_{i = 0}^{m}\frac{m!}{i!}\sum_{l = 0}^{i}\frac{i!}{l!}(yx)^{m + 1 - l}y^{2l} + y^{2m}xy\\
	&= \sum_{i = 0}^{m}\frac{m!}{i!}\sum_{l = 0}^{i}\frac{i!}{l!}(yx)^{m + 1 - l}y^{2l}
		+ \sum_{i = 0}^{m}\frac{m!}{i!}x(yx)^{m - i}y^{2i + 1}\\
	&= \sum_{i = 0}^{m}\frac{m!}{i!}\left(\sum_{l = 0}^{i}\frac{i!}{l!}(yx)^{m + 1 - l}y^{2l}
		+ x(yx)^{m - i}y^{2i + 1}\right).
\end{align*}
Sea $\alpha_{b, i} \in \Hom_{A^e}\left(P_{2p}A, A\right)$ tal que
\[
	\alpha_{b, i}\left(1 \ox x^{2p} \ox 1\right) = 0 \text{ y }
		\alpha_{b, i}\left(1 \ox y^2x^{2p - 1}\ox 1\right) = -(yx)^by^{2i},
\]
luego
\[
	d^{2p}(\alpha)\left(1 \ox x^{2p + 1} \ox 1\right) = 0 \text{ y }
d^{2p}(\alpha)\left(1 \ox y^2x^{2p} \ox 1\right) = \sum_{l = 0}^{i}\frac{i!}{l!}(yx)^{b + i + 1 - l}y^{2l}
		+ x(yx)^{b}y^{2i + 1}
\]
y por lo tanto $u_{m}^{2p} \smile c = \sum_{i =0}^{m}\frac{m!}{i!}d^{2p}\left(\alpha_{m - i, i}\right) = 0$.

Antes de continuar calculando los productos vamos a probar una proposición que será
útil.

\begin{prop}\label{combinatorio_recursivo}
	Sean $n, m\geq 0$, resulta que
	$\sum_{i = 0}^{m}\frac{(n + i)!}{i!} = \frac{(m + n + 1)!}{m!(n + 1)}$.
\end{prop}
\begin{proof}
	Si dividimos por $n!$ ambos lados de la igualdad que aparece enunciado obtenemos
	la siguiente expresión $\sum_{i = 0}^{m}\binom{n + i}{i} = \binom{m + n + 1}{m}$.
	Probaremos por inducción en $m$ que esta nueva igualdad vale para todo $n,m \geq 0$.
	Si $m = 0$, resulta que
	\begin{align*}
		\sum_{i = 0}^{m}\binom{n + i}{i} = \binom{n}{0} = 1 = \binom{n + 1}{0}.
	\end{align*}
	Si $m \geq 1$,
	\begin{align*}
		\sum_{i = 0}^{m}\binom{n + i}{i} &= \sum_{i = 0}^{m - 1}\binom{n + i}{i} + \binom{n + m}{m}
			= \binom{m + n}{m - 1} + \binom{n + m}{m} = \binom{m + n + 1}{m}.
	\end{align*}
\end{proof}

\subsubsection{7) $u_m^{2p} \smile s_n$:}
\begin{align*}
	u_m^{2p}& \smile s_n\left(1 \ox x^{2p + 1} \ox 1\right)
		= u_m^{2p}\left(1 \ox x^{2p} \ox 1\right)s_n(1 \ox x \ox 1)\\
	&= \sum_{i = 0}^{m}(yx)^{m  -i}y^{2i}(2n + 1)xy^{2n}
		= (2n + 1)\sum_{i = 0}^{m}\frac{m!}{i!} x(yx)^{m  -i}y^{2(i + n)},
\end{align*}
\begin{align*}
	u_m^{2p}& \smile s_n\left(1 \ox y^2x^{2p} \ox 1\right)
		= u_m^{2p}\left(1 \ox y^2x^{2p - 1} \ox 1\right)s_n(1 \ox x \ox 1)\\
	&\qquad + u_m^{2p}\left(1 \ox x^{2p} \ox 1\right)s_n(y \ox y \ox 1 + 1 \ox y \ox y)\\
	&\qquad + u_m^{2p}\left(1 \ox x^{2p} \ox 1\right)s_n(y \ox x \ox 1 + 1 \ox x \ox y)\\
	&= -y^{2m + 1}(2n + 1)xy^{2n} + 2\sum_{i = 0}^{m}\frac{m!}{i!}(yx)^{m - i}y^{2i}y^{2n + 2}\\
	&\qquad + \sum_{i = 0}^{m}\frac{m!}{i!}(yx)^{m - i}y^{2i}\left((2n + 1)yxy^{2n} + y^{2n + 1}x\right)\\
	&= 2\sum_{i = 0}^{m}\frac{m!}{i!}(yx)^{m - i}y^{2(i + n + 1)}
		+ (2n + 1)\sum_{i = 0}^{m - 1}\frac{m!}{i!}(yx)^{m - i}y^{2i + 1}xy^{2n}\\
	&\qquad+ \sum_{i = 0}^{m}\frac{m!}{i!}(yx)^{m - i}y^{2(i + n) + 1}x.
\end{align*}
Calcularemos cada una de las sumas por separado:
\begin{align*}
	2\sum_{i = 0}^{m} & \frac{m!}{i!}(yx)^{m - i}y^{2(i + n + 1)} =
		2\sum_{l = n + 1}^{m + n + 1}\frac{m!}{(l - n - 1)!}(yx)^{m + n + 1 - l}y^2l,
\end{align*}
\begin{align*}
	(2n + 1)\sum_{i = 0}^{m - 1} &\frac{m!}{i!}(yx)^{m - i}y^{2i + 1}xy^{2n} =
		\sum_{i = 0}^{m - 1}\sum_{l = 0}^{i}\frac{m!}{l!}(yx)^{m + 1 - l}y^{l + n}\\
	&=(2n + 1)\sum_{l = 0}^{m - 1}\frac{m!}{l!}(m - l)(yx)^{m + 1 - l}y^{2(l + n)}\\
	&= (2n + 1)\sum_{l = n}^{m + n - 1}\frac{m!}{(l - n)!}(m + n - l)(yx)^{m + n + 1 - l}y^{2l},
\end{align*}
\begin{align*}
	\sum_{i = 0}^{m} & \frac{m!}{i!}(yx)^{m - i}y^{2(i + n) + 1}x =
		\sum_{i = 0}^{m}\frac{m!}{i!}(yx)^{m - i}\sum_{l = 0}^{i + n}\frac{(i +n)!}{l!}(yx)^{i + n + 1 - l}y^{2l}\\
	&= \sum_{i = 0}^{m}\sum_{l = 0}^{i + n}\frac{m!}{l!}\frac{(i + n)!}{i!}(yx)^{m + n + 1 - l}y^{2l}
		= \sum_{l = 0}^{n}\frac{m!}{l!}\left(\sum_{i = 0}^{m}\frac{(i + n!)}{i!}\right)(yx)^{m + n + 1 - l}y^{2l}\\
		&\qquad +\sum_{l = n + 1}^{m + n}\frac{m!}{l!}\left(\sum_{i = l - n}^{m}\frac{(i + n!)}{i!}\right)(yx)^{m + n + 1 - l}y^{2l}
\end{align*}
Debido a la Proposición \ref{combinatorio_recursivo}, resulta que
\begin{align*}
	\sum_{i = 0}^{m} & \frac{m!}{i!}(yx)^{m - i}y^{2(i + n) + 1}x
		= \sum_{l = 0}^{n + m}\frac{m!}{l!}\left(\frac{(n + m + 1)!}{m!(n + 1)}\right)(yx)^{m + n + 1 - l}y^{2l}\\
	&\qquad -\sum_{l = n + 1}^{m + n}\frac{m!}{l!}\left(\frac{l!}{(l - n - 1)!(n + 1)}\right)(yx)^{m + n + 1 - l}y^{2l}
\end{align*}
A partir de ahora denotaremos por $\lambda_l$ a $(yx)^{m + n + 1 - l}y^{2l}$. Con los cálculos que hicimos obtuvimos que
\begin{align*}
	u_m^{2p}& \smile s_n\left(1 \ox y^2x^{2p} \ox 1\right) =
		2\sum_{l = n + 1}^{m + n + 1}\frac{m!}{(l - n - 1)!}\lambda_l
		+\sum_{l = 0}^{n + m}\frac{(n + m + 1)!}{l!(n + 1)}\lambda_l\\
	&\qquad+ (2n + 1)\sum_{l = n}^{m + n - 1}\frac{m!}{(l - n)!}(m + n - l)\lambda_l
		-\sum_{l = n + 1}^{m + n}\frac{m!}{(l - n - 1)!(n + 1)}\lambda_l.
\end{align*}
Sea $i$, $0 \leq i < m$ y sea $\alpha_i \in \Hom_{A^e}\left(P_{2p}A, A\right)$ tal que
\[
	\alpha_i\left(1\ox x^{2p} \ox 1\right) = (2n + 1)\frac{m!}{i!}(yx)^{m - i}y^{2(i + n)}
	\text{ y } \alpha_i\left(1\ox y^2x^{2p - 1} \ox 1\right) = 0,
\]
resulta que
\begin{align*}
	d^{2p}(\alpha_i)\left(1 \ox x^{2p + 1}\ox 1\right) = (2n + 1)\frac{m!}{i!}x(yx)^{m - i}y^{2(i + n)}
\end{align*}
y
\begin{align*}
	d^{2p}&(\alpha_i)\left(1 \ox y^2x^{2p}\ox 1\right) =
		(2n + 1)\frac{m!(m  - i)}{i!}(yx)^{m - i + 1}y^{2(i + n)}\\
	&\qquad- (2n + 1)\frac{m!}{i!}\sum_{i = 0}^{i + n}\frac{(i + n)!}{i!}(yx)^{m + n + 1 - l}y^{2l}.
\end{align*}
Reduciendo a $u_m^{2p} \smile s_n$ módulo borde, obtenemos que
\begin{align*}
	u_m^{2p}& \smile s_n\left(1 \ox x^{2p + 1} \ox 1\right) = 
		\left(u_m^{2p} \smile s_n - \sum_{i =0}^{m - 1}d^{2p}(\alpha_i)\right)
		\left(1 \ox x^{2p + 1} \ox 1\right)\\ &= (2n + 1)xy^{2(n + m)}
\end{align*}
y
\begin{align*}
	u_m^{2p}& \smile s_n\left(1 \ox y^2x^{2p} \ox 1\right) =
		\left(u_m^{2p} \smile s_n - \sum_{i =0}^{m - 1}d^{2p}(\alpha_i)\right)
		\left(1 \ox y^2x^{2p} \ox 1\right)\\
	&=2\sum_{l = n + 1}^{m + n + 1}\frac{m!}{(l - n - 1)!}\lambda_l
		+\sum_{l = 0}^{n + m}\frac{(n + m + 1)!}{l!(n + 1)}\lambda_l\\
	&\qquad -\sum_{l = n + 1}^{m + n}\frac{m!}{(l - n - 1)!(n + 1)}\lambda_l
		+(2n + 1)\sum_{i = 0}^{m - 1}\frac{m!}{i!}\sum_{l = 0}^{i + n}\frac{(i + n)!}{l!}\lambda_l\\
	&= 2\sum_{l = n + 1}^{m + n + 1}\frac{m!}{(l - n - 1)!}\lambda_l
		+\sum_{l = 0}^{n + m}\frac{(n + m + 1)!}{l!(n + 1)}\lambda_l
	 	-\sum_{l = n + 1}^{m + n}\frac{m!}{(l - n - 1)!(n + 1)}\lambda_l\\
	&\qquad+(2n + 1)\sum_{i = 0}^{m}\sum_{l = 0}^{i + n}\frac{m!}{i!}\frac{(i + n)!}{l!}\lambda_l
		-(2n + 1)\sum_{l = 0}^{m + n}\frac{(m + n)!}{l!}\lambda_l\\
	&= 2\sum_{l = n + 1}^{m + n + 1}\frac{m!}{(l - n - 1)!}\lambda_l
		+(2n + 2)\sum_{l = 0}^{n + m}\frac{(n + m + 1)!}{l!(n + 1)}\lambda_l\\
	&\qquad -(2n + 2)\sum_{l = n + 1}^{m + n}\frac{m!}{(l - n - 1)!(n + 1)}\lambda_l
		-(2n + 1)\sum_{l = 0}^{m + n}\frac{(m + n)!}{l!}\lambda_l\\
	&= 2\lambda_{n + m + 1} + (2m + 1)\sum_{l = 0}^{n + m}\frac{(n + m)!}{l!}\lambda_l\\
	&= 2\left(\sum_{l = 0}^{n + m + 1}\frac{(n + m + 1)!}{l!} \lambda_l
		- \sum_{l = 0}^{n + m}\frac{(n + m + 1)!}{l!} \lambda_l\right)
		+ (2m + 1)\sum_{l = 0}^{n + m}\frac{(n + m)!}{l!}\lambda_l\\
	&= 	2\sum_{l = 0}^{n + m + 1}\frac{(n + m + 1)!}{l!} \lambda_l
		- (2n + 1)\sum_{l = 0}^{n + m}\frac{(n + m)!}{l!}\lambda_l
\end{align*}
Sea $\beta \in \Hom_{A^e}\left(P_{2p}A, A\right)$ tal que
\[
	\beta\left(1\ox x^{2p} \ox 1\right) = 0 \text{ y } \beta\left(1\ox y^2x^{2p - 1} \ox 1\right) = -y^{2(n + m)+ 1},
\]
resulta que
\[
	d^{2p}(\beta)(1\ox x^{2p + 1} \ox 1) =0 \text{ y }
		d^{2p}(\beta)(1\ox y^2x^{2p} \ox 1) = \sum_{l = 0}^{n + m}\frac{(n + m)!}{l!}\lambda_l + xy^{2(n + m) + 1}
\]
y por lo tanto
\[
	u_m^{2p} \smile s_n = u_m^{2p} \smile s_n - \sum_{i =0}^{m - 1}d^{2p}(\alpha_i) + d^{2p}((2n + 1)\beta)
		= 2v_{n + m + 1}^{2p + 1} + (2n + 1)w_{n + m}^{2p + 1}.
\]
\subsubsection{8) $v_m^{2p + 1} \smile c$:}
\begin{align*}
	v_m^{2p + 1} \smile c\left(1 \ox x^{2p + 2} \ox 1\right)
		= v_m^{2p + 1}\left(1 \ox x^{2p + 1} \ox 1\right)c(1 \ox x \ox 1) = 0,
\end{align*}
\begin{align*}
	v_m^{2p + 1}& \smile c\left(1 \ox y^2x^{2p + 1} \ox 1\right)
		= v_m^{2p + 1}\left(1 \ox y^2 x^{2p} \ox 1\right)c(1 \ox x \ox 1)\\
		&\qquad- v_m^{2p + 1}\left(1 \ox x^{2p + 1} \ox 1\right)c(y \ox y \ox 1 + 1 \ox y \ox y)\\
		&\qquad -v_m^{2p + 1}\left(1 \ox x^{2p + 1} \ox 1\right)c(y \ox x \ox 1 + 1 \ox y \ox x) = 0
\end{align*}
Por lo tanto $v_m^{2p + 1} \smile c$ es igual a cero.
\subsubsection{9) $v_m^{2p + 1} \smile s_n$:}
\begin{align*}
	v_m^{2p + 1} \smile s_n\left(1 \ox x^{2p + 2} \ox 1\right)
		= v_m^{2p + 1}\left(1 \ox x^{2p + 1} \ox 1\right)s_n(1 \ox x \ox 1) = 0,
\end{align*}	
\begin{align*}
	v_m^{2p + 1}& \smile s_n\left(1 \ox y^2x^{2p + 1} \ox 1\right)
		= v_m^{2p + 1}\left(1 \ox y^2 x^{2p} \ox 1\right)s_n(1 \ox x \ox 1)\\
	&\qquad- v_m^{2p + 1}\left(1 \ox x^{2p + 1} \ox 1\right)s_n(y \ox y \ox 1 + 1 \ox y \ox y)\\
	&\qquad -v_m^{2p + 1}\left(1 \ox x^{2p + 1} \ox 1\right)s_n(y \ox x \ox 1 + 1 \ox y \ox x)\\
	&= \sum_{i = 0}^{m}\frac{m!}{i!}(yx)^{m - i}y^{2i}(2n + 1)xy^{2n}
		=(2n + 1)\sum_{l = 0}^{m}\frac{m!}{l!}x(yx)^{m - l}y^{2(l + n)}.
\end{align*}
Como $x(yx)^{b}y^{2i}$ pertenece a $\Ima\left(\partial\right)$ para todo $b\geq 1$ y para todo
$i \geq 0$, resulta que $v_m^{2p + 1}\smile s_n\left(1 \ox y^2x^{2p + 1} \ox 1\right) = (2n + 1)xy^{2(n + m)}$
y por lo tanto \[v_m^{2p + 1}\smile s_n =(2n + 1)t_{m + n}^{2p + 2}.\]
Es importante notar que este argumento solo puede ser utilizado para términos que aparezcan en la segunda coordenada
del producto que estemos calculando.
\subsubsection{10) $w_m^{2p + 1} \smile c$:}
\begin{align*}
	w_m^{2p + 1} \smile c\left(1 \ox x^{2p + 2} \ox 1\right)
		= w_m^{2p + 1}\left(1 \ox x^{2p + 1} \ox 1\right)c(1 \ox x \ox 1) = 0,
\end{align*}
\begin{align*}
	w_m^{2p + 1}& \smile c\left(1 \ox y^2x^{2p + 1} \ox 1\right)
		= w_m^{2p + 1}\left(1 \ox y^2 x^{2p} \ox 1\right)c(1 \ox x \ox 1)\\
	&\qquad- w_m^{2p + 1}\left(1 \ox x^{2p + 1} \ox 1\right)c(y \ox y \ox 1 + 1 \ox y \ox y)\\
	&\qquad -w_m^{2p + 1}\left(1 \ox x^{2p + 1} \ox 1\right)c(y \ox x \ox 1 + 1 \ox y \ox x)\\
	&= -xy^{2m}(yx+ xy) = \sum_{i = 0}^{m}\frac{m!}{i!}x(yx)^{m + 1 - i}y^{2i}.
\end{align*}
Como $x(yx)^{b}y^{2i}$ pertenece a $\Ima\left(\partial\right)$ para todo $b\geq 1$ y para todo
$i \geq 0$, resulta que $w_m^{2p + 1}\smile s_n\left(1 \ox y^2x^{2p + 1} \ox 1\right) = 0$
y por lo tanto $w_m^{2p + 1} \smile c = 0$
\subsubsection{11) $w_m^{2p + 1} \smile s_n$:}
\begin{align*}
	w_m^{2p + 1} &\smile s_n\left(1 \ox x^{2p + 2} \ox 1\right)
		= w_m^{2p + 1}\left(1 \ox x^{2p + 1} \ox 1\right)s_n(1 \ox x \ox 1)\\
	&= xy^{2m}(2n + 1)xy^{2n} = 0,
\end{align*}	
\begin{align*}
	w_m^{2p + 1}& \smile s_n\left(1 \ox y^2x^{2p + 1} \ox 1\right)
		= w_m^{2p + 1}\left(1 \ox y^2 x^{2p} \ox 1\right)s_n(1 \ox x \ox 1)\\
	&\qquad -w_m^{2p + 1}\left(1 \ox x^{2p + 1} \ox 1\right)s_n(y \ox y \ox 1 + 1 \ox y \ox y)\\
	&\qquad -w_m^{2p + 1}\left(1 \ox x^{2p + 1} \ox 1\right)s_n(y \ox x \ox 1 + 1 \ox y \ox x)\\
	&= xy^{2m + 1}(2n + 1)xy^{2n} -2xy^{2m}y^{2n + 2} - xy^{2m}\left((2n + 1)yxy^{2n} + y^{2n + 1}x\right)\\
	&= -2xy^{2m}y^{2n + 2} -xy^{2(n + m) + 1}x\\
	&= -2xy^{2(m + n + 1)}
		-\sum_{i = 0}^{n +m}\frac{(n + m)!}{i!}x(yx)^{m  + n  + 1 -i}y^{2i}.
\end{align*}
Como $x(yx)^{m  + n  + 1 -i}y^{2i}$ pertenece a $\Ima\left(\partial\right)$ para todo $i$, $0 \leq i \leq n + m$,
resulta que \[w_m^{2p + 1} \smile s_n\left(1 \ox y^2x^{2p + 1} \ox 1\right) = 2xy^{2(n + m + 1)}\]
y por lo tanto $w_m^{2p + 1} \smile s_n = -2t_{n + m + 1}^{2p + 2}$.
\subsubsection{12) $t_m^{2p} \smile t_n^{2q}$:}
\begin{align*}
	t_m^{2p}\smile t_n^{2q}\left(1\ox x^{2p + 2q} \ox 1\right)
		= t_m^{2p}\left(1\ox x^{2p}\ox 1\right)t_n^{2q}\left(1\ox x^{2q}\ox 1\right) = 0,
\end{align*}
\begin{align*}
	t_m^{2p} & \smile t_n^{2q}\left(1\ox y^2x^{2p + 2q - 1} \ox 1\right)
		= t_m^{2p}\left(1\ox y^2x^{2p - 1}\ox 1\right)t_n^{2q}\left(1\ox x^{2q}\ox 1\right)\\
	&\qquad + t_m^{2p}\left(1 \ox x^{2p} \ox 1\right)t_n^{2q}\left(1\ox y^2x^{2q - 1}\ox 1\right)\\
	&\qquad + t_m^{2p}\left(1 \ox x^{2p} \ox 1\right)t_n^{2q}\left(y\ox x^{2q}\ox 1\right) = 0.
\end{align*}
Por lo tanto $t_m^{2p}\smile t_n^{2q}$ es igual a cero.
\subsubsection{13) $u_m^{2p} \smile t_n^{2q}$:}
\begin{align*}
	u_m^{2p}\smile t_n^{2q}\left(1\ox x^{2p + 2q} \ox 1\right)
		= u_m^{2p}\left(1\ox x^{2p}\ox 1\right)t_n^{2q}\left(1\ox x^{2q}\ox 1\right) = 0,
\end{align*}
\begin{align*}
	u_m^{2p} & \smile t_n^{2q}\left(1\ox y^2x^{2p + 2q - 1} \ox 1\right)
		= u_m^{2p}\left(1\ox y^2x^{2p - 1}\ox 1\right)t_n^{2q}\left(1\ox x^{2q}\ox 1\right)\\
	&\qquad + u_m^{2p}\left(1 \ox x^{2p} \ox 1\right)t_n^{2q}\left(1\ox y^2x^{2q - 1}\ox 1\right)\\
	&\qquad + u_m^{2p}\left(1 \ox x^{2p} \ox 1\right)t_n^{2q}\left(y\ox x^{2q}\ox 1\right)\\
	&= \sum_{i = 0}^{m}\frac{m!}{i!}(yx)^{m - l}y^{2i}xy^{2n}
		= \sum_{l = 0}^{m}\frac{m!}{l!}x(yx)^{m - l}y^{2(l + n)}
\end{align*}
Como $x(yx)^{m - l}y^{2(l + n)}$ pertenece a $\Ima\left(\partial\right)$ para todo $l$, $0 \leq l \leq m - 1$,
resulta que \[u_m^{2p}  \smile t_n^{2q} = t_{n + m}^{2p + 2q}.\]

Antes de continuar calculando productos vamos a probar la siguiente proposición que
será útil.
\begin{prop}
	Sean $n, m \geq 0$, resulta que
	\begin{align*}
		\left(\sum_{i = 0}^{m}\frac{m!}{i!}(yx)^{m - i}y^{2i}\right)
			\left(\sum_{j = 0}^{n}\frac{n!}{j!}(yx)^{n - j}y^{2j}\right)
			= \sum_{k = 0}^{m + n}\frac{(m + n)!}{k!}(yx)^{m + n - k}y^{2k}
	\end{align*}
\end{prop}
\begin{proof}
	Sean $n ,m \geq 0$, luego
	\begin{align*}
		&\left(\sum_{i = 0}^{m}\frac{m!}{i!}(yx)^{m - i}y^{2i}\right)
			\left(\sum_{j = 0}^{n}\frac{n!}{j!}(yx)^{n - j}y^{2j}\right)
			= \sum_{i = 0}^{m}\sum_{j = 0}^{n}\frac{m!n!}{i!j!}(yx)^{m - i}y^{2i}(yx)^{n - j}y^{2j}\\
		&=\sum_{i = 0}^{m}\sum_{j = 0}^{n - 1}\frac{m!n!}{i!j!}(yx)^{m - i}y^{2i}(yx)^{n - j}y^{2j}
			+\sum_{i = 0}^{m}\frac{m!}{i!}(yx)^{m - i}y^{2(i + n)}.
	\end{align*}
	Debido a la Proposición \ref{prop_conmutatividad2}, obtenemos que
	\begin{align*}
		&\left(\sum_{i = 0}^{m}\frac{m!}{i!}(yx)^{m - i}y^{2i}\right)
			\left(\sum_{j = 0}^{n}\frac{n!}{j!}(yx)^{n - j}y^{2j}\right)\\
		&=\sum_{i = 0}^{m}\sum_{j = 0}^{n - 1}\frac{m!n!}{i!j!}(yx)^{m - i}
			\sum_{l = 0}^{i}\binom{i}{l}\frac{(n - j + i - l - 1)!}{(n -j - 1)!}(yx)^{n - j + i - l}y^{2(l + j)}\\
		&\qquad + \sum_{i = 0}^{m}\frac{m!}{i!}(yx)^{m - i}y^{2(i + n)}\\
		&=\sum_{i = 0}^{m}\sum_{j = 0}^{n - 1}\sum_{l = 0}^{i}\frac{m!n!}{i!j!}
			\binom{i}{l}\frac{(n - j + i - l - 1)!}{(n -j - 1)!}(yx)^{m + n - (j + l)}y^{2(l + j)}\\
		&\qquad + \sum_{i = 0}^{m}\frac{m!}{i!}(yx)^{m - i}y^{2(i + n)}\\
		&=\sum_{j = 0}^{n - 1}\sum_{l = 0}^{m}
			\left(\sum_{i = l}^{m}\frac{(n - j + i - l - 1)!}{(n -j - 1)!(i - l)!}\right)\frac{m!n!}{l!j!}
			(yx)^{m + n - (j + l)}y^{2(l + j)}\\
		&\qquad + \sum_{i = 0}^{m}\frac{m!}{i!}(yx)^{m - i}y^{2(i + n)}\\
		&=\sum_{j = 0}^{n - 1}\sum_{l = 0}^{m}
			\left(\sum_{r = 0}^{m - l}\binom{n - 1 - j + r}{r}\right)\frac{m!n!}{l!j!}
			(yx)^{m + n - (j + l)}y^{2(l + j)}
		 + \sum_{i = 0}^{m}\frac{m!}{i!}(yx)^{m - i}y^{2(i + n)}\\
	\end{align*}
	Debido a la proposición \ref{combinatorio_recursivo}, resulta que
	$\sum_{r = 0}^{m - l}\binom{n - 1 - j + r}{r} = \binom{m + n - (j + l)}{n - j}$ y por lo tanto
	\begin{align*}
		&\left(\sum_{i = 0}^{m}\frac{m!}{i!}(yx)^{m - i}y^{2i}\right)
			\left(\sum_{j = 0}^{n}\frac{n!}{j!}(yx)^{n - j}y^{2j}\right)\\
		&=\sum_{j = 0}^{n - 1}\sum_{l = 0}^{m}
			\binom{m + n - (j + l)}{n - j}\frac{m!n!}{l!j!}(yx)^{m + n - (j + l)}y^{2(l + j)}
		+ \sum_{i = 0}^{m}\frac{m!}{i!}(yx)^{m - i}y^{2(i + n)}\\
		&= \sum_{j = 0}^{n}\sum_{l = 0}^{m}
			\binom{m}{l}\binom{n}{j}(n + m - (j + l))!(yx)^{m + n - (j + l)}y^{2(l + j)}\\
	\end{align*}
	Como la igualdad que queremos probar es simétrica en $m$ y en $n$, podemos
	suponer, sin perdida de generalidad, que $n \leq m$ y escribir $m = n + t$ con $t \geq 0$.
	Realizamos el cambio de índice $k = j + l$
	y denotamos por $\lambda_k$ a $(yx)^{m + n - k}y^{2k}$ para simplificar la escritura.
	Separamos la suma en tres, según el valor de $k$
	\begin{align*}
		&\sum_{j = 0}^{n}\sum_{l = 0}^{m}
			\binom{m}{l}\binom{n}{j}(n + m - (j + l))!(yx)^{m + n - (j + l)}y^{2(l + j)}\\
		&= \sum_{k = 0}^{n}\sum_{j = 0}^{k}
			\binom{n + t}{k - j}\binom{n}{j}(2n + t -k)!\lambda_k
			 +\sum_{k = n + 1}^{n + t}\sum_{j = 0}^{n}
			\binom{n + t}{k - j}\binom{n}{j}(2n + t -k)!\lambda_k\\
		&\qquad  +\sum_{k = n + t  + 1}^{2n + t}\sum_{j = k - (n + t)}^{n}
			\binom{n + t}{k - j}\binom{n}{j}(2n + t -k)!\lambda_k.
	\end{align*}
	Para probar la proposición lo único que falta verificar es que se cumplen las siguientes igualdades
	\begin{align*}
		\sum_{j = 0}^{k}\binom{n + t}{k - j}\binom{n}{j} &= \binom{2n + t}{k} \text{ para todo }k, 0\leq k \leq n,\\
		\sum_{j = 0}^{n}\binom{n + t}{k - j}\binom{n}{j} &= \binom{2n + t}{k}
			\text{ para todo }k, n + 1 \leq k \leq n + t,\\
		\sum_{j = k - (n + t)}^{n}\binom{n + t}{k - j}\binom{n}{j} &= \binom{2n + t}{k}
			\text{ para todo }k, n + t + 1 \leq k \leq 2n + t.\\
	\end{align*}
	Una manera de probar estas igualdades es mediante un argumento combinatorio.
	El coeficiente binomial $\binom{2n + t}{k}$ cuenta de cuantas maneras es posible elegir
	$k$ elementos de un conjunto de tamaño $2n + t$. Una manera de calcular esto es dividir
	el conjunto en dos subconjuntos, uno de tamaño $n$ y el otro de tamaño $n + t$, y elegir
	$j$ elementos del primer conjunto y $k - j$ del segundo.
\end{proof}
\subsubsection{14) $u_m^{2p} \smile u_n^{2q}$:}
\begin{align*}
	u_m^{2p} &\smile u_n^{2q}(1 \ox x^{2p + 2q} \ox 1) = \left(\sum_{i = 0}^{m}\frac{m!}{i!}(yx)^{m - i}y^{2i}\right)
			\left(\sum_{j = 0}^{n}\frac{n!}{j!}(yx)^{n - j}y^{2j}\right)\\
	&= \sum_{k = 0}^{m + n}\frac{(m + n)!}{k!}(yx)^{m + n - k}y^{2k},
\end{align*}
\begin{align*}
	u_m^{2p} &\smile u_n^{2q}(1 \ox y^2x^{2p + 2q - 1} \ox 1) = 
		u_m^{2p}\left(1\ox y^2x^{2p - 1}\ox 1\right)u_n^{2q}\left(1\ox x^{2q}\ox 1\right)\\
	&\qquad + u_m^{2p}\left(1 \ox x^{2p} \ox 1\right)u_n^{2q}\left(1\ox y^2x^{2q - 1}\ox 1\right)\\
	&\qquad + u_m^{2p}\left(1 \ox x^{2p} \ox 1\right)u_n^{2q}\left(y\ox x^{2q}\ox 1\right)\\
	&= -\sum_{l = 0}^{n}\frac{n!}{l!}y^{2m + 1}(yx)^{n - l}y^{2l}
		-\sum_{l = 0}^{m}\frac{m!}{l!}(yx)^{n - l}y^{2l}y^{2n + 1}\\
	&\qquad + \sum_{l = 0}^{m}\frac{m!}{l!}(yx)^{m - l}y^{2l + 1}\sum_{i = 0}^{n}\frac{n!}{i!}(yx)^{n - i}y^{2i}\\
	&= -\sum_{l = 0}^{n}\frac{n!}{l!}y^{2m + 1}(yx)^{n - l}y^{2l}
		-\sum_{l = 0}^{m}\frac{m!}{l!}(yx)^{n - l}y^{2(n + l) + 1}\\
	&\qquad +\sum_{l = 0}^{m- 1}\frac{m!}{l!}(yx)^{n - l}y^{2(n + l) + 1}
		+ \sum_{l = 0}^{m - 1}\sum_{l = 0}^{n -1}\frac{m!n!}{l!i!}(yx)^{m - l}y^{2l + 1}(yx)^{n - i}y^{2i}\\
	&\qquad + \sum_{l = 0}^{n}\frac{n!}{l!}y^{2m + 1}(yx)^{n - l}y^{2l} = -y^{2(m + n) + 1}.
\end{align*}
Por lo tanto $u_m^{2p} \smile u_n^{2q}$ es igual a $u_{m + n}^{2p + 2q}$.
\subsubsection{15) $t_m^{2p} \smile v_n^{2q + 1}$:}
\begin{align*}
	t_m^{2p}\smile v_n^{2q + 1}\left(1\ox x^{2p + 2q + 1} \ox 1\right)
		= t_m^{2p}\left(1\ox x^{2p}\ox 1\right)v_n^{2q + 1}\left(1\ox x^{2q + 1}\ox 1\right) = 0,
\end{align*}
\begin{align*}
	t_m^{2p} & \smile v_n^{2q + 1}\left(1\ox y^2x^{2p + 2q} \ox 1\right)
		= t_m^{2p}\left(1\ox y^2x^{2p - 1}\ox 1\right)v_n^{2q + 1}\left(1\ox x^{2q + 1}\ox 1\right)\\
	&\qquad + t_m^{2p}\left(1 \ox x^{2p} \ox 1\right)v_n^{2q + 1}\left(1\ox y^2x^{2q}\ox 1\right)\\
	&\qquad + t_m^{2p}\left(1 \ox x^{2p} \ox 1\right)v_n^{2q + 1}\left(y\ox x^{2q +1}\ox 1\right) = 0.
\end{align*}
Por lo tanto $t_m^{2p}\smile v_n^{2q + 1}$ es igual a cero.
\subsubsection{16) $t_m^{2p} \smile w_n^{2q + 1}$:}
\begin{align*}
	t_m^{2p}\smile w_n^{2q + 1}\left(1\ox x^{2p + 2q + 1} \ox 1\right)
		= t_m^{2p}\left(1\ox x^{2p}\ox 1\right)w_n^{2q + 1}\left(1\ox x^{2q + 1}\ox 1\right) = 0,
\end{align*}
\begin{align*}
	t_m^{2p} & \smile w_n^{2q + 1}\left(1\ox y^2x^{2p + 2q} \ox 1\right)
		= t_m^{2p}\left(1\ox y^2x^{2p - 1}\ox 1\right)w_n^{2q + 1}\left(1\ox x^{2q + 1}\ox 1\right)\\
	&\qquad + t_m^{2p}\left(1 \ox x^{2p} \ox 1\right)w_n^{2q + 1}\left(1\ox y^2x^{2q}\ox 1\right)\\
	&\qquad + t_m^{2p}\left(1 \ox x^{2p} \ox 1\right)w_n^{2q + 1}\left(y\ox x^{2q +1}\ox 1\right)
	= xy^{2m}xy^{2m} = 0
\end{align*}
Por lo tanto $t_m^{2p}\smile w_n^{2q + 1}$ es igual a cero.
\subsubsection{17) $v_m^{2p + 1} \smile u_n^{2q}$:}
\begin{align*}
	v_m^{2p + 1} \smile u_n^{2q}(1 \ox x^{2q + 2p + 1} \ox 1) = 0,
\end{align*}
\begin{align*}
	v_m^{2p + 1} &\smile u_n^{2q}(1 \ox y^2x^{2q + 2p} \ox 1)
		= v_m^{2p + 1}\left(1\ox y^2x^{2p}\ox 1\right)u_n^{2q}\left(1\ox x^{2q}\ox 1\right)\\
	&\qquad - v_m^{2p + 1}\left(1 \ox x^{2p + 1}  \ox 1\right)u_n^{2q}\left(1\ox y^2x^{2q - 1}\ox 1\right)\\
	&\qquad - v_m^{2p + 1}\left(1 \ox x^{2p + 1} \ox 1\right)u_n^{2q}\left(y\ox x^{2q}\ox 1\right)\\
	&= \left(\sum_{i = 0}^{m}\frac{m!}{i!}(yx)^{m - i}y^{2i}\right)
		\left(\sum_{j = 0}^{n}\frac{n!}{j!}(yx)^{m - j}y^{2j}\right)\\
	&= \sum_{i = 0}^{m + n}\frac{(m + n)!}{i!}(yx)^{m + n - i}y^{2i}.
\end{align*}
Por lo tanto $v_m^{2p + 1} \smile u_n^{2q}$ es igual a $v_{m + n}^{2p + 2q + 1}$
\subsubsection{18) $u_m^{2p} \smile w_n^{2q + 1}$:}
\begin{align*}
	u_m^{2p} & \smile w_n^{2q + 1}\left(1\ox x^{2p + 2q + 1} \ox 1\right)
		= u_m^{2p}\left(1\ox x^{2p}\ox 1\right)w_n^{2q + 1}\left(1\ox x^{2q + 1}\ox 1\right)\\
	&= \sum_{i = 0}^{m}\frac{m!}{i!}(yx)^{m - i}y^{2i}xy^{2n} = y^{2m}xy^{2n}
		= \sum_{l = 0}^{m}\frac{m!}{l!}x(yx)^{m - l}y^{2(l + n)},
\end{align*}
\begin{align*}
	u_m^{2p} & \smile w_n^{2q + 1}\left(1\ox y^2x^{2p + 2q} \ox 1\right)
		= u_m^{2p}\left(1\ox y^2x^{2p - 1}\ox 1\right)w_n^{2q + 1}\left(1\ox x^{2q + 1}\ox 1\right)\\
	&\qquad + u_m^{2p}\left(1 \ox x^{2p} \ox 1\right)w_n^{2q + 1}\left(1\ox y^2x^{2q}\ox 1\right)\\
	&\qquad + u_m^{2p}\left(1 \ox x^{2p} \ox 1\right)w_n^{2q + 1}\left(y\ox x^{2q +1}\ox 1\right)\\
	&= -y^{2m + 1}xy^{2n} + \sum_{i = 0}^{m}\frac{m!}{i!}(yx)^{m - i}y^{2i}xy^{2n + 1}
		+ \sum_{i = 0}^{m}\frac{m!}{i!}(yx)^{m - i}y^{2i + 1}xy^{2n}\\
	&= y^{2m}xy^{2n + 1}
		+ \sum_{i = 0}^{m - 1}\sum_{l =0}^{i}\frac{m!}{l!}(yx)^{m + 1 - l}y^{2(l + n)}\\
	&= \sum_{l = 0}^{m}\frac{m!}{l!}x(yx)^{m - l}y^{2(l + n) + 1}
		+ \sum_{i = 0}^{m - 1}\frac{m!(m - l)}{l!}(yx)^{m + 1 - l}y^{2(l + n)}\\
\end{align*}
Como $x(yx)^{m - l}y^{2(l + n) + 1}
	+ \sum_{i = 0}^{l + n}\frac{(l + n)!}{i!}(yx)^{m + n + 1 -i}y^{2i}$ pertenece a $\Ima(\delta)$
	para todo \\ $l$, $0 \leq l \leq m - 1$, resulta que
\begin{align*}
	u_m^{2p} & \smile w_n^{2q + 1}\left(1\ox y^2x^{2p + 2q} \ox 1\right) = xy^{2(m + n) + 1}\\
		&\qquad - \sum_{l = 0}^{m - 1}\sum_{i = 0}^{l + n}\frac{m!(l + n)!}{l!i!}(yx)^{m + n + 1 -i}y^{2i}
		+\sum_{l = 0}^{m - 1}\frac{m!(m - l)}{l!}(yx)^{m + 1 - l}y^{2(l + n)}.
\end{align*}
Sea $l$, $0\leq l \leq m - 1$ y $\alpha_l \in \Hom_{A^{e}}\left(P_{2p + 2q}A, A\right)$, tal que
\[
	\alpha_l\left(1 \ox x^{2p + 2q} \ox 1\right) = (yx)^{m -l}y^{2(l + n)} \text{ y }
		\alpha_l\left(1 \ox y^2x^{2p + 2q - 1} \ox 1\right) = 0,
\]
resulta que
\begin{align*}
	&d^{2p + 2q}(\alpha_l)\left(1 \ox x^{2p + 2q + 1} \ox 1\right) = x(yx)^{m - l}y^{2(l + n)},\\
	&d^{2p + 2q}(\alpha_l)\left(1 \ox y^2x^{2p + 2q - 1} \ox 1\right)
		=(m - l)(yx)^{m  + 1 - l}y^{2(l + n)}\\
	&\qquad- \sum_{i = 0}^{l + n}\frac{(l + n)!}{i!}(yx)^{m + n  +1 -l}y^{2l}
\end{align*}
y por lo tanto
\[
	u_m^{2p}  \smile w_n^{2q + 1} = u_m^{2p}  \smile w_n^{2q + 1}
		- \sum_{l = 0}^{m - 1}d^{2p + 2q}(\alpha_l) = w_{n + m}^{2p + 2q + 1}.
\]

\subsubsection{19) $v_m^{2p} \smile v_n^{2q}$:}
\begin{align*}
	v_m^{2p + 1}\smile v_n^{2q + 1}\left(1\ox x^{2p + 2q + 2} \ox 1\right)
		= v_m^{2p + 1}\left(1\ox x^{2p + 1}\ox 1\right)v_n^{2q + 1}\left(1\ox x^{2q + 1}\ox 1\right)
		= 0,
\end{align*}
\begin{align*}
	v_m^{2p + 1} &\smile v_n^{2q + 1}\left(1\ox y^2x^{2p + 2q  + 1} \ox 1\right)
		= v_m^{2p + 1}\left(1\ox y^2x^{2p}\ox 1\right)v_n^{2q} \left(1\ox x^{2q + 1}\ox 1\right)\\
	&\qquad - v_m^{2p + 1}\left(1 \ox x^{2p + 1} \ox 1\right)
		v_n^{2q + 1}\left(1\ox y^2x^{2q}\ox 1\right)\\
	&\qquad - v_m^{2p + 1}\left(1 \ox x^{2p + 1} \ox 1\right)
		v_n^{2q + 1}\left(y\ox x^{2q + 1}\ox 1\right) = 0.
\end{align*}
Por lo tanto $v_m^{2p + 1}\smile v_n^{2q + 1}$ es igual a cero.
\subsubsection{20) $v_m^{2p} \smile w_n^{2q}$:}
\begin{align*}
	v_m^{2p + 1} &\smile w_n^{2q + 1}\left(1\ox x^{2p + 2q + 2} \ox 1\right)
		= v_m^{2p + 1}\left(1\ox x^{2p + 1}\ox 1\right)w_n^{2q + 1}\left(1\ox x^{2q + 1}\ox 1\right)\\
		&= 0,
\end{align*}
\begin{align*}
	v_m^{2p + 1} &\smile w_n^{2q + 1}\left(1\ox y^2x^{2p + 2q  + 1} \ox 1\right)
		= v_m^{2p + 1}\left(1\ox y^2x^{2p}\ox 1\right)w_n^{2q} \left(1\ox x^{2q + 1}\ox 1\right)\\
	&\qquad - v_m^{2p + 1}\left(1 \ox x^{2p + 1} \ox 1\right)
		w_n^{2q + 1}\left(1\ox y^2x^{2q}\ox 1\right)\\
	&\qquad - v_m^{2p + 1}\left(1 \ox x^{2p + 1} \ox 1\right)
		w_n^{2q + 1}\left(y\ox x^{2q + 1}\ox 1\right)\\
	&= \sum_{i = 0}^{m}\frac{m!}{i!}(yx)^{m - i}y^{2i}xy^{2n}
		=  \sum_{l = 0}^{m}\frac{m!}{l!}x(yx)^{m - l}y^{2(l + n)}
\end{align*}
Por lo tanto $v_m^{2p + 1}\smile w_n^{2q + 1}$ es igual a cero $t_{n + m}^{2p + 2q + 2}$.
\subsubsection{21) $w_m^{2p} \smile w_n^{2q}$:}
\begin{align*}
	w_m^{2p + 1}&\smile w_n^{2q + 1}\left(1\ox x^{2p + 2q + 2} \ox 1\right)
		= w_m^{2p + 1}\left(1\ox x^{2p + 1}\ox 1\right)w_n^{2q + 1}\left(1\ox x^{2q + 1}\ox 1\right)\\
	&= xy^{2m}xy^{2n} = 0,
\end{align*}
\begin{align*}
	w_m^{2p + 1} &\smile w_n^{2q + 1}\left(1\ox y^2x^{2p + 2q  + 1} \ox 1\right)
		= w_m^{2p + 1}\left(1\ox y^2x^{2p}\ox 1\right)w_n^{2q} \left(1\ox x^{2q + 1}\ox 1\right)\\
	&\qquad - w_m^{2p + 1}\left(1 \ox x^{2p + 1} \ox 1\right)
		w_n^{2q + 1}\left(1\ox y^2x^{2q}\ox 1\right)\\
	&\qquad - w_m^{2p + 1}\left(1 \ox x^{2p + 1} \ox 1\right)
		w_n^{2q + 1}\left(y\ox x^{2q + 1}\ox 1\right)\\
	&= xy^{2m +1}xy^{2n} -xy^{2m}xy^{2n + 1} - xy^{2m + 2}xy^{2n}\\
	&= \sum_{i = 0}^{m }\frac{m!}{i!}x(yx)^{m  + 1 - i}y^{2(i + n)}
\end{align*}
Como $x(yx)^{m + 1 - i}y^{2(i + n)}$ pertenece a $\Ima\left(\partial\right)$ para todo
$i$, $0 \leq i \leq m$,
resulta que \[w_m^{2p + 1} \smile w_n^{2q + 1} = 0.\]

La siguiente tabla incluye todos los productos entre los elementos de la base de $\Hy^{\bullet}(A, A)$.
\begin{center}
\begin{tabular}{ |c|c|c|c|c|c|c| } 
\hline
& $c$ & $s_n$ & $t_n^{2q}$ & $u_n^{2q}$ & $v_n^{2q + 1}$ & $w_n^{2q + 1}$ \\
\hline
$c$ & $0$ & $0$ & $ 0$ & $ 0$ & $0$ & $0$ \\
\hline
$s_m$ & $0$ & $4(n - m)t^{2}_{n + m + 1}$ & $0$ &
	$2v^{2q +1}_{n + m + 1} + (2n + 1)w_{n + m}^{2q + 1}$
		& $-(2n + 1)t_{n + m}^{2p + 2}$ & $2t_{n + m + 1}^{2p + 2}$ \\
\hline
$t_m^{2p}$ & $0$ & $0$ & $0$ & $t_{m + n}^{2p + 2q}$ & $0$ & $0$ \\
\hline
$u_m^{2p}$ & $0$ &$2v^{2p +1}_{n + m + 1} + (2n + 1)w_{n + m}^{2p + 1}$ & $t_{n + m}^{2p + 2q}$ &
	$u_{m + n}^{2p + 2q}$
		& $v_{n + m}^{2p + 2q + 1}$ & $w_{n + m}^{2p + 2q +1}$ \\
\hline
$v_m^{2p + 1}$ & $0$ &$(2n + 1)t_{n + m}^{2p + 2}$ & $0$ &
	$v_{m + n}^{2p + 2q + 1}$
		& $0$ & $t_{m + n}^{2p + 2q + 2}$ \\
\hline
$w_m^{2p + 1}$ & $0$ &$-2t_{n + m + 1}^{2p + 2}$ & $0$ &
	$w_{m + n}^{2p + 2q + 1}$
		& $-t_{m +n}^{2p + 2q + 2}$ & $0$\\
\hline
\end{tabular}
\end{center}
\subsection{Estructura de álgebra de Lie de $\Hy^{1}(A, A)$}
Si restringimos el corchete de Gerstenhaber a $\Hy^{1}(A, A)$ obtenemos un álgebra de Lie.
En esta subsección vamos calcular esta estructura que resultara ser isomorfa a una
subálgebra de Lie del álgebra de \emph{Virasoro}. Al igual que para el producto solo vamos a calcular
el corchete entre los elementos de la base de $\Hy^{1}(A, A)$. Vamos a empezar con el corchete
entre $c$ y $s_n$ y para eso debemos calcular los respectivos asociadores:
\begin{align*}
	c \circ & s_n(1 \ox x \ox 1) =  \left( cg_1 \circ s_n g_ 1\right)f_1(1 \ox x \ox 1)
		= \left( cg_1 \circ s_n g_ 1\right)(1 \ox x \ox 1)\\
	&= cg_1\left(1 \ox s_n(1\ox x \ox 1) \ox 1\right) = cg_1\left(1 \ox (2n + 1)xy^{2n} \ox 1\right)\\
	&= (2n+1)c\left(1 \ox x \ox y^{2n} + \sum_{i = 0}^{2n - 1}xy^i \ox y \ox y^{2n -1 -i}\right)\\
	&= (2n + 1)\sum_{i = 0}^{2n - 1}xy^i x y^{2n -1 -i}
		= (2n + 1)\sum_{i = 0}^{n - 1}xy^{2i + 1} x y^{2(n -i -1)}\\
	&=(2n + 1)\sum_{i = 0}^{n - 1}x\sum_{l = 0}^{n}\frac{j!}{l!}(yx)^{j - l + 1}y^{2(n - j + l - 1)}\\
	&=(2n +1)\sum_{j = 0}^{n - 1}\sum_{k = 0}^{j}\frac{j!}{(k - j)!}x(yx)^{k + 1}y^{2(n - 1 -k)}\\
	&=(2n + 1)\sum_{k = 0}^{n -1}
		\left(\sum_{j = k}^{n - 1}\frac{j!}{(j - k)!}\right)x(yx)^{k + 1}y^{2(n - 1 - k)}\\
	&=(2n + 1)\sum_{k = 0}^{n -1}
		\left(\sum_{l = 0}^{n - 1 - k}\frac{(k + l)!}{l!}\right)x(yx)^{k + 1}y^{2(n - 1 - k)}\\
	&=(2n + 1)\sum_{k = 0}^{n -1}\frac{n!}{l!(n - l)} x(yx)^{n - l}y^{2l},
\end{align*}
\begin{align*}
	c \circ & s_n(1 \ox y \ox 1) =  \left( cg_1 \circ s_n g_1\right)f_1(1 \ox y \ox 1)
		= \left( cg_1 \circ s_n g_ 1\right)(1 \ox y \ox 1)\\
	&= cg_1\left(1 \ox s_n(1\ox y\ox 1) \ox 1\right) = cg_1\left(1 \ox y^{2n + 1} \ox 1\right)\\
	&= c\left(\sum_{i = 0}^{2n}y^{i}\ox y \ox y^{2n - i}\right)
		= \sum_{i = 0}^{2n}y^{i} x y^{2n - i}
		= \sum_{i = 0}^{n}y^{2i}xy^{2(n - i)} + \sum_{i = 0}^{n - 1}y^{2i + 1}xy^{2(n - i) - 1}\\
	&= \sum_{i = 0}^{n}\sum_{l = 0}^{i}\frac{i!}{l!}x(yx)^{i - l}y^{2(n - i + l)}
		+ \sum_{i = 0}^{n - 1}\sum_{i =0}^{l}\frac{i!}{l!}(yx)^{i - l + 1}y^{2(n - i + l) -1}\\
	&= \sum_{i = 0}^{n}\sum_{k = 0}^{i}\frac{i!}{(i - k)!}x(yx)^{k}y^{2(n - k)}
		+ \sum_{i = 0}^{n - 1}\sum_{k = 1}^{i + 1}\frac{i!}{(i - k + 1)!}(yx)^{k}y^{2(n - k) +1}\\
	&= \sum_{k = 0}^{n}\left(\sum_{i = k}^{n}\frac{i!}{(i - k)!}\right)x(yx)^{k}y^{2(n - k)}
		+ \sum_{k = 1}^{n}
			\left(\sum_{i = k - 1}^{n - 1}\frac{i!}{(i - k + 1)!}\right)(yx)^{k}y^{2(n - k) +1}\\
	&= \sum_{k = 0}^{n}\left(\sum_{l = 0}^{n - k}\frac{(k + l)!}{l!}\right)x(yx)^{k}y^{2(n - k)}
		+ \sum_{k = 1}^{n}
			\left(\sum_{l = 0}^{n - k}\frac{(k + l - 1)!}{l!}\right)(yx)^{k}y^{2(n - k) +1}\\
	&= \sum_{i = 0}^{n}\frac{(n + 1)!}{i!(n -i + 1)} x(yx)^{k - i}y^{2i}
		+ \sum_{i = 0}^{n - 1}\frac{n!}{i!(n -i)} (yx)^{k}y^{2i +1},
\end{align*}
\begin{align*}
	s_n \circ & c(1 \ox x \ox 1) =  \left( s_ng_1 \circ c g_1\right)f_1(1 \ox x \ox 1)
		= \left( s_ng_1 \circ c g_ 1\right)(1 \ox x \ox 1)\\
	&= s_ng_1\left(1 \ox c(1\ox x \ox 1) \ox 1\right) = s_ng_1\left(1 \ox 0 \ox 1\right) = 0,
\end{align*}
\begin{align*}
	s_n \circ & c(1 \ox y \ox 1) =  \left( s_ng_1 \circ c g_1\right)f_1(1 \ox y \ox 1)
		= \left( s_ng_1 \circ c g_ 1\right)(1 \ox y \ox 1)\\
	&= s_ng_1\left(1 \ox c(1\ox y \ox 1) \ox 1\right) = s_ng_1\left(1 \ox x \ox 1\right) = (2n + 1)xy^{2n},
\end{align*}
luego
\begin{align*}
	&\left[c, s_n\right](1\ox x \ox 1)
		= (2n + 1)\sum_{k = 0}^{n -1}\frac{n!}{l!(n - l)} x(yx)^{n - l}y^{2l},\\
	&\left[c, s_n\right](1\ox y \ox 1)
		= \sum_{l = 0}^{n}\frac{(n + 1)!}{l!(n -l + 1)} x(yx)^{k - l}y^{2l}
		+ \sum_{l = 0}^{n - 1}\frac{n!}{l!(n -l)} (yx)^{k}y^{2l +1}\\
	&\qquad- (2n +  1)xy^{2n}.
\end{align*}
Utilizando la notación de la Proposición \ref{coh_ima_d01}, sea $l$, $0 \leq l \leq n - 1$ y sean
\begin{align*}
	&\theta_{n - l - 1, l} = \left(x(yx)^{n - l}y^{2l}, x(yx)^{n - l - 1}y^{2l + 2}
		 + (n - l)x(yx)^{n - l}y^{2l} -(yx)^{n - l}y^{2l + 1}\right),\\
	&\xi_{n - l - 1, l} = \left(-\sum_{i = 0}^{l - 1}\frac{l!}{i!}x(yx)^{n - i}y^{2i},
		(n - l)x(yx)^{n - l}y^{2l}\right),\\
	&\eta_n = \left(\sum_{l = 0}^{n - 1}\frac{n!}{l!}x(yx)^{n -l}y^{2l}, 0\right)
\end{align*}
pertenecientes a $\Ima\left(d^{0}\right)$, resulta que
\begin{align*}
	\sum_{l = 0}^{n - 1}\frac{n!}{l!(n - l)}\theta_{n - l - 1, l}(1 \ox x \ox 1)
		= \sum_{l = 0}^{n - 1}\frac{n!}{l!(n - l)}x(yx)^{n - l}y^{2l},
\end{align*}
\begin{align*}
	&\sum_{l = 0}^{n - 1}\frac{n!}{l!(n - l)}\theta_{n - l - 1, l}(1 \ox y \ox 1)
		= \sum_{l = 0}^{n - 1}\frac{n!}{l!(n - l)}\bigg(x(yx)^{n - l - 1}y^{2l + 2}\\
	&\qquad+ (n - l)x(yx)^{n - l}y^{2l}-(yx)^{n - l}y^{2l + 1}\bigg)\\
	&= \sum_{l = 1}^{n}\frac{n!}{(l - 1)!(n - l + 1)}x(yx)^{n - l}y^{2l}
		 + \sum_{l = 0}^{n - 1}\frac{n!}{l!}x(yx)^{n - l}y^{2l}\\
	&\qquad -\sum_{l = 0}^{n - 1}\frac{n!}{l!(n - l)}(yx)^{n - l}y^{2l + 1}\\
	&= nxy^{2n} + \sum_{l = 1}^{n -1}\left(\frac{n!}{(l - 1)!(n - l + 1)}
		+ \frac{n!}{l!}\right)x(yx)^{n - l}y^{2l} + n!x(yx)^{n}\\
	&\qquad -\sum_{l = 0}^{n - 1}\frac{n!}{l!(n - l)}(yx)^{n - l}y^{2l + 1}\\
	&= -nxy^{2n} + \sum_{l = 0}^{n}\frac{(n + 1)!}{l!(n - l + 1)} x(yx)^{n - l}y^{2l}
		-\sum_{l = 0}^{n - 1}\frac{n!}{l!(n - l)}(yx)^{n - l}y^{2l + 1},
\end{align*}
\begin{align*}
	&\sum_{l = 0}^{n - 1}\frac{2(n + 1)!}{l!(n - l + 1)(n -l)}\xi_{n - l - 1, l}(1\ox y\ox 1)
		=\sum_{l = 0}^{n - 1}\frac{2(n + 1)!}{l!(n - l + 1)(n -l)}x(yx)^{n - l}y^{2l},
\end{align*}
\begin{align*}
	&\sum_{l = 0}^{n - 1}\frac{2(n + 1)!}{l!(n - l + 1)(n -l)}\xi_{n - l - 1, l}(1\ox x\ox 1)\\
	&=-\sum_{l = 0}^{n - 1}\sum_{i = 0}^{l-1}\frac{2(n + 1)!}{(n -l + 1)(n - l)i!}x(yx)^{n -i}y^{2i}\\
	&=-\sum_{i = 0}^{n - 2}\left(\sum_{l = i + 1}^{n - 1}\frac{1}{(n -l + 1)(n - l)}\right)
		\frac{2(n + 1)!}{i!}x(yx)^{n - i}y^{2i}\\
	&=-\sum_{i = 0}^{n - 2}\left(\sum_{k = 1}^{n - i- 1}\frac{1}{(k + 1)(k)}\right)
		\frac{2(n + 1)!}{i!}x(yx)^{n - i}y^{2i}\\
	&=-\sum_{i = 0}^{n - 2}\frac{(n - i - 1)2(n + 1)!}{(n - i)i!}x(yx)^{n - i}y^{2i},
\end{align*}
donde en la última igualdad usamos que $\sum_{i = 1 }^{n}\frac{1}{(i + 1)i} = \frac{n}{n + 1}$. Es
sencillo verificar que
\[
	\left[c, s_n\right] + \sum_{l = 0}^{n - 1}\frac{n!}{l!(n - l)}\theta_{n - l - 1, l}
	 - \sum_{l = 0}^{n - 1}\frac{2(n + 1)!}{l!(n - l + 1)(n -l)}\xi_{n - l - 1, l}= \eta_n
\]
y por lo tanto $\left[c, s_n\right] = 0$.

Ahora vamos a calcular el corchete entre $s_m$ y $s_n$ para todo $m, n \geq 0$. Empezamos
con los asociadores:
\begin{align*}
	s_m \circ & s_n(1 \ox x \ox 1)
		= (2n + 1)s_m\left(1 \ox x \ox y^{2n}
			+ \sum_{i = 0}^{2n - 1}xy^i \ox y \ox y^{2n -1 -i}\right)\\
	&= (2n + 1)(2m + 1)xy^{2(n + m)} + (2n + 1)\sum_{i = 0}^{2n - 1}xy^{i}y^{2m + 1}y^{2n - 1 -i}\\
	&= (2n + 1)(2m + 1)xy^{2(n + m)} + (2n + 1)2nxy^{2(m + n)}\\
	&= (2n + 1)(2(m +n) + 1)xy^{2(m + n)},
\end{align*}
\begin{align*}
	s_m \circ & s_n(1 \ox y \ox 1)
		=s_m\left(\sum_{i = 0}^{2n}y^{i}\ox y \ox y^{2n - i}\right) = (2n + 1)y^{2(n + m)+ 1},
\end{align*}
luego $\left[s_m, s_n\right] = 2(n - m)s_{m + n}$.

El \emph{álgebra de Virasoro} es el álgebra de Lie con base
	$\left\lbrace L_n, c \mid n \in \ZZ\right\rbrace$ y corchete definido, para todo $n, m \in \ZZ$, como
\[
	\left[L_m, L_n\right] = (n - m)L_{m + n} + \delta_{m, -n}\frac{m^3 - m}{12}c
		\text{ y } \left[L_m, c\right] = 0.	
\]
Para identificar a $\Hy^{1}(A,A)$ con una subálgebra de Lie del álgebra Virasoro va a ser conveniente
hacer un cambio de base. Llamamos $L_m = 2^{m - 1}s_m$ para todo $m \geq 0$. Es claro
que el conjunto $\left\lbrace L_m, c \mid m \geq 0 \right\rbrace$ resulta ser una base $\Hy^{1}(A,A)$
y es sencillo verificar que el corchete cumple las siguientes igualdades
\[
	\left[L_m, L_n\right] = (n - m)L_{m + n} \text{ y } \left[L_m, c\right] = 0.	
\]
A partir de esta observación es claro que $\Hy^{1}(A,A)$ es una subálgebra de Lie del álgebra Virasoro.
\end{document}