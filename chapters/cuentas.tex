\documentclass[fleqn,../tesis.tex]{subfiles}
\begin{document}

\section{Cálculos preliminares}

\begin{prop}
\label{cp_conmutatividad}
Para todo $c \geq 0$,
\[
	y^cx = \left\{\begin{array}{cc}
			\sum_{i = 0}^k\frac{k!}{i!}x(yx)^{k - i}y^{2i} \quad&\text{si } c = 2k,\\
			&\\
			\sum_{i = 0}^k \frac{k!}{i!}(yx)^{k - i + 1}y^{2i} \quad&\text{si } c = 2k + 1.
	\end{array}\right.
\]
\end{prop}
\begin{proof}
Lo probaremos por inducción. Los casos $c = 0, 1, 2$ son claros. 
Si $k \geq 1$ y $c = 2k + 1$,
\begin{align*}
	y^{2k + 1}x = y(y^{2k}x) = y(\sum_{i = 0}^k \frac{k!}{i!}x(yx)^{k - i}y^{2i}) = \sum_{i = 0}^k \frac{k!}{i!}(yx)^{k - i + 1}y^{2i}.
\end{align*}
Si $k \geq 1$ y $c = 2k + 2$,
\begin{align*}
	y^{2k + 2}x &= y^{2k}y^2x = y^{2k}(xy^2 + xyx) \\ 
	&= y^{2k}x(y^2 + yx) = \sum_{i = 0}^k \frac{k!}{i!}x(yx)^{k - i}y^{2i}(y^2 + yx) \\
	&= \sum_{i = 0}^k \frac{k!}{i!}x(yx)^{k - i}y^{2i + 2} + \sum_{i = 0}^k\frac{k!}{i!} x(yx)^{k - i}y^{2i + 1}x \\
	&= \sum_{i = 0}^k \frac{k!}{i!}x(yx)^{k - i}y^{2i + 2}
		+ \sum_{i = 0}^k\frac{k!}{i!} x(yx)^{k - i}\left(\sum_{j = 0}^i\frac{i!}{j!}(yx)^{i - j + 1}y^{2j}\right) \\
	&= \sum_{i = 0}^k \frac{k!}{i!}x(yx)^{k - i}y^{2i + 2}
		+ \sum_{i = 0}^k \sum_{j = 0}^i \frac{k!}{j!} x(yx)^{k - j + 1}y^{2j} \\
	&= \sum_{j = 1}^{k + 1} \frac{k!}{(j-1)!} x(yx)^{k - i + 1}y^{2j}
		+ \sum_{i = 0}^k \sum_{j = 0}^i \frac{k!}{j!} x(yx)^{k - j + 1}y^{2j} \\
	&= \sum_{j = 1}^{k + 1} \frac{k!}{(j-1)!} x(yx)^{k - i + 1}y^{2j}
		+ \sum_{j = 0}^k (k - j + 1)\frac{k!}{j!} x(yx)^{k - j + 1}y^{2j} \\
	&= xy^{2k + 1} + \sum_{j = 1}^{k} \frac{k!}{(j-1)!} x(yx)^{k - i + 1}y^{2j} \\
		&\hspace{70pt}+ (k+1)!x(yx)^{k + 1} + \sum_{j = 1}^k (k - j + 1)\frac{k!}{j!} (yx)^{k - j + 1}y^{2j} \\
	&= xy^{2k + 1} + \sum_{j = 1}^k \left( \frac{k!}{(j-1)!} + (k - j + 1)\frac{k!}{j!} \right) (yx)^{k - j + 1}y^{2j} 
		+ (k+1)!x(yx)^{k + 1}\\
	&= xy^{2k + 1} +\sum_{j = 1}^k \frac{(k + 1)!}{j!} (yx)^{k - j + 1}y^{2j} + (k+1)!x(yx)^{k + 1} \\
	&= \sum_{j = 0}^{k + 1} \frac{(k + 1)!}{j!} (yx)^{k - j + 1}y^{2j}.
\end{align*}
\end{proof}
\section{Homología de Hochschild}
Sea $\mathbf{X}_{\bullet,\bullet}$ el siguiente complejo doble ubicado en el primer cuadrante
\begin{align*}
\xymatrix{
	& & \\
	& A \ox k\{x^4\} \ox A \ar[d]^{\delta} \ar@{<--}[u]_{\partial} & A \ox k\{y^2x^4\} \ox A \ar[l]_d \ar[d]^{\delta'} \ar@{<--}[u]_{\partial'} \\
	& A \ox k\{x^3\} \ox A \ar[d]^{\partial} & A \ox k\{y^2x^3\} \ox A \ar[l]_d \ar[d]^{\partial'} \\
	& A \ox k\{x^2\} \ox A \ar[d]^{\delta} & A \ox k \{y^2x^2\} \ox A \ar[l]_d \ar[d]^{\delta'} \\
	A \ox A & A \ox k\{x, y\} \ox A \ar[l]_{d_0} & A \ox k \{y^2x\} \ox A \ar[l]_{d_1} \\
}
\end{align*}
donde los diferenciales de $A$-bimódulos se definen de la siguiente manera:
\begin{align*}
	d_0(1 \ox v \ox 1) &= v \ox 1 - 1 \ox v, \quad\text{ para todo } v \in k\{x, y\}, \\
	\\
	d_1(1 \ox y^2x \ox 1) &= y^2 \ox x \ox 1 + y \ox y \ox x + 1 \ox y \ox yx \\
		&-(xy \ox y \ox 1 + x \ox y \ox y + 1 \ox x \ox y^2) \\
		&-(xy \ox x \ox 1 + x \ox y \ox x + 1 \ox x \ox yx), \\
	\\	
	d(1 \ox y^2x^n \ox 1) &= y^2 \ox x^n \ox 1 - xy \ox x^n \ox 1 - 1 \ox x^n \ox y^2 - 1 \ox x^n \ox yx,\\
	\\	
	\delta(1 \ox x^n \ox 1) &= x \ox x^{n -1} \ox 1 + 1 \ox x^{n - 1} \ox x,\\
	\delta'(1 \ox y^2x^n \ox 1) &= - (x \ox y^2 x^{n -1} \ox 1 + 1 \ox y^2x^{n - 1} \ox x),\\
	\\	
	\partial(1 \ox x^n \ox 1) &= x \ox x^{n -1} \ox 1 - 1 \ox x^{n - 1} \ox x,\\
	\partial'(1 \ox y^2x^n \ox 1) &= -(x \ox y^2 x^{n -1} \ox 1 - 1 \ox y^2x^{n - 1} \ox x). \\
\end{align*}

Aplicamos el funtor $A \ox_{A^e}(-)$ a $\mathbf{X}_{\bullet,\bullet}$ y obtenemos un nuevo
complejo doble tal que la homología de su complejo total es $HH_*(A)$. El complejo es el siguiente
\begin{align*}
\xymatrix{
	& & \\
	& A \ox_{A^e} (A \ox k\{x^4\} \ox A) \ar[d]^{id\ox\delta} \ar@{<--}[u]_{id\ox\partial}
		& A \ox_{A^e} (A \ox k\{y^2x^4\} \ox A) \ar[l]_{id\ox d} \ar[d]^{id\ox\delta'} \ar@{<--}[u]_{id\ox\partial'} \\
	& A \ox_{A^e} (A \ox k\{x^3\} \ox A) \ar[d]^{id\ox\partial}
		& A \ox_{A^e} (A \ox k\{y^2x^3\} \ox A) \ar[l]_{id\ox d} \ar[d]^{id\ox\partial'} \\
	& A \ox_{A^e} (A \ox k\{x^2\} \ox A) \ar[d]^{id\ox\delta}
		& A \ox_{A^e} (A \ox k \{y^2x^2\} \ox A) \ar[l]_{id\ox d} \ar[d]^{id\ox\delta'} \\
	A \ox_{A^e} (A \ox A) & A \ox_{A^e} (A \ox k\{x, y\} \ox A) \ar[l]_{id\ox d_0} & A \ox k \{y^2x\} \ox A \ar[l]_{id\ox d_1} \\
}
\end{align*}
Identificando de manera natural $A \ox_{A^e} (A \ox W \ox A)$ con $A \ox W$ para todo espacio vectorial $W$ resulta
el complejo doble
\begin{align*}
\xymatrix{
	& & \\
	& A \ar[d]^{\delta} \ar@{--}[u]_{\partial} & A\ar[l]_{d} \ar[d]^{-\delta} \ar@{--}[u]_{-\partial} \\
	& A \ar[d]^{\partial} & A \ar[l]_{d} \ar[d]^{-\partial} \\
	& A \ar[d]^{\overline{\delta}} & A \ar[l]_{d} \ar[d]^{-\delta} \\
	A & A \oplus A \ar[l]_{d_0} & A \ar[l]_{d_1} \\
}
\end{align*}
con los siguientes diferenciales $k$-lineales
\begin{align*}
&d_0(a,b) = \left[a,x\right] + \left[b,y\right(],\\
&d_1(a) = (\left[a,y^2\right] - (axy + yxa), \left[x,a\right]y + y\left[x,a\right] - xax),\\
&d(a) = \left[a,y^2\right] - (axy + yxa),\\
&\overline{\delta}(a) = (ax + xa, 0),\\
&\delta(a) = ax + xa,\\
&\partial(a) = \left[a, x\right].
\end{align*}
Para calcular la homología de Hochschild vamos a utilizar una sucesión espectral inducida por la filtración por columnas del bicomplejo.
Denotamos por $E$ a la sucesión espectral.


\subsection{Primera página}
Para poder calcular los espacios de homología de la primera página de la sucesión espectral
vamos a necesitar saber qué valores toman los morfismos $\overline{\delta}$, $\delta$ y $\partial$
en los elementos de la base $\base$.

Empecemos con $\delta$. Sea $z = x^a(yx)^by^c \in \base$, luego $\delta(z) = x^a(yx)^by^cx + x^{a + 1}(yx)^by^c$. Nos va a
resultar conveniente escribir esta última expresión como combinación lineal de elementos de $\base$. Para eso vamos a necesitar considerar
diferentes casos.
\begin{itemize}
\item Si $a = 0$ y $c = 2k$,
\begin{align*}
	\delta(z) &= (yx)^by^{2k}x + x(yx)^by^{2k} = (yx)^b\left( \sum_{i = 0}^k\frac{k!}{i!}x(yx)^{k - i}y^{2i}\right) + x(yx)^by^{2k} \\
	&= \sum_{i = 0}^k\frac{k!}{i!}(yx)^bx(yx)^{k - i}y^{2i} + x(yx)^by^{2k}. 
\end{align*}
\begin{itemize}
\item Si $b = 0$,
\[
	\delta(z) = \sum_{i = 0}^k\frac{k!}{i!}x(yx)^{k - i}y^{2i} + xy^{2k}.
\]
\item Si $b \geq 1$, como $x^2 = 0$,
\begin{align*}
	\delta(z) &= \sum_{i = 0}^k\frac{k!}{i!}(yx)^{b-1}(yx)x(yx)^{k - i}y^{2i} + x(yx)^by^{2k} \\
	&= \sum_{i = 0}^k\frac{k!}{i!}(yx)^{b-1}yx^2(yx)^{k - i}y^{2i} + x(yx)^by^{2k} = x(yx)^by^{2k}.
\end{align*}
\end{itemize}
\item Si $a = 0$ y $c = 2k + 1$,
\begin{align*}
	\delta(z) &= (yx)^by^{2k + 1}x + x(yx)^by^{2k + 1} \\
	&= (yx)^b\left( \sum_{i = 0}^k\frac{k!}{i!}(yx)^{k - i + 1}y^{2i}\right) + x(yx)^by^{2k + 1} \\
	&= \sum_{i = 0}^k\frac{k!}{i!}(yx)^{b + k - i + 1}y^{2i} + x(yx)^by^{2k + 1}.
\end{align*}
\item Si $a = 1$ y $c = 2k$,
	\begin{align*}
		\delta(z) &= x(yx)^by^{2k}x + x^2(yx)^by^c = x(yx)^by^{2k}x \\
		&= x(yx)^b\left( \sum_{i = 0}^k\frac{k!}{i!}x(yx)^{k - i}y^{2i}\right) \\
		&= \sum_{i = 0}^k\frac{k!}{i!}x(yx)^bx(yx)^{k - i}y^{2i}. \\
		& = 0,
	\end{align*}
	ya que como $x^2 = 0$, resulta que $x(yx)^bx = 0$ para todo $b \geq 0$.
\item Si $a = 1$ y $c = 2k + 1$
	\begin{align*}
		\delta(x) &= x(yx)^by^{2k + 1}x = x(yx)^b\left( \sum_{i = 0}^k\frac{k!}{i!}(yx)^{k - i + 1}y^{2i}\right) \\
		&= \sum_{i = 0}^k\frac{k!}{i!}x(yx)^{b + k - i + 1}y^{2i}.
	\end{align*}
\end{itemize}
Con este cálculo probamos que
\begin{align*}
	\Ima(\delta) = \Bigg\langle &\sum_{i = 0}^k\frac{k!}{i!}x(yx)^{k - i}y^{2i} + xy^{2k},\ x(yx)^{b + 1}y^{2k},
		 \sum_{i = 0}^k\frac{k!}{i!}(yx)^{b + k - i + 1}y^{2i} + x(yx)^by^{2k + 1}, \\
		&\hspace{100pt}\sum_{i = 0}^k\frac{k!}{i!}x(yx)^{b + k - i + 1}y^{2i}\ :\ b,k \geq 0	\Bigg\rangle.
\end{align*}
Ahora que ya tenemos un conjunto de generadores de $\Ima(\delta)$, queremos extraer una base.
\begin{prop}
\label{imdelta}
El conjunto
\begin{align*}
		\left\{ x(yx)^by^{2k},
			\sum_{i = 0}^k\frac{k!}{i!}(yx)^{b + k - i + 1}y^{2i} + x(yx)^by^{2k + 1} :b, k \geq 0 \right\}
\end{align*}
es una base de $\Ima(\delta)$.
\end{prop}
\begin{proof}
Vamos a usar las siguientes notaciones:
\begin{align*}
	\eta_k &=  \sum_{i = 0}^k\frac{k!}{i!}x(yx)^{k - i}y^{2i} + xy^{2k}, \\
	\theta_{b,k} &= x(yx)^{b + 1}y^{2k},\\
	\lambda_{b,k} &= \sum_{i = 0}^k\frac{k!}{i!}(yx)^{b + k - i + 1}y^{2i} + x(yx)^by^{2k + 1}, \\
	\mu_{b,k} &= \sum_{i = 0}^k\frac{k!}{i!}x(yx)^{b + k - i + 1}y^{2i}.
\end{align*}
Sabemos que el conjunto $\{\eta_k,\ \theta_{b,k},\ \lambda_{b,k},\ \mu_{b,k}\ :\ b,k \geq 0\}$ genera $\Ima(\delta)$.
Como
\begin{align*}
	\eta_k &= \sum_{i = 0}^{k}\frac{k!}{i!}x(yx)^{k - i}y^{2i} + xy^{2k}
		= \sum_{i = 0}^{k-1}\frac{k!}{i!}x(yx)^{k - i}y^{2i} + 2xy^{2k} \\
	&\hspace{100pt}= \sum_{i = 0}^{k -1}\frac{k!}{i!}\theta_{k - i, i} + 2xy^{2k}
\end{align*}
y la característica de $\field$ es cero, resulta que $xy^{2k}$ pertenece a $\Ima(\delta)$ para todo $k \geq 0$. Por otro lado,
\begin{align*}
	\mu_{b,k} &= \sum_{i = 0}^k\frac{k!}{i!}x(yx)^{b + k - i + 1}y^{2i}
		= \sum_{i = 0}^k\frac{k!}{i!}\theta_{b + k - i + 1, i}.
\end{align*}
Por lo tanto
\begin{align*}
	&\left\{\theta_{b,k},\ xy^{2k},\ \lambda_{b,k}\ :\ b,k \geq 0 \right\} \\
	&\hspace{60pt}= \left\{ x(yx)^by^{2k},
			\sum_{i = 0}^k\frac{k!}{i!}(yx)^{b + k - i + 1}y^{2i} + x(yx)^by^{2k + 1} :b, k \geq 0 \right\}
\end{align*}
genera $\Ima(\delta)$. Más aún, se trata de una base ya que cualquier combinación lineal de elementos de este conjunto es una
combinación lineal de elementos distintos de $\base$.
\end{proof}
\begin{prop}
El conjunto
\begin{align*}
		\left\{ \left(x(yx)^by^{2k}, 0\right),
			\left(\sum_{i = 0}^k\frac{k!}{i!}(yx)^{b + k - i + 1}y^{2i} + x(yx)^by^{2k + 1},0\right) :b, k \geq 0 \right\}
\end{align*}
es una base de $\Ima(\overline{\delta})$.
\end{prop}
\begin{proof}
	Se deduce de que para todo $z \in A$, $\overline{\delta}(z) = (\delta(z), 0)$.
\end{proof}

Nos resta calcular los valores que toma $\partial$ en los elementos de la base y obtener una base de su imagen.
Calculemos $\partial(z)$ para $z = x^a(yx)^by^c \in \mathcal{B}$. Como \[\partial(z) = x^a(yx)^by^cx - x^{a + 1}(yx)^by^c,\] resulta que
\begin{itemize}
\item si $a = 0$ y $c = 2k$,
	\begin{align*}
		\partial(z) &= (yx)^by^{2k}x - x(yx)^by^{2k} = \sum_{i = 0}^k\frac{k!}{i!}(yx)^bx(yx)^{k - i}y^{2i} - x(yx)^by^{2k}.
	\end{align*}
	\begin{itemize}
	\item Si $b = 0$,
		\begin{align*}
			\partial(z) = \sum_{i = 0}^k\frac{k!}{i!}x(yx)^{k - i}y^{2i} - xy^{2k} = \sum_{i = 0}^{k - 1}\frac{k!}{i!}x(yx)^{k - i}y^{2i}.
		\end{align*}
	\item Si $b \geq 1$, usando nuevamente que $x^2 = 0,$
		\begin{align*}
			\partial(z) &= \sum_{i = 0}^k\frac{k!}{i!}(yx)^{b-1}(yx)x(yx)^{k - i}y^{2i} - x(yx)^by^{2k} \\
			&=  -x(yx)^by^{2k}.
		\end{align*}
	\end{itemize}
\item Si $a = 0$ y $c = 2k + 1$,
	\begin{align*}
		\partial(z) &= (yx)^by^{2k + 1}x - x(yx)^by^{2k + 1}
			= \sum_{i = 0}^k\frac{k!}{i!}(yx)^{b + k - i + 1}y^{2i} - x(yx)^by^{2k + 1}.
	\end{align*}
\item Si $a = 1$ y  $c = 2k$,
	\begin{align*}
		\partial(z) &= x(yx)^by^{2k}x - x^2(yx)^by^{2k} 
			= \sum_{i = 0}^k\frac{k!}{i!}x(yx)^bx(yx)^{k - i}y^{2i} = 0,
	\end{align*}
ya que 	$x(yx)^bx = 0$ para todo $b \geq 0$.
\item Si $a = 1$ y  $c = 2k + 1$,
	\begin{align*}
		\partial(z) &= x(yx)^by^{2k}x - x^2(yx)^by^{2k} = \sum_{i = 0}^k\frac{k!}{i!}x(yx)^{b + k - i + 1}y^{2i}.
	\end{align*}
\end{itemize}
Con este cálculo probamos que
\begin{align*}
	\Ima(\partial) = \Bigg\langle & \sum_{i = 0}^{k - 1}\frac{k!}{i!}x(yx)^{k - i}y^{2i},
		\ x(yx)^{b + 1}y^{2k},
		\ \sum_{i = 0}^k\frac{k!}{i!}(yx)^{b + k - i + 1}y^{2i} - x(yx)^by^{2k + 1}, \\
		&\hspace{100pt}\sum_{i = 0}^k\frac{k!}{i!}x(yx)^{b + k - i + 1}y^{2i} : b,k \geq 0 \Bigg\rangle.
\end{align*}
\begin{prop}
\label{impartial}
El conjunto
\begin{align*}
		\left\{ x(yx)^{b + 1}y^{2k},
			\sum_{i = 0}^k\frac{k!}{i!}(yx)^{b + k - i + 1}y^{2i} - x(yx)^by^{2k + 1} :b, k \geq 0 \right\}
\end{align*}
es una base de $\Ima(\partial)$.
\end{prop}
\begin{proof}
Llamamos como antes
\begin{align*}
	\eta_{k - 1} &=  \sum_{i = 0}^{k-1}\frac{k!}{i!}x(yx)^{k - i}y^{2i}, \\
	\theta_{b,k} &= x(yx)^{b + 1}y^{2k},\\
	\lambda_{b,k} &= \sum_{i = 0}^k\frac{k!}{i!}(yx)^{b + k - i + 1}y^{2i} - x(yx)^by^{2k + 1}, \\
	\mu_{b,k} &= \sum_{i = 0}^k\frac{k!}{i!}x(yx)^{b + k - i + 1}y^{2i}.
\end{align*}
Como $k - i \geq 1$ para todo $i$,  $0 \leq i \leq k -1$, entonces
\begin{align*}
	\eta_{k - 1} &= \sum_{i = 0}^{k-1}\frac{k!}{i!}x(yx)^{k - i}y^{2i} = \sum_{i = 0}^{k-1}\frac{k!}{i!}\theta_{k-1,i}.
\end{align*}
Por otro lado como $b + k - i + 1 \geq 1$ para todo $i$, $0 \leq i \leq k$, entonces
\begin{align*}
	\mu_{b,k} &= \sum_{i = 0}^k\frac{k!}{i!}x(yx)^{b + k - i + 1}y^{2i}
		= \sum_{i = 0}^k\frac{k!}{i!}\theta_{b + k - i + 1, i}.
\end{align*}
Luego el conjunto
\begin{align*}
	\left\{ x(yx)^{b + 1}y^{2k},
		\sum_{i = 0}^k\frac{k!}{i!}(yx)^{b + k - i + 1}y^{2i} - x(yx)^by^{2k + 1} :b,k \geq 0	\right\}
\end{align*}
 genera $\Ima(\partial)$ y es fácil ver que es una base.
\end{proof}

Ya estamos en condiciones de calcular la primera página de la sucesión espectral. 
\begin{prop}
	El espacio vectorial $E_{1,0}^1 = \frac{A}{\Ima(\overline{\delta})}$
	tiene como base al conjunto
	\begin{align*}
		\left\{ \left[((yx)^by^c, 0)\right]\ :\ b,c \geq 0\right\}
			\cup \left\{ \left[(0, x^a(yx)^by^c)\right]\ :\ a \in \{0, 1\}\ y\ b,c \geq 0\right\}. 
	\end{align*}
\end{prop}
\begin{proof}
	Debido a que
	\begin{align*}
		&\left[(x(yx)^by^{2k}, 0)\right] = 0, \\
		&\left[(-x(yx)^by^{2k + 1}, 0)\right] = \sum_{i = 0}^{k}\frac{k!}{i!}\left[(yx)^{b + k - i + 1}y^{2i}, 0)\right],
	\end{align*}
	es claro que el conjunto que aparece en el enunciado de la proposición genera $E_{1, 0}^1$. Es fácil verificar
	que también es linealmente independiente.
\end{proof}
\begin{prop}
	El espacio vectorial $E_{2,0}^1 = \frac{A}{\Ima(\delta)}$ tiene como base al conjunto
	\begin{align*}
		\left\{ \left[(yx)^by^c\right] \ :\ b,c \geq 0\right\}. 
	\end{align*}
\end{prop}
\begin{proof}
	Se deduce con un razonamiento análogo al anterior.
\end{proof}
\begin{prop}
	Los espacios vectoriales
	\begin{align*}
		E_{1,2i + 1}^1 = E_{2, 2i + 1}^1 = \frac{\Ker(\delta)}{\Ima(\partial)}
	\end{align*}
	tienen como base al conjunto $\left\{ \left[xy^{2n}\right] \ :\ n \geq 0\right\}$ , para todo $i \geq 0$.
\end{prop}
\begin{proof}
	Es claro que el núcleo y la imagen de $\delta$ coinciden respectivamente con el núcleo
	y la imagen de $-\delta$. Lo mismo ocurre con los morfismos
	$\partial$ y $-\partial$. Luego $E_{1,2i + 1}^1 = E_{2, 2i + 1}^1$, para todo $i \geq 0$.
	Sea $z \in \Ker(\delta)$, como $\delta(w) = 0$ o $\gr(\delta(w)) = \gr(w) + 1$ para todo $w \in A$
	y las relaciones que definen a $A$ son homogéneas, podemos suponer que $z$ es homogéneo.
	Separamos el cálculo de $z$ según la paridad de su grado.
	\begin{itemize}
		\item Si $\gr(z) = 2n$, entonces
		\begin{align*}
			z = \sum_{l = 0}^n\alpha_l(yx)^{n - l}y^{2l} + \sum_{i = 1}^n\beta_ix(yx)^{n - i}y^{2i - 1}.
		\end{align*}
		Por la Proposición \ref{impartial}, $x(yx)^{n - i}y^{2i - 1} - \sum_{j = 0}^{i - 1}\frac{(i-1)!}{j!}(yx)^{n - j}y^{2j}$
		pertenece a la imagen de $\partial$ para todo $1 \leq i \leq n$.
		Por lo tanto, reduciendo módulo borde, podemos suponer que $z = \sum_{l = 0}^n\alpha_l(yx)^{n - l}y^{2l}$.
		Si $z$ pertenece al núcleo de $\delta$, entonces
		\begin{align*}
			0 &= \delta(z) = \delta\left(\sum_{l = 0}^n\alpha_l(yx)^{n - l}y^{2l}\right) = \sum_{l = 0}^n\alpha_l\delta((yx)^{n - l}y^{2l}) \\
			&= \sum_{l = 0}^{n - 1}\alpha_l\delta((yx)^{n - l}y^{2l}) + \alpha_n\delta(y^{2n}) \\
				&= \sum_{l = 0}^{n - 1}\alpha_lx(yx)^{n - l}y^{2l} + \alpha_n\sum_{l = 0}^n \frac{n!}{l!}x(yx)^{n - l}y^{2l} + \alpha_nxy^{2n} \\
				&= \sum_{l = 0}^{n - 1}(\alpha_l - \alpha_n\frac{n!}{l!}) x(yx)^{n - l}y^{2l} + 2\alpha_nxy^{2n}.
		\end{align*}	
		Como el monomio $xy^{2n}$ no aparece en la sumatoria, resulta que $\alpha_n = 0$.
		Luego
		\[
			\sum_{l = 0}^{n - 1}\alpha_lx(yx)^{n - l}y^{2l} = 0
		\]
		y como los monomios que aparecen en la suma son linealmente independientes, entonces $\alpha_l = 0$ para todo $l$, $0 \leq l \leq n -1$.
		Es decir, $z = 0$.
		\item Si $\gr(z) = 2n + 1$, entonces
		\begin{align*}
			z = \sum_{l = 0}^n\alpha_l(yx)^{n - l}y^{2l + 1} + \sum_{l = 0}^n\beta_lx(yx)^{n - l}y^{2l}.
		\end{align*}
		Por la Proposición \ref{impartial}, $x(yx)^{n - l}y^{2l} \in \Ima(\partial)$ para todo $0 \leq l \leq n - 1$.
		Por lo tanto, reduciendo módulo borde, podemos suponer que $z = \sum_{l = 0}^n\alpha_l(yx)^{n - l}y^{2l + 1} + \beta xy^{2n}$.
		Si $z$ pertenece a el núcleo de $\delta$, entonces
		\begin{align*}
			0 &= \delta(z) = \sum_{l = 0}^n\alpha_l\delta((yx)^{n - l}y^{2l + 1}) + \beta \delta(xy^{2n}) \\
			&= \sum_{l = 0}^n\alpha_l \left( \sum_{i = 0}^l\frac{l!}{i!}(yx)^{n + 1 - i}y^{2i} + x(yx)^{n - l}y^{2l + 1} \right) \\
			&= \sum_{l = 0}^n \sum_{i = 0}^l\alpha_l\frac{l!}{i!}(yx)^{n + 1 - i}y^{2i}
				+ \sum_{l = 0}^n\alpha_l x(yx)^{n - l}y^{2l + 1}.
		\end{align*}	
		En particular $\sum_{l = 0}^n\alpha_l x(yx)^{n - l}y^{2l + 1}  = 0$, por lo tanto $\alpha_l = 0$ para todo $l$, $0 \leq l \leq n$.
		Es decir, $z = \beta xy^{2n}$ y
		\[
			E_{1,2i + 1}^1 = E_{2, 2i + 1}^1 = \left\langle \left[xy^{2n}\right]\ :\ n \geq 0 \right\rangle.
		\]
	\end{itemize}
		Veamos que los generadores que encontramos forman una base. Sea $[w] \in \frac{\Ker(\delta)}{\Ima(\partial)}$ una combinación
		lineal de elementos de la forma $[xy^{2n}]$ y supongamos que $[w] = 0$, es decir, $w \in \Ima(\partial)$.
		Como los elementos que generan $\Ima(\delta)$ son homogéneos podemos suponer que $w$ también lo es y por lo tanto $w = \lambda xy^{2n}$
		para algún $n \geq 0$ y para algún $\lambda \in k$.
		Pero los únicos elementos de la forma $x(yx)^by^c$ que pertenecen a $\Ima(\delta)$ cumplen que $b \geq 1$.
		Por lo tanto $\lambda = 0$ y el conjunto $\left\{ [xy^{2n}] \ :\ n \geq 0\right\}$ es una base de $\frac{\Ker(\delta)}{\Ima(\partial)}$.	 
\end{proof}
\begin{prop}
	Para todo $i \geq 1$, los espacios vectoriales
	\begin{align*}
		E_{1,2i}^1 = E_{2, 2i}^1 = \frac{\Ker(\partial)}{\Ima(\delta)}
	\end{align*}
	tienen como base al conjunto $\left\{ \sum_{l = 0}^n \frac{n!}{l!}\left[(yx)^{n - l}y^{2l}\right] \ :\ n \geq 0\right\}$.
\end{prop}
\begin{proof}
	Sea $z \in \Ker(\partial)$, nuevamente podemos suponer que $z$ es homogéneo por la misma razón que antes.
	Separamos el cálculo de $z$ según la paridad de su grado.
	\begin{itemize}
		\item Si $\gr(z) = 2n, n \geq 0$,
		\begin{align*}
			z = \sum_{l = 0}^n\alpha_l(yx)^{n - l}y^{2l} + \sum_{i = 1}^n\beta_ix(yx)^{n - i}y^{2i - 1}.
		\end{align*}
		Por la Proposición \ref{imdelta}, $x(yx)^{n - i}y^{2i - 1} + \sum_{j = 0}^{i - 1}\frac{(i-1)!}{j!}(yx)^{n - j}y^{2j} \in \Ima(\delta)$
		para todo $1 \leq i \leq n$.
		Por lo tanto, reduciendo módulo borde, podemos suponer que $z = \sum_{l = 0}^n\alpha_l(yx)^{n - l}y^{2l}$.
		Si $z \in \Ker(\partial)$, entonces
		\begin{align*}
			0 &= \sum_{l = 0}^n\alpha_l\partial((yx)^{n - l}y^{2l})
				= \sum_{l = 0}^{n - 1}\alpha_l\partial((yx)^{n - l}y^{2l}) + \alpha_n\partial(y^{2n})\\
			&= \sum_{l = 0}^{n - 1}\alpha_l(-x(yx)^{n - l}y^{2l}) + \alpha_n\sum_{l = 0}^{n - 1}\frac{n!}{l!}x(yx)^{n - l}y^{2l}
				= \sum_{l = 0}^{n - 1}(-\alpha_l + \alpha_n\frac{n!}{l!})x(yx)^{n - l}y^{2l}.
		\end{align*}
		Por lo tanto $\alpha_l = \alpha_n\frac{n!}{l!}$ para todo $0 \leq l \leq n$ y
		\[
			z = 	\sum_{l = 0}^n\alpha_l(yx)^{n - l}y^{2l} = \alpha_n\sum_{l = 0}^n\frac{n!}{l!}(yx)^{n - l}y^{2l}.
		\]
		\item Si $\gr(z) = 2n + 1, n \geq 0$,
		\begin{align*}
			z = \sum_{l = 0}^n\alpha_l(yx)^{n - l}y^{2l + 1} + \sum_{l = 0}^n\beta_lx(yx)^{n - l}y^{2l}.
		\end{align*}
		Por la Proposición \ref{imdelta}, $x(yx)^{n - l}y^{2l} \in \Ima(\delta)$ para todo $l$, $0 \leq l \leq n$.
		Por lo tanto, reduciendo módulo borde, podemos suponer que $z = \sum_{l = 0}^n\alpha_l(yx)^{n - l}y^{2l + 1}$.
		Si $z$ pertenece al núcleo de $\partial$, entonces
		\begin{align*}
			0 &= \sum_{l = 0}^n\alpha_l\partial((yx)^{n - l}y^{2l + 1})
				= \sum_{l = 0}^n\alpha_l\left(\sum_{i = 0}^l\frac{l!}{i!}(yx)^{n - i + 1}y^{2i} -x(yx)^{n - l}y^{2l + 1}\right)\\
			&= \sum_{l = 0}^n\sum_{i = 0}^l\alpha_l\frac{l!}{i!}(yx)^{n - i + 1}y^{2i} - \sum_{l = 0}^n\alpha_lx(yx)^{n - l}y^{2l + 1}.
		\end{align*}
		En particular $\sum_{l = 0}^n\alpha_lx(yx)^{n - l}y^{2l + 1} = 0$ y por lo tanto $\alpha_l = 0$ para todo $0 \leq l \leq n$.
		Luego
		\[
			E_{1,2i}^1 = E_{2, 2i}^1 = \left\langle \sum_{l = 0}^n \frac{n!}{l!}\left[(yx)^{n - l}y^{2l}\right] \ :\ n \geq 0\right\rangle.
		\]
	\end{itemize}
	Nuevamente para verificar que estos generadores forman una base alcanza con probar que para  todo $n \geq 0$,
	$w = \sum_{l = 0}^n \frac{n!}{l!}(yx)^{n - l}y^{2l}$ no pertenece al subespacio $\Ima(\delta)$.
	Como el grado de $w$ es par, si $w \in \Ima(\delta)$, entonces necesariamente $w$ debería pertenecer al subespacio
	\[
		\left\langle \sum_{i = 0}^k\frac{k!}{i!}(yx)^{b + k - i + 1}y^{2i} + x(yx)^by^{2k + 1} : b,k \geq 0 \right\rangle.
	\]
	Pero esto es absurdo debido a que aparecen términos de la forma $x(yx)^by^{2k + 1}$.
\end{proof}

El siguiente diagrama describe la primera página de la sucesión espectral:
\begin{align*}
\xymatrix{
	& & \\
	& \left\langle \left[xy^{2n}\right] \right\rangle \ar@{--}[u] & \left\langle \left[xy^{2n}\right] \right\rangle \ar[l]_{d^{(1)}} \ar@{--}[u]\\
	& \left\langle \sum_{l = 0}^n \frac{n!}{l!}\left[(yx)^{n - l}y^{2l}\right] \right\rangle \ar@{--}[u]
		& \left\langle \sum_{l = 0}^n \frac{n!}{l!}\left[(yx)^{n - l}y^{2l}\right] \right\rangle \ar[l]_{d^{(1)}} \ar@{--}[u]\\
	& \left\langle \left[xy^{2n}\right] \right\rangle \ar@{--}[u] & \left\langle \left[xy^{2n}\right] \right\rangle \ar[l]_{d^{(1)}} \ar@{--}[u]\\
	A & \left\langle \left[\left((yx)^by^c, 0\right)\right] \right\rangle \bigoplus A \ar[l]_(.7){d_0^{(1)}} \ar@{--}[u]
		& \left\langle \left[(yx)^by^c\right] \right\rangle \ar[l]_{d_1^{(1)}} \ar@{--}[u] \\
}
\end{align*}
donde $d_0^{(1)}$, $d_1^{(1)}$ y $d^{(1)}$ son los morfismos inducidos en la homología por $d_0$, $d_1$ y $d$ respectivamente.



\subsection{Segunda Página}
Debido a la forma del complejo doble la sucesión espectral se estaciona rápidamente y $E^2 = E^{\infty}$. Para el cálculo de la segunda página procederemos de igual manera que para la primera. Calcularemos
qué valores toman los morfismos $d_0^{(1)}, d_1^{(1)}$ y $d^{(1)}$ en los elementos de las bases correspondientes y luego obtendremos una descripción
de las imágenes de estos morfismos. Por último calcularemos las homologías en cada lugar.
\begin{prop} Para todo $k \geq 0$, el morfismo $d^{(1)}: E_{2, 2k + 1}^1 \to E_{1, 2k + 1}^1$ es nulo.
\end{prop}
\begin{proof}
	Sea $\left[xy^{2n}\right]$ un elemento de la base de $E_{2, 2k + 1}^1$, entonces
	\begin{align*}
		d(xy^{2n}) &= \left[ xy^{2n}, y^2 \right] - xy^{2n}xy
			= \left[ xy^{2n}, y^2 \right] - x\sum_{l = 0}^n\frac{n!}{l!}x(yx)^{n - l}y^{2l + 1}  \\
		&= \left[ xy^{2n}, y^2 \right] = \left[x, y^2 \right]y^{2n} + x\left[y^{2n}, y^2\right] =  \left[x, y^2 \right]y^{2n} \\
		&= (xy^2 - y^2x)y^{2n} = (-xyx)y^{2n}. 
	\end{align*}
	Luego $d(xy^{2n}) \in \Ima(\partial)$ y por lo tanto $d^{(1)}(\left[xy^{2n}\right]) = 0$
	\end{proof}
\begin{prop} Para todo $k \geq 1$, el morfismo $d^{(1)}: E_{2, 2k}^1 \to E_{1, 2k}^1$ es nulo.
\end{prop}
\begin{proof}
	Sea $w = \sum_{l = 0}^n \frac{n!}{l!}(yx)^{n - l}y^{2l}$, de modo que $[w]$ es un elemento de la base de $E_{2,2k}^1$.
	\begin{itemize}
		\item Si $n = 0$, entonces
			\[
				d(w) = d(1) = \left[1, y^2 \right] - (xy + yx) = -xy -yx.			
			\]
			Luego $d(w) \in \Ima(\delta)$ y $d^{(1)}(\left[w\right]) = 0$.
			De hecho, utilizando la notación de la demostración de la Proposición \ref{imdelta}, resulta que
			$d(w) = -\lambda_{0,0}$.
		\item Si $n \geq 1$ podemos suponer que $n = m + 1$, donde $m \geq 0$. Debido a la Proposición \ref{cp_conmutatividad}
			\begin{align*}
				w &= \sum_{l = 0}^{m + 1} \frac{(m + 1)!}{l!}(yx)^{m + 1 - l}y^{2l}
					= \left((m + 1)\sum_{l = 0}^m\frac{m!}{l!}(yx)^{m + 1 - l}y^{2l}\right) + y^{2m + 2} \\
				&= (m + 1)y^{2m + 1}x + y^{2m + 2}.
			\end{align*}	
			Aplicando $d$ obtenemos que
			\begin{align*}
				d(w) &= d((m + 1)y^{2m + 1}x + y^{2m + 2}) = (m + 1)d(y^{2m + 1}x) + d(y^{2m + 2}).		
			\end{align*}
			Calculemos $d(y^{2m + 1}x)$ y $d(y^{2m + 2})$ por separado:
			\begin{align*}
				d(y^{2m + 1}x) &= \left[y^{2m + 1}x, y^2\right] - yxy^{2m + 1}x = y^{2m + 1}\left[x,y^2\right] - yxy^{2m + 1}x\\
				&= y^{2m + 1}(-xyx) - yxy^{2m + 1}x = -y^{2m + 1}xyx - yxy^{2m + 1}x\\
				&= -\sum_{i = 0}^m\frac{m!}{i!}(yx)^{m -i + 1}y^{2i +1}x - \sum_{j = 0}^{m}\frac{m!}{j!}(yx)^{m -j + 2}y^{2j} \\
				&= -\sum_{i = 0}^m\frac{m!}{i!}(yx)^{m -i + 1}\left(\sum_{j = 0}^{i}\frac{i!}{j!}(yx)^{i - j + 1}y^{2j}\right)
					- \sum_{j = 0}^{m}\frac{m!}{j!}(yx)^{m -j + 2}y^{2j} \\
				&= -\sum_{i = 0}^m\sum_{j = 0}^i\frac{m!}{j!}(yx)^{m -j + 2}y^{2j}
					- \sum_{j = 0}^{m}\frac{m!}{j!}(yx)^{m -j + 2}y^{2j} \\
				&= -\sum_{j = 0}^m\sum_{i = j}^m\frac{m!}{j!}(yx)^{m -j + 2}y^{2j}
					- \sum_{j = 0}^{m}\frac{m!}{j!}(yx)^{m -j + 2}y^{2j} \\
				&= -\sum_{j = 0}^m(m - j + 1)\frac{m!}{j!}(yx)^{m -j + 2}y^{2j}
					- \sum_{j = 0}^{m}\frac{m!}{j!}(yx)^{m -j + 2}y^{2j} \\
				&= -\sum_{j = 0}^m(m - j + 2)\frac{m!}{j!}(yx)^{m -j + 2}y^{2j}
			\end{align*}
			y
			\begin{align*}
				d(y^{2m + 2}) &= \left[y^{2m + 2}, y^2\right] - (y^{2m +2}xy + yxy^{2m + 2}) = - y^{2m +2}xy - yxy^{2m + 2} \\
				&= -\sum_{l = 0}^{m + 1}\frac{(m + 1)!}{l!}x(yx)^{m - l + 1}y^{2l + 1} - yxy^{2m + 2}.
			\end{align*}
			Luego
			\begin{align*}
				d^{(1)}(\left[w\right]) &=
					-\sum_{j = 0}^m(m - j + 2)\frac{(m + 1)!}{j!}\left[(yx)^{m -j + 2}y^{2j}\right]\\
				&\hspace{70pt}-\sum_{l = 0}^{m + 1}\frac{(m + 1)!}{l!}\left[x(yx)^{m - l + 1}y^{2l + 1}\right]
						- \left[yxy^{2m + 2}\right]\\
				&= -\sum_{j = 0}^{m + 1}(m - j + 2)\frac{(m + 1)!}{j!}\left[(yx)^{m -j + 2}y^{2j}\right]\\
				& \hspace{70pt}-\sum_{l = 0}^{m + 1}\frac{(m + 1)!}{l!}\left[x(yx)^{m - l + 1}y^{2l + 1}\right]			
			\end{align*}
			Como $\sum_{i = 0}^k\frac{k!}{i!}(yx)^{b + k - i + 1}y^{2i} + x(yx)^by^{2k + 1} \in \Ima(\delta)$
			para todo $b,k \geq 0$, eligiendo $k = l$, $b = m - l + 1$, obtenemos que
				$-\left[x(yx)^{m - l + 1}y^{2l + 1}\right]
					= \sum_{j = 0}^l\frac{l!}{j!}\left[(yx)^{m - j + 2}y^{2j}\right].$
			Por lo tanto
			
			%& \hspace{70pt}-\sum_{l = 0}^{m + 1}\frac{(m + 1)!}{l!}\left[x(yx)^{m - l + 1}y^{2l + 1}\right] \\
			\begin{align*}
				d^1(\left[w\right]) &=
					\sum_{j = 0}^{m + 1}(m - j + 2)\frac{(m + 1)!}{j!}\left[(yx)^{m -j + 2}y^{2j}\right]\\
				& \hspace{70pt}-\sum_{l = 0}^{m + 1}\frac{(m + 1)!}{l!}\left[x(yx)^{m - l + 1}y^{2l + 1}\right] \\
				&=-\sum_{j = 0}^{m + 1}(m - j + 2)\frac{(m + 1)!}{j!}\left[(yx)^{m -j + 2}y^{2j}\right]\\
				& \hspace{70pt} + \sum_{l = 0}^{m + 1}\frac{(m + 1)!}{l!}
					\left(\sum_{j = 0}^l\frac{l!}{j!}\left[(yx)^{m - j + 2}y^{2j}\right]\right) \\
				&= -	\sum_{j = 0}^{m + 1}(m - j + 2)\frac{(m + 1)!}{j!}\left[(yx)^{m -j + 2}y^{2j}\right]\\
				& \hspace{70pt} + \sum_{l = 0}^{m + 1}\sum_{j = 0}^{l}\frac{(m + 1)!}{j!}
					\left[(yx)^{m - j + 2}y^{2j}\right]\\
				&= -	\sum_{j = 0}^{m + 1}(m - j + 2)\frac{(m + 1)!}{j!}\left[(yx)^{m -j + 2}y^{2j}\right]\\
				& \hspace{70pt} + \sum_{j = 0}^{m + 1}\sum_{l = j}^{m + 1}\frac{(m + 1)!}{j!}
					\left[(yx)^{m - j + 2}y^{2j}\right]\\
				&= -	\sum_{j = 0}^{m + 1}(m - j + 2)\frac{(m + 1)!}{j!}\left[(yx)^{m -j + 2}y^{2j}\right]\\
				& \hspace{70pt} + \sum_{j = 0}^{m + 1}(m - j + 2)\frac{(m + 1)!}{j!}
					\left[(yx)^{m - j + 2}y^{2j}\right]\\
				&= 0.
			\end{align*}	 
	\end{itemize}
\end{proof}

\myworries{Mover al apéndice}

El siguiente paso va a ser calcular que valores toma $d_1^{(1)}$ en los elementos de la base de $E^1_{2, 0}$, pero antes
vamos a probar dos resultado que nos serán útiles.
\begin{prop}
\label{conmutary}
Sea $b \geq 1$, entonces $y(yx)^b = x(yx)^{b - 1}y^2 + bx(yx)^b$.
\end{prop}
\begin{proof}
Lo probaremos por inducción. El caso $b = 1$ es claro. Sea $b \geq 1$, luego
\begin{align*}
	y(yx)^{b + 1} &= y(yx)^b(yx) = (x(yx)^{b - 1}y^2 + bx(yx)^b)yx = x(yx)^{b - 1}y^3x + bx(yx)^{b + 1}\\
	&= x(yx)^{b - 1}(yxy^2 + (yx)^2) + bx(yx)^{b + 1} = x(yx)^{b}y^2 + (b + 1)x(yx)^{b + 1}.
\end{align*}
\end{proof}
El siguiente corolario es una consecuencia inmediata de la Proposición \ref{conmutary}:
\begin{coro}
\label{conmutary^2}
Si $b \geq 0$, entonces $y^2(yx)^b = (yx)^{b}y^2 + b(yx)^{b + 1}$.
\end{coro}

Ya estamos en condiciones de calcular la imagen de $d_1^{(1)}$. Llamemos
$f$ a la composición de $d_1$ con la proyección en la primera coordenada y llamemos $g$ a la composición de $d_1$ con la proyección en la segunda coordenada, de modo que
\[
	d_1(a) =  \left(\left[a,y^2\right] - (axy + yxa), \left[x,a\right]y + y\left[x,a\right] - xax\right) = (f(a), g(a)), \text{ para todo } a \in A.
\]
Para no complicar la notación notaremos $f$ al morfismo inducido por $f$ en la homología. Lo mismo haremos con $g$.
Empecemos calculando que valores toma $f$ en los elementos de la base de $E^1_{2, 0}$. Sean $b, c \geq 0$ y $z = (yx)^by^c$, de modo que $\left[z\right]$
sea un elemento de la base de $E^1_{2, 0}$. Luego
\begin{align*}
	f(z) &= \left[(yx)^by^c, y^2\right] - \left((yx)^by^cxy + yx(yx)^by^c\right)\\
	&= \left[(yx)^b, y^2\right]y^c - (yx)^b(y^cx)y - (yx)^{b + 1}y^c \\
	&= (yx)^by^{c + 2} - y^2(yx)^by^c - (yx)^b(y^cx)y - (yx)^{b + 1}y^c \\
\end{align*}
y debido a la Proposición \ref{conmutary^2},
\begin{align*}
	f(z) &= (yx)^by^{c + 2} - ((yx)^by^2 + b(yx)^{b + 1})y^c - (yx)^b(y^cx)y - (yx)^{b + 1}y^c \\
	&= - (b + 1)(yx)^{b + 1}y^c - (yx)^b(y^cx)y.
\end{align*}
En la homología, resulta que
\begin{itemize}
	\item si $c = 2k$,
	\[
		f([z]) = -(b + 1)\left[(yx)^{b + 1}y^{2k}\right] - \sum_{i  = 0}^k\frac{k!}{i!}\left[(yx)^bx(yx)^{k - i}y^{2i + 1}\right].
	\]	
	
	\begin{itemize}
		\item Si $b = 0$
		\[
			f([z]) = -\left[(yx)y^{2k}\right] - \sum_{i  = 0}^k\frac{k!}{i!}\left[x(yx)^{k - i}y^{2i + 1}\right].
		\]
		Como $\left(\sum_{l = 0}^n\frac{n!}{l!}(yx)^{d + n - l + 1}y^{2l} + x(yx)^by^{2n + 1}, 0\right) \in \Ima(\overline{\delta})$
		para todo $n,d \geq 0$, podemos elegir $n = i$, $d = k - i$ y obtenemos que 
		\[
			-\left[x(yx)^{k - i}y^{2i + 1}\right] = \sum_{l = 0}^i\frac{i!}{l!}\left[(yx)^{k - l + 1}y^{2l}\right].
		\]
		Luego
		\begin{align*}
			f([z]) &= -\left[(yx)y^{2k}\right] + \sum_{i  = 0}^k\sum_{l = 0}^i\frac{k!}{l!}\left[(yx)^{k - l + 1}y^{2l}\right]\\
			&= -\left[(yx)y^{2k}\right] + \sum_{l = 0}^{k}(k - l + 1)\frac{k!}{l!}\left[(yx)^{k - l + 1}y^{2l}\right]\\
			&= \sum_{l = 0}^{k - 1}(k - l + 1)\frac{k!}{l!}\left[(yx)^{k - l + 1}y^{2l}\right].
		\end{align*}
		\item Si $b \geq 1$,
		\[		
		f([z]) = -(b + 1)\left[(yx)^{b + 1}y^{2k}\right].
		\]	
	\end{itemize}
	\item Si $c = 2k + 1$,
	\[
		f([z]) = -(b + 1)\left[(yx)^{b + 1}y^{2k + 1}\right] - \sum_{i  = 0}^k\frac{k!}{i!}\left[(yx)^{k + b - i + 1}y^{2i + 1}\right].
	\]
\end{itemize}

Ahora vamos a calcular qué valores toma $g$. Dados $b, c$ y $z$ como antes. Luego
\[
	g(z) = \left[x,z\right]y + y\left[x,z\right] - xzx
	\]


%Para alivianar la notación, de ahora en más dejaremos de escribir los corchetes
%cuando hablemos de clases de elementos en los espacios cociente.
\end{document}
