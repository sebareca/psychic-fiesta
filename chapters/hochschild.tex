\documentclass[a4paper,oneside,fleqn,11pt,../tesis.tex]{subfiles}

\begin{document}

\section{Definiciones básicas}
Esta sección se basa en \myworries{Weibel} y en \myworries{Ginzburg}.

Antes de definir homología y cohomología de Hochschild es necesario hacer la siguiente observación.
Sea $A$ una $\field$-álgebra, llamamos \emph{álgebra envolvente de $A$} a la $\field$-álgebra
\[
	A^{e} = A \ox A^{op}.
\]
Debido a que el producto en $A^{op}$ está invertido es lo mismo considerar un $A$-módulo a derecha que un $A^{op}$-módulo a izquierda.
Por lo tanto un $A-A$ bimódulo $M$ es lo mismo que un $A^{e}$-módulo a izquierda, mediante la acción dada por $(a\ox b)\cdot m = a\cdot m \cdot b$
para todo $a, b \in A$ y para todo $m \in M$ y también es lo mismo que un $A^{e}$-módulo a derecha, mediante la acción dada por $m \cdot (a \ox b) = b\cdot m \cdot a$.
En particular, esto permite ver a la categoría de $A-A$ bimódulos como la categoría de $A^e$-módulos a izquierda.
Esta observación es fundamental para la siguiente definición ya que nos dice que la categoría de $A-A$ bimódulos
tiene suficientes proyectivos.

\begin{definition}
	Sea $A$ una $\field$-álgebra y $M$ un $A-A$ bimódulo. Se define la \emph{homología de Hochschild de $A$ con coeficientes en $M$} como
		\[
			\Hy_{\bullet}(A,M) = \Tor_{\bullet}^{A^e}(A, M)	
		\]
	y se define la \emph{cohomología de Hochschild de $A$ con coeficientes en $M$} como
	\[
			\Hy^{\bullet}(A,M) = \Ext^{\bullet}_{A^e}(A, M).	
	\]
\end{definition}

Para calcular los grupos de homología y cohomología de Hochschild es necesario elegir alguna resolución de $A$ como $A-A$ bimódulo,
o lo que es lo mismo, como $A^e$-módulos a izquierda. Sabemos que siempre existe una resolución porque la categoría tiene suficientes
proyectivos. La resolución que fue usada en la definición original de homología y cohomología de Hochschild es la \emph{resolución bar}.
\begin{definition}
	Sea $A$ una $\field$ álgebra. Se define el \emph{complejo bar de $A$ de $A-A$ bimódulos}
	\begin{align*}
		\xymatrix{
			B_{\bullet}A:\ldots \ar[r]^{b'_3} &  A \ox A^{\ox 3} A \ar[r]^{b'_2} & A \ox A^{\ox 2} A \ar[r]^{b'_1}
				& A \ox A \ox A \ar[r]^{b'_0} & A \ox A \ar[r] & 0 
		}
	\end{align*}
	donde, para cada $n \geq 0$, el diferencial $b_n: A \ox A^{\ox (n + 1)} \ox A \to A \ox A^{\ox n} \ox A$ está dado por
	\begin{align*}
		b'_n&(1\ox a_0 \ox \cdots \ox a_n \ox 1) = a_0 \ox \cdots \ox a_n \ox 1 \\
		 & \qquad + \sum_{i = 0}^{n -1}(-1)^{i + 1} \ox a_0 \ox \cdots \ox a_i a_{i + 1} \ox \cdots \ox a_n \ox 1 + (-1)^{n + 1} \ox \cdots \ox a_n.
	\end{align*}
\end{definition}
El complejo bar juntos con la multiplicación $m_A: A \ox A \to A$ es una resolución libre de $A$ como $A-A$ bimódulo. Una demostración de este hecho se puede encontrar en \myworries{[CE]}.

Para calcular los espacios $\Hy_i(A,M)$ aplicamos el funtor $M\ox_{A^e}-$ al complejo bar y obtenemos
\begin{align*}
	\xymatrix{
		M \ox_{A^e} B_{\bullet}A:\ldots \ar[r]^{id\ox b'_{2}} & M\ox_{A^e}A^{\ox 4} \ar[r]^{id\ox b'_{1}} & M \ox_{A^e}A^{\ox 3} \ar[r]^{id\ox b'_{0}} & M\ox_{A^e}A^{\ox 2} \ar[r] & 0.
	}
\end{align*}
Es posible realizar la siguiente simplificación. Sea
\[
	m\ox_{A^e}(a_0 \ox \cdots \ox a_n) \in  M\ox_{A^e}A^{\ox n}
\]
y escribamos
\begin{align*}
	m\ox_{A^e}(a_0 \ox \cdots \ox a_n) &= m\ox_{A^e}(a_0 \ox a_{n}^{op}) \cdot (1 \ox a_1 \cdots \ox a_{n -1} \ox 1) \\
		&=m \cdot(a_0 \ox a_{n}^{op})\ox_{A^e}(1 \ox a_1 \cdots \ox a_{n -1} \ox 1) \\
		&= \left( a_n \cdot m \cdot a_0\right) \ox_{A^e}(1 \ox a_1 \cdots \ox a_{n -1} \ox 1).
\end{align*}
Esta igualdad sugiere el siguiente isomorfismo $\field$-lineal
\[
	m\ox_{A^e}(a_0 \ox \cdots \ox a_n) \in  M\ox_{A^e}A^{\ox n} \mapsto
		\left( a_n \cdot m \cdot a_0\right) \ox_{A^e}(a_1 \cdots \ox a_{n -1}) \in M\ox_{A^e}A^{\ox n - 2}
\]
que identifica $ M\ox_{A^e}A^{\ox n}$ con $M\ox_{A^e}A^{\ox n - 2}$ e induce el siguiente complejo isomorfo
\begin{align*}
	\xymatrix{
		\ldots \ar[r]^{b_3} & M\ox A^{\ox 3} \ar[r]^{b_2}  & M\ox A^{\ox 2} \ar[r]^{b_1} & M \ox A \ar[r]^{b_0} & M \ar[r] & 0.
	}
\end{align*}
donde los diferenciales están dados por
\begin{align*}
	&b_0(m\ox a) = m\cdot a - a\cdot m,\\
	&b_n(m\ox a_1 \ox \ldots \ox a_n) = m \cdot a_1 \ox \cdots \ox a_n \\
		 &\qquad + \sum_{i = 1}^{n -1}(-1)^{i} m \ox a_1 \ox \cdots \ox a_i a_{i + 1} \ox \cdots \ox a_n
		 	+ (-1)^n a_n \cdot m \ox a_1 \ox \cdots \ox a_{n-1}.
\end{align*}
para todo $n \geq 1$.

Para calcular la cohomología de Hochschild aplicamos el funtor $\Hom_{A^e}(-,M)$ al complejo bar y obtenemos
\begin{align*}
	\xymatrix{
		\ldots & \Hom_{A^e}\left(A^{\ox 4}, M\right) \ar[l]_{d'_2} & \Hom_{A^e}\left(A^{\ox 3}, M\right) \ar[l]_{d'_1}
			& \Hom_{A^e}\left(A^{\ox 2}, M\right) \ar[l]_{d'_0} & 0 \ar[l],
	}
\end{align*}
donde $d'_n(f) = f \circ b'_n$ para todo $f \in \Hom_{A^e}\left(A^{\ox n + 2}, M\right)$ y para todo $n \geq 0$.

Nuevamente es posible realizar una simplificación. Para todo
$n \geq 0$,
\[
	\Hom_{A^e}\left(A^{\ox (n + 2)}, M\right) \cong \Hom_{\field}\left(A^{\ox n}, M\right),
\]
mediante el isomorfismo que asocia $\varphi \in \Hom_{A^e}\left(A^{\ox (n + 2)}, M \right)$ con la transformación $\field$-lineal
\[
	(a_1 \ox \ldots \ox a_n) \in A^{\ox n} \mapsto \varphi(1 \ox a_1 \ox \ldots \ox a_n \ox 1) \in M.
\]
Esto reduce el cálculo de la cohomología de Hochschild al siguiente complejo:
\begin{align*}
	\xymatrix{
		\ldots & \Hom_{\field}\left(A^{\ox 2}, M\right) \ar[l]_{d_2} & \Hom_{\field}\left(A, M\right) \ar[l]_{d_1}
			& M \ar[l]_{d_0} & 0 \ar[l],
	}
\end{align*}
donde los diferenciales están dados por
\begin{align*}
	&d_0(m)(a) = a\cdot m - m\cdot a,\\
	&d_n(f)(a_1 \ox \ldots \ox a_{n +1}) = a_1 \cdot f(a_2 \ox \ldots \ox a_{n + 1})\\
	&\qquad + \sum_{i = 1}^{n}(-1)^{i} f(a_1 \ox \cdots \ox a_i a_{i + 1} \ox \cdots \ox a_{n+1})
		 	+ (-1)^{n + 1} f(a_1 \ox \cdots \ox a_{n}) \cdot a_{n + 1}
\end{align*}
para todo $f \in \Hom_{\field}\left(A^{\ox n}, M\right)$ y para todo $n \geq 1$.

Calculemos los primeros dos espacios de cohomología.
\begin{itemize}
	\item El grupo $\Hy^{0}(A, M)$ es isomorfo $\Ker(d_0)$.
	Un elemento $m$ pertenece a $\Ker(d_0)$ si y solo si $a\cdot m - m \cdot a = 0$ para todo $a \in A$. En particular,
	si $M = A$, resulta que $\Hy^{0}(A, A)$ es el centro del álgebra.
	
	\item Para obtener una descripción del espacio $\Hy^{1}(A, M)$ vamos a calcular por separado $\Ker(d_1)$
	e $\Inn(d_0)$. Dado $f \in \Hom_{\field}\left(A, M\right)$, $f$ pertenece a $\Ker(d_1)$ si y solo si
	\[
		0 = d_1(f)(a_1\ox a_2) = a_1\cdot f(a_2) - f(a_1 a_2) + f(a_1)\cdot a_2
	\]
	para todo $a_1, a_2 \in A$. Por lo tanto, $\Ker(d_1)$ es el conjunto de \emph{derivaciones} $\Der_{\field}(A, M)$. Por otro lado
	la imagen de $d_0$ es el conjunto de \emph{derivaciones interiores} $\Inn_{\field}(A, M)$, es decir, los morfismos de la forma
	$a \mapsto a\cdot m - m\cdot a$ para algún $m \in M$. Luego $\Hy^{1}(A, M)$ resulta ser isomorfo al conjunto de \emph{derivaciones exteriores}
	\[
		\Der_{\field}(A, M) / \Inn(A, M).
	\]
	Notar que si $M = A$, entonces $\Der_{\field}(A) = \Der_{\field}(A,A)$ es un álgebra de Lie e $\Inn(A)$ es un ideal de Lie.
	Por lo tanto $\Hy^{1}(A, A)$ es un álgebra de Lie.
	
\end{itemize}

\subsection{Estructura de álgebra de Gerstenhaber}

Sea $A$ una $\field$-álgebra. La cohomología de Hochschild de $A$ con coeficientes en $A$ tiene estructura
de álgebra graduada super conmutativa y estructura de super álgebra de Lie junto con cierta relación de compatibilidad.

\begin{definition}
	Un \emph{álgebra de Gerstenhaber} es un álgebra graduada super conmutativa $G^{\bullet} = \oplus_{i}G^i$ junto con
	un corchete $\left[\cdot, \cdot \right]: G^{p} \times G^{q} \to G^{p + q - 1}$
	tal que $\left(G^{\bullet}, \left[\cdot, \cdot \right]\right)$ es una super álgebra de Lie y para todo $a \in G^{\bullet}$,
	el morfismo $b \in G^{\bullet} \mapsto \left[a,b\right] \in G^{\bullet}$
	es una super derivación con respecto al producto, es decir,
	\[
		\left[a, b c\right] = \left[a,b\right]c + (-1)^{(\gr(a) - 1)\gr(b)}b\left[a, c\right]
	\]
	para todo $a, b, c \in G^{\bullet}$.
\end{definition}

Empecemos dandole estructura de álgebra graduada a $\oplus_{n \geq 0}\Hom_{\field}\left(A^{\ox n}, A\right)$.
\begin{definition}
	Sean $f \in \Hom_{\field}\left(A^{\ox n}, A\right)$ y $g \in \Hom_{\field}\left(A^{\ox m}\right)$, se define el \emph{producto cup}
	$f \smile g \in \Hom_{\field}\left(A^{\ox n + m}, A\right)$ como la única transformación $\field$-lineal que cumple que
	\[
		f\smile g(a_1\ox \ldots \ox a_n \ox a_{n + 1} \ox \ldots \ox a_{n + m}) = f(a_1\ox \ldots \ox a_n)g(a_{n + 1} \ox \ldots \ox a_{n + m})
	\]
	para todo $a_1, \ldots, a_{n + m} \in A$.
\end{definition}
Los diferenciales del complejo resultan ser super derivaciones con respecto a este producto, es decir, si $f$ y $g$ son dos
elementos homogeneos de grados $n$ y $m$ respectivamente, entonces
\[
	d_{n + m}(f \smile g) = d_{n}(f) \smile g + (-1)^{nm}f \smile d_{m}(g).
\]
Esta formula muestra que el producto de dos cociclos es un cociclo y que el producto de un coborde con un cociclo
es un coborde. Por lo tanto $\Hy^{\bullet}(A,A)$ resulta ser una $\field$-álgebra graduada con el producto inducido por el producto cup.

También existe en $\oplus_{n \geq 0}\Hom_{\field}\left(A^{\ox n}, A\right)$ una estructura de super álgebra de Lie.
\begin{definition}
Sean $f \in \Hom_{\field}\left(A^{\ox n}, A\right)$ y $g \in \Hom_{\field}\left(A^{\ox m}\right)$, se define el \emph{asociador}
entre $f$ y $g$, $f \circ g \in \Hom_{\field}\left(A^{\ox n + m - 1}, A\right)$ como la única transformación $\field$-lineal que cumple que
	\begin{align*}
		&f \circ g (a_1\ox \ldots \ox a_{n + m - 1 }) = \\
		&\quad\sum_{i = 1}^n (-1)^{(i - 1)(m -1)}
				f\left(a_1\ox \ldots \ox a_{i - 1} \ox g(a_{i} \ox \ldots \ox a_{i + m - 1}) \ox a_{i + m} \ox \ldots \ox a_{n + m - 1}\right)
	\end{align*}
	para todo $a_1, \ldots, a_{n + m - 1} \in A$.
Se define el \emph{corchete de Gerstenhaber} entre estos dos elementos como
\[
	\left[f, g\right] = f \circ g  - (-1)^{(n -1)(m -1)}g \circ f.
\] 
\end{definition}

El producto cup y el corchete de Gerstenhaber hacen de $\Hy^{\bullet}(A, A)$ un álgebra de Gerstenhaber.

\end{document}








