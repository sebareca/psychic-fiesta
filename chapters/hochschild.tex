\documentclass[a4paper,oneside,fleqn,11pt,../tesis.tex]{subfiles}

\begin{document}

En este capítulo daremos ciertas definiciones y resultados del álgebra homológica que serán necesarios
para el próximo capítulo. En la primera sección daremos una definición de la homología y la cohomología
de Hochschild. En la segunda sección describiremos ciertas estructuras algebraicas que tiene
la cohomología de Hochschild. En la última sección resumiremos algunos
resultados de sucesiones espectrales que serán utilizados para facilitar los cálculos del siguiente capítulo.

\section{Homología y cohomología de Hochschild}
Esta sección se basa en \cite{We} y en \cite{Gi}.

Antes de definir homología y cohomología de Hochschild es necesario observar el siguiente hecho.
Debido a que el producto en $A^{op}$ está invertido, es lo mismo considerar un $A$-módulo a derecha que un $A^{op}$-módulo a izquierda.
Por lo tanto darse un $A-A$ bimódulo $M$ es lo mismo que darse un $A^{e}$-módulo a izquierda, con acción de $A^e$
dada por $(a\ox b)\cdot m = a\cdot m \cdot b$ para todo $a, b \in A$ y para todo $m \in M$; también es lo mismo que un $A^{e}$-módulo a derecha,
con acción dada por $m \cdot (a \ox b) = b\cdot m \cdot a$.
Esto permite identificar a la categoría de $A-A$ bimódulos con la categoría de $A^e$-módulos a izquierda,
lo cual asegura que la categoría de $A-A$ bimódulos tiene suficientes proyectivos. Recordemos
que $A^e = A\ox A^{op}$ es la $\field$-álgebra envolvente de $A$.

\begin{definition}
	Sea $A$ una $\field$-álgebra y $M$ un $A-A$ bimódulo. Se define la \emph{homología de Hochschild de $A$ con coeficientes en $M$} como
		\[
			\Hy_{\bullet}(A,M) = \Tor_{\bullet}^{A^e}(A, M)	
		\]
	y se define la \emph{cohomología de Hochschild de $A$ con coeficientes en $M$} como
	\[
			\Hy^{\bullet}(A,M) = \Ext^{\bullet}_{A^e}(A, M).	
	\]
\end{definition}

Para calcular los grupos de homología y cohomología de Hochschild es necesario elegir alguna resolución de $A$ como $A-A$ bimódulo,
o lo que es lo mismo, como $A^e$-módulo a izquierda. Sabemos que siempre existe una tal resolución porque la categoría tiene suficientes
proyectivos. La resolución que fue usada en la definición original de homología y cohomología de Hochschild es la \emph{resolución bar}.
\begin{definition}
	Sea $A$ una $\field$-álgebra. Se define el \emph{complejo bar de $A$ de $A-A$ bimódulos}
	\begin{align*}
		\xymatrix{
			B_{\bullet}A:\ldots \ar[r]^{b'_3} &  A \ox A^{\ox 3} A \ar[r]^{b'_2} & A \ox A^{\ox 2} A \ar[r]^{b'_1}
				& A \ox A \ox A \ar[r]^{b'_0} & A \ox A \ar[r] & 0 
		}
	\end{align*}
	donde, para cada $n \geq 0$, el diferencial $b_n: A \ox A^{\ox (n + 1)} \ox A \to A \ox A^{\ox n} \ox A$ está dado por
	\begin{align*}
		b'_n&(1\ox a_0 \ox \cdots \ox a_n \ox 1) = a_0 \ox \cdots \ox a_n \ox 1 \\
		 & \qquad + \sum_{i = 0}^{n -1}(-1)^{i + 1} \ox a_0 \ox \cdots \ox a_i a_{i + 1} \ox \cdots \ox a_n \ox 1 + (-1)^{n + 1} \ox \cdots \ox a_n.
	\end{align*}
\end{definition}
El complejo bar junto con la multiplicación $m_A: A \ox A \to A$ es una resolución libre de $A$ como $A^e$-módulo a izquierda.
Una demostración de este hecho se puede encontrar en \cite[Capitulo 9]{CE}.

Para calcular los espacios $\Hy_i(A,M)$ aplicamos el funtor $M\ox_{A^e}-$ al complejo bar y obtenemos
\begin{align*}
	\xymatrix{
		M \ox_{A^e} B_{\bullet}A:\ldots \ar[r]^{id\ox b'_{2}} & M\ox_{A^e}A^{\ox 4} \ar[r]^{id\ox b'_{1}} & M \ox_{A^e}A^{\ox 3} \ar[r]^{id\ox b'_{0}} & M\ox_{A^e}A^{\ox 2} \ar[r] & 0.
	}
\end{align*}
Es posible realizar la siguiente simplificación. Dado
\[
	m\ox_{A^e}(a_0 \ox \cdots \ox a_n) \in  M\ox_{A^e}A^{\ox n},
\]
observemos que:
\begin{align*}
	m\ox_{A^e}(a_0 \ox \cdots \ox a_n) &= m\ox_{A^e}(a_0 \ox a_{n}^{op}) \cdot (1 \ox a_1 \cdots \ox a_{n -1} \ox 1) \\
		&=m \cdot(a_0 \ox a_{n}^{op})\ox_{A^e}(1 \ox a_1 \cdots \ox a_{n -1} \ox 1) \\
		&= \left( a_n \cdot m \cdot a_0\right) \ox_{A^e}(1 \ox a_1 \cdots \ox a_{n -1} \ox 1).
\end{align*}
Esta igualdad sugiere el siguiente isomorfismo $\field$-lineal
\[
	m\ox_{A^e}(a_0 \ox \cdots \ox a_n) \in  M\ox_{A^e}A^{\ox n} \mapsto
		\left( a_n \cdot m \cdot a_0\right) \ox_{A^e}(a_1 \cdots \ox a_{n -1}) \in M\ox_{A^e}A^{\ox n - 2}
\]
que identifica $ M\ox_{A^e}A^{\ox n}$ con $M\ox_{A^e}A^{\ox n - 2}$ e induce el siguiente complejo isomorfo
\begin{align*}
	\xymatrix{
		\ldots \ar[r]^(.4){b_3} & M\ox A^{\ox 3} \ar[r]^{b_2}  & M\ox A^{\ox 2} \ar[r]^{b_1} & M \ox A \ar[r]^{b_0} & M \ar[r] & 0.
	}
\end{align*}
donde los diferenciales están dados por
\begin{align*}
	&b_0(m\ox a) = m\cdot a - a\cdot m,\\
	&b_n(m\ox a_1 \ox \ldots \ox a_n) = m \cdot a_1 \ox \cdots \ox a_n \\
		 &\qquad + \sum_{i = 1}^{n -1}(-1)^{i} m \ox a_1 \ox \cdots \ox a_i a_{i + 1} \ox \cdots \ox a_n
		 	+ (-1)^n a_n \cdot m \ox a_1 \ox \cdots \ox a_{n-1}.
\end{align*}
para todo $n \geq 1$.

Para calcular la cohomología de Hochschild aplicamos el funtor $\Hom_{A^e}(-,M)$ al complejo bar y obtenemos
\begin{align*}
	\xymatrix{
		\ldots & \Hom_{A^e}\left(A^{\ox 4}, M\right) \ar[l]_(.7){d'_2} & \Hom_{A^e}\left(A^{\ox 3}, M\right) \ar[l]_{d'_1}
			& \Hom_{A^e}\left(A^{\ox 2}, M\right) \ar[l]_{d'_0} & 0 \ar[l],
	}
\end{align*}
donde $d'_n(f) = f b'_n$ para todo $f \in \Hom_{A^e}\left(A^{\ox n + 2}, M\right)$ y para todo $n \geq 0$.

Nuevamente es posible realizar una simplificación. Para todo
$n \geq 0$,
\[
	\Hom_{A^e}\left(A^{\ox (n + 2)}, M\right) \cong \Hom_{\field}\left(A^{\ox n}, M\right),
\]
mediante el isomorfismo que asocia $\varphi \in \Hom_{A^e}\left(A^{\ox (n + 2)}, M \right)$ con la transformación $\field$-lineal
\[
	(a_1 \ox \ldots \ox a_n) \in A^{\ox n} \mapsto \varphi(1 \ox a_1 \ox \ldots \ox a_n \ox 1) \in M.
\]
Esto reduce el cálculo de la cohomología de Hochschild al cálculo de la homología del siguiente complejo:
\begin{align*}
	\xymatrix{
		\ldots & \Hom_{\field}\left(A^{\ox 2}, M\right) \ar[l]_(.7){d_2} & \Hom_{\field}\left(A, M\right) \ar[l]_(.45){d_1}
			& M \ar[l]_(.3){d_0} & 0 \ar[l],
	}
\end{align*}
donde los diferenciales están dados por
\begin{align*}
	&d_0(m)(a) = a\cdot m - m\cdot a,\\
	&d_n(f)(a_1 \ox \ldots \ox a_{n +1}) = a_1 \cdot f(a_2 \ox \ldots \ox a_{n + 1})\\
	&\qquad + \sum_{i = 1}^{n}(-1)^{i} f(a_1 \ox \cdots \ox a_i a_{i + 1} \ox \cdots \ox a_{n+1})
		 	+ (-1)^{n + 1} f(a_1 \ox \cdots \ox a_{n}) \cdot a_{n + 1}
\end{align*}
para todo $f \in \Hom_{\field}\left(A^{\ox n}, M\right)$ y para todo $n \geq 1$.

A partir de la definición, es claro que si $A$ es proyectivo como $A^e$-módulo, resulta que $\Hy^{n}(A, M) = \Hy_{n}(A,M) = 0$ para todo $n \geq 0$
y para todo $A^e$-módulo $M$. Por lo tanto, es importante poder determinar cuando $A$ es $A^e$-proyectivo. Al igual
que en el capitulo \ref{hopf}, llamamos $m_A :A^e \to A$ a la multiplicación del álgebra.
\begin{prop}
	El álgebra $A$ es $A^e$-proyectiva si y solo si existe un elemento $e \in A^e$ tal que $m_A(e) = 1$ y
	$ae = ea$ para todo $a \in A$.
\end{prop}
\begin{proof}
	Supongamos primero que $A$ es un $A^e$-módulo proyectivo. Como el morfismo $m_A$ es sobreyectivo, existe un morfismo
	$\sigma: A \to A^e$ tal que $m_A \sigma = id_A$. Llamamos $e = \sigma(1)$, luego $m_A(e) = 1$ y
	$ae = a\sigma(1) = \sigma(a) = \sigma(1)a = ea$ para todo $a \in A$.
	
	Ahora supongamos que existe un elemento $e$ perteneciente a $A^e$ que cumple las condiciones del enunciado.
	Sea $\sigma: A \to A^e$ la función dada por $\sigma(a) = ae$. Debido a que $ae = ea$ para todo $a \in A$, resulta que
	$\sigma$ es un morfismo de $A^e$-módulos y como $m_A(e) = 1$, entonces $m_A \sigma = id_A$. Veamos que $A$ es
	$A^e$-proyectivo. Sean $M$ y $N$ dos $A^e$-bimódulos y sean $f:M \to N$ y $g: A \to M$ dos morfismos de $A^e$-módulos,
	con $f$ sobreyectivo. Como $A^e$ es $A^e$-proyectivo, existe un morfismo $h: A^e \to M$ tal que
	$fh = gm_A$ y por lo tanto $fh\sigma = g$.
\end{proof}

\begin{example}
	Sea $A = M_n(\field)$, el álgebra de matrices cuadradas de tamaño $n$ con coeficientes en un cuerpo $\field$ y 
	sea $e = \sum_{i = 1}^{n}E_{i1} \ox E_{1i} \in A^e$. Veamos que $e$ verifica las condiciones de la proposición
	anterior. Dado $B \in A$, resulta que
	\begin{align*}
		Be &= \sum_{i = 1}^{n} B E_{i1} \ox E_{1i} = \sum_{i = 1}^{n} \left(\sum_{j = 1}^{n} B_{ji} E_{j1}\right) \ox E_{1i}
			= \sum_{i = 1}^{n} \sum_{j = 1}^{n} E_{j1}  \ox B_{ji}  E_{1i}\\
		&= \sum_{j = 1}^{n} \sum_{i = 1}^{n} E_{j1}  \ox   E_{1i} B_{ji} = \sum_{j = 1}^{n} \sum_{i = 1}^{n} E_{j1}  \ox   E_{1i} B_{ji}
			= \sum_{j = 1}^{n} E_{j1}  \ox    \left(\sum_{i = 1}^{n}E_{1i} B_{ji}\right)\\
		&= \sum_{j = 1}^{n} E_{j1} \ox E_{1j} B = eB.
	\end{align*}
	Por otro lado, $m_A(e) = \sum_{i = 1}^{n}E_{i1} E_{1i} = Id_A$ y por lo tanto $\Hy^{m}( M_n(\field), M)$ y $\Hy_{m}( M_n(\field),M)$
	son nulos para todo $m \geq 1$ y para todo $ M_n(\field)$-módulo $M$.
\end{example}
\begin{example}
	Sea $A =\field G$, donde $G$ es un grupo finito tal que su cardinal es inversible en $\field$.
	El elemento $e = \frac{1}{|G|}\sum_{x\in G}x^{-1}\ox x \in A^e$ verifica las condiciones de la proposición anterior
	y por lo tanto $\Hy^{m}(\field G, M)$ y $\Hy_{m}( \field G,M)$ son nulos
	para todo $m \geq 1$ y para todo $\field G$-módulo $M$.
\end{example}

Describamos ahora los primeros dos espacios de cohomología.
\begin{itemize}
	\item $\Hy^{0}(A, M)$ es isomorfo $\Ker(d_0)$.
	Un elemento $m$ pertenece a $\Ker\left(d_0\right)$ si y solo si $a\cdot m - m \cdot a = 0$ para todo $a \in A$. En particular,
	si $M = A$, resulta que $\Hy^{0}(A, A)$ es el centro del álgebra.
	
	\item Para obtener una descripción de $\Hy^{1}(A, M)$ vamos a calcular por separado $\Ker(d_1)$
	e $\Ima(d_0)$. Dado $f \in \Hom_{\field}\left(A, M\right)$, $f$ pertenece a $\Ker(d_1)$ si y solo si
	\[
		0 = d_1(f)(a_1\ox a_2) = a_1\cdot f(a_2) - f(a_1 a_2) + f(a_1)\cdot a_2
	\]
	para todo $a_1, a_2 \in A$. Por lo tanto, $\Ker(d_1)$ es el subespacio de $\Hom_{\field}\left(A, M\right)$
	cuyos elementos son las \emph{derivaciones}, $\Der_{\field}(A, M)$. Por otro lado
	la imagen de $d_0$ es el conjunto de \emph{derivaciones interiores} $\Inn_{\field}(A, M)$, es decir, los morfismos de la forma
	$a \mapsto a\cdot m - m\cdot a$ para algún $m \in M$. Luego $\Hy^{1}(A, M)$ resulta ser isomorfo al conjunto de \emph{derivaciones exteriores}
	\[
		\Der_{\field}(A, M) / \Inn_{\field}(A, M).
	\]
	Notar que si $M = A$, entonces $\Der_{\field}(A) := \Der_{\field}(A,A)$ es un álgebra de Lie e $\Inn_{\field}(A)$ es un ideal de Lie.
	Por lo tanto $\Hy^{1}(A, A)$ es un álgebra de Lie.
\end{itemize}
\begin{example}
	Sea $Q$ un carcaj finito y sea $\field Q$ el álgebra de caminos.
	Sea $A = \field Q / J^2$, donde $J$ es el ideal generado por las flechas.
	Vamos a probar que si $\Hy^1(A, A)$ es igual	a cero, entonces $Q$ es un árbol. Sea $Q_0 = \lbrace e_1, \ldots, e_n\rbrace$
	el conjunto de vértices, sea $Q_1 = \lbrace \alpha_1, \ldots, \alpha_r \rbrace$ el conjunto de flechas
	y llamamos $G$ al grafo que resulta de ignorar el sentido de las flechas en $Q$. Dada una flecha $\alpha$, llamamos
	$G \setminus \alpha$ al grafo que resulta de eliminar la arista correspondiente.
	Si $Q$ no es un árbol, entonces existe una flecha $\alpha \in Q_1$ tal que $G \setminus \alpha$ es conexo.
	Supongamos que $\alpha = \alpha_1$. Sea $\delta : A\to A$ la derivación definida como
	\begin{align*}
		\delta(e_i) &= 0 \text{ para todo } i,\text{ }1\leq 1 \leq n,\\
		\delta(\alpha_1) &= \alpha_1,\\
		\delta(\alpha_i) &= 0 \text{ para todo } i,\text{ } 2 \leq i \leq r.
	\end{align*}
	Si vemos que $\delta$ no es una derivación interior obtendremos el resultado buscado
	ya que $\Hy^1(A, A)$ es isomorfo a $\Der_{\field}(A, M) / \Inn_{\field}(A, M)$.  Supongamos que
	$\delta$ es una derivación interior y sea $x \in A$
	tal que $\delta(a) = ax - xa$ para todo $a \in A$. Escribimos $x = \sum_{i = 1}^{n}\lambda_i e_i + \sum_{j = 1}^r\mu_j \alpha_j$,
	donde $\lambda_i$ y $\mu_j$ pertenecen a $\field$ para todo $i$ y para todo $j$. Llamamos, para todo $\alpha \in Q_1$,
	$s(\alpha)$ al vértice donde empieza $\alpha$ y $t(\alpha)$ al vértice donde termina. Como $\delta(a) = ax - xa$ para todo $a \in A$, resulta que
	\begin{align*}
		\alpha_1 &= \delta(\alpha_1) = \alpha_1 x - x\alpha_1 = \left(\lambda_{s(\alpha_1)} - \lambda_{t(\alpha_1)}\right)\alpha_1,\\
		0 &= \delta(\alpha_i) = \alpha_i x - x\alpha_i
			= \left(\lambda_{s(\alpha_i)} - \lambda_{t(\alpha_i)}\right)\alpha_i \text{ para todo } i, \text{ }2\leq i \leq r.
	\end{align*}
	Por lo tanto $\lambda_{s(\alpha_1)} - \lambda_{t(\alpha_1)} = 1$ y $\lambda_{s(\alpha_i)}
			- \lambda_{t(\alpha_i)} = 0$ para todo $i$, $2\leq i \leq r$. 
	Como $G \setminus \alpha_1$ es conexo, existe un camino en $G \setminus \alpha_1$ que conecta $s(\alpha_1)$ con $t(\alpha_1)$.
	Sean $\alpha_{i_1}, \alpha_{i_2}, \ldots, \alpha_{i_k}$ las flechas que componen este camino.
	\begin{align*}
		\xymatrix{
			s(\alpha_1) \ar[rd]^{\alpha_{i_1}} & & & \cdots \ar@{-->}[r]& \cdots\ar[dr]^{\alpha_{i_{k - 1}}} & & t(\alpha_1) \ar[ld]^{\alpha_{i_k}}\\
			& \cdot & \cdot \ar[l]^{\alpha_{i_2}} \ar[ru]^{\alpha_{i_3}}& & & \cdot &
		}
	\end{align*}
	Al recorrer el camino empezando en $s(\alpha_1)$ podemos asignarle a cada flecha un número, $1$ ó $-1$, según si la flecha apunta en el sentido del
	recorrido o no, respectivamente. Para cada $j$, llamamos $sg(\alpha_{i_j})$ a este número. Resulta que
	\[
		1 = 	\lambda_{s(\alpha_1)} - \lambda_{t(\alpha_1)}
			= \sum_{j = 1}^{k}sg(\alpha_{i_j})\left(\lambda_{s(\alpha_{i_j})} - \lambda_{t(\alpha_{i_j})}\right)
				=  \sum_{j = 1}^{k} 0  = 0,
	\]
	lo cual es absurdo. Esta contradicción proviene de suponer que $\delta$ es una derivación interior.
\end{example}

\begin{example}\label{example_x2}
	Sea $\field$ un cuerpo de característica distinta de $2$ y sea $A = \field[x]/(x^2)$. En este ejemplo vamos
	a calcular la homología y la cohomología de Hochschild de $A$ con coeficientes en $A$. Consideramos
	el siguiente complejo de $A^e$-módulos
	\begin{align*}
		\xymatrix{
		\cdots \ar[r] & A^e \ar[r]^{\delta_n} & A^e \ar[r] &\cdots \ar[r] & A^e \ar[r]^{\delta_2} & A^e \ar[r]^{\delta_1}
			& A^e \ar[r]^{\delta_0} & A^e \ar[r]^{m_A} & A \ar[r] & 0,
		}
	\end{align*}	
	donde $m_A$ es la multiplicación del álgebra y para todo $i \geq 0$, $\delta_{2i}$ es la multiplicación por $ x \ox 1 - 1\ox x$  y
	$\delta_{2i + 1}$ es la multiplicación por $1 \ox x + x \ox 1$. Veamos que el complejo es una resolución proyectiva de $A$ como $A^e$-módulo.
	El conjunto $\lbrace 1,x \rbrace$ es una base de $A$ como $\field$-espacio vectorial
	y por lo tanto $\lbrace 1\ox 1, 1 \ox x, x \ox 1, x \ox x\rbrace$ es una base de $A^e$ como $\field$-espacio vectorial. Resulta que
	\begin{align*}
		m_A\left( a \cdot 1\ox 1 + b \cdot 1 \ox x +  c \cdot x \ox 1 + d \cdot x \ox x\right) &= a \cdot 1 + (b + c) \cdot x,\\
		\delta_{2i}\left( a \cdot 1\ox 1 + b \cdot 1 \ox x +  c \cdot x \ox 1 + d \cdot x \ox x\right) &= a\cdot ( x \ox 1 - 1\ox x)
			+ (b - c)(x \ox x),\\
		\delta_{2i + 1}\left( a \cdot 1\ox 1 + b \cdot 1 \ox x +  c \cdot x \ox 1 + d \cdot x \ox x\right) &= a\cdot ( x \ox 1 + 1\ox x)
			+ (b + c)(x \ox x).
	\end{align*}
	A partir de este cálculo es sencillo verificar que el complejo es efectivamente una resolución.
	Empecemos con el cálculo de la homología de Hochschild. Aplicamos el funtor $A\ox_{A^e}-$ al complejo truncado
	y obtenemos el siguiente complejo
	\begin{align*}
		\xymatrix{
		\cdots \ar[r] & A\ox_{A^e}A^e \ar[r]^{id \ox \delta_2} & A\ox_{A^e}A^e \ar[r]^{id \ox \delta_1}
			& A\ox_{A^e}A^e \ar[r]^{id \ox \delta_0} & A\ox_{A^e}A^e \ar[r] & 0,
		}
	\end{align*}	
	que a su vez resulta ser isomorfo al complejo
	\begin{align*}
		\xymatrix{
			\cdots \ar[r] & A \ar[r]^{d_2} & A \ar[r]^{d_1}
			& A \ar[r]^{d_0} & A \ar[r] & 0,
		}
	\end{align*}
	donde los diferenciales están dados por $d_{2i}(a) = ax - xa = [a, x]$ y $d_{2i + 1}(a) = ax + xa$ para todo $i \geq 0$ y para
	todo $a \in A$. Es fácil verificar que $\Ker(d_{2i})$ es igual a $A$ y que $\Ima(d_{2i + 1}) = \Ker(d_{2i + 1})$ esta generado por $x$, luego
	se tienen los siguientes isomorfismos
	\begin{align*}
		\Hy_0(A,A) \cong A, \quad \Hy_n(A,A) \cong \field \text{ para todo } n \geq 1.
	\end{align*}	
	Continuemos con la cohomología de Hochschild. Aplicamos el funtor $\Hom_{A^e}(-,A)$ a la resolución proyectiva y obtenemos 
	el siguiente complejo
	\begin{align*}
		\xymatrix{
		\cdots & \Hom_{A^e}(A^e,A) \ar[l]_(.65){\delta_2^{\ast}} & \Hom_{A^e}(A^e,A) \ar[l]_{\delta_1^{\ast}}
			& \Hom_{A^e}(A^e,A) \ar[l]_{\delta_0^{\ast}} & 0 \ar[l],
		}
	\end{align*}	
	donde $\delta_i^{\ast}(f) = f \delta_i$ para todo $i \geq 0$ y para todo $f \in \Hom_{A^e}(A^e,A)$. A su vez, este nuevo
	complejo es isomorfo al siguiente complejo
	\begin{align*}
		\xymatrix{
		\cdots & A \ar[l]_{d^2} & A \ar[l]_{d^1}
			& A \ar[l]_{d^0} & 0 \ar[l],
		}
	\end{align*}	
	donde los diferenciales están dados por $d^{2i + 1} = d_{2i + 1}$ y $d^{2i} = -d_{2i} = 0$. Debido a que los diferenciales
	son los mismos que para la homología, se tienen los siguientes isomorfismos
	\begin{align*}
		\Hy^0(A,A) \cong A, \quad \Hy^n(A,A) \cong \field \text{ para todo } n \geq 1.
	\end{align*}	
	Es importante aclarar que si bien los espacios $\Hy^{2i}(A,A)$ y $\Hy^{2i + 1}(A,A)$ son isomorfos a $\field$ para todo $i \geq 1$,
	el primero está generado por el morfismo $1 \ox 1 \mapsto 1$, mientras que el segundo está generado por el morfismo $1\ox 1 \mapsto x$.
\end{example}

\section{Estructura de álgebra de Gerstenhaber}\label{hochschild_gerstenhaber}

Sea $A$ una $\field$-álgebra. La cohomología de Hochschild de $A$ con coeficientes en $A$ tiene estructura
de álgebra conmutativa graduada y de álgebra de Lie graduada junto con cierta relación de compatibilidad.

\begin{definition}
	Un \emph{álgebra de Gerstenhaber} es un álgebra conmutativa graduada $G^{\bullet} = \oplus_{i}G^i$ junto con
	un corchete $\left[\cdot, \cdot \right]: G^{p} \times G^{q} \to G^{p + q - 1}$
	tal que $\left(G^{\bullet}, \left[\cdot, \cdot \right]\right)$ es una álgebra de Lie graduada y para todo $a \in G^{\bullet}$,
	el morfismo $b \in G^{\bullet} \mapsto \left[a,b\right] \in G^{\bullet}$
	es una derivación graduada con respecto al producto, es decir,
	\[
		\left[a, b c\right] = \left[a,b\right]c + (-1)^{(\gr(a) - 1)\gr(b)}b\left[a, c\right]
	\]
	para todo $a, b, c \in G^{\bullet}$.
\end{definition}

Empecemos dándole estructura de álgebra graduada a $\oplus_{n \geq 0}\Hom_{\field}\left(A^{\ox n}, A\right)$.
\begin{definition}
	Sean $f \in \Hom_{\field}\left(A^{\ox n}, A\right)$ y $g \in \Hom_{\field}\left(A^{\ox m}\right)$, se define el \emph{producto cup}
	$f \smile g \in \Hom_{\field}\left(A^{\ox n + m}, A\right)$ como la única transformación $\field$-lineal que cumple que
	\[
		f\smile g(a_1\ox \ldots \ox a_n \ox a_{n + 1} \ox \ldots \ox a_{n + m}) = f(a_1\ox \ldots \ox a_n)g(a_{n + 1} \ox \ldots \ox a_{n + m})
	\]
	para todo $a_1, \ldots, a_{n + m} \in A$.
\end{definition}
Los diferenciales del complejo resultan ser derivaciones graduadas con respecto a este producto, es decir, si $f$ y $g$ son dos
elementos homogéneos de grados $n$ y $m$ respectivamente, entonces
\[
	d_{n + m}(f \smile g) = d_{n}(f) \smile g + (-1)^{nm}f \smile d_{m}(g).
\]
Esta fórmula muestra que el producto de dos cociclos es un cociclo y que el producto de un coborde con un cociclo
es un coborde. Por lo tanto $\Hy^{\bullet}(A,A)$ resulta ser una $\field$-álgebra graduada con el producto inducido por el producto cup.

También existe en $\oplus_{n \geq 0}\Hom_{\field}\left(A^{\ox n}, A\right)$ una estructura de álgebra de Lie graduada.
\begin{definition}
Sean $f \in \Hom_{\field}\left(A^{\ox n}, A\right)$ y $g \in \Hom_{\field}\left(A^{\ox m}\right)$, se define el \emph{asociador}
entre $f$ y $g$, $f \circ g \in \Hom_{\field}\left(A^{\ox n + m - 1}, A\right)$ como la única transformación $\field$-lineal que cumple que
	\begin{align*}
		&f \circ g (a_1\ox \ldots \ox a_{n + m - 1 }) = \\
		&\quad\sum_{i = 1}^n (-1)^{(i - 1)(m -1)}
				f\left(a_1\ox \ldots \ox a_{i - 1} \ox g(a_{i} \ox \ldots \ox a_{i + m - 1}) \ox a_{i + m} \ox \ldots \ox a_{n + m - 1}\right)
	\end{align*}
	para todo $a_1, \ldots, a_{n + m - 1} \in A$.
Se define el \emph{corchete de Gerstenhaber} entre estos dos elementos como
\[
	\left[f, g\right] = f \circ g  - (-1)^{(n -1)(m -1)}g \circ f.
\] 
\end{definition}

El producto cup y el corchete de Gerstenhaber hacen de $\Hy^{\bullet}(A, A)$ un álgebra de Gerstenhaber.

\begin{example}
	Vamos a calcular la estructura de álgebra de Gerstenhaber de la cohomología de Hochschild del álgebra $A = \field[x]/(x^2)$, donde
	$\field$ es un cuerpo de característica distinta de dos. Si bien para calcular la cohomología podemos usar cualquier resolución proyectiva,
	el producto cup y el corchete de Gerstenhaber están definidos a partir de la resolución bar. Para transportar la estructura de álgebra
	de Gerstenhaber de la cohomología calculada a partir de la resolución bar a la cohomología calculada en el Ejemplo \ref{example_x2}
	es necesario encontrar morfismos de comparación $f$ y $g$ entre las dos resoluciones:
	\begin{align*}
		\xymatrix{
			\cdots \ar[r] &  A^e \ar[r]^{\delta_2} \ar[d]^{f_3} & A^e \ar[r]^{\delta_1} \ar[d]^{f_2}
				& A^e \ar[r]^{\delta_0} \ar[d]^{f_1} & A^e \ar[r]^{m_A} \ar@{=}[d]^{id_{A\ox A}} & A \ar[r] \ar@{=}[d]^{id_A} & 0\\
			\cdots \ar[r] & A \ox A^{\ox 3} \ox A \ar[r]^{b_2'} \ar@<1ex>[u]^{g_3} & A \ox A^{\ox 2} \ox A \ar[r]^{b_1'} \ar@<1ex>[u]^{g_2}
				& A \ox A \ox A \ar[r]^{b_0'} \ar@<1ex>[u]^{g_1} & A^e \ar[r]^{m_A} & A \ar[r] & 0\\
		}
	\end{align*}
	Es sencillo verificar que la sucesión de morfismos de $A^e$-módulos $\lbrace f_{n}: A^e \to A\ox A^{\ox n} \ox A\rbrace_{n \geq 0}$ definida como
	$f_0 = id_{A\ox A}$ y $f_n(1\ox 1) = 1 \ox x^{\ox n} \ox 1$ para todo $n \geq 1$ resulta ser un morfismo de complejos
	y un levantado de la identidad. Para todo $n \geq 0$, definimos el morfismo de $A^e$-módulos $g_n: A \ox A^{\ox n} \ox A \to A^e$ como
	\begin{align*}
		&g_n(1\ox x^{\ox n}\ox 1) = 1\ox 1,\\
		&g_n(1\ox a_1 \ox \ldots \ox a_{i - 1} \ox 1 \ox a_{i + 1}\ox \ldots \ox a_n) = 0
	\end{align*}
	para todo $i$, $1 \leq i \leq n$ y para todo $a_1, \ldots a_{i - 1}, a_{i + 1}, \ldots a_n \in A$.
 	La sucesión $\lbrace g_n \rbrace_{n \geq 0}$ resulta ser un levantado de la identidad. Para encontrar estos morfismos
	nos basamos en \cite{GGRSV}.
	
	Ya estamos en condiciones de calcular el product cup. Debido a que le producto es bilineal, alcanza con calcular
	los productos entre los elementos de la base de $\Hy^{\bullet}(A,A)$. El espacio $\Hy^{0}(A,A)$ tiene como
	base al conjunto formado por el morfismo $\alpha_0: 1\ox 1 \mapsto 1$ y el morfismo $\alpha_1: 1\ox 1 \mapsto x$.
	Denotamos por
	\[
		\beta^{2n} : 1\ox 1 \mapsto 1 \in \Hy^{2n}(A, A) \text{ y } \gamma^{2m + 1} : 1\ox 1 \mapsto x\in \Hy^{2m + 1}(A, A)
	\]
	a los generadores
	de sus respectivos espacios de cohomología para todo $n \geq 1$ y para todo $m \geq 0$. Por definición,
	si $\varphi \in \Hy^{i}(A,A)$ y $\psi \in \Hy^{j}(A,A)$, entonces $\varphi \smile \psi \in \Hy^{i + j}(A,A)$. Por lo tanto,
	para calcular $\varphi \smile \psi$ necesitamos saber que valor toma en el elemento $1\ox 1$. Resulta que
	\begin{align*}
		\varphi \smile \psi(1 \ox 1) &= \left(\varphi g_i \smile \psi g_j\right) f_{i + j}(1 \ox 1)
			=  \left(\varphi g_i \smile \psi g_j\right)\left(1\ox x^{\ox i + j} \ox 1\right)\\
		&= \varphi g_i\left(1\ox x^{\ox i} \ox 1\right)\psi g_j\left(1\ox x^{\ox j} \ox 1\right) = \varphi(1 \ox 1)\psi(1 \ox 1)
	\end{align*}
	y por lo tanto es sencillo verificar las siguientes igualdades:
	\begin{center}
	\begin{tabular}{ c c }
 		$\alpha_0 \smile \alpha_0 = \alpha_0$ & $\alpha_1 \smile \beta^{2i} = 0$ \\ 
 		$\alpha_0 \smile \alpha_1 = \alpha_1$ & $\alpha_1 \smile \gamma^{2j + 1} = 0$\\  
 		$\alpha_0 \smile \beta^{2i} = \beta^{2i}$ & $\beta^{2i} \smile \beta^{2j} = \beta^{2(i + j)}$ \\
 		$\alpha_0 \smile \gamma^{2j + 1} = \gamma^{2j + 1}$ & $\beta^{2i} \smile \gamma^{2j + 1} = \gamma^{2(i + j) + 1}$ \\
 		$\alpha_1 \smile \alpha_1 = 0$ & $\gamma^{2i + 1} \smile \gamma^{2j + 1} = 0$ \\  
	\end{tabular}
	\end{center}
	para todo $i$ y para todo $j$.
	
	Por último vamos a calcular la estructura de álgebra de Lie graduada. Al igual que para el producto, solo es necesario
	conocer los valores que toma el corchete de Gerstenhaber entre los elementos de la base. Empecemos calculando los asociadores.
	Si $\varphi \in \Hy^{m}(A,A)$ y $\psi \in \Hy^{n}(A,A)$, resulta que
	\begin{align*}
		\varphi \circ \psi(1\ox 1) &= \left(\varphi g_m \circ \psi g_n\right) f_{m + n - 1}(1\ox 1)
			= \left(\varphi g_m \circ \psi g_n\right) \left(1\ox x^{n + m - 1} \ox 1\right)\\
		&= \sum_{i = 1}^{m}(-1)^{(i -1)(n - 1)}
			\varphi g_m\left(1\ox x^{\ox i - 1}\ox \psi g_n\left(1 \ox x^{\ox n} \ox 1\right)\ox x^{\ox m -i}\ox 1\right)\\
		&= \sum_{i = 1}^{m}(-1)^{(i -1)(n - 1)}\varphi g_m\left(1\ox x^{\ox i - 1}\ox \psi (1\ox 1)\ox x^{\ox m -i}\ox 1\right).
	\end{align*}
	Separamos el calculo del asociador en casos.
	\begin{itemize}
		\item Si $n$ es par,
		\begin{align*}
			\varphi \circ \psi(1\ox 1) &=
				\sum_{i = 1}^{m}(-1)^{(i -1)}\varphi g_m\left(1\ox x^{\ox i - 1}\ox \psi (1\ox 1)\ox x^{\ox m -i}\ox 1\right),
		\end{align*}
		\begin{itemize}
			\item si $m$ es par,
				\begin{align*}
					\varphi \circ \psi(1\ox 1) = 0,
				\end{align*}
			\item si $m$ es impar y $\psi(1\ox 1) = 1$,
				\begin{align*}
					\varphi \circ \psi(1\ox 1) = 0,
				\end{align*}
			\item si $m$ es impar y $\psi(1\ox 1) = x$,
				\begin{align*}
					\varphi \circ \psi(1\ox 1) = \varphi(1 \ox 1),
				\end{align*}
		\end{itemize}
	\item si $n$ es impar,
	\begin{align*}
		\varphi \circ \psi(1\ox 1) &=
			\sum_{i = 1}^{m} \varphi g_m\left(1\ox x^{\ox m}\ox 1\right) = m\varphi(1\ox 1).
	\end{align*}
	\end{itemize}
	A partir de este cálculo es fácil verificar las siguientes igualdades:
	\begin{center}
	\begin{tabular}{ c c }
 		$\left[\alpha_0, \alpha_0\right] = 0$ & $\left[\alpha_1, \beta^{2i}\right] = 0$ \\ 
 		$\left[\alpha_0, \alpha_1\right] = 0$ & $\left[\alpha_1, \gamma^{2j + 1}\right] = 0$\\  
 		$\left[\alpha_0, \beta^{2i}\right] = 0$ & $\left[\beta^{2i}, \beta^{2j}\right] = 0$ \\
 		$\left[\alpha_0, \gamma^{2j + 1}\right] = 0$ & $\left[\beta^{2i}, \gamma^{2j + 1}\right] = 2i\beta^{2(i + j)}$ \\
 		$\left[\alpha_1, \alpha_1\right] = 0$ & $\left[\gamma^{2i + 1}, \gamma^{2j + 1}\right] = 2(i - j)\gamma^{2(i + j) + 1}$ \\  
	\end{tabular}
	\end{center}
	para todo $i$ y para todo $j$.
\end{example}

\section{Sucesiones espectrales}\label{sucesiones_espectrales}

En esta sección resumiremos algunos resultados sobre sucesiones espectrales que provienen de una filtración de un complejo.

Empecemos considerando el problema de calcular la homología del complejo total $T_{\bullet}$ de un complejo doble
$E_{\bullet, \bullet}$ de espacios vectoriales, cuyas únicas columnas no necesariamente nulas son  $E_{0, \bullet}$ y $E_{1, \bullet}$.
\begin{align*}
    \xymatrix@=3em{
        &\ar@{-->}[d]& \ar@{-->}[d]& \ar@{-->}[d]& \ar@{-->}[d]& \\
        & 0 \ar@{-->}[l] \ar[d]& E_{0, q + 2} \ar[l]_{d^{h}_{0, q + 2}} \ar[d]_{d^{v}_{0, q + 2}}
            & E_{1,q + 2} \ar[l]_{d^{h}_{1, q + 2}} \ar[d]_{d^{v}_{1, q + 2}} & 0 \ar[l]_{d^{h}_{2, q + 2}} \ar[d] & \ar@{-->}[l]\\
        & 0 \ar@{-->}[l] \ar[d]& E_{0, q + 1} \ar[l]_{d^{h}_{0, q + 1}} \ar[d]_{d^{v}_{0, q + 1}}
            & E_{1,q + 1} \ar[l]_{d^{h}_{1, q + 1}} \ar[d]_{d^{v}_{1, q + 1}} & 0 \ar[l]_{d^{h}_{2, q + 1}} \ar[d]& \ar@{-->}[l]\\
        & 0 \ar@{-->}[l] \ar[d]& E_{0, q} \ar[l]_{d^{h}_{0, q}} \ar[d]_{d^{v}_{0, q}}
            & E_{1,q} \ar[l]_{d^{h}_{1, q}} \ar[d]_{d^{v}_{1, q}} & 0 \ar[l]_{d^{h}_{2, q}} \ar[d] &\ar@{-->}[l]\\
        & 0  \ar@{-->}[l] \ar@{-->}[d] & E_{0, q - 1} \ar[l]_{d^{h}_{0, q - 1}} \ar@{-->}[d]
            & E_{1, q- 1} \ar[l]_{d^{h}_{1 , q -1}} \ar@{-->}[d] & 0 \ar[l]_{d^{h}_{2, q - 1}} \ar@{-->}[d] &  \ar@{-->}[l]\\
        &&&&&
    }
\end{align*}

Como primer paso, calculamos la homología vertical $\Hy_{\bullet}(E_{p, \bullet})$ para todo $p$. Denotamos por $E_{p, q}^{1}$
al espacio vectorial $\Hy_{q}\left(E_{p, \bullet}\right)$ y lo ubicamos en el punto $(p, q)$  de un nuevo diagrama junto con los diferenciales horizontales inducidos. 

\begin{align*}
    \xymatrix@=3em{
        & 0 \ar@{-->}[l] & E^1_{0, q + 2} \ar[l]_{d^{h}_{0, q + 2}}
            & E^1_{1,q + 2} \ar[l]_{d^{h}_{1, q + 2}} & 0 \ar[l]_{d^{h}_{2, q + 2}}  & \ar@{-->}[l]\\
        & 0 \ar@{-->}[l] & E^1_{0, q + 1} \ar[l]_{d^{h}_{0, q + 1}} 
            & E^1_{1,q + 1} \ar[l]_{d^{h}_{1, q + 1}} & 0 \ar[l]_{d^{h}_{2, q + 1}} & \ar@{-->}[l]\\
        & 0 \ar@{-->}[l] & E^1_{0, q} \ar[l]_{d^{h}_{0, q}} 
            & E^1_{1,q} \ar[l]_{d^{h}_{1, q}}  & 0 \ar[l]_{d^{h}_{2, q}}  &\ar@{-->}[l]\\
        & 0  \ar@{-->}[l]  & E^{1}_{0, q - 1} \ar[l]_{d^{h}_{0, q - 1}}  & E^{1}_{1, q - 1} \ar[l]_{d^{h}_{1, q - 1}} & 0 \ar[l]_{d^{h}_{2, q - 1}}  &  \ar@{-->}[l]
    }
\end{align*}

Denotamos por $E_{p, q}^2$ al espacio de homología horizontal $\Hy_{p}\left(E_{\bullet, q}^{1}\right)$. El objetivo
es encontrar, para cada $q$, la relación entre los espacios $E_{0, q + 1}^{2}$,  $E_{1, q}^{2}$ y $H_{q + 1}(T_{\bullet})$.
Como solo las columnas $E_{0, \bullet}$ y $E_{1, \bullet}$ son no necesariamente nulas, escribimos $(a, b) \in E_{0, q + 1} \times E_{1, q}$ cuando nos referimos
a un elemento de $T_{q + 1}$ y llamamos $d_{\bullet}$ a los diferenciales del complejo total.
Si $a$ pertenece a $\Ker\left(d_{0,q + 1}^{v}\right)$, resulta que
\[
    d_{q + 1}(a, 0) = \left(d_{0, q + 1}^{v}(a) + d_{1, q}^{h}(0), d_{1, q}^{v}(0)\right) = (0, 0).
\]
Esta observación permite definir el morfismo
\[
    \hat{\iota}: a \in \Ker\left(d_{0,q + 1}^{v}\right) \mapsto  [(a, 0)] \in \Hy_{q + 1}(T).
\]
Sea $a \in \Ker\left(d_{0,q + 1}^{v}\right)$. Si $a = d_{0, q + 2}^{v}(b)$ para algún $b \in E_{0, q + 2}$, entonces $(a, 0) = d_{q + 2}(b, 0)$,
y si $a = d_{1, q + 1}^{h}(c)$ para algún $c \in \Ker\left(d_{1, q + 1}^{v}\right)$, resulta que $(a, 0) = d_{q + 2}(0, c)$. Luego 
el morfismo $\hat{\iota}$ pasa al cociente y obtenemos un nuevo morfismo inyectivo
\[
    \iota: E_{0, q + 1}^{2} \to H_{q + 1}(T).
\]
Por otro lado, el elemento $(a, b)$ pertenece a $\Ker\left(d_{q + 1}\right)$ si y solo si $b$ pertenece a $\Ker\left(d_{1, q}^{v}\right)$
y $d_{0, q + 1}^{v}(a) = - d_{1, q}^{h}(b)$. Esto permite definir un morfismo
\[
    \pi: [(a, b)] \in \Hy_{q + 1}(T)  \mapsto  [b] \in E_{1, q}^{2}.
\]
Es claro que $\Ima(\iota)$ esta incluida en $\Ker(\pi)$. Sea $[(a, b)] \in \Ker(\pi)$, queremos encontrar un elemento $a'$ tal que
$(a' - a, b)$ pertenezca a $\Ima\left(d_{q + 2}\right)$. Como $\pi([(a, b)]) = 0$, existe $c \in E_{1, q + 1}$ tal que $d_{1, q + 1}^{v}(c) = b$,
y como $(a, b)$ pertenece a $\Ker(d_{q + 1})$, se cumple que
\[
    d_{0, q + 1}^{v}(a) = -d_{1, q}^{h}(b) = -d_{1, q}^{h} d_{1, q + 1}^{v}(c) = d_{0, q + 1}^{v} d_{1, q+ 1}^{h}(c).
\]
Por lo tanto $d_{0, q + 1}^{v} \left(a - d_{1, q+ 1}^{h}(c)\right) = 0$ y si llamamos $a' = a - d_{1, q+ 1}^{h}(c)$, obtenemos lo que buscábamos.
Acabamos de probar que hay una sucesión exacta corta
\begin{align*}
\xymatrix{
    0 \ar[r] & E_{0, q + 1}^{2} \ar[r]^{\iota} & H_{q + 1}(T) \ar[r]^{\pi} & E_{1, q }^{2} \ar[r] & 0,
}
\end{align*}
que relaciona $E_{0, q + 1}^{2}$ y $E_{1, q}^{2}$ con $H_{q + 1}(T)$. Como estamos trabajando con espacios vectoriales, $\pi$ es una retracción y
resulta que
\[
    H_{q + 1}(T) \cong E_{0, q + 1}^{2} \oplus E_{1, q}^{2}.
\]

\begin{definition}
    Una \emph{sucesión espectral de homología (que empieza en s)} en una categoría abeliana $\mathfrak{A}$ consiste de los siguientes objetos:
    \begin{itemize}
        \item Una familia $\left\lbrace E_{p, q}^{r} \right\rbrace$ de objetos de $\mathfrak{A}$ definidos para todo $p, q, r\in \ZZ$, $r\geq s$.
        \item Una familia de morfismos  $\left\lbrace d_{p, q}^{r}: E_{p, q}^{r} \to E_{p - r, q + r - 1}^{r} \right\rbrace$ que resultan
        ser diferenciales, es decir $d_{p, q}^{r} \circ d_{p + r, q - r + 1}^{r} = 0$, de manera que la recta de pendiente $-(r + 1)/r$ en el reticulado $E_{\bullet, \bullet}^{r}$ resulte ser un complejo.
        \item Isomorfismos entre $E_{p, q}^{r + 1}$ y la homología de $E^{r}_{\bullet, \bullet}$ en el punto $E_{p, q}^{r}$:
        \[
           E_{p, q}^{r + 1} \cong \Ker\left(d^{r}_{p, q}\right) / \Ima\left(d^{r}_{p + r, q - r + 1}\right).
        \]
    \end{itemize}
    El \emph{grado total} de $E_{p, q}^{r}$ es $n = p + q$. Los términos de grado total $n$ yacen en una recta de pendiente $-1$ y cada
    diferencial hace decrecer el grado total por uno. Llamamos \emph{páginas} a cada reticulado $E_{\bullet, \bullet}^{r}$ para todo $r\geq s$.
\end{definition}

Una sucesión espectral de homología ubicada en \emph{el primer cuadrante}  es una sucesión espectral tal que $E_{p, q}^{r} = 0$ a menos que
$p \geq 0$ y $q \geq 0$. Si fijamos $p$ y $q$, entonces $E_{p, q}^{r} = E_{p, q}^{r + 1}$ para todo $r$ suficientemente grande. Esto se debe
a que eventualmente el dominio de $d^{r}_{p + r, q - r + 1}$ va a estar ubicado en el cuarto cuadrante y el codominio de $d^{r}_{p, q}$ en el segundo cuadrante. Denotamos por $E_{p, q}^{\infty}$ a este valor estable de $E_{p, q}^{r}$.

\begin{definition}
    Una \emph{sucesión espectral de cohomología (que arranca en s)} en $\mathfrak{A}$ es una familia $\left\lbrace E^{p, q}_{r} \right\rbrace$
    de objetos de $\mathfrak{A}$ junto con morfismos $\left\lbrace d^{p, q}_{r}: E^{p, q}_{r} \to E^{p + r, q - r + 1}_{r} \right\rbrace$
    que son diferenciales, es decir $d_r \circ d_r = 0$, e isomorfismos entre $E_{r + 1}$ y la homología de $E_r$.
\end{definition}

\begin{definition}
    Una sucesión espectral de homología  se dice \emph{acotada} si para todo $n$ existe una cantidad finita del términos no nulos de grado total
    $n$ en $E^{s}_{\bullet,\bullet}$. Recordar que $E^{s}_{\bullet,\bullet}$ es la primera página. Si este es el caso, para cada $p, q$ existe 
    un $r_0$ tal que $E_{p, q}^{r} = E_{p, q}^{r_0}$ para todo $r \geq r_0$.
    Denotamos por $E_{p, q}^{\infty}$ a este valor estable de $E_{p, q}^{r}$. Decimos que una sucesión espectral acotada \emph{converge}
    a $\Hy_{\bullet}$ si existe una familia de objetos $\left\lbrace \Hy_{n} \right\rbrace_{n \in \ZZ}$ en $\mathfrak{A}$, cada uno con una
    filtración finita
    \[
        0 = F_s \Hy_n \subseteq \cdots \subseteq F_{p - 1}\Hy_n \subseteq
            F_{p}\Hy_n \subseteq F_{p + 1}\Hy_n \subseteq \cdots \subseteq F_t \Hy_n = 0
    \]
    tal que $E_{p, q}^{\infty} \cong F_{p}\Hy_{p + q} / F_{p - 1} \Hy_{p + q}$. Si la sucesión espectral $E$ converge a $\Hy_{\bullet}$
    escribimos $E_{p, q}^{s} \Rightarrow H_{p + q}$.
    
    Una sucesión espectral de cohomología  se dice \emph{acotada} si para todo $n$ existe una cantidad finita del terminos no nulos de grado total
    $n$ en $E_{s}^{\bullet,\bullet}$. Decimos que una sucesión espectral acotada converge a $\Hy^{\bullet}$ si existe una familia de objetos 
    $\left\lbrace \Hy^{n} \right\rbrace_{n \in \ZZ}$ en $\mathfrak{A}$, cada uno con una
    filtración finita
    \[
        0 = F^t \Hy^n \subseteq \cdots \subseteq F^{p + 1}\Hy^n \subseteq
            F^{p}\Hy^n \subseteq F^{p - 1}\Hy^n \subseteq \cdots \subseteq F^s \Hy^n = 0
    \]
    tal que $E^{p, q}_{\infty} \cong F^{p}\Hy^{p + q} / F^{p + 1} \Hy^{p + q}$.
\end{definition}

$ $\\
Una \emph{filtración} $F$ de un complejo $C$ es una familia ordenada de subcomplejos de  $C$
\[
        \cdots \subseteq F_{p - 1}C \subseteq F_{p}C \subseteq \cdots .
\]
Como en los cálculos que se harán en esta
tesis las sucesiones espectrales se estabilizarán en la segunda página, al igual que en el ejemplo introductorio de esta sección,
omitimos la demostración del siguiente teorema. Ver el capitulo 5 de \cite{We}.

\begin{Teorema} Una filtración $F$ de un complejo $C$ determina una sucesión espectral que empieza con
$E_{p, q}^{0} = F_{p} C_{p + q} / F_{p - 1}C_{p + q}$ y continúa con $E_{p, q}^{1} = \Hy_{p + q}(E_{p, \bullet}^{0})$.
\end{Teorema}

Una filtración de un complejo $C$ induce una filtración en la homología de $C$, el espacio $F_p \Hy_{n}(C)$ es la imagen
del morfismo $\Hy_{n}(F_p C) \to \Hy_{n}(C)$.

\begin{Teorema}
    Si $C$ un complejo provisto de una filtración $F$ acotada, entonces la sucesión espectral asociada es acotada y converge a $\Hy_{\bullet}(C)$:
    \[
        E_{p, q}^{1} = H_{p + q}(F_p C / F_{p - 1}(C)) \Rightarrow H_{p + q}(C).
    \]
\end{Teorema}

Hay dos filtraciones usuales asociadas a cualquier complejo doble $C$ que resultan en sucesiones espectrales relacionadas con la homología
del complejo total $\Tot(C)$. En esta tesis vamos a hacer uso de una de estas.

\begin{definition} (Filtración por columnas) Si $C$ es un complejo doble, se puede filtrar el complejo total $\Tot(C)$ de la siguiente manera. Sea $\prescript{1}{}{F_{n}}\Tot(C)$ el complejo total del subcomplejo doble de $C$
\[
	\left(\prescript{}{\tau \leq n}{C}\right)_{p, q} = \left\{\begin{array}{cc}
    C_{p, q} &\text{ si } p \leq n,\\
    &\\
    0 & \text{ si no}. \\
	\end{array}\right.
\]
Esta filtración da lugar a una sucesión espectral $\left\lbrace E_{p, q}^{r}\right\rbrace$ que empieza con $E_{p, q}^{0} = C_{p, q}$. Los morfismos
$d^{0}$ son los morfismos verticales $d^{v}$ de $C$ y por lo tanto $E^{1}_{p, q} = \Hy_{q}^{v}(C_{p, \bullet})$. Los morfismos $d^{1}$
son los morfismos horizontales $d^{h}$ inducidos en al homología. Si $C$ es un complejo doble ubicado en el primer cuadrante
la filtración es acotada y por lo tanto la sucesión espectral es convergente: $E_{p, q}^{0} \Rightarrow H_{p + q}(\Tot(C))$. 
\end{definition}

La otra sucesión espectral usual es la que surge de filtrar por filas y se define de manera análoga. 

\end{document}








