\documentclass[a4paper,oneside,fleqn,11pt,../tesis.tex]{subfiles}

\begin{document}

\section{Definiciones básicas}
Esta sección se basa en \myworries{Weibel} y en \myworries{Ginzburg}.

Antes de definir homología y cohomología de Hochschild es necesario hacer la siguiente observación.
Sea $A$ una $\field$-álgebra, llamamos \emph{álgebra envolvente de $A$} a la $\field$-álgebra
\[
	A^{e} = A \ox A^{op}.
\]
Debido a que el producto en $A^{op}$ está invertido es lo mismo considerar un $A$-módulo a derecha que un $A^{op}$-módulo a izquierda.
Por lo tanto un $A-A$ bimódulo $M$ es lo mismo que un $A^{e}$-módulo a izquierda, mediante la acción dada por $(a\ox b)\cdot m = a\cdot m \cdot b$
para todo $a, b \in A$ y para todo $m \in M$ y también es lo mismo que un $A^{e}$-módulo a derecha, mediante la acción dada por $m \cdot (a \ox b) = b\cdot m \cdot a$.
En particular, esto permite ver a la categoría de $A-A$ bimódulos como la categoría de $A^e$-módulos a izquierda.
Esta observación es fundamental para la siguiente definición ya que nos dice que la categoría de $A-A$ bimódulos
tiene suficientes proyectivos.

\begin{definition}
	Sea $A$ una $\field$-álgebra y $M$ un $A-A$ bimódulo. Se define la \emph{homología de Hochschild de $A$ con coeficientes en $M$} como
		\[
			\Hy_{\bullet}(A,M) = \Tor_{\bullet}^{A^e}(A, M)	
		\]
	y se define la \emph{cohomología de Hochschild de $A$ con coeficientes en $M$} como
	\[
			\Hy^{\bullet}(A,M) = \Ext^{\bullet}_{A^e}(A, M).	
	\]
\end{definition}

Para calcular los grupos de homología y cohomología de Hochschild es necesario elegir alguna resolución de $A$ como $A-A$ bimódulo,
o lo que es lo mismo, como $A^e$-módulos a izquierda. Sabemos que siempre existe una resolución porque la categoría tiene suficientes
proyectivos. La resolución que fue usada en la definición original de homología y cohomología de Hochschild es la \emph{resolución bar}.
\begin{definition}
	Sea $A$ una $\field$ álgebra. Se define el \emph{complejo bar de $A$ de $A-A$ bimódulos}
	\begin{align*}
		\xymatrix{
			B_{\bullet}A:\ldots \ar[r]^{b'_3} &  A \ox A^{\ox 3} A \ar[r]^{b'_2} & A \ox A^{\ox 2} A \ar[r]^{b'_1}
				& A \ox A \ox A \ar[r]^{b'_0} & A \ox A \ar[r] & 0 
		}
	\end{align*}
	donde, para cada $n \geq 0$, el diferencial $b_n: A \ox A^{\ox (n + 1)} \ox A \to A \ox A^{\ox n} \ox A$ está dado por
	\begin{align*}
		b'_n&(1\ox a_0 \ox \cdots \ox a_n \ox 1) = a_0 \ox \cdots \ox a_n \ox 1 \\
		 & \qquad + \sum_{i = 0}^{n -1}(-1)^{i + 1} \ox a_0 \ox \cdots \ox a_i a_{i + 1} \ox \cdots \ox a_n \ox 1 + (-1)^{n + 1} \ox \cdots \ox a_n.
	\end{align*}
\end{definition}
El complejo bar juntos con la multiplicación $m_A: A \ox A \to A$ es una resolución libre de $A$ como $A-A$ bimódulo. Una demostración de este hecho se puede encontrar en \myworries{[CE]}.

Para calcular los espacios $\Hy_i(A,M)$ aplicamos el funtor $M\ox_{A^e}-$ al complejo bar y obtenemos
\begin{align*}
	\xymatrix{
		M \ox_{A^e} B_{\bullet}A:\ldots \ar[r]^{id\ox b'_{2}} & M\ox_{A^e}A^{\ox 4} \ar[r]^{id\ox b'_{1}} & M \ox_{A^e}A^{\ox 3} \ar[r]^{id\ox b'_{0}} & M\ox_{A^e}A^{\ox 2} \ar[r] & 0.
	}
\end{align*}
Es posible realizar la siguiente simplificación. Sea
\[
	m\ox_{A^e}(a_0 \ox \cdots \ox a_n) \in  M\ox_{A^e}A^{\ox n}
\]
y escribamos
\begin{align*}
	m\ox_{A^e}(a_0 \ox \cdots \ox a_n) &= m\ox_{A^e}(a_0 \ox a_{n}^{op}) \cdot (1 \ox a_1 \cdots \ox a_{n -1} \ox 1) \\
		&=m \cdot(a_0 \ox a_{n}^{op})\ox_{A^e}(1 \ox a_1 \cdots \ox a_{n -1} \ox 1) \\
		&= \left( a_n \cdot m \cdot a_0\right) \ox_{A^e}(1 \ox a_1 \cdots \ox a_{n -1} \ox 1).
\end{align*}
Esta igualdad sugiere el siguiente isomorfismo $\field$-lineal
\[
	m\ox_{A^e}(a_0 \ox \cdots \ox a_n) \in  M\ox_{A^e}A^{\ox n} \mapsto
		\left( a_n \cdot m \cdot a_0\right) \ox_{A^e}(a_1 \cdots \ox a_{n -1}) \in M\ox_{A^e}A^{\ox n - 2}
\]
que identifica $ M\ox_{A^e}A^{\ox n}$ con $M\ox_{A^e}A^{\ox n - 2}$ e induce el siguiente complejo isomorfo
\begin{align*}
	\xymatrix{
		\ldots \ar[r]^{b_3} & M\ox A^{\ox 3} \ar[r]^{b_2}  & M\ox A^{\ox 2} \ar[r]^{b_1} & M \ox A \ar[r]^{b_0} & M \ar[r] & 0.
	}
\end{align*}
donde los diferenciales están dados por
\begin{align*}
	&b_0(m\ox a) = m\cdot a - a\cdot m,\\
	&b_n(m\ox a_1 \ox \ldots \ox a_n) = m \cdot a_1 \ox \cdots \ox a_n \\
		 &\qquad + \sum_{i = 1}^{n -1}(-1)^{i} m \ox a_1 \ox \cdots \ox a_i a_{i + 1} \ox \cdots \ox a_n
		 	+ (-1)^n a_n \cdot m \ox a_1 \ox \cdots \ox a_{n-1}.
\end{align*}
para todo $n \geq 1$.

Para calcular la cohomología de Hochschild aplicamos el funtor $\Hom_{A^e}(-,M)$ al complejo bar y obtenemos
\begin{align*}
	\xymatrix{
		\ldots & \Hom_{A^e}\left(A^{\ox 4}, M\right) \ar[l]_{d'_2} & \Hom_{A^e}\left(A^{\ox 3}, M\right) \ar[l]_{d'_1}
			& \Hom_{A^e}\left(A^{\ox 2}, M\right) \ar[l]_{d'_0} & 0 \ar[l],
	}
\end{align*}
donde $d'_n(f) = f \circ b'_n$ para todo $f \in \Hom_{A^e}\left(A^{\ox n + 2}, M\right)$ y para todo $n \geq 0$.

Nuevamente es posible realizar una simplificación. Para todo
$n \geq 0$,
\[
	\Hom_{A^e}\left(A^{\ox (n + 2)}, M\right) \cong \Hom_{\field}\left(A^{\ox n}, M\right),
\]
mediante el isomorfismo que asocia $\varphi \in \Hom_{A^e}\left(A^{\ox (n + 2)}, M \right)$ con la transformación $\field$-lineal
\[
	(a_1 \ox \ldots \ox a_n) \in A^{\ox n} \mapsto \varphi(1 \ox a_1 \ox \ldots \ox a_n \ox 1) \in M.
\]
Esto reduce el cálculo de la cohomología de Hochschild al siguiente complejo:
\begin{align*}
	\xymatrix{
		\ldots & \Hom_{\field}\left(A^{\ox 2}, M\right) \ar[l]_{d_2} & \Hom_{\field}\left(A, M\right) \ar[l]_{d_1}
			& M \ar[l]_{d_0} & 0 \ar[l],
	}
\end{align*}
donde los diferenciales están dados por
\begin{align*}
	&d_0(m)(a) = a\cdot m - m\cdot a,\\
	&d_n(f)(a_1 \ox \ldots \ox a_{n +1}) = a_1 \cdot f(a_2 \ox \ldots \ox a_{n + 1})\\
	&\qquad + \sum_{i = 1}^{n}(-1)^{i} f(a_1 \ox \cdots \ox a_i a_{i + 1} \ox \cdots \ox a_{n+1})
		 	+ (-1)^{n + 1} f(a_1 \ox \cdots \ox a_{n}) \cdot a_{n + 1}
\end{align*}
para todo $f \in \Hom_{\field}\left(A^{\ox n}, M\right)$ y para todo $n \geq 1$.

Calculemos los primeros dos espacios de cohomología.
\begin{itemize}
	\item El grupo $\Hy^{0}(A, M)$ es isomorfo $\Ker(d_0)$.
	Un elemento $m$ pertenece a $\Ker(d_0)$ si y solo si $a\cdot m - m \cdot a = 0$ para todo $a \in A$. En particular,
	si $M = A$, resulta que $\Hy^{0}(A, A)$ es el centro del álgebra.
	
	\item Para obtener una descripción del espacio $\Hy^{1}(A, M)$ vamos a calcular por separado $\Ker(d_1)$
	e $\Inn(d_0)$. Dado $f \in \Hom_{\field}\left(A, M\right)$, $f$ pertenece a $\Ker(d_1)$ si y solo si
	\[
		0 = d_1(f)(a_1\ox a_2) = a_1\cdot f(a_2) - f(a_1 a_2) + f(a_1)\cdot a_2
	\]
	para todo $a_1, a_2 \in A$. Por lo tanto, $\Ker(d_1)$ es el conjunto de \emph{derivaciones} $\Der_{\field}(A, M)$. Por otro lado
	la imagen de $d_0$ es el conjunto de \emph{derivaciones interiores} $\Inn_{\field}(A, M)$, es decir, los morfismos de la forma
	$a \mapsto a\cdot m - m\cdot a$ para algún $m \in M$. Luego $\Hy^{1}(A, M)$ resulta ser isomorfo al conjunto de \emph{derivaciones exteriores}
	\[
		\Der_{\field}(A, M) / \Inn(A, M).
	\]
	Notar que si $M = A$, entonces $\Der_{\field}(A) = \Der_{\field}(A,A)$ es un álgebra de Lie e $\Inn(A)$ es un ideal de Lie.
	Por lo tanto $\Hy^{1}(A, A)$ es un álgebra de Lie.
	
\end{itemize}

\section{Estructura de álgebra de Gerstenhaber}\label{hochschild_gerstenhaber}

Sea $A$ una $\field$-álgebra. La cohomología de Hochschild de $A$ con coeficientes en $A$ tiene estructura
de álgebra graduada super conmutativa y estructura de super álgebra de Lie junto con cierta relación de compatibilidad.

\begin{definition}
	Un \emph{álgebra de Gerstenhaber} es un álgebra graduada super conmutativa $G^{\bullet} = \oplus_{i}G^i$ junto con
	un corchete $\left[\cdot, \cdot \right]: G^{p} \times G^{q} \to G^{p + q - 1}$
	tal que $\left(G^{\bullet}, \left[\cdot, \cdot \right]\right)$ es una super álgebra de Lie y para todo $a \in G^{\bullet}$,
	el morfismo $b \in G^{\bullet} \mapsto \left[a,b\right] \in G^{\bullet}$
	es una super derivación con respecto al producto, es decir,
	\[
		\left[a, b c\right] = \left[a,b\right]c + (-1)^{(\gr(a) - 1)\gr(b)}b\left[a, c\right]
	\]
	para todo $a, b, c \in G^{\bullet}$.
\end{definition}

Empecemos dandole estructura de álgebra graduada a $\oplus_{n \geq 0}\Hom_{\field}\left(A^{\ox n}, A\right)$.
\begin{definition}
	Sean $f \in \Hom_{\field}\left(A^{\ox n}, A\right)$ y $g \in \Hom_{\field}\left(A^{\ox m}\right)$, se define el \emph{producto cup}
	$f \smile g \in \Hom_{\field}\left(A^{\ox n + m}, A\right)$ como la única transformación $\field$-lineal que cumple que
	\[
		f\smile g(a_1\ox \ldots \ox a_n \ox a_{n + 1} \ox \ldots \ox a_{n + m}) = f(a_1\ox \ldots \ox a_n)g(a_{n + 1} \ox \ldots \ox a_{n + m})
	\]
	para todo $a_1, \ldots, a_{n + m} \in A$.
\end{definition}
Los diferenciales del complejo resultan ser super derivaciones con respecto a este producto, es decir, si $f$ y $g$ son dos
elementos homogeneos de grados $n$ y $m$ respectivamente, entonces
\[
	d_{n + m}(f \smile g) = d_{n}(f) \smile g + (-1)^{nm}f \smile d_{m}(g).
\]
Esta formula muestra que el producto de dos cociclos es un cociclo y que el producto de un coborde con un cociclo
es un coborde. Por lo tanto $\Hy^{\bullet}(A,A)$ resulta ser una $\field$-álgebra graduada con el producto inducido por el producto cup.

También existe en $\oplus_{n \geq 0}\Hom_{\field}\left(A^{\ox n}, A\right)$ una estructura de super álgebra de Lie.
\begin{definition}
Sean $f \in \Hom_{\field}\left(A^{\ox n}, A\right)$ y $g \in \Hom_{\field}\left(A^{\ox m}\right)$, se define el \emph{asociador}
entre $f$ y $g$, $f \circ g \in \Hom_{\field}\left(A^{\ox n + m - 1}, A\right)$ como la única transformación $\field$-lineal que cumple que
	\begin{align*}
		&f \circ g (a_1\ox \ldots \ox a_{n + m - 1 }) = \\
		&\quad\sum_{i = 1}^n (-1)^{(i - 1)(m -1)}
				f\left(a_1\ox \ldots \ox a_{i - 1} \ox g(a_{i} \ox \ldots \ox a_{i + m - 1}) \ox a_{i + m} \ox \ldots \ox a_{n + m - 1}\right)
	\end{align*}
	para todo $a_1, \ldots, a_{n + m - 1} \in A$.
Se define el \emph{corchete de Gerstenhaber} entre estos dos elementos como
\[
	\left[f, g\right] = f \circ g  - (-1)^{(n -1)(m -1)}g \circ f.
\] 
\end{definition}

El producto cup y el corchete de Gerstenhaber hacen de $\Hy^{\bullet}(A, A)$ un álgebra de Gerstenhaber.

\section{Sucesiones espectrales}\label{sucesiones_espectrales}

En esta sección daremos una introducción a las sucesiones espectrales que provienen de una filtración de un complejo.


\subsection{Definiciones básicas}
Empecemos considerando el problema de calcular la homología del complejo total $T_{\bullet}$ de un complejo doble
$E_{\bullet, \bullet}$ de espacios vectoriales, cuyas únicas columnas no necesariamente nulas son  $E_{0, \bullet}$ y $E_{1, \bullet}$.
\begin{align*}
    \xymatrix@=3em{
        &\ar@{-->}[d]& \ar@{-->}[d]& \ar@{-->}[d]& \ar@{-->}[d]& \\
        & 0 \ar@{-->}[l] \ar[d]& E_{0, q + 2} \ar[l]_{d^{h}_{0, q + 2}} \ar[d]_{d^{v}_{0, q + 2}}
            & E_{1,q + 2} \ar[l]_{d^{h}_{1, q + 2}} \ar[d]_{d^{v}_{1, q + 2}} & 0 \ar[l]_{d^{h}_{2, q + 2}} \ar[d] & \ar@{-->}[l]\\
        & 0 \ar@{-->}[l] \ar[d]& E_{0, q + 1} \ar[l]_{d^{h}_{0, q + 1}} \ar[d]_{d^{v}_{0, q + 1}}
            & E_{1,q + 1} \ar[l]_{d^{h}_{1, q + 1}} \ar[d]_{d^{v}_{1, q + 1}} & 0 \ar[l]_{d^{h}_{2, q + 1}} \ar[d]& \ar@{-->}[l]\\
        & 0 \ar@{-->}[l] \ar[d]& E_{0, q} \ar[l]_{d^{h}_{0, q}} \ar[d]_{d^{v}_{0, q}}
            & E_{1,q} \ar[l]_{d^{h}_{1, q}} \ar[d]_{d^{v}_{1, q}} & 0 \ar[l]_{d^{h}_{2, q}} \ar[d] &\ar@{-->}[l]\\
        & 0  \ar@{-->}[l] \ar@{-->}[d] & E_{0, q - 1} \ar[l]_{d^{h}_{0, q - 1}} \ar@{-->}[d]
            & E_{1, q- 1} \ar[l]_{d^{h}_{1 , q -1}} \ar@{-->}[d] & 0 \ar[l]_{d^{h}_{2, q - 1}} \ar@{-->}[d] &  \ar@{-->}[l]\\
        &&&&&
    }
\end{align*}

Como primer paso, calculamos la homología vertical $\Hy_{\bullet}(E_{p, \bullet})$ para todo $p$. Denotamos por $E_{p, q}^{1}$
al espacio vectorial $\Hy_{q}\left(E_{p, \bullet}\right)$ y lo ubicamos en el punto $(p, q)$  de un nuevo reticulado junto con los diferenciales horizontales inducidos. 

\begin{align*}
    \xymatrix@=3em{
        && & & & \\
        & 0 \ar@{-->}[l] & E^1_{0, q + 2} \ar[l]_{d^{h}_{0, q + 2}}
            & E^1_{1,q + 2} \ar[l]_{d^{h}_{1, q + 2}} & 0 \ar[l]_{d^{h}_{2, q + 2}}  & \ar@{-->}[l]\\
        & 0 \ar@{-->}[l] & E^1_{0, q + 1} \ar[l]_{d^{h}_{0, q + 1}} 
            & E^1_{1,q + 1} \ar[l]_{d^{h}_{1, q + 1}} & 0 \ar[l]_{d^{h}_{2, q + 1}} & \ar@{-->}[l]\\
        & 0 \ar@{-->}[l] & E^1_{0, q} \ar[l]_{d^{h}_{0, q}} 
            & E^1_{1,q} \ar[l]_{d^{h}_{1, q}}  & 0 \ar[l]_{d^{h}_{2, q}}  &\ar@{-->}[l]\\
        & 0  \ar@{-->}[l]  & E^{1}_{0, q - 1} \ar[l]_{d^{h}_{0, q - 1}}  & E^{1}_{1, q - 1} \ar[l]_{d^{h}_{1, q - 1}} & 0 \ar[l]  &  \ar@{-->}[l]\\
        &&&&&
    }
\end{align*}

Denotamos por $E_{p, q}^2$ al espacio de homología horizontal $\Hy_{p}\left(E_{\bullet, q}^{1}\right)$. El objetivo
es encontrar, para cada $q$, la relación entre los espacios $E_{0, q + 1}^{2}$,  $E_{1, q}^{2}$ y $H_{q + 1}(T_{\bullet})$.
Como solo las primeras dos columnas no son nulas, escribimos $(a, b) \in E_{0, q + 1} \times E_{1, q}$ cuando nos referimos
a un elemento de $T_{q + 1}$ y llamamos $d_{\bullet}$ a los diferenciales del complejo total.
Si $a$ pertenece a $\Ker\left(d_{0,q + 1}^{v}\right)$, resulta que
\[
    d_{q + 1}(a, 0) = \left(d_{0, q + 1}^{v}(a) + d_{1, q}^{h}(0), d_{1, q}^{v}(0)\right) = (0, 0).
\]
Esta observación permite definir el morfismo
\[
    \hat{\iota}: a \in \Ker\left(d_{0,q + 1}^{v}\right) \mapsto  [(a, 0)] \in \Hy_{q + 1}(T).
\]
Sea $a \in \Ker\left(d_{0,q + 1}^{v}\right)$. Si $a = d_{0, q + 2}^{v}(b)$ para algún $b \in E_{0, q + 2}$, entonces $(a, 0) = d_{q + 2}(b, 0)$,
y si $a = d_{1, q + 1}^{h}(c)$ para algún $c \in \Ker\left(d_{1, q + 1}^{v}\right)$, resulta que $(a, 0) = d_{q + 2}(0, c)$. Luego 
el morfismo $\hat{\iota}$ pasa al cociente y obtenemos un nuevo morfismo inyectivo
\[
    \iota: E_{0, q + 1}^{2} \to H_{q + 1}(T).
\]
Por otro lado, el elemento $(a, b)$ pertenece a $\Ker\left(d_{q + 1}\right)$ si y solo si $b$ pertence a $\Ker\left(d_{1, q}^{v}\right)$
y $d_{0, q + 1}^{v}(a) = - d_{1, q}^{h}(b)$. Esto permite definir un morfismo
\[
    \pi: [(a, b)] \in \Hy_{q + 1}(T)  \mapsto  [b] \in E_{1, q}^{2}.
\]
Es claro que $\Ima(\iota)$ esta incluida en $\Ker(\pi)$. Sea $[(a, b)] \in \Ker(\pi)$, queremos encontrar un elemento $a'$ tal que
$(a' - a, b)$ pertenezca a $\Ima\left(d_{q + 2}\right)$. Como $\pi([(a, b)]) = 0$, existe $c \in E_{1, q + 1}$ tal que $d_{1, q + 1}^{v}(c) = b$,
y como $(a, b)$ pertenece a $\ker(d_{q + 1})$, se cumple que
\[
    d_{0, q + 1}^{v}(a) = -d_{1, q}^{h}(b) = -d_{1, q}^{h} d_{1, q + 1}^{v}(c) = d_{0, q + 1}^{v} d_{1, q+ 1}^{h}(c).
\]
Por lo tanto $d_{0, q + 1}^{v} \left(a - d_{1, q+ 1}^{h}(c)\right) = 0$ y si llamamos $a' = a - d_{1, q+ 1}^{h}(c)$, obtenemos lo que buscabamos.
Acabamos de probar que existe la siguiente sucesión exacta corta
\begin{align*}
\xymatrix{
    0 \ar[r] & E_{0, q + 1}^{2} \ar[r]^{\iota} & H_{q + 1}(T) \ar[r]^{\pi} & E_{1, q }^{2} \ar[r] & 0,
}
\end{align*}
que relaciona $E_{0, q + 1}^{2}$ y $E_{1, q}^{2}$ con $H_{q + 1}(T)$. Como estamos trabajando con espacios vectoriales, $\pi$ es una retracción y
resulta que
\[
    H_{q + 1}(T) \cong E_{0, q + 1}^{2} \oplus E_{1, q}^{2}.
\]

\begin{definition}
    Una \emph{sucesión espectral de homología (que empieza en a)} en una categoría abeliana $\mathfrak{A}$ consiste de los siguientes objetos:
    \begin{itemize}
        \item Una familia $\left\lbrace E_{p, q}^{r} \right\rbrace$ de objetos de $\mathfrak{A}$ definidos para todo $p, q, r\in \ZZ$, $r\geq a$.
        \item Una familia de morfismos  $\left\lbrace d_{p, q}^{r}: E_{p, q}^{r} \to E_{p - r, q + r - 1}^{r} \right\rbrace$ que resultan
        ser diferenciales, es decir $d_{p, q}^{r} \circ d_{p + r, q - r + 1}^{r} = 0$, de manera que la recta de pendiente $-(r + 1)/r$ en el reticulado $E_{\bullet, \bullet}^{r}$ resulte ser un complejo.
        \item Isomorfismos entre $E_{p, q}^{r + 1}$ y la homología de $E^{r}_{\bullet, \bullet}$ en el punto $E_{p, q}^{r}$:
        \[
           E_{p, q}^{r + 1} \cong \Ker\left(d^{r}_{p, q}\right) / \Ima\left(d^{r}_{p + r, q - r + 1}\right).
        \]
    \end{itemize}
    El \emph{grado total} de $E_{p, q}^{r}$ es $n = p + q$. Los términos de grado total $n$ yacen en una recta de pendiente $-1$ y cada
    diferencial hace decrecer el grado total por uno. Llamamos \emph{páginas} a cada reticulado $E_{\bullet, \bullet}^{r}$ para todo $r\geq a$.
\end{definition}

Una sucesión espectral de homología ubicada en \emph{el primer cuadrante}  es una sucesión espectral tal que $E_{p, q}^{r} = 0$ a menos que
$p \geq 0$ y $q \geq 0$. Si fijamos $p$ y $q$, entonces $E_{p, q}^{r} = E_{p, q}^{r + 1}$ para todo $r$ los suficientemente grande. Esto se debe
a que eventualmente el dominio de $d^{r}_{p + r, q - r + 1}$ va a estar ubicado en el cuarto cuadrante y el codominio de $d^{r}_{p, q}$ en el segundo cuadrante. Denotamos por $E_{p, q}^{\infty}$ a este valor estable de $E_{p, q}^{r}$.

\begin{definition}
    Una \emph{sucesión espectral de cohomología (que arranca en a)} en $\mathfrak{A}$ es una familia $\left\lbrace E^{p, q}_{r} \right\rbrace$
    de objetos de $\mathfrak{A}$ junto con morfismos $\left\lbrace d^{p, q}_{r}: E^{p, q}_{r} \to E^{p + r, q - r + 1}_{r} \right\rbrace$
    que son diferenciales, es decir $d_r \circ d_r = 0$, e isomorfismos entre $E_{r + 1}$ y la homología de $E_r$.
\end{definition}

\begin{definition}
    Una sucesión espectral de homología  se dice \emph{acotada} si para todo $n$ existe una cantidad finita del terminos no nulos de grado total
    $n$ en $E^{a}_{\bullet,\bullet}$. Recordar que $E^{a}_{\bullet,\bullet}$ es la primer página. Si este es el caso, para cada $p, q$ existe 
    un $r_0$ tal que $E_{p, q}^{r} = E_{p, q}^{r_0}$ para todo $r \geq r_0$.
    Denotamos por $E_{p, q}^{\infty}$ a este valor estable de $E_{p, q}^{r}$. Decimos que una sucesión espectral acotada \emph{converge}
    a $\Hy_{\bullet}$ si existe una familia de objetos $\left\lbrace \Hy_{n} \right\rbrace_{n \in \ZZ}$ en $\mathfrak{A}$, cada uno con una
    filtración finita
    \[
        0 = F_s \Hy_n \subseteq \cdots \subseteq F_{p - 1}\Hy_n \subseteq
            F_{p}\Hy_n \subseteq F_{p + 1}\Hy_n \subseteq \cdots \subseteq F_t \Hy_n = 0
    \]
    tal que $E_{p, q}^{\infty} \cong F_{p}\Hy_{p + q} / F_{p - 1} \Hy_{p + q}$. Si la sucesión espectral $E$ converge a $\Hy_{\bullet}$
    escribimos $E_{p, q}^{a} \Rightarrow H_{p + q}$.
    
    Una sucesión espectral de cohomología  se dice \emph{acotada} si para todo $n$ existe una cantidad finita del terminos no nulos de grado total
    $n$ en $E_{a}^{\bullet,\bullet}$. Decimos que una sucesión espectral acotada converge a $\Hy^{\bullet}$ si existe una familia de objetos 
    $\left\lbrace \Hy^{n} \right\rbrace_{n \in \ZZ}$ en $\mathfrak{A}$, cada uno con una
    filtración finita
    \[
        0 = F^t \Hy^n \subseteq \cdots \subseteq F^{p + 1}\Hy^n \subseteq
            F^{p}\Hy^n \subseteq F^{p - 1}\Hy^n \subseteq \cdots \subseteq F^s \Hy^n = 0
    \]
    tal que $E^{p, q}_{\infty} \cong F^{p}\Hy^{p + q} / F^{p + 1} \Hy^{p + q}$.
\end{definition}


Una \emph{filtración} $F$ de un complejo $C$ es una familia ordenada de subcomplejos de  $C$
\[
        \cdots \subseteq F_{p - 1}C \subseteq F_{p}C \subseteq \cdots .
\]
Como en los cálculos que se harán en esta
tesis las sucesiones espectrales se estabilizaran en la segunda página, al igual que en el ejemplo introductorio de esta sección,
omitimos la demostración del siguiente teorema. Ver el capitulo 5 de \myworries{[W]}.

\begin{Teorema} Una filtración $F$ de un complejo $C$ determina una sucesión espectral que empieza con
$E_{p, q}^{0} = F_{p} C_{p + q} / F_{p - 1}C_{p + q}$ y continua con $E_{p, q}^{1} = \Hy_{p + q}(E_{p, \bullet}^{0})$.
\end{Teorema}

Una filtración de un complejo $C$ induce una filtración en la homología de $C$, el espacio $F_p \Hy_{n}(C)$ es la imagen
del morfismo $\Hy_{n}(F_p C) \to \Hy_{n}(C)$.

\begin{Teorema}
    Si $C$ un complejo y $F$ una filtración de $C$ acotada, entonces la sucesión espectral es acotada y converge a $\Hy_{\bullet}(C)$:
    \[
        E_{p, q}^{1} = H_{p + q}(F_p C / F_{p - 1}(C)) \Rightarrow H_{p + q}(C).
    \]
\end{Teorema}

Hay dos filtraciones usuales asociadas a cualquier complejo doble $C$ que resultan en sucesiones espectrales relacionadas con la homología
del complejo total $\Tot(C)$. En esta tesis vamos a hacer uso de una de estas.

\begin{definition} (Filtración por columnas) Si $C$ es un complejo doble, se puede filtrar el complejo total $\Tot(C)$ de la siguiente manera. Sea $\prescript{1}{}{F_{n}}\Tot(C)$ el complejo total del subcomplejo doble de $C$
\[
	\left(\prescript{}{\tau \leq n}{C}\right)_{p, q} = \left\{\begin{array}{cc}
    C_{p, q} &\text{ si } p \leq n,\\
    &\\
    0 & \text{ si no}. \\
	\end{array}\right.
\]
Esta filtración da lugar a una sucesión espectral $\left\lbrace E_{p, q}^{r}\right\rbrace$ que empieza con $E_{p, q}^{0} = C_{p, q}$. Los morfismos
$d^{0}$ son los morfismos verticales $d^{v}$ de $C$ y por lo tanto $E^{1}_{p, q} = \Hy_{q}^{v}(C_{p, \bullet})$. Los morfismos $d^{1}$
son los morfismos horizontales $d^{h}$ inducidos en al homología. Si $C$ es un complejo doble ubicado en el primer cuadrante
la filtración es acotada y por lo tanto la sucesión espectral es convergente: $E_{p, q}^{0} \Rightarrow H_{p + q}(\Tot(C))$. 
\end{definition}

La otra sucesión espectral usual es la que surge de filtrar por filas y se define de manera análoga. 

\end{document}








